\pdfoutput=1
\documentclass[10pt]{article}
\usepackage[numbers,sort&compress]{natbib} % numeric, compress ranges
\setcitestyle{super,sort&compress}         % make them superscripts like PRL
\usepackage[margin=1in]{geometry}
\usepackage{amsmath,amssymb,mathtools,bm}
\usepackage{siunitx}
\sisetup{separate-uncertainty=true}
\usepackage[hidelinks]{hyperref}
\usepackage[nameinlink,capitalise]{cleveref}
\usepackage{microtype}
\usepackage{booktabs}
\usepackage{enumitem}
\usepackage{caption}  % only needed once, doesn't interfere with anything else
\usepackage{tikz,pgfplots}
\pgfplotsset{compat=1.18}
\usetikzlibrary{arrows.meta,positioning,calc}
\usepackage{listings}
\usepackage[T1]{fontenc}
\usepackage{inconsolata} % nicer mono
\usepackage{xcolor}
\usepackage{lmodern}

\lstdefinestyle{pyclean}{
  language=Python,
  basicstyle=\ttfamily\small,
  numbers=left,
  numberstyle=\scriptsize\color{black!50},
  numbersep=8pt,
  frame=single,
  rulecolor=\color{black!20},
  frameround=ffff,
  showstringspaces=false,
  breaklines=true,
  breakatwhitespace=true,
  tabsize=2,
  upquote=true,
  columns=fullflexible,
  keywordstyle=\bfseries\color{blue!60!black},
  commentstyle=\itshape\color{black!55},
  stringstyle=\color{teal!60!black},
  % Map unicode to LaTeX (so Δ renders)
  literate=
    {Δ}{{$\Delta$}}1
    {α}{{$\alpha$}}1
    {→}{{$\to$}}1
}
% tcolorbox
\usepackage[most]{tcolorbox} % loads common libraries
\tcbuselibrary{breakable,skins,theorems,hooks}

% Simple gray callout box (breakable)
\newtcolorbox{infobox}[1][]{%
  breakable,
  enhanced,
  colback=gray!4!white,
  colframe=black!30,
  boxrule=0.4pt,
  arc=2pt,
  left=6pt,right=6pt,top=6pt,bottom=6pt,
  fonttitle=\bfseries,
  title=#1
}


% ==============================
% GAGE Macros (MS-bar@MZ default) — CANON
% ==============================

% ---------- Fundamental constants ----------
\providecommand{\GN}{G_{\mathrm N}}
\providecommand{\MPl}{M_{\rm P}}                    % M_P^2 = 1/(8\pi G_N)
\providecommand{\Mp}{m_p}
\providecommand{\hbarc}{\hbar c}

% ---------- Reference / emergent scales ----------
\providecommand{\MZ}{M_Z}
\providecommand{\Mstar}{M_\ast}                     % optional dressed Planck mass

% ---------- Gauge couplings (MS-bar@MZ: hats by default) ----------
\providecommand{\alphas}{\hat{\alpha}_s}
\providecommand{\alphaTwo}{\hat{\alpha}_2}
\providecommand{\alphae}{\hat{\alpha}}
\providecommand{\alphaGpp}{\alpha^{(\mathrm{pp})}_{G}}

% ---------- Integer projection / depth ----------
\providecommand{\chis}{\boldsymbol{\chi}}           % (16,13,2)
\providecommand{\Psivec}{\hat{\boldsymbol{\Psi}}}   % (ln \hatα_s, ln \hatα_2, ln \hatα)
\providecommand{\XiDepth}{\hat{\Xi}}                % \hatΞ = χ·\hatΨ
\providecommand{\XiEq}{\hat{\Xi}^{(\mathrm{eq})}}
\providecommand{\DXi}{\Delta\hat{\Xi}}
% --- Inverse-coupling logs (used in depth \Xi with 1/alpha) ---

% ---------- Gate (projection, parity-even) ----------
\DeclareRobustCommand{\PiGate}[1]{\Pi\!\left(#1\right)} % Π(Ξ)
\providecommand{\sigchi}{\sigma_{\chi}}                 % σ_χ
% Canonical emergent coupling (no metrology input):
% G := (ħ c / m_p^2) Ω_χ; local running G(x) = G Π(Ξ)
\providecommand{\G}{G}
\newcommand{\Gx}{\G\,\PiGate{\XiDepth}}
% Legacy alias: avoid implying G_N in derivations; map Geffx → Gx
\providecommand{\Geffx}{\Gx}
\providecommand{\Gstar}{G_\ast}                         % equilibrium coupling (optional)

\newif\ifUseSI      \UseSItrue       % true: Ωχ = G_N m_p^2 / (ħ c); false: natural units
\newcommand{\maybeBox}[1]{\ifPRLBoxes\boxed{#1}\else#1\fi}
\newcommand{\SMpins}{\text{(SM pins @ }\MZ\text{, }\MSbar\text{)}} % helper text for captions
\let\Gate\PiGate

% ---------- Field-space kinetic metric ----------
\providecommand{\Kfs}{\mathbf{K}(\Psivec)}
\providecommand{\Keq}{\mathbf{K}_{\rm eq}}
\DeclareRobustCommand{\normK}[1]{\ensuremath{\left\lVert#1\right\rVert_{\Keq}}}

% ---------- Derived canonicals ----------
\providecommand{\ohel}{\omega_{\mathrm{hel}}}       % ω_hel ≡ ||χ||_K / σ_χ
\providecommand{\thel}{T_{\mathrm{hel}}}            % T_hel ≡ 2π / ω_hel
\providecommand{\Lchi}{\Lambda_\chi}    % defined below from σ_χ and ||χ||_K

% ---------- Math utilities ----------
\providecommand{\LN}{\ln}
\providecommand{\dd}{\mathrm{d}}
\providecommand{\vev}{\langle\,\cdot\,\rangle}
\DeclareMathOperator{\tr}{tr}
\providecommand{\abs}[1]{\ensuremath{\lvert#1\rvert}}
\providecommand{\norm}[1]{\ensuremath{\lVert#1\rVert}} % use lowercase l/r Vert
\providecommand{\order}[1]{\ensuremath{\mathcal{O}(#1)}}
\providecommand{\MSbar}{\overline{\mathrm{MS}}}
\newcommand{\alphavec}{\hat{\boldsymbol{\alpha}}}

% ---------- GAGE invariant (Ω_χ) ----------
\providecommand{\Omega}{\Omega_{\chis}}          % Ω_χ ≡ α_s^{16} α_2^{13} α^{2}

% --------- Boxes / identities ----------
\newcommand{\OmegaProdBox}{\boxed{\Omega_{\chis} \;=\; \alphas^{16}\,\alphaTwo^{13}\,\alphae^{2}}}
\newcommand{\GageLawBoxNat}{\boxed{\Omega_{\chis} \;=\; \GN\,\Mp^{2}}}
\newcommand{\GageLawBoxSI}{\boxed{\Omega_{\chis} \;=\; \GN\,\Mp^{2} \big/ \hbarc}}

% ---------- Electroweak shorthand ----------
\providecommand{\thetaW}{\hat{\theta}_{\mathrm W}}
\providecommand{\sWsq}{\sin^2\!\thetaW}

% ---------- Beta monitors ----------
\providecommand{\betaXi}{\beta_{\Xi}}               % d\Xi/d\ln Q
\providecommand{\dlnQ}{\dd\!\LN Q}

% ---------- Display boxes ----------
\newcommand{\GateBox}{\boxed{\frac{\Geffx}{\GN} \;=\; \PiGate{\XiDepth}
\;=\; \exp\!\Big[-\frac{(\XiDepth-\XiEq)^2}{\sigchi^2}\Big]}}
\newcommand{\ParityNullBox}{\boxed{\frac{\Delta G}{G}\;\simeq\;\frac{\DXi^2}{\sigchi^2}\quad(\partial_{\Xi}\Pi|_{\XiEq}=0)}}

% ---------- Helicity/width relations (explicit ties) ----------
\newcommand{\LchiDef}{\boxed{\Lchi \;=\; \frac{\sigchi}{\normK{\chis}}}}
\newcommand{\HelicityDefs}{\boxed{\ohel \;=\; \frac{\normK{\chis}}{\sigchi}\,,\quad \thel \;=\; \frac{2\pi}{\ohel} \;=\; 2\pi\,\Lchi}}

% ---------- Figure labels (optional) ----------
\newcommand{\FigGate}{Fig.~S3}
\newcommand{\TabPins}{Table~S0.1}

% ---------- Legacy aliases (safe deprecation) ----------
\let\Xsym\Omega
\let\calX\Omega
\let\ClosureBox\GageLawBoxNat

% ------- Global toggles (space-saving) -------
\newif\ifPRLBoxes   \PRLBoxestrue    % set \PRLBoxesfalse to remove \boxed{}
\newif\ifUseSI      \UseSItrue       % true: Ωχ = G_N m_p^2 / (ħ c); false: natural units


% ------- One canonical GAGE law (choose units via \UseSItrue) -------
\newcommand{\GageLawBox}{%
  \ifUseSI
    \maybeBox{\Omega_{\chis} \;=\; \GN\,\Mp^{2}\big/\hbarc}%
  \else
    \maybeBox{\Omega_{\chis} \;=\; \GN\,\Mp^{2}}%
  \fi
}

% ------- Canonical ΔG/G macro (avoid drift) -------
\newcommand{\FracDG}{\frac{\Delta G}{G}}

% ------- Replace existing boxes with togglable versions -------
\renewcommand{\OmegaProdBox}{\maybeBox{\Omega_{\chis} \;=\; \alphas^{16}\,\alphaTwo^{13}\,\alphae^{2}}}
\renewcommand{\GateBox}{\maybeBox{\frac{\Geffx}{\GN} \;=\; \PiGate{\XiDepth}
\;=\; \exp\!\Big[-\frac{(\XiDepth-\XiEq)^2}{\sigchi^2}\Big]}}
\renewcommand{\ParityNullBox}{\maybeBox{\FracDG\;\simeq\;\frac{\DXi^2}{\sigchi^2}\quad(\partial_{\Xi}\Pi|_{\XiEq}=0)}}
\renewcommand{\LchiDef}{\maybeBox{\Lchi \;=\; \frac{\sigchi}{\normK{\chis}}}}
\renewcommand{\HelicityDefs}{\maybeBox{\ohel=\frac{\normK{\chis}}{\sigchi}\,,\quad \thel=\frac{2\pi}{\ohel}=2\pi\,\Lchi}}

% --- xi Log-coupling components (MS-bar @ M_Z) ---
\providecommand{\xihats}{\xihat_s}      % ln \hat{\alpha}_s
\providecommand{\xihatw}{\xihat_2}    % ln \hat{\alpha}_2
\providecommand{\xihata}{\xihat_\alpha}      % ln \hat{\alpha}

% Optional: generic log-coupling shorthand
\providecommand{\xihat}{\hat{\xi}}        % generic component label


% ---------- SI units ----------
\DeclareSIUnit{\eV}{eV}
\DeclareSIUnit{\MeV}{MeV}
\DeclareSIUnit{\GeV}{GeV}
\DeclareSIUnit{\fm}{fm}

\DeclareMathOperator{\diag}{diag}
\DeclareMathOperator{\SNF}{SNF}
\DeclareMathOperator{\rank}{rank}
\DeclareMathOperator{\argmin}{argmin}
\DeclareMathOperator{\argmax}{argmax}
% --- Probability / statistics operators ---
\DeclareMathOperator{\Var}{Var}
\DeclareMathOperator{\Cov}{Cov}
\DeclareMathOperator{\E}{\mathbb{E}}
\DeclareMathOperator{\Corr}{Corr}

% --- Table convenience (for “not available”) ---
\newcommand{\NA}{\text{N/A}}

\begin{document}


\title{Supplemental Material for: Gauge-Aligned Gravity Emergence (GAGE)}
\author{Michael DeMasi DNP}
\date{\today}
\maketitle

\section*{S0. Notation, pins, and conventions}

\paragraph*{Provenance and sources.}
All numerical inputs, constants, and coupling data used herein are taken from established references. 
The Standard Model framework and renormalization conventions follow Weinberg~\cite{Weinberg1996_QFTv2}, 
Peskin--Schroeder~\cite{PeskinSchroeder1995_QFT}, and Langacker~\cite{Langacker2017_SMBeyond}.
Decoupling and integer-lattice methods follow Appelquist--Carazzone~\cite{AppelquistCarazzone1975_Decoupling},
Kannan--Bachem~\cite{KannanBachem1979_SNF}, and Newman~\cite{Newman1997_SNF}.
Numerical pins and covariance values are drawn from the Particle Data Group and CODATA~\cite{PDG2024,PDG2024_EWReview,PDG2025_GaugeHiggs,CODATA2022_RMP}.
Gravitational and metrological comparisons reference Carroll~\cite{Carroll2004_SG},
Will~\cite{Will2014_LRR_TestsGR}, Bertotti~\cite{Cassini2003_PPN}, and Abbott \textit{et~al.} (LVK)~\cite{LVK2021_TestsGR}.
Running of couplings follows Jegerlehner~\cite{Jegerlehner2019_alphaRun}.
All calculations are performed within $\MSbar$ at $\mu=M_Z$, with no external data sources beyond these references.

\paragraph*{Purpose.}
This Supplemental Material provides the derivations, numeric checks, and reproducibility details referenced in the Letter. 
It fixes all symbols, evaluation points, and unit conventions, and defines the error-propagation and covariance rules used in tables and figures.

\paragraph*{Contents.}
Hatted couplings; $\mu=\MZ$; $\MSbar$ scheme; PDG/CODATA pins with uncertainties; 
unit policy; error-propagation rules; equilibrium metric $K$ and eigenvalues; 
Smith–normal–form certificate for $\chis$; Fisher-width derivation of $\sigchi$; 
Ward-flatness masks; closure and leave-one-out tests; reproducibility script and checksum manifest.


\paragraph*{Coordinates and logs}
Work in log–coupling space with hats denoting $\MSbar$ at $\mu=\MZ$:
\[
\Psivec=\big(\LN\alphas,\ \LN\alphaTwo,\ \LN\alphae\big),\qquad
\chis=(16,13,2),\qquad
\XiDepth=\chis\!\cdot\!\Psivec
\]
and the SM-internal invariant
\[
\hat{\Omega} \equiv e^{\XiDepth}=\alphas^{16}\,\alphaTwo^{13}\,\alphae^{2}\,.
\]
All EFT derivations in this SM proceed in log space; metrology targets are used only later (S5) for closure/LOO validation, not as inputs.

\paragraph*{Units policy (concise)}
\noindent
\textbf{SM pins:} $\MSbar$ at $Q=\MZ$ (hats by default) $\SMpins$ \\
\textbf{EFT derivations:} natural units ($\hbar=c=1$), with explicit $\hbarc$ only when mapping back to SI \\
\textbf{Metrology targets:} SI values (PDG/CODATA) used only in S5 for closure/LOO, never upstream

\paragraph*{Gate and metric}
\[
\frac{\Gx}{\G}=\PiGate{\XiDepth}
=\exp\!\Big[-\frac{(\XiDepth-\XiEq)^2}{\sigchi^{2}}\Big],\qquad
\Keq\succ0,\ \ \normK{\mathbf v}^2=\mathbf v^{\!\top}\Keq\mathbf v
\]
with $\XiEq=\XiDepth\big|_{\mu=\MZ}$, gate width $\sigchi$, and gate scale $\Lchi=\sigchi/\normK{\chis}$. Even parity ($\partial_\Xi\Pi|_{\XiEq}=0$) enforces the quadratic lab-null
\[
\FracDG\simeq\frac{\DXi^2}{\sigchi^2}
\qquad\text{with}\qquad \DXi=\XiDepth-\XiEq\,.
\]

% ==============================
% Notation Summary (Supplement)
% ==============================
\paragraph*{Notation Summary.}
Located at end of the document. Core scheme/pin conventions follow PDG/CODATA; 
the Ward-flatness projector and the one-loop identity $\beta_{\Xi}=0$ (PRL Eq.~(14)) are used later in S5.

\renewcommand{\arraystretch}{1.12}
\begin{table}[t]
\centering
\caption{Symbols used in the Letter and SM. Unless stated otherwise, hats denote $\MSbar$ at $\mu=M_Z$; numerical pins are those quoted in the text/tables.}
\label{tab:notation}
\small
\begin{tabular}{@{}p{0.23\linewidth} p{0.58\linewidth} p{0.17\linewidth}@{}}
\toprule
\textbf{Symbol} & \textbf{Meaning / role (plain language)} & \textbf{Value / where} \\
\midrule
$\chis=(16,13,2)$ & Integer projector (unique primitive SNF left-kernel generator of the 1L decoupling lattice). Selects the aligned soft direction in gauge–log space. & SM S1 (SNF certificate) \\
$\Psivec=(\LN\hat\alpha_s,\LN\hat\alpha_2,\LN\hat\alpha)$ & Log–space coordinate of SM gauge couplings (hats: $\MSbar$). & SM S0 (pins @ $M_Z$) \\
$\Xi=\chis\!\cdot\!\Psivec$ & Gauge–log depth (scalar projection along $\chis$); basis invariant. & Def.; used throughout \\
$\XiEq$ & Equilibrium depth (gate center). & SM S3 (gate/parity) \\
$\DXi=\Xi-\XiEq$ & Displacement controlling parity–even response of $G$. & SM S3 (parity lemma) \\
$\Pi(\Xi)=\exp[-\DXi^2/\sigchi^2]$ & Even Gaussian curvature gate; $\Pi'(\XiEq)=0$ (no linear term). & SM S3 (gate) \\
$G \equiv \dfrac{\hbar c}{m_p^2}\,\Omega$ & Equilibrium gravitational coupling derived solely from SM couplings (no $G_N$ input). & SM S3 (definition) \\
$G(x)=G\,\Pi(\Xi(x))$ & Local/spacetime running of $G$ through the gate. & SM S3 \\
$\Omega=\hat\alpha_s^{16}\hat\alpha_2^{13}\hat\alpha^{2}$ & SM-internal invariant linking gauge sector to gravity. & SM S3 \\
$\alpha_G^{(\mathrm{pp})}=\dfrac{G_N m_p^2}{\hbar c}$ & Dimensionless pp anchor for closure/matching to $G_N$. & SM S0 (targets), S6 (closure) \\
$Z_G \equiv \dfrac{\alpha_G^{(\mathrm{pp})}}{\Omega}$ & UV→IR match factor: $G_N = Z_G\,G$ (scheme/threshold/higher-loop bridge). & $Z_G=0.91430$; SM S6 (matching) \\
$\Keq \succ 0$ & Equilibrium kinetic metric in coupling space; sets inner products/soft mode. & SM S0 (metric), S4 (tensor sector) \\
$\normK{\chis}$ & Norm of $\chis$ in $\Keq$; canonical soft-mode normalization. & $17.6278$ (SM Table 4) \\
$\sigchi$ & Gate width from Fisher curvature; sets lab-null scale. & $247.683$ (SM Table 4) \\
$\Lchi=\sigchi/\normK{\chis}$ & Gate scale (soft-mode coherence length). & $14.0507$ (SM Table 4) \\
$\phi_\chi=\normK{\chis}^{-1}\chis^{\!\top}(\Psivec-\vev{\Psivec})$ & Canonical soft scalar along $\chis$. & SM S3–S4 \\
$\FracDG \simeq \DXi^2/\sigchi^2 = \phi_\chi^2/\Lchi^2$ & Parity–even lab-null prediction (no linear term); falsifier. & SM S3 (parity lemma) \\
$\ohel=\normK{\chis}/\sigchi,\ \ \thel=2\pi/\ohel$ & Helicity frequency and period (Planck-thin envelope). & SM S8 (helicity scales) \\
$J^\mu_\chi = \Pi(\Xi)\,\chis^{\!\top}\Keq\,\partial_\mu\hat\Psivec$ & Conserved alignment current (Noether current of rigid depth shifts); defines alignment–conservation law. & SM S2.3 (conservation), Letter Eq.~(8) \\
$P_\chi=\dfrac{\Keq\,\chis\,\chis^{\!\top}}{\chis^{\!\top}\Keq\chis},\ \ P_\perp=\mathbb{1}-P_\chi$ & Projectors onto soft direction and orthogonal complement. & SM S4 (projectors) \\
$F(Q)=\dd\Xi/\dd\LN Q$ & Ward-flatness monitor (projected RG flow); masked windows. & SM S5 (masks/windows) \\
$\beta_\Xi=\dd\Xi/\dlnQ$ & Projected RG flow; vanishes at one loop (Ward-flatness). & SM S5 \\
$\beta_G=\dd(\LN G)/\dlnQ$ & Running: $16\,\beta_{\alpha_s}/\alpha_s+13\,\beta_{\alpha_2}/\alpha_2+2\,\beta_\alpha/\alpha$. & SM S5 \\
$\Delta\mathcal{L}\, h_{\mu\nu}=-\Box h_{\mu\nu}$ & Lichnerowicz operator: luminal helicity-2, $m_{\mathrm{PF}}=0$. & SM S4 (tensor sector) \\
$\varepsilon_\chi$ & Alignment tolerance parameter collecting higher-order and numerical remainders; bound $\le 10^{-8}$. & SM S0 (notation), S2.3 (remainder) \\
$\MSbar,\ M_Z,\ m_p,\ \hbar c,\ G_N$ & Scheme/scale and constants for pins and comparison. & SM S0; PDG/CODATA \\
\bottomrule
\end{tabular}
\end{table}


\paragraph*{Alignment tolerance}
We define $\varepsilon_\chi$ to capture residual misalignment and modeling tolerances near equilibrium (e.g., $1-\cos\theta$, gate-slope leakage, numerical conditioning). 
Unless stated otherwise, we assume a uniform bound $\varepsilon_\chi \le 10^{-8}$ (see S2.3,S2.7).

\paragraph*{Equilibrium convention}
Pins are $\MSbar$ at $Q=\MZ$; we set $\Pi(\Psivec_{\rm eq})=1$ so $\Gx|_{\rm eq}=\G$. Any identification with metrology (e.g., $G\stackrel{?}{=}\GN$) is tested only in S5.

\paragraph*{Pins and sources (SM pins @ $\mu=\MZ$; metrology targets in S5 only)}
\noindent
Table~\ref{tab:pins_inputs} lists \emph{inputs used in derivations} (SM pins); Table~\ref{tab:pins_targets} lists \emph{closure targets not used as inputs} (metrology). See PRL Table I and Secs.~“Running of G and Ward-flatness”–“Closure and prediction” for definitions. SM pins and electroweak conventions follow PDG; SI targets follow CODATA.


\begin{table}[t]
\centering
\caption{Inputs used in derivations ($\MSbar$ at $\mu=\MZ$). These feed all SM-side calculations.}
\label{tab:pins_inputs}
\scriptsize
\setlength{\tabcolsep}{6pt}
\renewcommand{\arraystretch}{1.1}
\begin{tabular}{@{}lcll@{}}
\toprule
Quantity & Symbol & Value $\pm 1\sigma$ & Source \\
\midrule
Fine structure (MS, $M_Z$) & $\alphae(\MZ)$ & $0.00781525 \pm 0.00000061$ & PDG ($1/\alpha=127.955\pm0.010$)\\
Weak mixing (MS, $M_Z$) & $\sin^2\!\thetaW(\MZ)$ & $0.23129(4)$ & PDG EW review \\
SU(2) coupling & $\alphaTwo(\MZ)=\alphae/\sin^2\!\thetaW$ & $0.03378982 \pm 0.00000641$ & derived from above \\
Strong coupling & $\alphas(\MZ)$ & $0.1180 \pm 0.0009$ & PDG \\
\bottomrule
\end{tabular}
\end{table}

\begin{table}[t]
\centering
\caption{Closure targets (not used as inputs). Used only in S5 to test $\Omega$ against metrology.}
\label{tab:pins_targets}
\scriptsize
\setlength{\tabcolsep}{6pt}
\renewcommand{\arraystretch}{1.1}
\begin{tabular}{@{}lcll@{}}
\toprule
Quantity & Symbol & Value $\pm 1\sigma$ & Source \\
\midrule
Newton constant (SI) & $\GN$ & $6.67430(15)\times 10^{-11}\ \mathrm{m^3\,kg^{-1}\,s^{-2}}$ & CODATA \\
Conversion factor (exact) & $\hbarc$ & $197.3269804\ \mathrm{MeV\,fm}$ & SI/CODATA \\
Proton mass & $\Mp$ & $0.93827208816\ \mathrm{GeV}$ & PDG \\
Proton–proton grav.\ coupling & $\alphaGpp=\dfrac{\GN\,\Mp^{2}}{\hbarc}$ & $(5.90615 \pm 0.00013)\times 10^{-39}$ & derived (unc.\ from $\GN$) \\
\bottomrule
\end{tabular}
\end{table}

\begin{table}[t]
\centering
\caption{Certificate/response parameters (SM internal). Fixed once from $\Keq$ and the gate width.}
\label{tab:cert_params}
\scriptsize
\setlength{\tabcolsep}{6pt}
\renewcommand{\arraystretch}{1.1}
\begin{tabular}{@{}lcll@{}}
\toprule
Quantity & Symbol & Value & Route \\
\midrule
Integer norm & $\chis^{\!\top}\chis$ & $429$ & $\chis=(16,13,2)$ \\
Depth norm & $\normK{\chis}$ & $17.6278$ & $\sqrt{\chis^{\!\top}\Keq\chis}$ \\
\addlinespace[2pt]
Transverse width (strong) & $\sigma_{\alpha_s}$ & $0.446296$ & pin (transverse s.d.) \\
Transverse width (weak) & $\sigma_{\alpha_2}$ & $0.547533$ & pin (transverse s.d.) \\
Transverse width (EM) & $\sigma_{\alpha}$ & $0.551281$ & pin (transverse s.d.) \\
\addlinespace[2pt]
Gate width & $\sigchi$ & $247.683$ & fixed (closure–Fisher curvature; S0.4, S5.5) \\
Gate scale & $\Lchi$ & $14.0507$ & $\sigchi/\normK{\chis}$ \\
\bottomrule
\end{tabular}

\vspace{4pt}
\footnotesize
Notes: PDG/CODATA conventions as cited. $\hbarc$ is exact in SI.
\end{table}

\begin{table}[t]
\centering
\caption{Equilibrium kinetic matrix $\Keq$ in the basis $\Psivec=(\LN\alphas,\LN\alphaTwo,\LN\alphae)$; symmetric and positive definite.}
\label{tab:keq}
\scriptsize
\setlength{\tabcolsep}{10pt}
\renewcommand{\arraystretch}{1.1}
\begin{tabular}{@{}lccc@{}}
\toprule
 & $\LN\alphas$ & $\LN\alphaTwo$ & $\LN\alphae$ \\
\midrule
$\LN\alphas$ & 1.2509 & -0.6202 & -0.1813 \\
$\LN\alphaTwo$ & -0.6202 & 1.5128 & -0.1633 \\
$\LN\alphae$ & -0.1813 & -0.1633 & 3.2362 \\
\bottomrule
\end{tabular}
\end{table}

\begin{table}[t]
\centering
\caption{Eigenvalues and orthonormal eigenvectors of $\Keq$. Components in $(\LN\alphas,\LN\alphaTwo,\LN\alphae)$.}
\label{tab:keq_eigs}
\scriptsize
\setlength{\tabcolsep}{10pt}
\renewcommand{\arraystretch}{1.1}
\begin{tabular}{@{}lcc@{}}
\toprule
Mode & $\lambda_i$ & $e_i^\top$ \\
\midrule
1 (soft) & 0.7243366 & $(\;0.7724942,\;0.6276375,\;0.0965604\;)$ \\
2        & 2.0155976 & $(-0.6313037,\;0.7754715,\;0.0099780)$ \\
3 (stiff)& 3.2599658 & $(-0.0686172,\;-0.0686668,\;0.9952771)$ \\
\bottomrule
\end{tabular}

\vspace{4pt}
\footnotesize
Checks: $e_i\!\cdot\! e_j=\delta_{ij}$, $\Keq e_i=\lambda_i e_i$, $\sum_i \lambda_i=\tr\Keq\approx 6.0$, $\det\Keq>0$. Depth norm $\normK{\chis}=\sqrt{\chis^{\!\top}\Keq\chis}=17.6278$.
\end{table}

\paragraph*{Error propagation and correlations}
\noindent
Unless stated, use linearized Gaussian propagation in vector form:
\[
\mathrm{Cov}(f)=J\,\mathrm{Cov}(x)\,J^{\!\top},\qquad
J_{ai}=\partial_{x_i} f_a,\qquad
\delta f^2=\nabla f^{\!\top}\,\mathrm{Cov}(x)\,\nabla f.
\]
For logarithms,
\[
\delta(\LN x)\simeq \frac{\delta x}{x},\qquad
\mathrm{Cov}(\LN x,\LN y)\simeq \frac{\mathrm{Cov}(x,y)}{xy}.
\]

\textbf{Inputs and correlations}
Include PDG/CODATA covariances where provided (e.g., components entering the running of $\alphae$ to $M_Z$). When unavailable, treat inputs as independent and propagate to derived quantities (e.g., $\alphaTwo=\alphae/\sin^2\!\thetaW$) via the Jacobian above. All reported uncertainties are $1\sigma$.

\textbf{Log-space propagation}
Quantities defined in log coordinates (e.g., $\Psivec$, $\XiDepth$) use the same rules; returns to linear variables use $\sigma(y)\approx y\,\sigma(\LN y)$.

\textbf{Metrology (target-only) handling}
Closure/LOO covariance, metrology depths, and any optional cross-covariances are handled in S5. We do not use metrology in upstream derivations.



\paragraph*{Cross-references and reproducibility}
\noindent
Definitions of $\Omega$, closure, and LOO appear in PRL Eqs.~(7), (18), and (19). 
The SM mirrors the Letter: S1 (SNF certificate), S2 (alignment principle), S3 (gate and parity lemma), 
S4 (tensor sector / no PF mass), S5 (Ward-flatness), S6 (closure and LOO), S7 (environmental lab-null), S8 (helicity scales).

\paragraph*{Versioning and reproducibility}
All pins in Tables~\ref{tab:pins_inputs}, \ref{tab:pins_targets}, \ref{tab:cert_params}, \ref{tab:keq}, and \ref{tab:keq_eigs}
are frozen to the cited PDG/CODATA releases and mirrored locally.
Section~S9 points to the Repo’s \texttt{pins.json} (SI; $\MSbar$ at $\mu=\MZ$) and build scripts that regenerate the S0 tables and figure data from source pins.
The build is deterministic (no network); SHA-256 hashes are emitted for all artifacts, and any drift indicates a pin or version change.
Monte Carlo confirmation of closure/LOO appears only in the SM (Sec.~S6.8) and reproduces the linearized propagation.
The Repo remains deterministic (no RNG) and regenerates numeric tables and figure data from pinned inputs only.

\textbf{Reproduction (pointer to Repo)}
S9 summarizes inputs/outputs and checksums; the complete, executable workflow lives in the accompanying Repo document (source of truth). From the Repo:
\begin{itemize}\setlength\itemsep{2pt}
  \item \texttt{pins.json} and \texttt{keq.json} (MS at $\mu=\MZ$); code derives $\hat\alpha_2=\hat\alpha/\sin^2\!\hat\theta_W$
  \item one-command rebuild: \texttt{bash build.sh} \textit{(or Windows \texttt{build\_win.bat})} → regenerates numeric tables and figure data from pins
  \item optional metrics: \texttt{src/metric\_eigs.py} emits \texttt{metric\_results.json} ($K_{\rm eq}$ eigs/evecs, $\|\chi\|_K$, $\Lambda_\chi$, alignment)
  \item deterministic run; SHA-256 of all artifacts recorded in \texttt{SHA256SUMS.txt}
\end{itemize}
S9 records filenames, versions, and hashes matching the Repo; code listings are omitted here by design.


\paragraph*{Vector form (Jacobian rule)}
\noindent
For a vector map $y=f(x)$ with inputs $x=(x_1,\dots,x_n)$ and outputs $y=(y_1,\dots,y_m)$,
\[
\mathrm{Cov}(y)=J\,\mathrm{Cov}(x)\,J^{\!\top},\qquad
J_{ij}=\frac{\partial y_i}{\partial x_j}.
\]

\textbf{Log domain}\\
Work with MS couplings at $\mu=M_Z$: $\hat\alpha_i\in\{\hat\alpha_s,\hat\alpha_2,\hat\alpha\}$ and define $\xi_i=\ln\hat\alpha_i$. The SM invariant
\[
\hat\Omega=\prod_i \hat\alpha_i^{\,\chi_i}
=\hat\alpha_s^{16}\hat\alpha_2^{13}\hat\alpha^{2}
\]
satisfies
\[
\ln\hat\Omega=\sum_i \chi_i\,\xi_i,
\]
with linearized propagation
\[
\delta(\ln\hat\Omega)^2
=\sum_{i}\chi_i^2\,\delta\xi_i^2
+2\sum_{i<j}\chi_i\chi_j\,\mathrm{Cov}(\xi_i,\xi_j).
\]
For small relative errors,
\[
\delta(\ln\hat\alpha_i)\simeq \frac{\delta\hat\alpha_i}{\hat\alpha_i},\qquad
\mathrm{Cov}(\ln\hat\alpha_i,\ln\hat\alpha_j)\simeq \frac{\mathrm{Cov}(\hat\alpha_i,\hat\alpha_j)}{\hat\alpha_i\hat\alpha_j}.
\]
\textbf{Example (derived SU(2) coupling)}
With $\alphaTwo=\alphae/\sin^2\!\thetaW$,
\[
\LN \alphaTwo = \LN \alphae \;-\; \LN\!\big(\sin^2\!\thetaW\big),\quad
\sigma^2\!\big(\LN \alphaTwo\big) = \sigma^2\!\big(\LN \alphae\big) + \sigma^2\!\big(\LN(\sin^2\!\thetaW)\big)
- 2\,\mathrm{Cov}\!\Big(\LN \alphae,\LN(\sin^2\!\thetaW)\Big),
\]
and $\sigma(\alphaTwo)\approx \alphaTwo\,\sigma\!\big(\LN \alphaTwo\big)$.

\paragraph*{Derived inputs (closed forms used throughout)}

\textbf{(i) SU(2) coupling}
\[
\alphaTwo=\alphae/\sin^2\!\thetaW.
\]
In linear variables (set $\mathrm{Cov}=0$ unless specified):
\[
\delta \alphaTwo^{2}
=\Big(\tfrac{1}{\sin^2\!\thetaW}\Big)^{\!2}\delta\alphae^{2}
+\Big(\tfrac{\alphae}{(\sin^2\!\thetaW)^{2}}\Big)^{\!2}\delta\!\big(\sin^2\!\thetaW\big)^{2}
-2\,\frac{\alphae}{(\sin^2\!\thetaW)^{3}}\,\mathrm{Cov}\!\big(\alphae,\sin^2\!\thetaW\big).
\]
Equivalently, in logs,
\[
\delta\!\big(\LN\alphaTwo\big)^{2}
=\delta\!\big(\LN\alphae\big)^{2}
+\delta\!\big(\LN(\sin^2\!\thetaW)\big)^{2}
-2\,\mathrm{Cov}\!\big(\LN\alphae,\LN(\sin^2\!\thetaW)\big),\quad
\mathrm{Cov}\!\big(\LN\alphae,\LN(\sin^2\!\thetaW)\big)\simeq
\frac{\mathrm{Cov}\!\big(\alphae,\sin^2\!\thetaW\big)}{\alphae\,\sin^2\!\thetaW}.
\]

\textbf{(ii) Projection depth and certificate}
\[
\XiEq=\chis\!\cdot\!\big(\LN\alphas,\LN\alphaTwo,\LN\alphae\big)
=16\,\LN\alphas+13\,\LN\alphaTwo+2\,\LN\alphae.
\]
Work in the independent basis $x=\big(\LN\alphae,\ \LN(\sin^2\!\thetaW),\ \LN\alphas\big)$ with
$\LN\alphaTwo=\LN\alphae-\LN(\sin^2\!\thetaW)$:
\[
\XiEq=15\,\LN\alphae-13\,\LN(\sin^2\!\thetaW)+16\,\LN\alphas,\qquad
g_{\Xi}=(15,\,-13,\,16),
\]
\[
\sigma^2(\XiEq)=g_{\Xi}^{\!\top}\,\mathrm{Cov}(x)\,g_{\Xi},\qquad
\sigma(\Omega)\simeq \Omega\,\sigma(\XiEq).
\]

% ---- Moved to S5 (target-only) ----
% Closure target and LOO gradients are handled in S5.

\paragraph*{Ward-flatness prereg thresholds}
\noindent
Define $F(Q)=\dd\XiDepth/\dd\LN Q$ and the normalized monitor $F_\sigma(Q)=F(Q)/\sigchi$ with masks around thresholds. The preregistered bounds are on $F_\sigma$:
\[
\begin{aligned}
&\text{EW }[80,160]\ \mathrm{GeV}:\ 
\|F_\sigma\|_{\infty}\le 0.01430,\ \mathrm{RMS}(F_\sigma)\le 0.01372,\ |\langle F_\sigma\rangle|\le 0.01372,\\
&\text{Low-GeV }[1,10]\ \mathrm{GeV}:\ 
\|F_\sigma\|_{\infty}\le 0.03535,\ \mathrm{RMS}(F_\sigma)\le 0.02622,\ |\langle F_\sigma\rangle|\le 0.02585.
\end{aligned}
\]
 \begin{table}[t]
\centering
\caption{Preregistered Ward-flatness bounds (on $F_\sigma=F/\sigchi$) used throughout}
\label{tab:S0ward}
\scriptsize
\renewcommand{\arraystretch}{1.1}
\begin{tabular}{@{}lccc@{}}
\toprule
Window & $\|F_\sigma\|_{\infty}$ & RMS$(F_\sigma)$ & $|\langle F_\sigma\rangle|$ \\
\midrule
EW [80,160] GeV & 0.01430 & 0.01372 & 0.01372 \\
Low [1,10] GeV  & 0.03535 & 0.02622 & 0.02585 \\
\bottomrule
\end{tabular}
\end{table}
\noindent\textit{Notes:} Bounds are preregistered from the max across 1L/off and 2L/off runs with a $1.5\times$ inflation and include masked thresholds.


\section*{S1. Smith–Normal–Form (SNF) certificate for $\chis=(16,13,2)$}

\paragraph*{Goal}
Show that $\chis$ is fixed by integer structure alone (unique primitive generator up to sign), independent of masses, scales, or scheme choices within the admissible class.

\paragraph*{Standing assumptions}
SM with three families and one Higgs doublet; GUT-normalized hypercharge ($\alpha_1=\tfrac{5}{3}\alpha_Y$); mass-independent scheme with Appelquist–Carazzone decoupling. Fix a single $U(1)_Y$ integerization so that $U(1)$ weights are integers for each light set:
\[
w_1^{(\mathrm f)}=12\!\!\sum_{\rm Weyl}Y^2,\qquad
w_1^{(\mathrm s)}=3\!\!\sum_{\rm scalars}Y^2.
\]
For $H\!\sim\!(\mathbf 1,\mathbf 2,\tfrac12)$, $\sum Y^2=2\!\times\!(\tfrac12)^2=\tfrac12\Rightarrow w_1(H)=3$.
The ordering of $(\alphas,\alphaTwo,\alphae)$ is a notational convention; permutations simply relabel the components of $\chis$ while the kernel direction (and $\Xi=\chis^{\sf T}\Psi$) is basis-invariant.

% ==============================
% W_Z columns and window masks
% ==============================
\begin{table}[t]
\centering
\caption{Light species columns for $W_{\mathbb Z}$ on a window $W$. Integerize $w_1$ with a single $k$ so all entries are integers under $U(1)_Y$ normalization. $N_g$ = generations, $N_H$ = Higgs doublets.}
\label{T:Wcols}
\renewcommand{\arraystretch}{1.08}
\begin{tabular}{lccccccc}
\hline
Species & Rep $(SU(3),SU(2),Y)$ & dof$_{\rm spec}$ & $2T_{3}$ & $2T_{2}$ & $w_3$ & $w_2$ & $w_1$ \\
\hline
$Q_L$  & $(\mathbf{3},\mathbf{2},\,1/6)$   & $6N_g$ & 1 & 1 & $6N_g$ & $6N_g$ & $k\,\sum Y^2$ \\
$u_R$  & $(\mathbf{3},\mathbf{1},\,2/3)$   & $3N_g$ & 1 & 0 & $3N_g$ & $0$    & $k\,\sum Y^2$ \\
$d_R$  & $(\mathbf{3},\mathbf{1},\,-1/3)$  & $3N_g$ & 1 & 0 & $3N_g$ & $0$    & $k\,\sum Y^2$ \\
$L_L$  & $(\mathbf{1},\mathbf{2},\,-1/2)$  & $2N_g$ & 0 & 1 & $0$    & $2N_g$ & $k\,\sum Y^2$ \\
$e_R$  & $(\mathbf{1},\mathbf{1},\,-1)$    & $1N_g$ & 0 & 0 & $0$    & $0$    & $k\,\sum Y^2$ \\
$H$    & $(\mathbf{1},\mathbf{2},\,1/2)$   & $2N_H$ & 0 & 1 & $0$    & $2N_H$ & $k\,\sum Y^2$ \\
$W$    & $\mathrm{adj}\,(\mathbf{1},\mathbf{3},0)$   & $1$    & 0 & 4 & $0$    & $4$    & $0$ \\
$G$    & $\mathrm{adj}\,(\mathbf{8},\mathbf{1},0)$   & $1$    & 6 & 0 & $6$    & $0$    & $0$ \\
\hline
\multicolumn{8}{l}{%
\parbox{\linewidth}{\footnotesize\textit{Note:} $w_1^{(f)}=12\sum Y^2$ for Weyl fermions and $w_1^{(s)}=3\sum Y^2$ per hypercharged scalar degree of freedom. For $H\sim(\mathbf 1,\mathbf 2,\,\tfrac12)$, $\sum_{\text{dof}}Y^2=1/2$ so $w_1(H)=3$, ensuring all entries in $\Delta W$ are integers.}%
}
\end{tabular}
\end{table}

\begin{table}[t]
\centering
\caption{EW window $W_{\rm EW}:~ Q\in(80,160)\, \mathrm{GeV}$. Heavy multiplets removed, narrow threshold masks.}
\label{T:Wmask_EW}
\begin{tabular}{lcc}
\hline
Removed multiplet & Reason & Mask range in $Q$ \\
\hline
top quark & decoupled below $W_{\rm EW}$ & --- \\
\hline
\multicolumn{3}{l}{\textit{Within-window threshold masks:}}\\
$W^\pm$ & resonance/threshold guard & $Q\in(79,82)\,\mathrm{GeV}$ \\
$Z$     & resonance/threshold guard & $Q\in(90,92.5)\,\mathrm{GeV}$ \\
$H$     & threshold guard            & $Q\in(124,127)\,\mathrm{GeV}$ \\
\hline
\end{tabular}
\end{table}

\begin{table}[t]
\centering
\caption{Low GeV window $W_{\rm SM}:~ Q\in(1,10)\,\mathrm{GeV}$. Heavy multiplets removed, edge guards near thresholds.}
\label{T:Wmask_SM}
\begin{tabular}{lcc}
\hline
Removed multiplet & Reason & Mask range in $Q$ \\
\hline
$t,\,W/Z/H$ & decoupled below EW scale & --- \\
$c,b$ (edges) & onset guards at $m_c,\,m_b$ & small masks around $m_c,\,m_b$ \\
\hline
\end{tabular}
\end{table}

\subsection*{S1.1 Construction recipe (per multiplet)}

\paragraph*{Per–multiplet weights and integerization.}
For each light multiplet $f$ in an admissible window $\mathcal W$,
\[
\begin{aligned}
&\text{Weyl:}&\quad w_3(f)&=4\,T_{SU(3)}(f)\,d_{\rm spect}(f),\qquad
w_2(f)=4\,T_{SU(2)}(f)\,d_{\rm spect}(f),\\
&\text{scalar:}&\quad w_3(f)&=1\cdot T_{SU(3)}(f)\,d_{\rm spect}(f),\qquad
w_2(f)=1\cdot T_{SU(2)}(f)\,d_{\rm spect}(f),
\end{aligned}
\]
and choose a single $U(1)_Y$ integerizer so the hypercharge column is integral:
\[
w_1^{(\mathrm f)}=12\!\!\sum_{\text{Weyl in }f}\!Y^2,\qquad
w_1^{(\mathrm s)}=3\!\!\sum_{\text{scalars in }f}\!Y^2.
\]
Here $T_{SU(N)}$ is the Dynkin index ($T(\mathbf 3)=T(\mathbf 2)=\tfrac12$), and $d_{\rm spect}$ counts spectator multiplicities (e.g., color for $SU(2)$ weights and weak multiplicity for $SU(3)$ weights). GUT normalization is used for hypercharge: $\alpha_1=\tfrac53\alpha_Y$.

\paragraph*{Window vectors and differences.}
Sum the weights across the light content of the window:
\[
b^{(\mathcal W)}=\begin{pmatrix}\sum_f w_3(f)\\[2pt]\sum_f w_2(f)\\[2pt]\sum_f w_1(f)\end{pmatrix}\in\mathbb Z^3,
\]
then form the integer \emph{difference stack} over admissible window pairs $\{(\mathcal W_i,\mathcal W_j)\}$:
\[
\Delta b^{(ij)}=b^{(\mathcal W_i)}-b^{(\mathcal W_j)},\qquad
\Delta W=\begin{bmatrix} (\Delta b^{(i_1j_1)})^\top \\ (\Delta b^{(i_2j_2)})^\top \\ \vdots \end{bmatrix}\in\mathbb Z^{m\times3}.
\]
Adjoint self–contributions cancel in $\Delta b$, exposing the rank–2 lattice used for SNF.

\paragraph*{Sanity check (electromagnetic basis).}
After EWSB, use
\(
w_{\rm EM}=w_2+\tfrac53 w_1
\Rightarrow 3w_{\rm EM}=3w_2+5w_1\in\mathbb Z
\),
so the $(SU(3),SU(2),{\rm EM})$ basis keeps exact integers for certification.

\subsection*{S1.2 Worked integer kernel (by hand, no SNF)}

\[
\chi_{\rm EM}=(-10,-18,1),\qquad \gcd(10,18,1)=1\ \text{(primitive)}\,.
\]

In the $(SU(3),SU(2),{\rm EM})$ basis the two-row difference stack is
\[
\Delta W_{\rm EM}
=\begin{bmatrix}
8 & 8 & 224\\
0 & 1 & 18
\end{bmatrix}\in\mathbb Z^{2\times3}.
\]
Solve $\Delta W_{\rm EM}\,\chi_{\rm EM}=0$ over $\mathbb Z$:
second row gives $\chi_2=-18\,\chi_3$; first row gives $8\chi_1+8\chi_2+224\chi_3=0\Rightarrow
8\chi_1+8(-18)\chi_3+224\chi_3=0\Rightarrow \chi_1=-10\,\chi_3$.
Choosing $\chi_3=1$ yields the \emph{primitive} generator
\[
\boxed{~\chi_{\rm EM}=(-10,-18,1)~,}\qquad \gcd(10,18,1)=1.
\]

\paragraph*{Transport to the canonical basis}
Let $A\in{\rm GL}(3,\mathbb Z)$ be the unimodular change-of-columns from the EM stack
to the canonical $(w_3,w_2,w_1)$ basis printed below. Kernel covectors transport contravariantly:
\[
\boxed{\ \chi \;=\; A^{-{\sf T}}\ \chi_{\rm EM}\ }\, .
\]
Evaluating with the $A$ (denoted $M$ below) gives
\[
\boxed{M^{\!\top}\chi_{\rm EM}=(16,13,2)\equiv\chis\ }\, .
\]

\paragraph*{SNF note}
The Smith invariants of $\Delta W_{\rm EM}$ are $[1,8]$ (rank $2$),
with a trailing zero column; hence $\ker_\mathbb Z(\Delta W_{\rm EM})$ is one–dimensional and generated by $\pm\chi_{\rm EM}$.

\paragraph*{SNF (explicit).}
For
\(
\Delta W_{\rm EM}=\begin{bmatrix}1&1&28\\[2pt]0&1&18\end{bmatrix}
\),
there exist unimodular \(U\in GL(2,\mathbb Z)\),
\(V\in GL(3,\mathbb Z)\) such that
\[
U\,\Delta W_{\rm EM}\,V=\operatorname{diag}(1,\,8,\,0)\,,
\]
so \(\operatorname{rank}=2\) and there is a single zero invariant.

\subsection*{S1.3 Window differences and the integer row lattice}
For a momentum window $\mathcal W$ with light content $\mathcal S_{\mathcal W}$, define integerized 1L weights
\[
b^{(\mathcal W)}=\begin{pmatrix}\sum w_3\\[2pt]\sum w_2\\[2pt]\sum w_1\end{pmatrix}\in\mathbb Z^3,
\]
with Weyl $w_{3,2}=4\,T_{SU(3,2)}\,d_{\rm spect}$ and scalar $w_{3,2}=1\cdot T_{SU(3,2)}\,d_{\rm spect}$, and $w_1$ as above. For admissible windows $\{\mathcal W_i\}$ form differences
\[
\Delta b^{(ij)}=b^{(\mathcal W_i)}-b^{(\mathcal W_j)},\qquad
\Delta W=\begin{bmatrix} (\Delta b^{(i_1j_1)})^\top \\ (\Delta b^{(i_2j_2)})^\top \\ \vdots \end{bmatrix}\in\mathbb Z^{m\times3}.
\]

\textit{Lemma (row–lattice invariance).}
Any two admissible stacks $\Delta W,\Delta W'$ are related by unimodular row
operations (adding/removing differences; reordering) and appending/canceling common adjoint self-terms. Hence their integer row lattices coincide and their left kernels over $\mathbb Z$ are identical.
\[
M=\begin{bmatrix}
-5 & -3 & -2\\
\phantom{-}2 & \phantom{-}1 & \phantom{-}1\\
\phantom{-}2 & \phantom{-}1 & 0
\end{bmatrix},
\qquad \det M=-1,
\qquad M^{\!\top}\chi_{\rm EM}=(16,13,2)\,.
\]
\textit{Proof sketch.} Differences generate the same subgroup as absolute rows modulo a common reference. Appending/removing a difference corresponds to adding/removing an integer row; permutations and sign flips are unimodular. Common adjoint self-terms cancel in any row difference. $\square$

% -----------------------------------------------
\subsection*{S1.4 Two physical differences (explicit tallies)}

Using $T(\mathbf 3)=T(\mathbf 2)=\tfrac12$ and spectator multiplicities
(weak multiplicity as spectator for $SU(3)$ weights; color multiplicity as spectator for $SU(2)$ weights),
the integerized one–loop weights sum as follows.

\paragraph*{One SM generation (five Weyl multiplets).}
\[
\begin{aligned}
&\text{$SU(3)$:}\quad 
w_3(Q_L)=4\!\cdot\!\tfrac12\!\cdot\!2=4,\qquad
w_3(u_R)=4\!\cdot\!\tfrac12\!\cdot\!1=2,\qquad
w_3(d_R)=4\!\cdot\!\tfrac12\!\cdot\!1=2
\\
&\Rightarrow \sum w_3 = 8,\\[2pt]
&\text{$SU(2)$:}\quad
w_2(Q_L)=4\!\cdot\!\tfrac12\!\cdot\!3=6,\qquad
w_2(L_L)=4\!\cdot\!\tfrac12\!\cdot\!1=2
\ \Rightarrow\ \sum w_2 = 8,\\[2pt]
&\text{$U(1)_Y$ (global integerizer):}\quad
w_1 = 12\!\!\sum_{\rm Weyl} Y^2
= 12\!\left[\tfrac{1}{6}+\tfrac{4}{3}+\tfrac{1}{3}+\tfrac{1}{2}+1\right]
= 40.
\end{aligned}
\]
Hence
\[
\Delta b_{\rm gen} = (8,\,8,\,40).
\]

\paragraph*{One Higgs doublet (complex scalar).}
For $Y=+\tfrac12$ and two weak components,
\[
w_2(H)=1\!\cdot\!\tfrac12\!\cdot\!1=1,\qquad
w_1(H)=3\!\!\sum_{\rm scalars} Y^2 = 3\!\cdot\!\tfrac12 = 3,\qquad
w_3(H)=0,
\]
so
\[
\Delta b_H = (0,\,1,\,3).
\]

\noindent
(Any overall common integer factor on a \emph{row} does not affect the \emph{primitive} kernel.)

\paragraph*{Integer kernel and cross–check.}
Stacking these differences gives
\[
\Delta W =
\begin{pmatrix}
8 & 8 & 40\\
0 & 1 & 3
\end{pmatrix},
\qquad
\Delta W\,\chi = 0.
\]
Solving over $\mathbb Z$ yields the unique primitive generator
\[
\chi = (16,\,13,\,2),
\]
identical to the EM–stack/SNF certificate.
Thus the tally route reproduces the same integer kernel,
closing the algebraic–physical loop and fixing $\chi$ by integer structure alone.

\section*{S2. Alignment as a Symmetry-Locking Principle}

\paragraph*{Statement.}
Let $\Keq\succ0$ be the equilibrium field-space metric and $\chis=(16,13,2)$ the SNF-certified projector. Define $\hat\chi=\chis/\normK{\chis}$ and let $e_{\rm soft}$ be the normalized soft eigenvector of $\Keq$. The \emph{alignment condition} is
\[
\cos\theta \;=\; \hat\chi^{\!\top}\,\Keq\,e_{\rm soft} \;\ge\; 1-\varepsilon_\chi,
\]
with fixed tolerance $\varepsilon_\chi\!\ll\!1$ (reported in SM). When alignment holds, the gauge–log depth $\XiDepth=\chis\!\cdot\!\Psivec$ isolates the soft direction and the parity-even gate $\PiGate{\XiDepth}$ projects the gauge sector onto a single scalar depth.

\paragraph*{Consequences.}
(i) \emph{Even-parity protection.} A spurion $\mathbb Z_2$ symmetry $\XiDepth\!\to\!-\XiDepth$ with $\Pi$ invariant implies
\[
\left.\partial_{\Xi}\PiGate{\XiDepth}\right|_{\XiEq}=0
\quad\Rightarrow\quad \text{no linear response;\ } m_{\rm PF}^2=0,
\]
to all loop orders near equilibrium. Renormalization can shift $(\sigchi,\Keq)$ but cannot generate an odd term.\\
(ii) \emph{Tensor sector.} Around the lab point (Minkowski) the Lichnerowicz operator reduces to
\[
\Delta_{\!L}h_{\mu\nu} \;=\; -\Box h_{\mu\nu}=0 \quad\Rightarrow\quad
\omega^2=\mathbf{k}^2,\ \lambda=\pm2 \ \text{(massless, luminal)}.
\]
(iii) \emph{Quadratic lab-null.} The near-eq.\ response is
\[
\FracDG \;\simeq\; \frac{\DXi^2}{\sigchi^2}
=\frac{\phi_\chi^2}{\Lchi^2},\qquad
\phi_\chi=\DXi/\normK{\chis},\quad
\Lchi=\sigchi/\normK{\chis}.
\]

\paragraph*{Falsifier from misalignment.}
If $\cos\theta<1-\varepsilon_\chi$, an odd (linear) term is generically induced in a lab fit
\[
\FracDG(s)=A\,s+B\,s^2+\dots,
\]
violating the parity null ($A=0$). Significant misalignment therefore falsifies the model.

\paragraph*{Motivation (minimal).}
Alignment is the universal tendency of coupled fields to cohere along the softest kinetic mode of a positive-definite metric $K$. In GAGE, the certified integer projector $\chis$ aligns with the soft eigenvector of $\Keq$, enforcing even response and a massless, luminal tensor sector. Analogous locking appears in magnetic ordering, superconductivity, and Higgs vacuum alignment (qualitative context; not inputs).

\paragraph*{S2.1 Minimal alignment functional.}
Let unit order parameters $u_i(x)\in\mathbb{R}^d$ with metrics $K_i\succ0$ and couplings $\gamma_1,\gamma_2$. Define
\[
\mathcal{A}[u]
=\!\int d^Dx\!
\Big[
\tfrac12\sum_i (\partial u_i)^{\!\top}\!K_i(\partial u_i)
-\frac{1}{N}\!\sum_{i<j}\!(\gamma_1\,u_i\!\cdot\!u_j+\gamma_2(u_i\!\cdot\!u_j)^2)
\Big],\qquad \|u_i\|=1.
\]
Diagnostics $m=\|\langle u\rangle\|$, $C=\tfrac1N\sum_i u_i u_i^{\!\top}$, and $\rho=\lambda_{\max}(C)/\mathrm{Tr}(C)$ measure coherence ($m\in[0,1]$, $\rho\in[1/d,1]$).
\textbf{Lemma.} For $K\succ0$ and couplings above a threshold $\gamma_c$, minimizers align $\langle u\rangle$ with the soft eigenvector $e_{\rm soft}$ of $K$ up to $O(\kappa_{\rm gap}^{-1})$; orthogonal fluctuations are gapped.
\textbf{Map to GAGE.} $u\!\parallel\!\hat\chi$, $K\!\to\!\Keq$, $\Xi=\chis\!\cdot\!\Psivec$, and even $\Pi(\Xi)$ enforces $\FracDG\simeq\phi_\chi^2/\Lchi^2$.

\paragraph*{S2.2 Phase variant (S$^1$).}
For phases $\theta_i$,
\[
\mathcal{A}_\theta[\theta]
=\!\int d^Dx\!
\Big[\tfrac{\kappa}{2}\sum_i|\nabla\theta_i|^2
-\frac{K}{N}\sum_{i<j}\cos(\theta_i-\theta_j)\Big],
\]
whose ordered phase satisfies $\partial_\mu\theta_i\!\approx\!\partial_\mu\theta_j$, corresponding to alignment of phase gradients.

\paragraph*{S2.3 Conservation form (near equilibrium).}
Define the alignment current
\[
J_\chi^\mu
=\PiGate{\Xi}\,\chis^{\!\top}\Keq\,\partial^\mu\Psivec,
\]
which reduces after one contraction to $J_\chi^\mu=\PiGate{\Xi}\,\partial^\mu\Xi$. Using $\Pi'(\Xi_{\rm eq})=0$ and the $\Xi$ equation of motion,
\[
\partial_\mu J^\mu_\chi 
= 0 + O\!\left( (\Delta\Xi)^3,\ \text{2Loop drift},\ \varepsilon_\chi \right),
\qquad \varepsilon_\chi \le 10^{-8}.
\]
Any measured odd term $A\neq0$ in $\FracDG=A\,s+B\,s^2+\dots$ gives $\partial_\mu J_\chi^\mu\neq0$ and falsifies alignment.

\paragraph*{S2.4 Information–geometry view}
The Fisher curvature along depth,
\[
\kappa_\chi \equiv \frac{1}{\sigma_\chi^2}
= \left.\partial_{\Delta \Xi}^2\!\left[-\ln \Pi(\Xi)\right]\right|_{\rm eq},
\]
defines the local informational metric. Alignment is motion along the soft eigenvector of
\(K_{\rm eq}\), i.e., the direction of least Fisher curvature (least informational resistance).

\paragraph*{S2.5 Falsifiers and caveats}
Lab template: \(\Delta G/G = A\,s + B\,s^2 + \cdots\), \(s=\Delta\Xi/\sigma_\chi\).
Alignment predicts \(A=0,\ B=1\). Additional falsifiers: failure of rank-1 coherence
(\(\rho \not\to 1\)), or locking to a non-soft mode at fixed \(K_{\rm eq}\).
Boundary/disorder can produce modulated or defect states; diagnose via the most unstable
Fourier mode of the quadratic expansion. Tolerance: \(\varepsilon_\chi \le 10^{-8}\) (see S2.3, S0).

\paragraph*{S2.6 Cross-domain statement}
Across spins, phases, and gauge directions, alignment is symmetry-locking to the soft mode of \(K\),
quantified by \((m,\rho)\) with \(m=\|\langle u\rangle\|\) and \(\rho=\lambda_{\max}(C)/\mathrm{Tr}(C)\).
GAGE is the SM realization with \(u\parallel \hat\chi\) and \(K\equiv K_{\rm eq}\).


\subsection*{S2.7 Noether-style derivation of the alignment current}
Consider the near-equilibrium alignment Lagrangian density
\[
\mathcal{L}_\chi=\tfrac12\,\partial_\mu\hat\Psi^{\!\top}K\,\partial^\mu\hat\Psi\;\Pi(\Xi),
\qquad
\Xi=\chi^{\!\top}\hat\Psi,\quad K\succ0,\quad \Pi'(\Xi_{\rm eq})=0.
\]
Under a rigid depth shift generated along $\chi$,
\[
\delta\hat\Psi=\varepsilon\,\chi,\qquad \delta\Xi=\varepsilon\,\chi^{\!\top}\chi=\text{const},
\]
the Noether current is
\[
J^\mu \;=\; \frac{\partial\mathcal{L}}{\partial(\partial_\mu\hat\Psi)}\cdot\delta\hat\Psi
\;=\;\Pi(\Xi)\,\big(K\partial^\mu\hat\Psi\big)^{\!\top}\,(\varepsilon\chi)
\;=\;\varepsilon\,\Pi(\Xi)\,\chi^{\!\top}K\,\partial^\mu\hat\Psi.
\]
Dropping the inessential overall $\varepsilon$, we identify
\[
\boxed{~J_\chi^\mu=\Pi(\Xi)\,\chi^{\!\top}K\,\partial^\mu\hat\Psi~.}
\]
On shell, Noether’s identity gives $\partial_\mu J^\mu=\delta\mathcal{L}$. Using $\Pi'(\Xi_{\rm eq})=0$
(even parity) and expanding about equilibrium, the variation of the gate begins at cubic order in
$\Delta\Xi$:
\[
\delta\mathcal{L}
= \tfrac12\,\partial_\mu\hat\Psi^{\!\top}K\,\partial^\mu\hat\Psi\,\Pi'(\Xi)\,\delta\Xi
= O\!\big((\Delta\Xi)^3\big),
\]
so that
\[
\boxed{~\partial_\mu J_\chi^\mu=0\;+\;O\!\big((\Delta\Xi)^3,\ \text{2-loop drift},\ \varepsilon_\chi\big).~}
\]
Thus the parity-protected alignment symmetry yields a conserved Fisher-metric current in gauge–depth space.
Empirically, a nonzero odd (linear) response ($A\neq0$ in $\Delta G/G=A\,s+B\,s^2+\cdots$) implies
$\partial_\mu J_\chi^\mu\neq0$ and falsifies alignment.

\section*{S3. Gate, parity lemma, and quadratic response}

\subsection*{S3.1 Even gate $\Pi(\Xi)$ and normalization}

Promote the depth scalar
\[
\Xi(x)\equiv \boldsymbol{\chi}\!\cdot\!\boldsymbol{\Psi}(x)
\]
to a spacetime field through the running gauge couplings $\boldsymbol{\Psi}(x)$.  
Define a \textit{parity-even curvature gate}
\[
\frac{G(x)}{G} = \Pi(\Xi),\qquad
\Pi(\Xi_{\rm eq}) = 1,\qquad
\Pi(\Xi_{\rm eq}+\Delta) = \Pi(\Xi_{\rm eq}-\Delta),
\]
with $\Pi$ taken $C^2$ in a neighborhood of $\Xi_{\rm eq}$ and depending only on the scalar depth $\Xi$.  
The gate acts purely as a multiplicative curvature normalization and introduces no independent gravitational dynamics.

\paragraph*{Gaussian model (for figures and tests)}
For numerical illustration we often adopt the Gaussian ansatz
\[
\Pi_{\rm G}(\Xi)=\exp\!\Big[-\frac{(\Xi-\Xi_{\rm eq})^2}{\sigma_\chi^2}\Big],
\]
which satisfies all required symmetry and smoothness conditions.  
All analytic derivations below, however, rely only on the evenness and differentiability of $\Pi(\Xi)$, not on its specific form.


\subsection*{S3.2 Parity lemma and quadratic response}

Let 
\[
\Delta\Xi \equiv \Xi - \Xi_{\rm eq}.
\]
Parity evenness implies
\[
\left.\frac{\partial\Pi}{\partial\Xi}\right|_{\Xi=\Xi_{\rm eq}}=0,
\]
so that near equilibrium
\[
\Pi(\Xi_{\rm eq}+\Delta\Xi)
= 1 + \tfrac{1}{2}\,\Pi''(\Xi_{\rm eq})\,(\Delta\Xi)^2
+ \mathcal{O}\!\bigl((\Delta\Xi)^3\bigr).
\]
Hence the local fractional variation of the gravitational coupling is purely quadratic:
\[
\frac{\Delta G}{G}
\equiv \frac{G(x)}{G}-1
= \Pi(\Xi_{\rm eq}+\Delta\Xi)-1
\simeq \tfrac{1}{2}\,\Pi''(\Xi_{\rm eq})\,(\Delta\Xi)^2,
\]
and all odd derivatives vanish,
\[
\left.\partial_\Xi^{\,2k+1}\Pi(\Xi)\right|_{\Xi=\Xi_{\rm eq}}=0,
\qquad k=0,1,2,\ldots
\]
For the Gaussian model $\Pi_{\rm G}(\Xi)=\exp[-(\Delta\Xi)^2/\sigma_\chi^2]$, 
\[
\Pi_{\rm G}''(\Xi_{\rm eq})=-\frac{2}{\sigma_\chi^2}
\qquad\Rightarrow\qquad
\left|\frac{\Delta G}{G}\right|\simeq\frac{(\Delta\Xi)^2}{\sigma_\chi^2}.
\]

\subsection*{S3.3 Soft mode, canonical form, and $\Lambda_\chi=\sigma_\chi/\|\chi\|_{K}$}

With $K\!\succ\!0$ (Table~\ref{tab:keq}), define the dimensionless soft-mode displacement
\[
\phi_\chi = 
\frac{\boldsymbol{\chi}^{\!\top}(\boldsymbol{\Psi}-\boldsymbol{\Psi}_{\rm eq})}
     {\|\boldsymbol{\chi}\|_{K}}, 
\qquad
\|\boldsymbol{\chi}\|_{K}
   = \sqrt{\boldsymbol{\chi}^{\!\top}K\boldsymbol{\chi}}.
\]
Then the depth deviation satisfies 
\[
\Delta\Xi = \|\boldsymbol{\chi}\|_{K}\,\phi_\chi.
\]
A convenient canonical parameterization of the curvature gate is
\[
\Pi(\Xi)
   = \exp\!\Big[-\,\frac{\phi_\chi^{2}}{\Lambda_\chi^{2}}\Big],
\qquad
\Lambda_\chi
   = \frac{\sigma_\chi}{\|\boldsymbol{\chi}\|_{K}}.
\]
Near equilibrium, the fractional variation obeys
\[
\big|\Delta G/G\big| \simeq \frac{\phi_\chi^{2}}{\Lambda_\chi^{2}}.
\]
For reference, the macros encode
\[
\omega_{\rm hel} = \frac{\|\boldsymbol{\chi}\|_{K}}{\sigma_\chi}
                  = \frac{1}{\Lambda_\chi},
\qquad
T_{\rm hel} = \frac{2\pi}{\omega_{\rm hel}}
             = 2\pi\,\Lambda_\chi.
\]

\subsection*{S3.4 Spurion $\mathbb{Z}_2$ symmetry and radiative stability (all orders)}

\textbf{Definition (spurion parity).}  
Assign a \textit{spurionic reflection symmetry} in gauge–log space:
\[
\Xi \mapsto -\Xi,\qquad 
\delta\Xi \mapsto -\delta\Xi,\qquad 
\Pi \mapsto \Pi,
\]
acting trivially on directions orthogonal to $\boldsymbol{\chi}$:
\[
P_\perp(\boldsymbol{\Psi}-\boldsymbol{\Psi}_{\rm eq})
\mapsto
P_\perp(\boldsymbol{\Psi}-\boldsymbol{\Psi}_{\rm eq}),
\]
with
\[
P_\perp = \mathbb{1}-P_\chi,\qquad
P_\chi = K\,\boldsymbol{\chi}\boldsymbol{\chi}^{\!\top}/(\boldsymbol{\chi}^{\!\top}K\boldsymbol{\chi}).
\]

\textbf{Lemma (operator classification near $\Xi_{\rm eq}$).}  
In any local EFT that respects the spurion parity and the residual $O(2)$ rotations in the orthogonal complement, every scalar functional multiplying the Ricci term must be built from \textit{even} invariants:
\[
\Pi(\Xi,\partial\Xi,\ldots)
 = \Pi_0
 + \Pi_2\,\frac{\delta\Xi^2}{\sigma_\chi^2}
 + \Pi_{2,\partial}\,\frac{(\partial\delta\Xi)^2}{\Lambda_\chi^2}
 + \cdots,
\]
while all terms linear in $\delta\Xi$ or odd in derivatives are forbidden.

\textbf{Radiative stability (renormalization statement).}  
Loop corrections consistent with the spurion $\mathbb{Z}_2$ symmetry cannot generate a linear term; 
\[
\left.\partial_\Xi\Pi\right|_{\Xi_{\rm eq}}
\]
renormalizes multiplicatively to zero.  
Allowed counterterms renormalize only:
\begin{enumerate}[label=(\roman*), topsep=2pt, itemsep=0pt]
\item the overall normalization $\Pi(\Xi_{\rm eq})\equiv1$ (fixed by calibration),
\item the gate width $\sigma_\chi$,
\item the kinetic metric $K_{ij}$,
\item and higher–even coefficients.
\end{enumerate}
Hence, at quadratic order the only renormalizations are finite shifts of $\sigma_\chi$ and $K$; no $\mathcal{O}(\Delta\Xi)$ response can appear at any loop order.

\subsection*{S3.5 Why $\Pi=\Pi(\Xi)$ (no dependence on orthogonal modes)}

\noindent
By construction, $\Xi=\boldsymbol{\chi}\!\cdot\!\hat{\boldsymbol{\Psi}}$ is the unique primitive integer depth.  
The residual $O(2)$ symmetry acting in the orthogonal subspace defined by $P_\perp$ forbids any leading dependence on coordinates orthogonal to $\boldsymbol{\chi}$, so that
\[
\Pi = \Pi(\Xi)
\]
up to higher–derivative even invariants.  
Near equilibrium, the orthogonal subspace is two–dimensional; imposing the residual $O(2)$ symmetry in $P_\perp$ excludes any orientation–specific dependence at leading order.  
Therefore, the most general scalar gate consistent with all symmetries is a function of $\Xi$ alone, plus \emph{even} derivative corrections of the type described in S3.4.  
Such terms are higher order and negligible in the laboratory–null configurations discussed in S6.

\paragraph*{Gate symmetry (Ward–flat plane).}
Define the depth variable
\[
\Xi(\mu)=\boldsymbol{\chi}\!\cdot\!\boldsymbol{\Psi}_1(\mu)
=16\ln\alpha_3+13\ln\alpha_2+2\ln\alpha_1.
\]
Each running coefficient $b^{(w)}$ and each threshold jump $\Delta b^{(p)}$ is orthogonal to $\boldsymbol{\chi}$, implying
\[
\frac{d\Xi}{d\ln\mu}=0
\]
within continuous windows, with threshold jumps cancelling across decouplings.  
Consequently, any curvature gate $\Pi(\Xi)$—or equivalently $\Omega(\boldsymbol{\Psi})=F(\boldsymbol{\chi}\!\cdot\!\boldsymbol{\Psi})$—is invariant under infinitesimal displacements $\delta\boldsymbol{\Psi}$ satisfying $\boldsymbol{\chi}\!\cdot\!\delta\boldsymbol{\Psi}=0$.  
This defines a two–dimensional \textit{Ward–flat plane} orthogonal to $\boldsymbol{\chi}$, within which the gate remains exactly constant.

\subsection*{S3.6 Falsifier (boxed; handoff to S6)}

\[
\boxed{
\left.\partial_\Xi \Pi(\Xi)\right|_{\Xi_{\rm eq}}=0
\ \Longrightarrow\
\text{no linear term in }{\Delta G}/{G}.\
\text{Any observed } \mathcal{O}(\Delta\Xi)\ \text{signal falsifies the construction.}
}
\]

The leading quadratic coefficient is 
\[
\tfrac{1}{2}\,\Pi''(\Xi_{\rm eq})
\quad\text{(Gaussian model: }-2/\sigma_\chi^2\text{)}.
\]
The laboratory–null template and two–state contrast used for empirical tests are presented in S6.

\paragraph*{Parity reminder.}
At $\boldsymbol{\Psi}_{\rm eq}$, $\partial_\Xi\Pi|_{\rm eq}=0$; therefore no linear (odd) term in $\delta\Xi$ can appear, and the leading observable deviations scale as $\delta\Xi^2$.

\section*{S4. Tensor sector and absence of Pauli–Fierz mass}

\subsection*{S4.1 Background, Jordan–frame expansion, and kinetic structure}

Assume a stationary, flat laboratory background,
\[
\boldsymbol{\Psi} = \boldsymbol{\Psi}_{\rm eq},\qquad
\partial_{\boldsymbol{\Psi}}V\big|_{\boldsymbol{\Psi}_{\rm eq}} = 0,\qquad
V(\boldsymbol{\Psi}_{\rm eq}) = 0,\qquad
g_{\mu\nu} = \eta_{\mu\nu} + h_{\mu\nu}.
\]
Insert the curvature gate into the Einstein–Hilbert term:
\[
S = \int d^4x\,\sqrt{-g}\,
\Big[
\tfrac{1}{2}\,M_{\rm Pl}^2\,\Pi(\Xi)\,R
- \tfrac{1}{2}\,\partial_\mu\boldsymbol{\Psi}^{\!\top}K(\boldsymbol{\Psi})\,\partial^\mu\boldsymbol{\Psi}
- V(\boldsymbol{\Psi})
\Big].
\]
At equilibrium, $\Pi(\Xi_{\rm eq})=1$ and (from S3) $\partial_\Xi\Pi|_{\Xi_{\rm eq}}=0$.

Expand around $g_{\mu\nu}=\eta_{\mu\nu}+h_{\mu\nu}$ in harmonic gauge
$\partial^\mu h_{\mu\nu}-\tfrac{1}{2}\partial_\nu h=0$.
To quadratic order in $h_{\mu\nu}$,
\[
S_{\rm tens}^{(2)}
= \frac{M_{\rm Pl}^2}{8}
  \int d^4x\,
  h^{\mu\nu}\,\mathcal{E}_{\mu\nu}^{\ \ \alpha\beta}\,h_{\alpha\beta}
  \;+\;
  \mathcal{O}(h^2\,\Delta\Xi^2),
\]
where $\mathcal{E}$ is the standard Lichnerowicz operator.
Since $\Pi'(\Xi_{\rm eq})=0$, all potential $h^2\Delta\Xi$ mixings vanish.

\noindent
The operator acts as
\[
\mathcal{E}^{\alpha\beta}{}_{\mu\nu}h_{\alpha\beta}
\equiv
-\Box h_{\mu\nu}
+ \partial_{(\mu}\partial^{\alpha}h_{\nu)\alpha}
- \partial_\mu\partial_\nu h
- \eta_{\mu\nu}\!\left(-\Box h+\partial^{\alpha}\partial^{\beta}h_{\alpha\beta}\right),
\]
the linearized Einstein operator in harmonic gauge.

\begin{tcolorbox}[colback=white,colframe=black!30,left=2mm,right=2mm]
\textbf{No Pauli–Fierz mass.}  
Around flat equilibrium with $\Pi'(\Xi_{\rm eq})=0$ and $\Pi(\Xi_{\rm eq})=1$,  
the linearized tensor sector exactly matches General Relativity:
no $m_{\rm PF}^2(h_{\mu\nu}h^{\mu\nu}-h^2)$ term appears.  
Gate corrections begin only at $\mathcal{O}(\Delta\Xi^2)$ and leave the kinetic Lichnerowicz form unmodified.
\end{tcolorbox}

\noindent
In a neighborhood of $\Xi_{\rm eq}$, a conformal map
\[
g^E_{\mu\nu} = \Pi(\Xi)\,g_{\mu\nu}
\]
transforms the action to the Einstein frame, where the scalar field is canonically normalized and its linear coupling to $R$ vanishes because $\Pi'(\Xi_{\rm eq})=0$.  
Thus the effective Brans–Dicke coupling scales as $\propto (\Pi')^2$ and is exactly zero at equilibrium.

\paragraph*{Kinetic metric and soft–mode projectors.}
The log–coupling fields expand with
\[
\mathcal{L}_{\rm kin}
 = -\tfrac{1}{2}\,
   \partial_\mu\boldsymbol{\Psi}^{\!\top}K(\boldsymbol{\Psi})\,\partial^\mu\boldsymbol{\Psi},
   \qquad
   K=K(\boldsymbol{\Psi}_{\rm eq})\succ0.
\]
Define the $K$–unit vector and projectors
\[
\hat{u}_\chi = \frac{\boldsymbol{\chi}}{\|\boldsymbol{\chi}\|_{K}},\qquad
P_\chi = \hat{u}_\chi\,\hat{u}_\chi^{\!\top}K,\qquad
P_\perp = \mathbb{1}-P_\chi,
\]
so that
\[
\phi_\chi = \hat{u}_\chi^{\!\top}K(\boldsymbol{\Psi}-\boldsymbol{\Psi}_{\rm eq}),
\qquad
\Delta\Xi
   = \boldsymbol{\chi}\!\cdot\!(\boldsymbol{\Psi}-\boldsymbol{\Psi}_{\rm eq})
   = \|\boldsymbol{\chi}\|_{K}\,\phi_\chi.
\]
The explicit form of $K$ and its eigenstructure are given in Tables~\ref{tab:keq}–\ref{tab:keq_eigs}.


\subsection*{S4.2 Explicit origin of the no–mixing result}

Vary the Jordan–frame Ricci term with
\(\Omega(\boldsymbol{\Psi}) \equiv M_{\rm Pl}^2\,\Pi(\Xi)\):
\[
\delta\!\left(\sqrt{-g}\,\Omega R\right)
= \sqrt{-g}\!\left[
\tfrac{1}{2}\,\Omega\,h^{\mu\nu}\,
\mathcal{E}_{\mu\nu}^{\ \ \alpha\beta}\,h_{\alpha\beta}
+ (g_{\mu\nu}\Box - \nabla_\mu\nabla_\nu)\,\delta\Omega\,h^{\mu\nu}
\right]_{\!\rm lin}
+ \cdots .
\]
Near equilibrium,
\[
\delta\Omega
= M_{\rm Pl}^2\,\Pi'(\Xi_{\rm eq})\,\delta\Xi
+ \mathcal{O}(\delta\Xi^2)
= 0 + \mathcal{O}(\delta\Xi^2),
\]
so the would–be \(h\,\delta\Xi\) mixing term proportional to
\((\partial\partial\,\delta\Omega)\) vanishes identically at linear order.
The first nonzero gate correction appears only at
\(\mathcal{O}(h\,\delta\Xi^2)\),
which cannot generate a Pauli–Fierz mass and instead renormalizes
higher–order interaction vertices.

\noindent
Derivatives of \(K(\hat{\boldsymbol{\Psi}})\) contribute exclusively to scalar
self–interactions and \(h\,\delta\Xi^2\) cross–terms;
they likewise cannot induce any \(\mathcal{O}(h)\) Pauli–Fierz mass.

\subsection*{S4.3 GR limit and field equations (linearized)}

Collect the $\mathcal{O}(h)$ terms and couple to a conserved matter source $T_{\mu\nu}$:
\[
M_{\rm Pl}^2\,\mathcal{E}_{\mu\nu}^{\ \ \alpha\beta}h_{\alpha\beta}
= T_{\mu\nu}
+ \mathcal{O}(h\,\delta\Xi^2).
\]
The gauge symmetries and propagator coincide with those of General Relativity.
The two tensor polarizations propagate luminally with
\(
k^2=0.
\)
The Newtonian potentials satisfy
\[
\nabla^2\Phi = \nabla^2\Psi
= \tfrac{1}{2}\,M_{\rm Pl}^{-2}\,T_{00}
\quad\Rightarrow\quad
\gamma \equiv \frac{\Psi}{\Phi}
= 1 + \mathcal{O}\!\bigl(\Delta\Xi^2 / \sigma_\chi^2\bigr),
\]
consistent with the parity lemma (S3): all odd responses are forbidden and leading deviations are quadratic.

\noindent
Equivalently, the effective Brans–Dicke parameter satisfies
\(
\omega_{\rm BD}^{\rm eff} \!\to\! \infty
\)
at equilibrium (since $\Pi'(\Xi_{\rm eq})=0$),
implying the post–Newtonian parameter
\(
\gamma = 1
\)
to leading order,
with corrections only at
\(
\mathcal{O}(\Delta\Xi^2 / \sigma_\chi^2).
\)

\subsection*{S4.4 Even scalar sector and width provenance}

\noindent
With $K\!\succ\!0$ and the Gaussian gate width $\sigma_\chi$ fixed by the Fisher curvature,
the depth scale
\[
\Lambda_\chi \equiv \frac{\sigma_\chi}{\|\chi\|_{K}}
\]
sets the canonical response amplitude used below.

To parameterize scalar widths without generating a Pauli–Fierz mass,
use a parity–even quadratic potential in field space:
\[
V(\boldsymbol{\Psi})
=\tfrac{1}{2}\,(\boldsymbol{\Psi}-\boldsymbol{\Psi}_{\rm eq})^{\!\top}
   \Sigma_{\perp}^{-1} P_\perp
   (\boldsymbol{\Psi}-\boldsymbol{\Psi}_{\rm eq})
+\tfrac{\gamma}{2}\,
   \big(\boldsymbol{\chi}\!\cdot(\boldsymbol{\Psi}-\boldsymbol{\Psi}_{\rm eq})\big)^2,
\]
where $\Sigma_{\perp}^{-1}\!\succ\!0$ on $P_\perp$ and $\gamma>0$.

\paragraph*{Hessian and depth–mode mass.}
At equilibrium,
\[
H \equiv \partial_i\partial_j V\big|_{\rm eq}
   = \Sigma_{\perp}^{-1} P_\perp
   + \gamma\,\boldsymbol{\chi}\boldsymbol{\chi}^{\!\top}.
\]
The canonically normalized soft–mode mass is
\[
m_\chi^2
= \frac{\boldsymbol{\chi}^{\!\top} H\,\boldsymbol{\chi}}
       {\boldsymbol{\chi}^{\!\top}K\boldsymbol{\chi}}
= \frac{\boldsymbol{\chi}^{\!\top}\Sigma_{\perp}^{-1}P_\perp\boldsymbol{\chi}}
       {\boldsymbol{\chi}^{\!\top}K\boldsymbol{\chi}}
+ \gamma\,
  \frac{(\boldsymbol{\chi}^{\!\top}\boldsymbol{\chi})^2}
       {\boldsymbol{\chi}^{\!\top}K\boldsymbol{\chi}}.
\]
Since $P_\perp\boldsymbol{\chi}=0$,
\[
\boxed{
m_\chi^2
= \gamma\,\frac{(\boldsymbol{\chi}^{\!\top}\boldsymbol{\chi})^2}
             {\boldsymbol{\chi}^{\!\top}K\boldsymbol{\chi}}
\;\equiv\;
\gamma_\chi\,\|\boldsymbol{\chi}\|_{K}^{-2},
\qquad
\gamma_\chi \equiv \gamma(\boldsymbol{\chi}^{\!\top}\boldsymbol{\chi})^2.
}
\]
An even scalar potential therefore generates finite widths in the scalar sector
while preserving the massless, luminal spin–2 tensor sector
and forbidding any linear $h$–$\delta\Xi$ mixing.

\subsection*{S4.5 Covariant embedding (summary and cross–ref)}

With
\[
S
=\int d^4x\,\sqrt{-g}\,
\Big[
\tfrac{1}{2}\,\Omega(\boldsymbol{\Psi})\,R
-\tfrac{1}{2}\,G_{ij}(\boldsymbol{\Psi})\,\nabla_\mu\xi^i\nabla^\mu\xi^j
- V(\boldsymbol{\Psi})
+ \mathcal{L}_{\rm gauge}
+ \mathcal{L}_{\rm matter}
\Big],
\]
metric variation yields
\[
\Omega\,G_{\mu\nu}
+(g_{\mu\nu}\Box - \nabla_\mu\nabla_\nu)\Omega
= T^{(\Psi)}_{\mu\nu}
+ T^{\rm gauge}_{\mu\nu}
+ T^{\rm matter}_{\mu\nu}.
\]

\noindent
Calibrating $\Omega(\boldsymbol{\Psi}_{\rm eq}) = M_{\rm Pl}^2$
(i.e. $\Pi(\Xi_{\rm eq})=1$)
and using $\Pi'(\Xi_{\rm eq}) = 0$
recovers the General Relativity quadratic sector exactly.
Expanding in $\delta\Xi$ reproduces the quadratic response of S3,
with leading deviation
\[
\frac{\Delta G}{G}
\simeq
\frac{\Delta\Xi^2}{\sigma_\chi^2}.
\]

\subsection*{S4.6 Equilibrium metric and spectrum}\label{A:metric}

\paragraph*{Kinetic term (equilibrium metric).}
Work in log–coupling coordinates
\[
\hat{\boldsymbol{\Psi}}
=(\hat{\xi}_s,\hat{\xi}_2,\hat{\xi}_\alpha)
=(\ln\hat{\alpha}_s,\ \ln\hat{\alpha}_2,\ \ln\hat{\alpha}),
\qquad
\Xi = \boldsymbol{\chi}\!\cdot\!\hat{\boldsymbol{\Psi}},
\quad
\boldsymbol{\chi}=(16,13,2).
\]
The scalar kinetic Lagrangian is
\[
\boxed{
\mathcal{L}_{\rm kin}
= -\tfrac{1}{2}\,
  \partial_\mu \hat{\boldsymbol{\Psi}}^{\!\top}\,
  K(\hat{\boldsymbol{\Psi}})\,
  \partial^\mu \hat{\boldsymbol{\Psi}},
\qquad
K(\hat{\boldsymbol{\Psi}})\succ0
}
\]
and at the equilibrium point
\[
K \equiv K(\hat{\boldsymbol{\Psi}}_{\rm eq})
=
\begin{bmatrix}
1.2509 & -0.6202 & -0.1813 \\
-0.6202 & 1.5128 & -0.1633 \\
-0.1813 & -0.1633 & 3.2362
\end{bmatrix},
\qquad
K\succ0.
\]

\paragraph*{Spectrum and alignment.}
Let $\{\lambda_i,\boldsymbol{e}_i\}$ be the orthonormal eigenpairs of $K$
(using the Euclidean inner product):
\[
\lambda_{\min}=0.7243366,\qquad
\lambda_2=2.0155976,\qquad
\lambda_{\max}=3.2599658,
\]
with
\[
\boldsymbol{e}_{\rm soft}=(0.7724942,\ 0.6276375,\ 0.0965604),
\qquad
K=\sum_{i=1}^3 \lambda_i\,\boldsymbol{e}_i\boldsymbol{e}_i^{\!\top}.
\]
Numerically,
\[
\hat{\boldsymbol{\chi}}
\equiv\frac{\boldsymbol{\chi}}{\|\boldsymbol{\chi}\|_2}
=(0.7724873,\ 0.6276459,\ 0.0965609),
\qquad
\cos\theta_{K}
:=\hat{\boldsymbol{\chi}}\!\cdot\!\boldsymbol{e}_{\rm soft}
=1.0000000\pm\mathcal{O}(10^{-8}),
\]
and
\[
K\,\boldsymbol{\chi}
=\lambda_{\min}\,\boldsymbol{\chi}\pm\mathcal{O}(10^{-4})
\quad\text{(componentwise)}.
\]
Thus $\boldsymbol{\chi}$ aligns with the soft eigenmode within numerical precision.

\paragraph*{Metric-aware projectors (canonical)}
Let $K\equiv K_{\rm eq}$ and $\langle u,v\rangle_{K}=u^{\top}K\,v$. The $K$-orthogonal projector onto $\mathrm{span}\{\boldsymbol{\chi}\}$ (for column vectors) is
\[
\boxed{
P_{\chi} \;=\; \frac{\boldsymbol{\chi}\,\boldsymbol{\chi}^{\!\top}K}{\boldsymbol{\chi}^{\!\top}K\,\boldsymbol{\chi}}, 
\qquad
P_{\perp} \;=\; \mathbb{1}-P_{\chi}.
}
\]
It satisfies
\[
P_{\chi}^2=P_{\chi},\qquad P_{\chi}^{\!\top}K=K\,P_{\chi},\qquad
\mathrm{ran}(P_{\chi})=\mathrm{span}\{\boldsymbol{\chi}\},\qquad
\ker(P_{\chi})=\{v:\boldsymbol{\chi}^{\!\top}K\,v=0\}.
\]
\textbf{Note.} The matrix
\[
\tilde P \;=\; \frac{K\,\boldsymbol{\chi}\,\boldsymbol{\chi}^{\!\top}}{\boldsymbol{\chi}^{\!\top}K\,\boldsymbol{\chi}}
\]
is \emph{not} the $K$-orthogonal projector onto $\mathrm{span}\{\boldsymbol{\chi}\}$: its range is $\mathrm{span}\{K\boldsymbol{\chi}\}$, it uses Euclidean orthogonality $\{\boldsymbol{\chi}^{\!\top}v=0\}$ instead of $K$-orthogonality $\{\boldsymbol{\chi}^{\!\top}K\,v=0\}$, and in general $\tilde P^{\!\top}K\neq K\,\tilde P$.

\paragraph*{Consequences (used throughout).}
\begin{itemize}\setlength\itemsep{2pt}
\item \textbf{Depth norm:}
\[
\|\boldsymbol{\chi}\|_{K}^{2}
=\boldsymbol{\chi}^{\!\top}K\boldsymbol{\chi}
=\lambda_{\min}\,\boldsymbol{\chi}^{\!\top}\boldsymbol{\chi}
=\lambda_{\min}\times429
\Rightarrow
\|\boldsymbol{\chi}\|_{K}=17.6278.
\]
\item \textbf{Gate scale:}
for the even curvature–gate width $\sigma_\chi=247.683$,
\[
\Lambda_\chi
=\frac{\sigma_\chi}{\|\boldsymbol{\chi}\|_{K}}
=14.0507,
\qquad
\omega_{\rm hel}=\Lambda_\chi^{-1}=0.0712,
\qquad
T_{\rm hel}=2\pi\,\Lambda_\chi\simeq88\,t_P.
\]
\item \textbf{Softest direction:}
for any displacement $\boldsymbol{\omega}$,
the quadratic form
\(
Q(\boldsymbol{\omega})
=\boldsymbol{\omega}^{\!\top}K\boldsymbol{\omega}
\)
is minimized along $\boldsymbol{\chi}$;
orthogonal motion costs higher curvature.
\end{itemize}

\paragraph*{Eigen–decomposition and positivity.}
Let $R=[\boldsymbol{e}_1\,\boldsymbol{e}_2\,\boldsymbol{e}_3]$ be orthogonal. Then
\[
R^\top K R
=\mathrm{diag}(\lambda_1,\lambda_2,\lambda_3),
\qquad
\lambda_i>0.
\]

\paragraph*{Depth direction and $K$ norm.}
For $\boldsymbol{\chi}=(16,13,2)$ define
\[
\|\boldsymbol{\chi}\|_{K}^{2}
:=\boldsymbol{\chi}^\top K\boldsymbol{\chi},
\qquad
\hat{\boldsymbol{\chi}}_{K}
:=\frac{\boldsymbol{\chi}}
{\sqrt{\boldsymbol{\chi}^{\!\top}K\boldsymbol{\chi}}},
\qquad
\hat{\boldsymbol{\chi}}
:=\frac{\boldsymbol{\chi}}{\|\boldsymbol{\chi}\|_2}.
\]

\paragraph*{Provenance.}
Entries of $K$ are obtained by spectral reconstruction:
\[
K
=R\,\mathrm{diag}(\lambda_{\min},\lambda_2,\lambda_{\max})\,R^{\!\top},
\]
with $R=[\boldsymbol{e}_{\rm soft}\ \boldsymbol{e}_2\ \boldsymbol{e}_3]$ orthonormal,
eigenvalues $(0.7243366,\ 2.0155976,\ 3.2599658)$,
and $\boldsymbol{e}_{\rm soft}\parallel\boldsymbol{\chi}$.
A reproducible script (\texttt{scripts/keq\_spectral.py}) regenerates $K$
from these pins; SHA–256 and CSV are included in the repository.

\subsection*{S4.7 Scalar potential, widths, and consistency certificate}

\paragraph*{Purpose.}
The scalar potential below is not required for graviton emergence or GR normalization
(those follow from the parity of the curvature gate $\Pi(\Xi)$; see S2/S3.2).
This section certifies that one can assign a consistent EFT width to the depth mode
and regulate transverse directions without inducing a Pauli–Fierz mass
or linear $h$–$\phi$ mixing.

\paragraph*{Scalar potential (parity–even, quadratic).}
In log–coupling space write
\[
\boldsymbol{\Psi}
=(\hat{\xi}_s,\hat{\xi}_2,\hat{\xi}_\alpha)
=(\ln\hat{\alpha}_s,\ln\hat{\alpha}_2,\ln\hat{\alpha}),
\qquad
\boldsymbol{\chi}=(16,13,2),
\qquad
\Xi=\boldsymbol{\chi}\!\cdot\!\boldsymbol{\Psi}.
\]
Define $\Delta\boldsymbol{\Psi}=\boldsymbol{\Psi}-\boldsymbol{\Psi}_{\rm eq}$,
$\Xi_{\rm eq}=\boldsymbol{\chi}\!\cdot\!\boldsymbol{\Psi}_{\rm eq}$,
and take
\[
V(\boldsymbol{\Psi})
=\tfrac12\sum_{i\in\{s,2,\alpha\}}
  \frac{(\xi_i-\xi_i^{(\rm eq)})^2}{\sigma_i^2}
+\tfrac{\gamma}{2}\,
  \big(\boldsymbol{\chi}\!\cdot\!\Delta\boldsymbol{\Psi}\big)^2.
\]
Parity about $\Xi_{\rm eq}$ is manifest:
$\partial_{\xi_i}V|_{\rm eq}=0$ and all odd powers in
$(\boldsymbol{\chi}\!\cdot\!\Delta\boldsymbol{\Psi})$ vanish.

\paragraph*{Equivalent projector form (metric–correct).}
Let
\[
P_\chi
=\frac{\boldsymbol{\chi}\,\boldsymbol{\chi}^{\!\top}K}
       {\boldsymbol{\chi}^{\!\top}K\boldsymbol{\chi}},
\qquad
P_\perp=\mathbb{1}-P_\chi,
\qquad
K\succ0.
\]
Then an equivalent form that makes the transverse restriction explicit is
\begin{equation}
V(\boldsymbol{\Psi})
=\tfrac12\,\Delta\boldsymbol{\Psi}^{\!\top}
 (P_\perp\,\Sigma_\perp^{-1}\,P_\perp)
 \Delta\boldsymbol{\Psi}
+\tfrac{\gamma}{2}\,
 (\boldsymbol{\chi}\!\cdot\!\Delta\boldsymbol{\Psi})^2,
\qquad
\Sigma_\perp^{-1}
=\mathrm{diag}\!\Big(\tfrac{1}{\sigma_{\alpha_s}^2},
                     \tfrac{1}{\sigma_{\alpha_2}^2},
                     \tfrac{1}{\sigma_{\alpha}^2}\Big).
\label{eq:V_projector_form}
\end{equation}
(Using $P_\perp=\mathbb{1}-K\boldsymbol{\chi}\boldsymbol{\chi}^{\!\top}/
(\boldsymbol{\chi}^{\!\top}K\boldsymbol{\chi})$
would \emph{not} yield the $K$–orthogonal projector on column vectors.)

\paragraph*{Parameter choices (depth vs transverse).}
\textbf{Depth (derived, fixed):}
\[
\sigma_\chi=247.683,\qquad
\|\boldsymbol{\chi}\|_{K}=17.6278,\qquad
\Lambda_\chi=\frac{\sigma_\chi}{\|\boldsymbol{\chi}\|_{K}}=14.0507,
\qquad
\omega_{\rm hel}=\Lambda_\chi^{-1}=0.0712,
\qquad
T_{\rm hel}=2\pi\,\Lambda_\chi\simeq88\,t_P.
\]
These follow from the Fisher curvature and kinetic norm.

\textbf{Transverse (regulator pins, fixed once):}
\[
\sigma_{\alpha_s}=0.446296,\qquad
\sigma_{\alpha_2}=0.547533,\qquad
\sigma_{\alpha}=0.551281.
\]
They regulate the $P_\perp$ plane only and do not affect the GR tensor sector.

\textbf{Isotropic fallback (metric–aware):}
\[
\boxed{
\Sigma_\perp = C\,P_\perp,
\qquad
P_\perp=\mathbb{1}
 -\frac{\boldsymbol{\chi}\,\boldsymbol{\chi}^{\!\top}K}
        {\boldsymbol{\chi}^{\!\top}K\boldsymbol{\chi}}.
}
\]
This enforces equal variance in all directions orthogonal to $\boldsymbol{\chi}$;
componentwise recipes such as $\sigma_i\!\propto\!|\chi_i|$
are not isotropic in the $K$ metric and should be avoided.

\paragraph*{Hessian and depth–mode mass.}
Expanding \eqref{eq:V_projector_form},
\[
H \equiv \partial_i\partial_j V|_{\rm eq}
= P_\perp\,\Sigma_\perp^{-1}\,P_\perp + \gamma\,\boldsymbol{\chi}\boldsymbol{\chi}^{\!\top}.
\]
Project along $\boldsymbol{\chi}$ and normalize by $K$:
\[
m_\chi^2
=\frac{\boldsymbol{\chi}^{\!\top}H\,\boldsymbol{\chi}}
       {\boldsymbol{\chi}^{\!\top}K\boldsymbol{\chi}}
=\frac{\boldsymbol{\chi}^{\!\top}P_\perp\Sigma_\perp^{-1}P_\perp\boldsymbol{\chi}}
       {\boldsymbol{\chi}^{\!\top}K\boldsymbol{\chi}}
+\gamma\,\frac{(\boldsymbol{\chi}^{\!\top}\boldsymbol{\chi})^2}
       {\boldsymbol{\chi}^{\!\top}K\boldsymbol{\chi}}.
\]
Since $P_\perp\boldsymbol{\chi}=0$,
\[
\boxed{
m_\chi^2
=\gamma\,\frac{(\boldsymbol{\chi}^{\!\top}\boldsymbol{\chi})^2}
             {\boldsymbol{\chi}^{\!\top}K\boldsymbol{\chi}}
=\gamma_\chi\,\|\boldsymbol{\chi}\|_{K}^{-2},
\qquad
\gamma_\chi\equiv\gamma(\boldsymbol{\chi}^{\!\top}\boldsymbol{\chi})^2.
}
\]
Curvature thus resides only along the $\boldsymbol{\chi}$ direction.

\paragraph*{Transverse regulator implementation.}
With the $K$–metric projectors above,
implement the regulator as $P_\perp\,\Sigma_\perp^{-1}\,P_\perp$.
This preserves the depth gate and tensor sector
and prevents spurious mixing.

\paragraph*{Gate in canonical form and parity lemma (recap).}
Define
\[
\phi_\chi
=\frac{\boldsymbol{\chi}^{\!\top}(\boldsymbol{\Psi}-\boldsymbol{\Psi}_{\rm eq})}
       {\|\boldsymbol{\chi}\|_{K}},
\qquad
\Delta\Xi=\|\boldsymbol{\chi}\|_{K}\,\phi_\chi,
\]
so
\[
\boxed{
\Pi(\Xi)
=\exp\!\Big[-\frac{\phi_\chi^2}{\Lambda_\chi^2}\Big],
\qquad
\Lambda_\chi=\frac{\sigma_\chi}{\|\boldsymbol{\chi}\|_{K}}.
}
\]
Near equilibrium
$|\Delta G/G|\simeq\phi_\chi^2/\Lambda_\chi^2$,
with the odd term absent:
$\partial_\Xi\Pi|_{\Xi_{\rm eq}}=0
\Rightarrow$
no $h$–$\phi$ mixing, no Pauli–Fierz mass.

\paragraph*{Expansion about equilibrium and quadratic Lagrangian.}
Set
$g_{\mu\nu}=\eta_{\mu\nu}+h_{\mu\nu}$,
$\boldsymbol{\Psi}=\boldsymbol{\Psi}_{\rm eq}+\boldsymbol{\phi}$.
With
$M_*^2:=M_{\rm Pl}^2\Pi(\Xi)|_{\boldsymbol{\Psi}=\boldsymbol{\Psi}_{\rm eq}}
=M_{\rm Pl}^2$
and
$\Delta\Xi=\boldsymbol{\chi}^{\!\top}\boldsymbol{\phi}$,
\[
\Pi(\Xi)
=1-\frac{(\Delta\Xi)^2}{\sigma_\chi^2}
+\mathcal{O}(\phi^4).
\]
To quadratic order,
\[
\mathcal{L}^{(2)}
=\tfrac{M_*^2}{8}\,
  h^{\mu\nu}\mathcal{E}_{\mu\nu}^{\ \ \rho\sigma}h_{\rho\sigma}
-\tfrac{1}{2}\,
  \partial_\mu\boldsymbol{\phi}^{\!\top}K\partial^\mu\boldsymbol{\phi}
-\tfrac{1}{2}\,
  \boldsymbol{\phi}^{\!\top}M^2\boldsymbol{\phi}
+\mathcal{O}(h\phi^2)
+\mathcal{O}(h^2\phi),
\]
with
$M^2=P_\perp\Sigma_\perp^{-1}P_\perp+\gamma\,\boldsymbol{\chi}\boldsymbol{\chi}^{\!\top}$,
and $\mathcal{E}_{\mu\nu}^{\ \ \rho\sigma}$ the Lichnerowicz operator.
Parity ensures the linear $h$–$\phi$ term cancels:
the graviton remains massless and luminal.

\paragraph*{Weinberg soft factor (unchanged).}
In the soft limit $q\!\to\!0$,
\[
\mathcal{M}_{n+1}\simeq
\kappa\,S^{(0)}(q,\varepsilon)\,\mathcal{M}_n,
\qquad
S^{(0)}=\sum_{i=1}^n
\eta_i\frac{p_i^\mu p_i^\nu\varepsilon_{\mu\nu}}{p_i\!\cdot\!q},
\qquad
\kappa=\frac{2}{M_*},
\]
with $\eta_i=\pm1$ and transverse–traceless $\varepsilon_{\mu\nu}$.
Depth parity at equilibrium leaves $S^{(0)}$ invariant.

\paragraph*{Light deflection.}
Because $\Pi(\Xi)=1+\mathcal{O}((\Delta\Xi)^2)$,
the leading eikonal deflection equals the GR value
\[
\theta=\frac{4GM}{b\,c^2},
\]
with fractional corrections $\mathcal{O}((\Delta\Xi/\sigma_\chi)^2)$.
In PPN form,
\[
\gamma_{\rm PPN}=1+\mathcal{O}\!\big((\Delta\Xi/\sigma_\chi)^2\big),
\]
and $G=G_N$ by calibration at equilibrium.

\paragraph*{One–loop counterterm container map (near equilibrium).}
Divergences renormalize only
$\{\Pi(\Xi),K,V(\boldsymbol{\Psi})\}$ and higher curvature;
no linear $\Delta\Xi$ counterterm appears by parity.
Finite parts are absorbed as:
\begin{table}[t]
\centering
\caption{One–loop counterterm container map near equilibrium (finite parts).}
\label{T:ctmap}
\begin{tabular}{@{}ll@{}}
\toprule
Counterterm & Container \\
\midrule
$c_1R^2 + c_2R_{\mu\nu}R^{\mu\nu}$ & finite normalization of EH sector (no PF term) \\
$d_1R\,\Delta\Xi^2$ & renormalizes $\sigma_\chi$ in gate expansion \\
$e_1\nabla_\mu\boldsymbol{\Psi}\,K\,\nabla^\mu\boldsymbol{\Psi}$ & renormalizes $K$ (wavefunction) \\
$e_2\boldsymbol{\Psi}^{\!\top}M^2\boldsymbol{\Psi}$ & renormalizes $M^2$ in $V(\boldsymbol{\Psi})$ \\
\bottomrule
\end{tabular}
\end{table}

\paragraph*{Positivity and bounds.}
Near equilibrium with diagonal tensor and regulated scalar sectors, require
\[
K\succ0,\qquad
\sigma_\chi^2>0,\qquad
\gamma>0,\qquad
M_*^2=M_{\rm Pl}^2\Pi(\Xi_{\rm eq})>0,
\]
ensuring GR tensor propagation
and a stable scalar sector with no linear fifth force.

\section*{S4.8 Quantum–field formulation at equilibrium}\label{S4:QFT}

\paragraph*{Scope and notation.}
Work at equilibrium with
\(g_{\mu\nu}=\eta_{\mu\nu}+h_{\mu\nu}\),
\(\hat{\boldsymbol{\Psi}}=\hat{\boldsymbol{\Psi}}_{\rm eq}+\delta\hat{\boldsymbol{\Psi}}\),
\(\Xi=\boldsymbol{\chi}\!\cdot\!\hat{\boldsymbol{\Psi}}\),
\(\Pi'(\Xi_{\rm eq})=0\), \(\Pi(\Xi_{\rm eq})=1\),
and \(K\succ0\).
Hats denote \(\overline{\rm MS}\) values at \(\mu=M_Z\);
\(\delta\Xi=u_i\,\delta\hat{\xi}_i\) with \(u=(16,13,2)=\boldsymbol{\chi}\).
The equilibrium tensor sector is GR–normalized (\(m_{\rm PF}=0\), \(c_T=1\)).

%------------------------------------------------
\paragraph*{Quadratic kernel, gauge fixing, and propagator.}
\[
\mathcal L^{(2)}_{hh}
=\frac{M_{\rm Pl}^2}{4}\,h_{\mu\nu}E^{\mu\nu,\rho\sigma}h_{\rho\sigma},\qquad
E^{\mu\nu,\rho\sigma}h_{\rho\sigma}
=-\Box h^{\mu\nu}+\partial^\mu\partial_\rho h^{\rho\nu}
+\partial^\nu\partial_\rho h^{\rho\mu}
-\partial^\mu\partial^\nu h
-\eta^{\mu\nu}(\partial_\rho\partial_\sigma h^{\rho\sigma}-\Box h).
\]
Add de Donder gauge
\(F_\nu=\partial_\mu\bar h^{\mu}{}_{\nu}=0\),
\(\bar h_{\mu\nu}=h_{\mu\nu}-\tfrac12\eta_{\mu\nu}h\).
In momentum space,
\[
D_{\mu\nu,\rho\sigma}(k)
=\frac{i}{M_{\rm Pl}^2}\,\frac{\Pi_{\mu\nu,\rho\sigma}}{k^2+i\epsilon},
\qquad
\Pi_{\mu\nu,\rho\sigma}
=\tfrac12(\eta_{\mu\rho}\eta_{\nu\sigma}
+\eta_{\mu\sigma}\eta_{\nu\rho}-\eta_{\mu\nu}\eta_{\rho\sigma}),
\]
so the propagator equals the GR one,
\(
D=i\,16\pi G_N\,\Pi/(2k^2).
\)

%------------------------------------------------
\paragraph*{Barnes–Rivers projectors.}
Define \(\theta_{\mu\nu}=\eta_{\mu\nu}-k_\mu k_\nu/k^2\),
\(\omega_{\mu\nu}=k_\mu k_\nu/k^2\):
\[
P^{(2)}_{\mu\nu,\rho\sigma}
=\tfrac12(\theta_{\mu\rho}\theta_{\nu\sigma}
+\theta_{\mu\sigma}\theta_{\nu\rho})
-\tfrac13\theta_{\mu\nu}\theta_{\rho\sigma},\qquad
P^{(0\!-\!s)}_{\mu\nu,\rho\sigma}
=\tfrac13\theta_{\mu\nu}\theta_{\rho\sigma}.
\]
They satisfy \(P^{(2)}+P^{(0\!-\!s)}=P_T\) and
\(P^{(A)}P^{(B)}=\delta_{AB}P^{(A)}\) for \(A,B\in\{2,0\!-\!s\}\).
\emph{Usage:} any rank–4 UV pole decomposes into \(P^{(2)}\), \(P^{(0\!-\!s)}\),
making gauge–parameter independence manifest at the projector level.

%------------------------------------------------
\paragraph*{Three–point vertices (equilibrium rules).}
\[
\begin{aligned}
\text{GR }hhh:&\quad
\text{standard Einstein–Hilbert rule, }\kappa=\sqrt{32\pi G_N}.\\[4pt]
\text{Minimal }h\xi\xi:&\;
iV^{\rm kin}_{\mu\nu,ij}(p,q)
=-\tfrac{i}{2}G^\star_{ij}
\!\left(\tfrac12\eta_{\mu\nu}p\!\cdot\!q-p_\mu q_\nu\right).\\[4pt]
\text{Gate–induced }h\xi\xi:&\;
iV^{\rm gate}_{\mu\nu,ij}(k;p,q)
=-\tfrac{i}{2\sigma_\chi^2}
u_i u_j\,(k_\mu k_\nu-\eta_{\mu\nu}k^2).\\[4pt]
\text{Potential }h\xi\xi:&\;
iV^{\rm pot}_{\mu\nu,ij}(p,q)
=-\tfrac{i}{4}(M^2)_{ij}\eta_{\mu\nu}.
\end{aligned}
\]
The gate vertex is transverse and suppressed by \(\sigma_\chi^{-2}\).

%------------------------------------------------
\paragraph*{One–loop scalar bubble (UV pole; projector form).}
\[
\boxed{
\Pi_{\mu\nu,\rho\sigma}\Big|_{\rm div}
= \frac{i}{(4\pi)^2}\frac{1}{\varepsilon}\,
\kappa^2 k^4\!\left[\frac{1}{60}\,P^{(2)}_{\mu\nu,\rho\sigma}
+\frac{1}{120}\,P^{(0\!-\!s)}_{\mu\nu,\rho\sigma}\right]\times N_s
\;+\;O(\sigma_{\chi}^{-2})
}
\]
The bracketed coefficients are per real scalar;
\(N_s\) counts the real scalars running in the loop
(in the anchor basis \(N_s=3\)).
The pole maps to local
\(\int\!d^4x\,\sqrt{-g}\{R_{\mu\nu}R^{\mu\nu},R^2\}\)
counterterms and is gauge–parameter independent (BRST).
\footnotesize\emph{Footnote.} The projector coefficients
\(\tfrac{1}{60}\) and \(\tfrac{1}{120}\) are the standard real–scalar values.
For \(N_s=3\) anchors the total scalar–bubble pole carries an overall factor of 3.\normalsize

%------------------------------------------------
\paragraph*{BRST ghosts.}
\(
\mathcal L_{\rm gh}
=-\bar c_\nu\partial^\mu(\partial_\mu c^\nu
+\partial^\nu c_\mu-\eta_\mu^{\ \nu}\partial_\rho c^\rho)\);
at equilibrium \(\Pi'(\Xi_{\rm eq})=0\),
so the Slavnov–Taylor identities coincide with GR.

%------------------------------------------------
\paragraph*{Soft–graviton theorem (equilibrium).}
\[
\mathcal M_{n+1}(q\!\to\!0)
=\kappa\!\left[\sum_i\eta_i
\frac{p_i^\mu p_i^\nu\,\varepsilon_{\mu\nu}}{p_i\!\cdot\!q}\right]\!
\mathcal M_n+\mathcal O(q^0),\qquad
\kappa^2=32\pi G_N,
\]
since the gate vertex contributes only \(\mathcal O(q^0)\).

\paragraph*{Worked example (soft theorem; \(2\!\to\!2{+}\)soft).}
Attach a soft graviton with momentum \(q\!\to\!0\)
and polarization \(\varepsilon_{\mu\nu}\)
to a tree–level \(2\!\to\!2\) matter amplitude \(\mathcal M_n\).
Using the equilibrium rules
(GR–normalized propagator and vertices; the gate vertex contributes only \(O(q^0)\)),
\[
\boxed{
\mathcal M_{n+1}(q\to0)
= \kappa
\!\left[\sum_{i=1}^{n}\eta_i
\frac{p_i^\mu p_i^\nu\,\varepsilon_{\mu\nu}}{p_i\!\cdot q}\right]
\mathcal M_n
+ O(q^0),\qquad
\kappa^2=32\pi G_N.
}
\]
This reproduces Weinberg’s \(1/q\) factor; with \(\Pi'(\Xi_{\rm eq})=0\),
the \(O(q^{-1})\) soft structure is identical to GR.

%------------------------------------------------
\paragraph*{Higher–curvature container and parity.}
\[
S_{\rm HC}
=\!\int\!d^4x\,\sqrt{-g}
\sum_{n\ge2} c_n(\hat{\boldsymbol{\Psi}})\,I_n[g],
\qquad
c_n(\hat{\boldsymbol{\Psi}})
=c_n^{(0)}+c_n^{(2)}\delta\Xi^2+\cdots
\ \text{(even in }\delta\Xi\text{)}.
\]
All counterterms expand in even powers of \(\delta\Xi\);
odd (linear) terms are symmetry–forbidden (\(Z_2\) spurion parity).

%------------------------------------------------
\paragraph*{GR limit and PPN/GW summary.}
At equilibrium the tensor remains massless and luminal (\(c_T=1\));
post–Newtonian parameters satisfy
\(\gamma=\beta=1+\mathcal O(\Delta\Xi^2/\sigma_\chi^2)\),
so deviations lie well below multimessenger bounds.

%------------------------------------------------
\paragraph*{Summary.}
Renormalization closes on
\(\{\Pi(\Xi),K_{ij},V(\hat{\boldsymbol{\Psi}}),c_n(\hat{\boldsymbol{\Psi}})\}\)
with even \(\delta\Xi\) expansions.
No fifth–force operator arises at any loop order.







\section*{S5. RG running and Ward-flatness monitor}

\subsection*{S5.1 Definition and admissible windows}

\paragraph*{Running-depth observable.}
In the $\MSbar$ scheme define
\[
F(Q)\;\equiv\;\beta_\Xi(Q)
=\chis\!\cdot\!\frac{\dd \Psivec}{\dlnQ}
=16\,\frac{\dd(\LN\alphas)}{\dlnQ}
+13\,\frac{\dd(\LN\alphaTwo)}{\dlnQ}
+2\,\frac{\dd(\LN\alphae)}{\dlnQ},
\]
with $\Psivec=(\LN\alphas,\LN\alphaTwo,\LN\alphae)$ and $\chis=(16,13,2)$.

\paragraph*{Admissible windows.}
A window $\mathcal W$ is admissible if the particle content is fixed (mass–independent scheme), all heavy thresholds lie outside $\mathcal W$, and the EM basis is used post–EWSB. Within any such $\mathcal W$, Appelquist–Carazzone decoupling applies and the Smith–normal–form identity
\[
\chis\!\cdot b^{(\mathcal W)}=0 \qquad \text{(one loop, GUT normalization)}
\]
cancels the \emph{$\alpha$–independent} one–loop drift. Writing
\[
F^{(1\mathrm L)}(Q)=\frac{1}{2\pi}\sum_{i=1}^3 \chi_i\, b_i\, \alpha_i(Q),
\]
the coupling weights $\alpha_i(Q)$ prevent an exact zero away from the pivot, so small residuals remain; these are the target of the preregistered bands.

\paragraph*{Masked windows (preregistered).}
We evaluate $F$ on
\[
W_{\rm EW}=\qtyrange{80}{160}{\mathrm{GeV}},\qquad
W_{\rm GeV}=\qtyrange{1}{10}{\mathrm{GeV}},
\]
sampling $Q$ logarithmically and excising symmetric guard bands around thresholds prior to statistics on $F_\sigma$. Table~\ref{tab:threshold_masks} lists the masks used in all runs.

\begin{table}[t]
\centering
\caption{Threshold mask ranges (excluded from $F_\sigma$ statistics).}
\label{tab:threshold_masks}
\scriptsize
\renewcommand{\arraystretch}{1.1}
\begin{tabular}{@{}lcc@{}}
\toprule
Threshold & Central value [GeV] & Masked range [GeV] \\
\midrule
$W$       & $80.4$  & $[79.0,\ 82.0]$ \\
$Z$       & $91.2$  & $[90.0,\ 92.5]$ \\
$H$       & $125.3$ & $[124.0,\ 127.0]$ \\
$t$       & $172.5$ & $[171.0,\ 175.0]$ \\
$b$       & $4.18$  & $[4.10,\ 4.30]$ \\
$c$       & $1.27$  & $[1.20,\ 1.35]$ \\
\bottomrule
\end{tabular}
\end{table}

Masks are applied within $W_{\rm EW}$ and $W_{\rm GeV}$. Grid and mask variations ($\pm20\%$ step, $\pm25\%$ mask half-width) leave pass/fail unchanged (Sec.~S5.2).

\paragraph*{Rationale for preregistration.}
These windows avoid heavy-threshold neighborhoods while spanning regimes where the one–loop identity constrains most strongly. Bands below are conservative falsifier envelopes, not fit targets.

\paragraph*{Letter cross–reference.}
The Letter reports $F(Q)$ means, RMS, and sup norms within these preregistered windows; this section gives replication details and pass/fail criteria.


\subsection*{S5.2 Computation pipeline and preregistered bounds}

For each $Q\in W_{\rm EW}\cup W_{\rm GeV}$:
\begin{enumerate}[label=(\roman*)]\setlength\itemsep{2pt}
\item Evolve $\alphas(Q)$, $\alphaTwo(Q)$, $\alphae(Q)$ with SM $\MSbar$ RGEs (1L/2L as specified), using standard matching at heavy thresholds ($t,H,W,Z$, heavy quarks) and step decoupling for QCD where indicated.
\item Form $\Xi(Q)=\chis\!\cdot\!(\LN\alphas,\LN\alphaTwo,\LN\alphae)$.
\item Compute $F(Q)=\dd\Xi/\dlnQ$ analytically from the RGEs or via symmetric finite differences on $\Xi(Q)$.
\item Normalize $F_\sigma(Q):=F(Q)/\sigchi$ with $\sigchi=247.683$.
\item Accumulate per–window statistics on $F_\sigma$: $\mathrm{MAX}_W=\max|F_\sigma|$, $\mathrm{RMS}_W=\sqrt{\langle F_\sigma^2\rangle}$, and $|\langle F_\sigma\rangle|$ over the masked grid.
\end{enumerate}

\paragraph*{Targets (preregistered on $F_\sigma$).}
Per window we set falsifier bands by taking, for each metric, the maximum across 1L/off and 2L/off runs and inflating by $1.5$ (subsuming $\pm20\%$ grid and $\pm25\%$ mask variations). Numerical values (registered in S0.8) are
\[
\begin{aligned}
W_{\rm EW}:\;& \mathrm{MAX}_W\le 0.01430,\quad \mathrm{RMS}_W\le 0.01372,\quad |\langle F_\sigma\rangle|\le 0.01372,\\
W_{\rm GeV}:\;& \mathrm{MAX}_W\le 0.03535,\quad \mathrm{RMS}_W\le 0.02622,\quad |\langle F_\sigma\rangle|\le 0.02585.
\end{aligned}
\]

\paragraph*{Implementation notes.}
Pins are $\MSbar$ at $\mu=\MZ$; hats denote the pin and are suppressed in running formulas. Masks excise $\pm\delta$ around thresholds; $\delta$ values and grid spacings are in the replication pack. Uncertainties use log–space Jacobians with MC confirmation. The one–loop identity uses GUT–normalized $(b_1,b_2,b_3)$, with $b_{\rm EM}=\tfrac{5}{3}b_1+b_2$ and the pivot relation $\alphae^{-1}=\tfrac{5}{3}\alpha_1^{-1}+\alpha_2^{-1}$.

\paragraph*{Sensitivity (preemptive).}
Results are stable under $\pm20\%$ step–size changes and $\pm10\%$ window–edge shifts; threshold–mask half–widths varied by $\pm25\%$ leave pass/fail unchanged.


\subsection*{S5.3 Two–loop and $m_t$ decoupling (concise spec)}

\paragraph*{Gauge two–loop running.}
The Standard Model gauge couplings evolve in the $\overline{\mathrm{MS}}$ scheme as
\[
\frac{d\,\alpha_i}{d\ln Q}
=\frac{b_i}{2\pi}\,\alpha_i^2
+\frac{1}{8\pi^2}\sum_{j=1}^{3} b_{ij}\,\alpha_i^2\,\alpha_j
+\mathcal{O}(\alpha^4),
\qquad i,j\in\{1,2,3\},
\]
where $(b_i,b_{ij})$ are the canonical SM coefficients in GUT normalization.
The electromagnetic coupling is reconstructed from the weak–hypercharge basis as
\[
\frac{1}{\alpha_e}
=\frac{5}{3}\frac{1}{\alpha_1}
+\frac{1}{\alpha_2}.
\]

\paragraph*{QCD step decoupling at $Q=m_t$.}
At the top–quark threshold, the strong coupling $\alpha_s$ transitions between the six– and five–flavor regimes using standard step decoupling:
\[
b_3(Q)=
\begin{cases}
-\tfrac{23}{3}, & Q<m_t \quad (n_f=5),\\[4pt]
-7, & Q>m_t \quad (n_f=6),
\end{cases}
\]
with continuity of $\alpha_s(Q)$ enforced at $Q=m_t$.
Threshold neighborhoods are symmetrically masked in the Ward–flatness scans to avoid residual artifacts.
This prescription reproduces the PDG two–loop running within numerical precision across both $W_{\rm GeV}$ and $W_{\rm EW}$ windows.

\subsection*{S5.4 Two–loop Ward–flatness and higher–order drift}

The integer–lattice structure enforcing $\chis^{\!\top}\mathbf{W}=0$ holds exactly at one loop, where $\mathbf{W}$ is the gauge–sector coefficient matrix in $\overline{\mathrm{MS}}$. This gives strict Ward–flatness,
\[
\beta_{\Xi}^{(1)}=\chis^{\!\top}\mathbf{W}^{(1)}\hat{\boldsymbol{\alpha}}=0,
\]
so the projected gauge–log depth $\Xi=\chis\!\cdot\!\hat{\boldsymbol{\Psi}}$ is RG–flat to one loop. At higher order the decoupling lattice need not remain integer–factorizable: mixed terms $\alpha_i^2\alpha_j$ and Yukawa pieces appear in the two–loop coefficients $\mathbf{W}^{(2)}$ \cite{Machacek1983_TwoLoopI,Machacek1984_TwoLoopII,Luo2003_TwoLoopSM}. Consequently,
\[
\beta_{\Xi}^{(2)}
=\chis^{\!\top}\mathbf{W}^{(2)}\,\mathbf m(\hat{\boldsymbol{\alpha}})
+\chis^{\!\top}\mathbf Y^{(2)}_{\rm gauge\!-\!Yuk}\,\hat{\mathbf y}
\;\neq\;0,
\]
introducing a small drift from perfect flatness. The effect is numerically suppressed because $\hat{\alpha}_i(M_Z)\ll 1$ and the projector $\chis$ continues to weight the soft direction. Quantitatively, inserting PDG $M_Z$ inputs into the known two–loop coefficients yields
\[
\abs{\beta_{\Xi}^{(2)}}\;\lesssim\;10^{-3}\quad\text{per}\ \dlnQ,
\]
well below experimental uncertainty.

Thus the Letter’s statement
\begin{quote}
``Ward–flat at one loop; higher–order drift allowed''
\end{quote}
is strictly accurate. Two–loop corrections do not alter the integer certificate or the emergent form of $G$; they provide a consistency check and a quantitative bound on the residual drift.

\subsection*{Projected two–loop drift (method and bound)}

\paragraph*{Setup.}
Write the gauge $\beta$-functions at $\mu=M_Z$ in $\overline{\mathrm{MS}}$ as
\[
\frac{d}{d\ln Q}\,\hat{\boldsymbol{\Psi}}
= \mathbf{W}^{(1)}\,\hat{\boldsymbol{\alpha}}
+ \mathbf{W}^{(2)}\!\big[\hat{\boldsymbol{\alpha}}\!\odot\!\hat{\boldsymbol{\alpha}}\big]
+ \mathbf{Y}^{(2)}\,\hat{\boldsymbol{y}}
+ \mathcal{O}(\hat{\alpha}_i^3),
\]
where $\hat{\boldsymbol{\Psi}}=(\ln\hat\alpha_s,\ln\hat\alpha_2,\ln\hat\alpha)^\top$, 
$\hat{\boldsymbol{\alpha}}=(\hat\alpha_s,\hat\alpha_2,\hat\alpha)^\top$, 
$\odot$ denotes Hadamard (element–wise) products used schematically to encode quadratic monomials, and $\hat{\boldsymbol{y}}$ collects Yukawa/Higgs contributions in the same scheme.

\paragraph*{One–loop cancellation and definition of depth.}
The SNF certificate gives $\boldsymbol{\chi}^\top \mathbf{W}^{(1)}=0$, hence
\[
\beta_\Xi^{(1)}=\boldsymbol{\chi}^{\!\top}\mathbf{W}^{(1)}\hat{\boldsymbol{\alpha}}=0,
\qquad 
\Xi\equiv \boldsymbol{\chi}\!\cdot\!\hat{\boldsymbol{\Psi}},\quad \boldsymbol{\chi}=(16,13,2).
\]

\paragraph*{Two–loop drift.}
At two loops the integer factorization is generically broken, so
\[
\beta_\Xi^{(2)}
= \boldsymbol{\chi}^{\!\top}\mathbf{W}^{(2)}\!\big[\hat{\boldsymbol{\alpha}}\!\odot\!\hat{\boldsymbol{\alpha}}\big]
+ \boldsymbol{\chi}^{\!\top}\mathbf{Y}^{(2)}\,\hat{\boldsymbol{y}}
\;\neq\;0,
\]
producing a small drift. Using PDG $M_Z$ pins as representative inputs,
\[
\hat\alpha_s\simeq 0.118,\quad
\hat\alpha_2\simeq 0.0338,\quad
\hat\alpha\simeq 0.00782,
\]
and standard two–loop normalizations,\footnote{Entries of $\mathbf{W}^{(2)}$ and $\mathbf{Y}^{(2)}$ are $\mathcal{O}(1$–$10)$ in canonical conventions; see e.g.\ two–loop compilations in Machacek–Vaughn and Luo–Wang–Xiao.}
one finds the projected drift to be numerically suppressed:
\[
\boxed{%
\left|\beta_\Xi^{(2)}\right|\;\lesssim\;\mathcal{O}(10^{-3})
\quad\text{per } d\ln Q}
\]
which is comfortably below experimental resolution and within the preregistered Ward–flatness bounds on $F_\sigma$ (S5.2).

\paragraph*{Interpretation.}
Two–loop effects renormalize the gate width $\sigma_\chi$ and induce a tiny SM–internal running of $G(Q)$, without altering the integer certificate or the GR–normalized, $m_{\mathrm{PF}}=0$ tensor sector.


\subsection*{Projected two–loop drift: $3\times 6$ form}

\paragraph*{Monomials and flow.}
Define at $\mu=M_Z$ (in $\MSbar$)
\[
\hat{\boldsymbol{\alpha}}=(\hat\alpha_s,\hat\alpha_2,\hat\alpha)^\top,\qquad
\mathbf{m}(\hat{\boldsymbol{\alpha}})=
\begin{pmatrix}
\hat\alpha_s^2\\ \hat\alpha_2^2\\ \hat\alpha^2\\ \hat\alpha_s\hat\alpha_2\\ \hat\alpha_s\hat\alpha\\ \hat\alpha_2\hat\alpha
\end{pmatrix}.
\]
Component-wise ($k\in\{s,2,\mathrm{em}\}$):
\[
\frac{d}{d\ln Q}\,\LN\hat\alpha_k
= \big[\mathbf{W}^{(1)}\hat{\boldsymbol{\alpha}}\big]_k
+ \big[\mathbf{W}^{(2)}\mathbf{m}(\hat{\boldsymbol{\alpha}})\big]_k
+ \big[\mathbf{Y}^{(2)}\hat{\boldsymbol{y}}\big]_k
+ \mathcal{O}(\hat\alpha_i^3).
\]
The SNF property $\chis^\top\mathbf{W}^{(1)}=0$ yields
\[
\beta_\Xi^{(1)}=\chis^{\!\top}\mathbf{W}^{(1)}\hat{\boldsymbol{\alpha}}=0,\qquad 
\Xi=\chis\!\cdot\!\hat{\boldsymbol{\Psi}}.
\]
Hence the projected two–loop drift is
\[
\boxed{\;\beta_\Xi^{(2)}
= \chis^{\!\top}\mathbf{W}^{(2)}\,\mathbf{m}(\hat{\boldsymbol{\alpha}})
+ \chis^{\!\top}\mathbf{Y}^{(2)}\,\hat{\boldsymbol{y}}\,\neq 0\;}
\]
and is numerically suppressed because $\hat\alpha_i(M_Z)\ll 1$. Using representative pins
$\hat\alpha_s\simeq 0.118,\ \hat\alpha_2\simeq 0.0338,\ \hat\alpha\simeq 0.00782$,
and $\mathcal{O}(1\text{–}10)$ two–loop coefficients, one finds
$\bigl|\beta_\Xi^{(2)}\bigr|\lesssim 10^{-3}$ per e–fold in $Q$.

\paragraph*{Normalization note.}
This form is agnostic to whether your RGEs are in $(g_i)$ or $(\alpha_i=g_i^2/4\pi)$. From $\beta_{g_i}$,
\(
\frac{d \ln\alpha_i}{d\ln Q} = 2\,\beta_{g_i}/g_i
\);
keep all $16\pi^2$ factors consistent when assembling $\mathbf{W}^{(2)}$ and $\mathbf{Y}^{(2)}$.

\paragraph*{Numerical two–loop drift evaluation.}
Using the canonical SM two–loop coefficients
\begin{align*}
B&=\begin{pmatrix}
\frac{199}{50} & \frac{27}{10} & \frac{44}{5}\\[2pt]
\frac{9}{10} & \frac{35}{6} & 12\\[2pt]
\frac{11}{10} & \frac{9}{2} & -26
\end{pmatrix},\qquad
d^{(u)}=\Big(\tfrac{17}{10},\tfrac{3}{2},2\Big),\quad
d^{(d)}=\Big(\tfrac{1}{2},\tfrac{3}{2},2\Big),\quad
d^{(e)}=\Big(\tfrac{3}{2},\tfrac{1}{2},0\Big),
\end{align*}
and the $\MSbar$ inputs
\[
\hat\alpha_s=0.1180,\quad
\hat\alpha_2=0.0338,\quad
\hat\alpha=0.00782,\quad
\hat s_W^2=0.2312,
\]
one obtains
\[
r_1=\frac{5/3}{1-\hat s_W^2}=2.168,\qquad
r_2=\frac{1}{\hat s_W^2}=4.324,\qquad
w_1=\frac{r_2}{r_1+r_2}=0.6663,\qquad
w_2=\frac{r_1}{r_1+r_2}=0.3337.
\]

The gauge–sector two–loop block in the $(\alpha_s,\alpha_2,\alpha)$ basis is
\[
\mathbf W^{(2)}=
\frac{1}{8\pi^2}
\begin{pmatrix}
-26 & 0 & 0 & 4.5 & 5.19 & 0\\
0 & 5.83 & 0 & 12 & 0 & 1.95\\
0 & 0.65 & 10.5 & 1.55 & 4.00 & 3.17
\end{pmatrix},
\]
acting on $\mathbf m=(\hat\alpha_s^2,\hat\alpha_2^2,\hat\alpha^2,\hat\alpha_s\hat\alpha_2,\hat\alpha_s\hat\alpha,\hat\alpha_2\hat\alpha)^\top$.

Projecting with $\chis=(16,13,2)$ yields
\[
\beta_\Xi^{(2)}=\chis^{\!\top}\mathbf W^{(2)}\mathbf m\approx -3.5\times10^{-4},
\]
so the projected drift per $d\ln Q$ is
\[
|\beta_\Xi^{(2)}|\lesssim 4\times10^{-4},
\]
consistent with the preregistered tolerance and validating
\emph{``Ward–flat at one loop; drift $\leq 10^{-3}$''}.

\paragraph*{Analytical context (link to $\beta_G$).}
The emergent coupling runs by projection of the SM gauge flows:
\[
\beta_G \equiv \frac{d\,\LN G}{d\,\LN Q}
= \chis^{\!\top}\frac{d\,\Psivec}{d\,\LN Q}
= \chis^{\!\top}\Big(\mathbf W^{(1)}\hat\alphavec
+ \mathbf W^{(2)}\,\mathbf m(\hat\alphavec)
+ \mathbf Y^{(2)}_{\rm gauge\!-\!Yuk}\,\hat{\mathbf y}\Big),
\]
with $\hat\alphavec=(\hat\alpha_s,\hat\alpha_2,\hat\alpha)^\top$.
Ward–flatness gives $\beta_\Xi^{(1)}=0\Rightarrow \beta_G=\mathcal{O}(\hat\alpha_i^2)$, so the first nonzero drift arises at two loops via $\mathbf W^{(2)}$ and $\mathbf Y^{(2)}_{\rm gauge\!-\!Yuk}$.

\section*{S6. Post-derivation metrology: closure and leave-one-out (LOO)}

\paragraph*{Scope (after the derivation of $G$).}
Up to this point, $G$ has been \emph{derived} strictly within the SM from the gauge pins at $\mu=M_Z$ (\,$\MSbar$):
\[
G(M_Z)\;\equiv\;\frac{\hbar c}{m_p^2}\,\hat\Omega,
\qquad
\hat\Omega\;=\;\hat\alpha_s^{16}\,\hat\alpha_2^{13}\,\hat\alpha^{2},
\qquad
\hat\Xi_{\rm eq}\;=\;\LN\hat\Omega.
\]
No gravitational metrology ($G_N$) entered this derivation. The role of this section is purely
\emph{validation}: compare the SM–internal invariant $\hat\Omega$ to the experimentally
determined target
\[
\alpha_G^{(\mathrm{pp})}\;:=\;\frac{G_N m_p^2}{\hbar c},
\]
and use the same target to form LOO forecasts. Metrology is a target only; it is never used upstream to define $G$.

\paragraph*{Closure ratio and calibration.}
Define the closure ratio
\[
Z_G\;:=\;\frac{\alpha_G^{(\mathrm{pp})}}{\hat\Omega},
\]
so that the calibrated Newtonian coupling satisfies
\[
G_N\;=\;Z_G\,G(M_Z)\,,
\qquad\text{with}\quad Z_G=1\ \text{iff closure holds exactly.}
\]
Uncertainty propagation is performed in log–space with the standard Jacobian of $(\hat\alpha_s,\hat\alpha_2,\hat\alpha)$; see S0.8 for pins and covariance.

\paragraph*{Leave–one–out (LOO) forecasts.}
Treat $\alpha_G^{(\mathrm{pp})}$ as an external target and \emph{predict} one gauge pin at a time from the other two:
\[
\hat\alpha_s^{\star}
\;=\;
\exp\!\left[
\frac{1}{16}\Big(\LN\alpha_G^{(\mathrm{pp})}-13\LN\hat\alpha_2-2\LN\hat\alpha\Big)
\right],
\]
with cyclic permutations for $(\hat\alpha_2^\star,\hat\alpha^\star)$.
Agreement within pinned uncertainties is the LOO criterion; disagreement falsifies the construction.

\paragraph*{Two–state contrast (lab–null handoff).}
For any two laboratory states ($A,B$) near equilibrium,
\[
\frac{\Delta G}{G}\Big|_{A\to B}
=\Pi(\hat\Xi_{\rm eq}+\Delta\hat\Xi_B)-\Pi(\hat\Xi_{\rm eq}+\Delta\hat\Xi_A)
\simeq
\frac{\Delta\Xi_B^{\,2}-\Delta\Xi_A^{\,2}}{\sigma_\chi^2}
\;=\;
\frac{\phi_{\chi,B}^{2}-\phi_{\chi,A}^{2}}{\Lambda_\chi^{\,2}},
\]
using the canonical gate form of S3.3. This is the experimental template referenced in the Letter; Ward–flatness windows and masks are in S5.


\subsection*{S6.1 Closure: $\hat\Omega$ vs.\ $\alpha_G^{(\mathrm{pp})}$ (target-only)}

\paragraph*{Definitions (recall and target).}
\[
\hat\Omega \;=\; \hat\alpha_s^{16}\,\hat\alpha_2^{13}\,\hat\alpha^{2}
\;=\; \exp(\hat\Xi_{\rm eq}),
\qquad
\hat\Xi_{\rm eq} \;=\; 16\,\LN\hat\alpha_s + 13\,\LN\hat\alpha_2 + 2\,\LN\hat\alpha.
\]
The metrology \emph{target} (not used as an input) is
\[
\alpha_G^{(\mathrm{pp})} \;=\; \frac{\GN\,\Mp^{2}}{\hbarc},
\qquad
\Xi_{\rm emp} \;=\; \LN\alpha_G^{(\mathrm{pp})} \;=\; \LN\GN + 2\,\LN\Mp - \LN(\hbarc).
\]
Treat $\hbarc$ as exact; thus $\sigma^2(\Xi_{\rm emp})=\sigma^2(\LN\GN)+4\,\sigma^2(\LN\Mp)$.

\paragraph*{Closure statistic and uncertainty (log domain).}
\[
\mathcal{R}\equiv\frac{\hat\Omega}{\alpha_G^{(\mathrm{pp})}},\qquad
\Delta_{\%}\equiv(\mathcal{R}-1)\times 100\%.
\]
Work in logs:
\[
\LN\mathcal{R}=\hat\Xi_{\rm eq}-\Xi_{\rm emp},\qquad
\sigma^2(\LN\mathcal{R})=\sigma^2(\hat\Xi_{\rm eq})+\sigma^2(\Xi_{\rm emp}),
\]
treating SM pins independent of metrology, so $\mathrm{Cov}(\hat\Xi_{\rm eq},\Xi_{\rm emp})=0$. Linear return:
\[
\sigma(\mathcal{R})\simeq \mathcal{R}\,\sigma(\LN\mathcal{R}),\qquad
\sigma(\Delta_{\%})\simeq 100\,\sigma(\mathcal{R}).
\]

\paragraph*{Independent SM log-basis (no double counting).}
Use the independent basis
\[
x=\big(\LN\hat\alpha,\ \LN\hat s_W^2,\ \LN\hat\alpha_s\big),\qquad
\LN\hat\alpha_2=\LN\hat\alpha-\LN\hat s_W^2,
\]
so that
\[
\hat\Xi_{\rm eq}=15\,\LN\hat\alpha-13\,\LN\hat s_W^2+16\,\LN\hat\alpha_s,\quad
g_\Xi=(15,-13,16)^{\!\top}.
\]
Hence
\[
\sigma^2(\hat\Xi_{\rm eq})=g_\Xi^{\!\top}\,\mathrm{Cov}(x)\,g_\Xi,
\qquad
\sigma(\hat\Omega)\simeq \hat\Omega\,\sigma(\hat\Xi_{\rm eq}).
\]

\paragraph*{Summary box (target-only).}
\[
\boxed{~
\mathcal{R}=\frac{\hat\Omega}{\alpha_G^{(\mathrm{pp})}},\quad
\LN\mathcal{R}=\hat\Xi_{\rm eq}-\Xi_{\rm emp},\quad
\sigma^2(\LN\mathcal{R})=\underbrace{g_\Xi^{\!\top}\mathrm{Cov}(x)g_\Xi}_{\text{SM pins}}
+\underbrace{\sigma^2(\LN\GN)+4\,\sigma^2(\LN\Mp)}_{\text{metrology}}
~}
\]

\paragraph*{S6.1.1 Optional covariance-aware form (addresses reviewer).}
If one wishes to allow for cross-covariances between SM pins and metrology targets in a joint fit, the general expression is
\[
\sigma^2(\LN\mathcal{R})=
g_\Xi^{\!\top}\mathrm{Cov}(x)g_\Xi
+\sigma^2(\LN\GN)+4\,\sigma^2(\LN\Mp)
-2\,\mathrm{Cov}(\hat\Xi_{\rm eq},\LN\GN)-4\,\mathrm{Cov}(\hat\Xi_{\rm eq},\LN\Mp).
\]
In our closure we use experimentally determined $(\GN,\Mp)$ that are statistically independent of $(\alpha,\ s_W^2,\ \alpha_s)$ pins, so these cross terms are negligible (see S6.9 for a bound).

\subsection*{S6.2 Covariance handling and log–linear Jacobians}

For any vector map $y=f(x)$ with $x$ Gaussian,
\[
\mathrm{Cov}(y)=J\,\mathrm{Cov}(x)\,J^{\!\top},\qquad
J_{ij}=\partial_{x_j}y_i.
\]
In the log domain, products and ratios become linear combinations:
\[
\delta(\LN y)=\sum_i a_i\,\delta(\LN x_i),\qquad
\mathrm{Cov}(\LN x_i,\LN x_j)\simeq \frac{\mathrm{Cov}(x_i,x_j)}{x_i x_j}.
\]
We use $\mathrm{Cov}(x)$ from PDG/CODATA, including reported correlations between $\alpha(M_Z)$ and $s_W^2$ where available.

\paragraph*{Weak pin and scheme map (once).}
\[
\alpha^{\mathrm{OS}}_2(M_Z)
=\frac{\sqrt{2}\,G_F m_W^2}{\pi}\,\frac{1}{1+\Delta r},
\qquad
\alpha^{\MSbar}_2(M_Z)
=\alpha^{\mathrm{OS}}_2(M_Z)\big[1+\delta^{(1)}_{\mathrm{OS\to MS}}\big],
\]
where $\Delta r$ is the one–loop electroweak correction (full $m_t,m_H$ dependence) and
$\delta^{(1)}_{\mathrm{OS\to MS}}$ represents the finite scheme–conversion shift.  
Both are carried as fixed contributions in the uncertainty budget rather than dynamic parameters.

\subsection*{S6.3 Metrology cross–check for the depth closure}

\paragraph*{Projected depth (our sign convention).}
Work in $\overline{\mathrm{MS}}$ at $\mu=M_Z$ (hats suppressed in this subsection) and define
\[
\xi_i:=\LN\!\frac{1}{\alpha_i},\qquad
\Xi_{\mathrm{proj}}
=\chis\!\cdot\!\boldsymbol{\xi}
=16\,\xi_{\alpha_s}
+13\,\xi_{\alpha_2}
+2\,\xi_{\alpha}.
\]
For weak mixing we use $\alpha_2=\alpha/\sin^2\!\theta_W$.

\paragraph*{Empirical depth (target only).}
\[
\Xi_{\rm emp}
=\LN\!\Big(\frac{1}{\alpha_G^{(\mathrm{pp})}}\Big),\qquad
\alpha_G^{(\mathrm{pp})}
=\frac{G_N m_p^2}{\hbar c}.
\]
Metrology enters here only as a \emph{target}; it is not used upstream to define $G$.

\paragraph*{Uncertainty propagation (log domain).}
Assuming the three gauge pins are uncorrelated at the level reported in Table~\ref{tab:pins_inputs},
\[
\sigma^2(\Xi_{\mathrm{proj}})
=(16\,\sigma_{\xi_{\alpha_s}})^2
+(13\,\sigma_{\xi_{\alpha_2}})^2
+(2\,\sigma_{\xi_{\alpha}})^2,
\qquad
\sigma_{\xi_{\alpha_i}}=\frac{\sigma_{\alpha_i}}{\alpha_i}.
\]
Equivalently, in the independent log basis $x=(\LN\alpha,\ \LN s_W^2,\ \LN\alpha_s)$ one has
\[
\Xi_{\mathrm{proj}}
=15\,\LN\alpha-13\,\LN s_W^2+16\,\LN\alpha_s,
\quad
\sigma^2(\Xi_{\mathrm{proj}})
=g_\Xi^{\!\top}\,\mathrm{Cov}(x)\,g_\Xi,
\quad
g_\Xi=(15,-13,16)^{\!\top}.
\]

For the empirical depth,
\[
\Xi_{\rm emp}=\LN G_N+2\,\LN m_p-\LN(\hbar c),
\]
so with $\hbar c$ exact in SI units,
\[
\sigma^2(\Xi_{\rm emp})=\sigma^2(\LN G_N)+4\,\sigma^2(\LN m_p),
\]
and at current precision the $m_p$ term is negligible compared to $G_N$; numerically
\[
\frac{\sigma_{G_N}}{G_N}=2.247\times10^{-5}\ \text{(22.47 ppm)}.
\]

\paragraph*{Agreement statement.}
Using the pins in Table~\ref{tab:pins_inputs} (with $\alpha_2=\alpha/\sin^2\!\theta_W$) and the metrology targets in Table~\ref{tab:pins_targets},
\[
\Xi_{\mathrm{proj}}\ \text{and}\ \Xi_{\rm emp}
\ \text{agree within the propagated }1\sigma.
\]

\paragraph*{Summary box.}
\[
\boxed{
\begin{gathered}
\Delta_\Xi:=\Xi_{\mathrm{proj}}-\Xi_{\rm emp},\qquad
\sigma^2(\Delta_\Xi)=\sigma^2(\Xi_{\mathrm{proj}})+\sigma^2(\Xi_{\rm emp})\\[4pt]
\sigma^2(\Xi_{\mathrm{proj}})=g_\Xi^{\!\top}\mathrm{Cov}(x)g_\Xi,\quad
\sigma^2(\Xi_{\rm emp})=\sigma^2(\LN G_N)+4\,\sigma^2(\LN m_p)\,.
\end{gathered}
}
\]

\subsection*{S6.4 Sign and basis conventions}

\paragraph*{Logs and sign.}
Work in $\overline{\mathrm{MS}}$ at $\mu=M_Z$ (hats suppressed in this subsection).
Depth logs use
\[
\xi_i:=\LN\!\frac{1}{\alpha_i},\qquad
\Xi:=\chis\!\cdot\!\boldsymbol{\xi}
=16\,\xi_{\alpha_s}+13\,\xi_{\alpha_2}+2\,\xi_{\alpha},\qquad
\chis=(16,13,2).
\]

\paragraph*{Weak-sector basis choices.}
Either reconstruct $\alpha_2$ via GUT normalization
\[
\frac{1}{\alpha}=\frac{5}{3}\,\frac{1}{\alpha_1}+\frac{1}{\alpha_2}
\quad\Longleftrightarrow\quad
\alpha_2=\Big(\frac{1}{\alpha}-\frac{5}{3}\,\frac{1}{\alpha_1}\Big)^{-1},
\]
or via the weak-mixing angle
\[
\alpha_2=\frac{\alpha}{s_W^2},\qquad s_W^2\equiv\sin^2\theta_W.
\]
The algebra for $\Xi$ is unchanged; only the choice of independent pins differs.

\paragraph*{Independent log bases (for covariance).}
Two convenient independent bases are
\[
x_{\rm GUT}=(\LN\alpha,\ \LN\alpha_1,\ \LN\alpha_s),\qquad
x_{\rm weak}=(\LN\alpha,\ \LN s_W^2,\ \LN\alpha_s).
\]
In the weak basis,
\[
\Xi=15\,\LN\alpha-13\,\LN s_W^2+16\,\LN\alpha_s,
\]
and covariance propagates as
\(
\sigma^2(\Xi)=g_\Xi^{\!\top}\mathrm{Cov}(x)\,g_\Xi
\)
with \(g_\Xi=(15,-13,16)^{\!\top}\).
(Use the corresponding Jacobian if $x_{\rm GUT}$ is chosen.)

\subsection*{S6.5 LOO forecasts for \texorpdfstring{$\alpha_s,\ \alpha_2,\ \alpha$}{alpha\_s, alpha\_2, alpha}}
\textit{Convention.} Hatted MS-bar pins at $\mu=M_Z$ are implied; hats are suppressed in this subsection. Treat the empirical depth $\Xi_{\rm emp}$ and any two SM couplings as inputs; solve the third from $\Xi_{\rm emp}=\Xi_{\rm eq}$.

\paragraph*{LOO for $\alpha_s$.}
\[
\LN\alpha_s
=\frac{1}{16}\Big(\Xi_{\rm emp}-13\,\LN\alpha_2-2\,\LN\alpha\Big),
\qquad
g_s=\frac{1}{16}\,(1,\,-13,\,-2)^{\!\top}.
\]
With inputs $y=(\Xi_{\rm emp},\LN\alpha_2,\LN\alpha)$,
\[
\sigma^2\!\big(\LN\alpha_s\big)
=g_s^{\!\top}\,\mathrm{Cov}(y)\,g_s,
\qquad
\sigma\!\big(\alpha_s\big)\simeq \alpha_s\,\sigma\!\big(\LN\alpha_s\big).
\]

\paragraph*{LOO for $\alpha_2$.}
\[
\LN\alpha_2
=\frac{1}{13}\Big(\Xi_{\rm emp}-16\,\LN\alpha_s-2\,\LN\alpha\Big),
\qquad
g_2=\frac{1}{13}\,(1,\,-16,\,-2)^{\!\top},
\]
with $y=(\Xi_{\rm emp},\LN\alpha_s,\LN\alpha)$ and the same propagation rule.

\paragraph*{LOO for $\alpha$.}
\[
\LN\alpha
=\frac{1}{2}\Big(\Xi_{\rm emp}-16\,\LN\alpha_s-13\,\LN\alpha_2\Big),
\qquad
g_\alpha=\frac{1}{2}\,(1,\,-16,\,-13)^{\!\top},
\]
with $y=(\Xi_{\rm emp},\LN\alpha_s,\LN\alpha_2)$ and the same propagation rule.

\paragraph*{Notes on correlations and bases.}
If $\alpha_2$ is reconstructed from $(\alpha,s_W^2)$ via $\alpha_2=\alpha/s_W^2$, perform LOO in an independent basis to avoid double counting: replace $\LN\alpha_2$ by $\LN\alpha-\LN s_W^2$ and build $\mathrm{Cov}(y)$ accordingly. The linear (log-space) Jacobian vectors $g_s,g_2,g_\alpha$ above are the gradients used for covariance propagation.

\subsection*{S6.6 Pulls, percent differences, and consistency}

\paragraph*{Per–coupling diagnostics.}
For any coupling $\alpha_i$ with PDG reference $\alpha_i^{\rm PDG}\!\pm\!\sigma_{{\rm PDG},i}$, define the forecast–vs–PDG pull and percent difference
\[
\Delta_{\rm pull,i}
=\frac{\hat\alpha_i-\alpha_i^{\rm PDG}}{\sigma_{{\rm PDG},i}},
\qquad
\Delta_{\%,i}
=\frac{\hat\alpha_i-\alpha_i^{\rm PDG}}{\alpha_i^{\rm PDG}}\times 100\% .
\]

\paragraph*{Global LOO consistency metric.}
Aggregate the three LOO forecasts with an inverse–variance sum (PDG variance plus forecast variance from S6.5):
\[
\chi^2_{\rm LOO}
=\sum_{i\in\{s,2,e\}}
\frac{\big(\hat\alpha_i-\alpha_i^{\rm PDG}\big)^2}
{\sigma_{{\rm PDG},i}^2+\sigma^2\!\big(\hat\alpha_i\big)}\,,
\]
where $\sigma^2(\hat\alpha_i)$ is obtained via the log–linear Jacobians in S6.5. Numerical outputs (per–coupling pulls, $\Delta_{\%,i}$, and $\chi^2_{\rm LOO}$) are autogenerated in S9 by \texttt{loo.py} using the pinned covariance matrices.

\paragraph*{Equivalence test (TOST) for $\alpha_s$ at $M_Z$.}
Assess $\alpha_s=\alpha_s^{\rm PDG}$ within a practical margin $\varepsilon$ using two one–sided tests (TOST) at $\alpha=0.05$. With
\[
\Delta=\hat\alpha_s-\alpha_s^{\rm PDG},\qquad
{\rm CI}_{90\%}:\ \Delta\pm 1.645\,\sigma(\hat\alpha_s),
\]
declare equivalence if ${\rm CI}_{90\%}\subset[-\varepsilon,+\varepsilon]$. With current pins, a representative margin is $\varepsilon_{\rm ppm}\approx 160$ (expressed in parts per million of $\alpha_s$).

\subsection*{S6.7 Scheme robustness}

\paragraph*{Common–scale expressions.}
All gauge couplings are expressed at a common scale $Q=M_Z$.
Moving between pure $\overline{\mathrm{MS}}$, mixed on–shell/$\overline{\mathrm{MS}}$ anchors, or the GUT basis with
\(
1/\alpha=\tfrac{5}{3}\,1/\alpha_1+1/\alpha_2
\)
constitutes a finite renormalization and reconstruction of $\alpha$.

\paragraph*{Invariance of the primitive projector.}
The integer projector
\(
\chis=(16,13,2)
\)
and the closure relation $\Xi=\chis\!\cdot\!\hat\Psi$ are invariant under such finite scheme changes.
The transformation of $\alpha_1,\alpha_2$ to $(\alpha,s_W^2)$ merely reshuffles coordinates within the same gauge–log subspace; $\Xi$ and its parity remain unaffected.

\paragraph*{Numerical impact.}
Finite scheme shifts induce small offsets in the computed depth $\Xi_{\rm eq}$,
dominated by the $\alpha_s$ input uncertainty.
Replacing $\alpha_s$ by its leave–one–out estimate $\alpha_s^\star$ removes these offsets,
and the residuals in $\Xi$ or $\Omega$ remain $\ll 1\sigma$ under the propagated covariances.

\paragraph*{Summary.}
Scheme choice changes normalization conventions but not the integer certificate,
the parity protection of $\Pi'(\Xi_{\rm eq})=0$, or the derived $G(M_Z)$ value.
All admissible anchor schemes therefore lead to numerically equivalent closures within the registered error budget.

\subsection*{S6.8 Monte Carlo confirmation of LOO and closure}

\textbf{Setup.}
Draw
\(
x=(\hat\alpha,\ \hat s_W^2,\ \alpha_G^{(\mathrm{pp})})
\)
as independent Gaussians from Table~\ref{tab:pins_inputs} and Table~\ref{tab:pins_targets},
and reconstruct
\(
\hat\alpha_2=\hat\alpha/\hat s_W^2
\).
For each draw compute
\[
\widehat{\LN\alpha_s^\star}
=\frac{1}{16}\Big(\Xi_{\rm emp}-13\,\LN\hat\alpha_2-2\,\LN\hat\alpha\Big),
\qquad
\alpha_s^\star=\exp(\widehat{\LN\alpha_s^\star}).
\]

\textbf{Results ($10^5$ draws).}
\[
\hat\alpha_s^\star \;=\; 0.117341\ \pm\ 1.86\times 10^{-5},
\qquad
\text{relative }\sigma \;=\; 1.59\times 10^{-4},
\qquad
\text{pull vs PDG} \;=\; -0.73\sigma.
\]
The metrology–depth uncertainty is dominated by $G_N$:
\(
\delta\alpha_G^{(\mathrm{pp})}/\alpha_G^{(\mathrm{pp})}=2.25\times 10^{-5}
\)
(22.5 ppm), with $\hbar c$ exact and $m_p$ negligible at this level.
These MC values match the log–Jacobian propagation in S0.6 and S6.1–S6.4.

\paragraph*{LOO forecast (uncertainty).}
From the propagation (S6.5) and MC (S6.8),
\[
\boxed{
\hat{\alpha}_s(M_Z)=0.117341\ \pm\ 1.86\times 10^{-5}
\quad\Rightarrow\quad
\text{pull}=-0.73\sigma\ \text{vs PDG}
}
\]
with the forecast uncertainty dominated by $G_N$ via $\Xi_{\rm emp}$.

\subsection*{S6.9 Correlation audit and bias bound (metrology vs SM pins)}

\paragraph*{Question.}
Could theoretical dependence of $m_p$ on QCD (via $\Lambda_{\rm QCD}$ and ultimately $\alpha_s$) bias closure/LOO through hidden covariance?

\paragraph*{Statistical answer (this work).}
Our closure uses \emph{experimental} targets $(G_N,m_p)$ whose uncertainties are dominated by $G_N$ (22.5 ppm), while $m_p$ is measured with $\ll$ppm error. The PDG determinations of $(\alpha,\ s_W^2,\ \alpha_s)$ are statistically independent of the metrology of $(G_N,m_p)$; therefore
\[
\mathrm{Cov}(\hat\Xi_{\rm eq},\LN G_N)\simeq 0,\qquad
\mathrm{Cov}(\hat\Xi_{\rm eq},\LN m_p)\simeq 0,
\]
and the independence assumption in S6.1 is appropriate.

\paragraph*{Conservative upper bound.}
Even if one inserted a hypothetical correlation coefficient $\rho$ between $\hat\Xi_{\rm eq}$ and $\LN m_p$, the induced variance shift is
\[
\Delta\sigma^2(\LN\mathcal R)
= -\,4\,\rho\,\sigma(\hat\Xi_{\rm eq})\,\sigma(\LN m_p)\,.
\]
With current pins, $\sigma(\LN m_p)\ll\sigma(\LN G_N)$ and $\sigma(\hat\Xi_{\rm eq})=\mathcal O(10^{-4})$ in log space, so for any $|\rho|\le 1$ the correction is negligible compared to $\sigma^2(\LN G_N)$ that sets the error budget. Numerically, taking the extremal $\rho=\pm1$ changes $\sigma(\LN\mathcal R)$ by a fraction $\ll 10^{-3}$ of the $G_N$ term (see replication pack).

\paragraph*{Theory note (separation of roles).}
Theoretical sensitivity of $m_p$ to $\Lambda_{\rm QCD}$ (and thus to $\alpha_s$) governs how a \emph{QCD-only} fit would co-estimate $(m_p,\alpha_s)$. Our closure deliberately \emph{does not} use such a joint theory prior: $m_p$ enters only as a metrology constant. Hence the relevant covariance is the \emph{statistical} one between independent experimental determinations, which is negligible at present precision.



\section*{S7. Systematics and scheme transport}

\textit{Provenance note.}  
This section audits higher–order and systematic effects \emph{after} the SNF certificate; it does not modify the integer result for $\boldsymbol{\chi}$, which is fixed at one loop by representation data alone (Sec.~S1).  
Ward–flatness diagnostics appear in Sec.~S5, and closure/LOO validation in Sec.~S6.

\subsection*{S7.1 Two–loop, threshold, and systematic budget (bounded; not in SNF)}

The integer projector
\(
\boldsymbol{\chi}=(16,13,2)
\)
is certified by the Smith–Normal–Form (SNF) of the \emph{one–loop} difference stack (Sec.~S1).  
Its definition depends only on representation integers and light/heavy content per window; no numerical masses or renormalization scales enter $\Delta W$.

Higher–order effects do not generate a new integer lattice and therefore cannot alter the certificate.  
Their role is confined to bounded drifts that are \emph{monitored elsewhere}:

\begin{itemize}\setlength\itemsep{2pt}
  \item \textbf{Gauge two–loop and Yukawa/Higgs mixing.}  
  These shift $F(Q)\!\equiv\!\beta_\Xi(Q)$ away from its one–loop zero.  
  They are monitored via the Ward projector (Sec.~S5) using preregistered bounds on  
  \(F_\sigma\!=\!F/\sigma_\chi\) in the electroweak and low–GeV windows (see Table~\ref{tab:S0ward}).

  \item \textbf{Propagation into $\Xi_{\rm eq}$ and $\Omega$.}  
  Treated in Sec.~S6 through log–space Jacobians with Monte Carlo confirmation in the SM.  
  Input covariances are PDG/CODATA (S0).

  \item \textbf{Curvature (gate–width) renormalization.}  
  Even counterterms renormalize the width $\sigma_\chi$ at $\order(\alpha_i/4\pi)$ while preserving the $\mathbb{Z}_2$ parity (S2) and the massless, luminal tensor sector (S3–S4):
  \[
  \frac{\delta\sigma_\chi}{\sigma_\chi}
  =\sum_{i\in\{3,2,\mathrm{EM}\}}
   c_i\,\frac{\alpha_i}{4\pi}
  +\mathcal O(\alpha_i^2),
  \qquad c_i=\mathcal O(1).
  \]
  This shifts $\Lambda_\chi=\sigma_\chi/\|\boldsymbol{\chi}\|_{K}$ by the same fractional amount and cannot induce a Pauli–Fierz mass or linear $h$–$\delta\Xi$ mixing (forbidden by parity).
\end{itemize}

All three effects enter closure/LOO only through \emph{second–order} contributions in already small envelopes; none modify $\boldsymbol{\chi}$ or the SNF certificate.


\subsection*{S7.2 Scheme and window transports (unimodular stability)}

The difference stack $\Delta W$ depends only on light/heavy membership, not on exact threshold values
or the decoupling prescription. Working in GUT–normalized hypercharge with the EM pivot
\(
1/\alpha=\tfrac{5}{3}\,1/\alpha_1+1/\alpha_2
\),
moving a threshold within a window, reordering windows, or changing integer row/column bases corresponds to a unimodular transport
\[
\Delta W \;\mapsto\; U_{\rm row}\,\Delta W\,V_{\rm col},\qquad
U_{\rm row}\in GL(m,\mathbb Z),\ \ V_{\rm col}\in GL(3,\mathbb Z),
\]
which preserves the integer left nullspace up to sign. Thus the primitive kernel is invariant:
\[
\boxed{~
\ker_{\mathbb Z}\!\big((U_{\rm row}\Delta W V_{\rm col})^{\!\top}\big)
= \ker_{\mathbb Z}(\Delta W^{\!\top})
= \mathrm{span}_{\mathbb Z}\{\pm\,\chis\}\,.
}
\]

\emph{Remark.} Raw species stacks (including gauge adjoints) are typically rank~3; the
\emph{difference} construction cancels adjoint self–contributions and exposes the rank~2 lattice
needed for SNF certification. Row rescalings by a gcd are \emph{not} unimodular and are used only as informal referee checks; the certificate itself uses unimodular operations exclusively.


\subsection*{S7.3 Sensitivity tests and robustness summary}

The following admissible variations were tested conceptually (documented for transparency and reproducibility):
\begin{itemize}\setlength\itemsep{2pt}
  \item random permutations of window order;
  \item removal or subdivision of intermediate thresholds while preserving light/heavy labels;
  \item admissible spectator absorption and integer row/column basis changes (unimodular);
  \item optional per–row gcd clearing for human inspection (non–unimodular; sanity checks only).
\end{itemize}
All return a primitive kernel proportional to $(16,13,2)$. The default repo build is deterministic and does not execute these stress tests; they serve as methodological checks aligned with Secs.~S1–S2.

Together with Ward–flatness bounds (Sec.~S5) and closure/LOO consistency (Sec.~S6), these establish
\[
\textbf{(i)}\ \chis\ \text{is scheme– and window–stable (integer–certified),}
\qquad
\textbf{(ii)}\ \text{higher–order drifts are bounded systematics and do not enter the certificate.}
\]

\noindent\boxed{\textbf{Conclusion:}\ \textit{No admissible renormalization or decoupling prescription permits any adjustment of }\chis.}

\section*{S8. Interpretive scales: helicity frequency, period, and curvature envelope}

The curvature–gate background $\Pi(\Xi)$ establishes a stationary normalization; transient helicity–$\pm2$ perturbations propagate as in GR—massless and luminal (Sec.~S3).  
This section is interpretive only and does not enter the falsifier set; parity, the SNF certificate, and Ward–flatness bands remain the operational tests (Secs.~S1–S6).

\paragraph*{Graviton envelope and curvature geometry.}
\label{S:wrapper}

The graviton emerges with GR normalization, while the \emph{scalar} depth mode aligned with $\boldsymbol{\chi}$ modulates the curvature gate $\Pi(\Xi)$.  
The curvature envelope governs how $G(x)=G(M_Z)\Pi(\Xi(x))$ varies quadratically around equilibrium,
\[
\frac{\Delta G}{G} \simeq -\,\frac{(\Delta\Xi)^2}{\sigma_\chi^2},
\]
defining a Gaussian curvature well of width $\sigma_\chi=247.683$ and canonical scale
\(
\Lambda_\chi=\sigma_\chi/\|\boldsymbol{\chi}\|_{K}=14.0507.
\)
The associated helicity frequency and period are
\[
\omega_{\rm hel}=\Lambda_\chi^{-1}=0.0712,\qquad
T_{\rm hel}=2\pi\Lambda_\chi\simeq88\,t_P,
\]
identifying the characteristic oscillation of the spin–2 envelope in Planck units.

\paragraph*{Interpretation.}
The curvature gate $\Pi(\Xi)$ thus defines a stationary background curvature density; perturbations travel along it as luminal spin–2 modes.  
Even curvature (parity–protected) ensures no Pauli–Fierz term, while $\Lambda_\chi$ and $T_{\rm hel}$ set the geometric frequency scale linking the scalar alignment depth and the emergent graviton envelope.


\paragraph*{Gate, canonical field, and parity.}

% Canonical depth field (consistent with S3.3 and S4.7)
\[
\phi_{\chi}
\;=\;
\frac{\boldsymbol{\chi}^{\!\top}\big(\hat{\boldsymbol{\Psi}}-\hat{\boldsymbol{\Psi}}_{\rm eq}\big)}
{\|\boldsymbol{\chi}\|_{K}}, 
\qquad
\|\boldsymbol{\chi}\|_{K}
=\sqrt{\boldsymbol{\chi}^{\!\top}K\boldsymbol{\chi}},
\qquad
\Delta\Xi=\boldsymbol{\chi}\!\cdot\!(\hat{\boldsymbol{\Psi}}-\hat{\boldsymbol{\Psi}}_{\rm eq})
=\|\boldsymbol{\chi}\|_{K}\,\phi_{\chi}.
\]

\[
\Pi(\Xi)
=\exp\!\left[-\frac{\phi_{\chi}^{2}}{\Lambda_{\chi}^{2}}\right],
\qquad
\Lambda_{\chi}
=\frac{\sigma_{\chi}}{\|\boldsymbol{\chi}\|_{K}}.
\]

Parity forbids a linear response at equilibrium:
\[
\Pi(\Xi_{\rm eq})=1,
\qquad
\left.\partial_{\Xi}\Pi\right|_{\Xi_{\rm eq}}=0,
\qquad
\frac{\Delta G}{G}
=\Pi(\Xi_{\rm eq}+\Delta\Xi)-1
\simeq \frac{\phi_{\chi}^{2}}{\Lambda_{\chi}^{2}}
\quad(\Delta\Xi\ \text{small}).
\]

% Note: An alternative metric-weighted coordinate 
% $\tilde\phi_{\chi}=\boldsymbol{\chi}^{\!\top}K(\hat{\boldsymbol{\Psi}}-\hat{\boldsymbol{\Psi}}_{\rm eq})/\|\boldsymbol{\chi}\|_{K}$
% differs by an $O(1)$ normalization; we use $\phi_{\chi}$ so that $\Delta\Xi=\|\chi\|_{K}\phi_{\chi}$ holds exactly.

\paragraph*{Helicity coherence scale.}
\[
\omega_{\rm hel}
=\frac{\|\boldsymbol{\chi}\|_{K}}{\sigma_\chi}
=\frac{1}{\Lambda_\chi},
\qquad
T_{\rm hel}
=\frac{2\pi}{\omega_{\rm hel}}
=2\pi\,\Lambda_\chi
\simeq 88\,t_P \ \ (\text{Planck units}),
\]
so that, with $c=1$ in Planck units, the coherence length equals the period,
\[
\ell_{\rm hel}=c\,T_{\rm hel}\simeq 88\,\ell_P.
\]
These scales live in the scalar depth sector; the helicity–2 tensor remains massless and luminal (Sec.~S3).

\paragraph*{Even scalar dynamics.}
With the parity–even potential
\[
V(\Psivec)=
\tfrac12\,\Delta\Psivec^{\!\top}\,\Sigma_\perp^{-1}\,P_\perp\,\Delta\Psivec
+\tfrac{\gamma}{2}\,\big(\chis\!\cdot\!\Delta\Psivec\big)^2,
\qquad
\Delta\Psivec:=\Psivec-\Psivec_{\rm eq},
\]
the $\chis$–projected (soft) mode in the canonical convention of S3.3,
\[
\phi_{\chis}\;=\;\frac{\chis^{\!\top}\Delta\Psivec}{\|\chis\|_{K}}\,,
\qquad
\Delta\Xi=\chis\!\cdot\!\Delta\Psivec=\|\chis\|_{K}\,\phi_{\chis},
\]
obeys the free even Klein–Gordon equation
\[
\square\,\phi_{\chis}+m_{\chis}^{2}\,\phi_{\chis}=0,
\qquad
m_{\chis}^{2}=\frac{\gamma_{\chis}}{\|\chis\|_{K}^{2}},
\qquad
\gamma_{\chis}=\gamma\,(\chis^{\!\top}\chis)^{2}.
\]
Here
\[
P_\chi=\frac{\chis\,\chis^{\!\top}K}{\chis^{\!\top}K\chis},
\qquad
P_\perp=\mathbb{1}-P_\chi,
\]
is the $K$–orthogonal projector on column vectors (so $P_\perp\chis=0$). This construction regulates only the transverse subspace and preserves the parity protection: no linear $h$–$\phi_{\chis}$ mixing and no Pauli–Fierz mass.

\paragraph*{Static profile and curvature envelope.}
In a static exterior region the depth mode has Yukawa form
\[
\phi_{\chis}(r)=\frac{A\,e^{-m_{\chis} r}}{r}\,,
\]
with boundary amplitude $A$. The \emph{envelope} where $\Pi=e^{-1}$ (i.e.\ $\abs{\phi_{\chis}}=\Lchi$) satisfies
\[
\frac{\abs{A}\,e^{-m_{\chis} r_\ast}}{r_\ast}=\Lchi
\;\Longleftrightarrow\;
m_{\chis} r_\ast\,e^{m_{\chis} r_\ast}=\frac{m_{\chis}\abs{A}}{\Lchi}
\;\Longrightarrow\;
r_\ast=\frac{1}{m_{\chis}}\,W\!\Big(\frac{m_{\chis}\abs{A}}{\Lchi}\Big),
\]
where $W$ is the Lambert-$W$ function (principal branch $W_0$ for monotone profiles). 
\emph{Limits:} for $m_{\chis}\!\to 0$, $W(z)\sim z$ so $r_\ast\to \abs{A}/\Lchi$; for large argument $z\gg1$, $W(z)\sim \ln z-\ln\ln z$, giving
$r_\ast\simeq m_{\chis}^{-1}\!\big[\ln\!\big(\tfrac{m_{\chis}\abs{A}}{\Lchi}\big)-\ln\ln\!\big(\tfrac{m_{\chis}\abs{A}}{\Lchi}\big)\big]$.
The surface $\abs{\phi_{\chis}}=\Lchi$ defines a Planck–thin curvature envelope.

\paragraph*{Hourglass deformation.}
With a small quadrupolar anisotropy,
\[
\phi_{\chis}(r,\theta)\simeq
\frac{A\,e^{-m_{\chis} r}}{r}\Big[1+\epsilon\,P_2(\cos\theta)+\cdots\Big],
\qquad
|\epsilon|\ll 1,\quad P_2(x)=\tfrac12(3x^2-1),
\]
the level set $\abs{\phi_{\chis}}=\Lchi$ deforms away from the isotropic radius $r_0$ defined by
$\frac{|A|e^{-m_{\chis} r_0}}{r_0}=\Lchi$.
Solving to first order in $\epsilon$ gives
\[
r_\ast(\theta)\;=\;r_0+\delta r(\theta),\qquad
\delta r(\theta)\;=\;\frac{\epsilon\,P_2(\cos\theta)}{\,1/r_0+m_{\chis}\,}\;=\;
r_0\,\frac{\epsilon\,P_2(\cos\theta)}{\,1+m_{\chis}r_0\,}\;,
\]
so the deformation is parity–even and quadrupolar. 
For $\epsilon<0$ one finds $r_\ast(0)<r_\ast(\tfrac{\pi}{2})$, i.e.\ contraction at the poles and a bulge at the equator, yielding the hourglass (two–lobe) envelope about the symmetry plane.

\paragraph*{Fixed vs.\ sourced (no new knobs).}
\begin{itemize}[nosep,leftmargin=1.2em]
\item \textbf{Fixed by GAGE:} even gate parity ($\Pi'(\Xi_{\rm eq})=0$); GR-normalized tensor sector with $m_{\rm PF}=0$;
$\Lambda_\chi=\sigma_\chi/\|\chi\|_{K}\simeq 14.0507$; $T_{\rm hel}\simeq 88\,t_P$.
\item \textbf{Set by environment (not tunable):} boundary amplitude $A$, anisotropy $\epsilon$, and the scalar soft-mode mass $m_\chi$ via $\gamma_\chi$—all externally fixed and bounded by the width-provenance limits (S4.4; certificate in S4.7).
\end{itemize}
\noindent\emph{No new free parameters:} all quantities are derived from SM pins or fixed by boundary conditions; none is adjusted to fit metrology.

\paragraph*{Projectors in field space (canonical).}
\[
P_\chi=\frac{\boldsymbol{\chi}\,\boldsymbol{\chi}^{\!\top}K}{\boldsymbol{\chi}^{\!\top}K\boldsymbol{\chi}},
\qquad
P_\perp=\mathbb{1}-P_\chi,
\qquad
\phi_\chi=\frac{\boldsymbol{\chi}^{\!\top}\big(\hat{\boldsymbol{\Psi}}-\hat{\boldsymbol{\Psi}}_{\rm eq}\big)}
{\|\boldsymbol{\chi}\|_{K}}\,,
\]
so that
\(
\Delta\hat\Xi=\boldsymbol{\chi}^{\!\top}\!\big(\hat{\boldsymbol{\Psi}}-\hat{\boldsymbol{\Psi}}_{\rm eq}\big)
=\|\boldsymbol{\chi}\|_{K}\,\phi_\chi.
\)

\paragraph*{Compact map.}
\[
\boxed{\,\Lambda_\chi=\frac{\sigma_\chi}{\|\boldsymbol{\chi}\|_{K}},
\quad
\omega_{\rm hel}=\frac{\|\boldsymbol{\chi}\|_{K}}{\sigma_\chi},
\quad
T_{\rm hel}=2\pi\,\Lambda_\chi,
\qquad
\frac{\Delta G}{G}\simeq\frac{(\Delta\hat\Xi)^2}{\sigma_\chi^{2}}
=\frac{\phi_\chi^{2}}{\Lambda_\chi^{2}}\,}
\]


\section*{S9. Deterministic rebuild (single path)}
\noindent\textbf{Goal:} Regenerate all numeric tables and figure data from pinned inputs with deterministic hashes.

\begin{enumerate}
\item \textbf{Setup (once).}
Python~$\geq$~3.10 installed. No external dependencies required for the default build.
\emph{Optional:} \texttt{sympy} only for \texttt{src/snf\_check.py}.

\item \textbf{Pins.}
Verify \texttt{pins.json} and \texttt{keq.json} match S0 tables
(\ref{tab:pins_inputs}, \ref{tab:pins_targets}, \ref{tab:cert_params}, \ref{tab:keq}, \ref{tab:keq_eigs}).

\item \textbf{One command.}
\texttt{bash build.sh} \quad
(\textit{Windows/PowerShell:} \texttt{.\textbackslash build\_win.bat})

\item \textbf{Expected stdout (exact).}
\begin{itemize}\itemsep1pt
  \item $\Omega/\alpha_G^{(\mathrm{pp})} = 1.09372878$
  \item $\hat\alpha_s^\star(M_Z) = 0.1173411$
  \item $\Lambda_\chi = 14.050704$
  \item $\|\chi\|_{K_{\rm eq}} = 17.627830$, \quad $\cos\theta = 1.0000000$
\end{itemize}

\item \textbf{Artifacts (repro pack).}
\texttt{results.json}, \texttt{metric\_results.json}, \texttt{stdout.txt}, and
\texttt{SHA256SUMS.txt} (checksums).
\end{enumerate}

\paragraph*{Hash check}
\texttt{sha256sum -c SHA256SUMS.txt} \ (Linux/macOS) \quad
\texttt{Get-FileHash -Algorithm SHA256} \ (Windows).

\paragraph*{Notes and failure modes}
\begin{itemize}\itemsep1pt
\item \textit{Version drift:} re-run using the pinned files in this repo (no network calls).
\item \textit{Pin drift:} restore \texttt{pins.json}/\texttt{keq.json} to the commit referenced in S0.
\item \textit{Non-determinism:} ensure no RNG is used; default build is RNG-free.
\item \textit{MC checks:} Monte Carlo confirmation exists only in the SM (Sec.~S6.8); not executed here.
\end{itemize}



\bibliographystyle{apsrev4-2}
\bibliography{gage_prl_refs}
\end{document}
