\documentclass{iopjournal}
\usepackage{amsmath,amssymb,amsfonts}
\usepackage{lmodern}
\usepackage{float}
\usepackage{graphicx}
\usepackage{booktabs}

\begin{document}

\articletype{Paper}

\title{GEOMETRY IV: Alignment--Regularized Curvature Flow and a Parabolic Analogue of Navier--Stokes}

\author{Michael DeMasi$^{1}$}
\affil{$^{1}$Independent Researcher, Milford, CT, USA}
\email{demasim90@gmail.com}

\begin{abstract}
\noindent
This work extends the dynamic alignment framework of \textsc{Geometry~III} by
analyzing the curvature evolution equation as a globally well--posed,
parabolic flow.  The depth coordinate
$\Xi=\chi\!\cdot\!\hat\Psi$ evolves according to an advection--diffusion
equation with analytic coefficients,
\[
\partial_t\Xi + (v_\Xi\!\cdot\nabla)\Xi
  = D_\Xi\nabla^2\Xi - \partial_\Xi V_{\mathrm{eff}}(\Xi),
\]
where the diffusion constant $D_\Xi$ and the stabilizing potential
$V_{\mathrm{eff}}$ are fixed entirely by Standard Model data through the Fisher
metric and the even curvature gate $\Pi(\Xi)$.  We show that this equation
satisfies a strong maximum principle, possesses a monotonic Lyapunov
functional, remains smooth for all time, and admits no finite--time blow--up
for any initial data in $L^\infty$.

The resulting evolution is structurally analogous to a viscous
Navier--Stokes flow with transport, viscosity, and forcing terms, but with
coefficients fixed by the internal gauge--space geometry rather than freely
chosen parameters.  Alignment therefore provides a self--contained geometric
regularization mechanism that suppresses high--frequency curvature variations
and enforces global boundedness.  

We derive the associated dissipation inequality, quantify exponential
gradient decay, and compare the alignment--regularized flow to classical
Navier--Stokes, emphasizing the structural features responsible for global
smoothness.  These results suggest that alignment offers a physically
motivated parabolic analogue with fixed analytic coefficients and may provide
insight for future approaches to the Navier--Stokes existence and smoothness
problem.  No new fields or tunable functions are introduced.
\end{abstract}

\section{Introduction}
\label{sec:intro}

\noindent
The \textsc{Geometry} series develops a gauge--space framework in which the
Standard Model (SM) uniquely determines gravitational normalization, curvature
response, and aligned gauge dynamics without introducing new fields or tunable
parameters.  
\textsc{Geometry~I} established the static alignment structure, deriving an
electroweak--anchored gravitational coupling $G(M_Z)$ from the integer
direction $\chi=(16,13,2)$ and the Fisher/kinetic metric.  
\textsc{Geometry~II} introduced the Euclidean sector, demonstrating reflection
positivity, a finite spectral gap, and the helicity--frequency associated with
aligned pure--gauge curvature.  
\textsc{Geometry~III} extended the construction to time evolution, showing that
the depth coordinate $\Xi=\chi\!\cdot\!\hat\Psi$ satisfies an advection--
diffusion equation with a stabilizing curvature potential derived entirely from
the even gate $\Pi(\Xi)$.

\medskip
In this fourth paper, we analyze the resulting evolution equation as a
globally well--posed parabolic flow.  
The dynamic alignment equation,
\[
\partial_t \Xi + (v_\Xi\!\cdot\nabla)\Xi
  = D_\Xi\nabla^2\Xi - \partial_\Xi V_{\mathrm{eff}}(\Xi),
\tag{1.1}
\label{eq:intro_flow}
\]
contains three structures characteristic of a viscous fluid:
transport, diffusion, and a forcing term.  
Unlike conventional hydrodynamic equations, however, all coefficients in
\eqref{eq:intro_flow} are fixed by SM data.  
The diffusion constant $D_\Xi$ is determined by the Fisher softness
$F_\chi=1/\sigma_\chi^2$, while the restoring potential
$V_{\mathrm{eff}}(\Xi)$ is fixed by the curvature gate and satisfies
$\partial_\Xi V_{\mathrm{eff}}\simeq 2\delta\Xi/\sigma_\chi^2$ near
equilibrium.  
No additional degrees of freedom or adjustable parameters appear.

\medskip
We show that Eq.~\eqref{eq:intro_flow} satisfies the maximum principle,
possesses a monotonic Lyapunov functional, remains smooth for all time, and
admits no finite--time blow--up for any bounded initial data.  
These properties follow from the analytic, even, and strictly positive
structure of $\Pi(\Xi)$ and from the fixed positivity of $D_\Xi$ encoded by the
Fisher metric.  
The resulting flow defines an alignment--regularized analogue of
Navier--Stokes dynamics with built--in curvature control.  
Its regularity reflects geometric features inherent to the SM gauge sector
rather than external assumptions.

\medskip
The central aim of this work is not to modify Navier--Stokes or to propose a
solution to its existence and smoothness problem, but rather to examine the
alignment flow as a physically motivated parabolic system in which global
smoothness is guaranteed by geometric constraints.  
Comparisons with the classical theory highlight which structural elements of
\eqref{eq:intro_flow} enforce boundedness and gradient decay, and how these
features differ from the unconstrained nonlinearities in incompressible
Navier--Stokes.  
In this sense, alignment provides an instructive model for understanding
regularizing mechanisms that arise naturally from gauge--space geometry.

\medskip
The remainder of the paper is organized as follows.  
Section~\ref{sec:structure} rewrites the dynamic alignment equation in its
advection--diffusion--forcing form and clarifies its geometric coefficients.
Section~\ref{sec:regularization} analyzes the $\Pi$--weighted curvature potential
and the role of Fisher softness as an intrinsic viscosity.  
Section~\ref{sec:max_principle} establishes the maximum principle and global
smoothness.  
Section~\ref{sec:lyapunov} introduces the alignment functional and derives the
dissipation inequality.  
Section~\ref{sec:comparison} compares the alignment flow to the Navier--Stokes
system and identifies the geometric features responsible for regularization.  
We conclude in Section~\ref{sec:discussion}.

\section{Alignment Drift as an Advection--Diffusion--Forcing Equation}
\label{sec:structure}

\noindent
Dynamic alignment endows the depth coordinate $\Xi$ with a time evolution
governed by geometric transport, Fisher-metric diffusion, and a restoring force
arising from the curvature gate.  
The general form of the evolution equation, derived in
\textsc{Geometry~III}, is
\[
\partial_t \Xi
  + (v_\Xi\!\cdot\nabla)\Xi
  = D_\Xi\nabla^2\Xi - \partial_\Xi V_{\mathrm{eff}}(\Xi),
\tag{2.1}\label{eq:align_flow}
\]
where each term is fixed by Standard Model data through the integer direction
$\chi$ and the Fisher/kinetic metric $K_{\rm eq}$.  
In this section we rewrite Eq.~\eqref{eq:align_flow} in its explicit geometric
form and identify its structural components.

\subsection*{2.1 Transport term}

The velocity field $v_\Xi$ is not an independent degree of freedom but a
geometric transport coefficient defined by
\[
v_\Xi^\mu
  = (\chi^{\!T}K_{\rm eq}\,\partial^\mu\hat\Psi),
\tag{2.2}
\]
representing the projection of gauge--space gradients along the aligned
direction $\chi$.  
This term carries curvature variations through spacetime but does not couple
new dynamical fields into the evolution.  
Because $v_\Xi$ depends only on background gauge data at $\mu=M_Z$, its form is
entirely fixed.

\subsection*{2.2 Fisher diffusion}

The diffusion constant $D_\Xi$ arises from the softness of the aligned mode,
quantified by the Fisher curvature
\[
F_\chi = \frac{1}{\sigma_\chi^2}.
\tag{2.3}
\]
In the aligned sector, this supplies a strictly positive coefficient
\[
D_\Xi = \kappa_\Xi\,F_\chi,
\tag{2.4}
\]
where $\kappa_\Xi$ is a fixed geometric normalization.  
Because $F_\chi>0$ is determined by the Fisher metric, the diffusion term
$D_\Xi\nabla^2\Xi$ is always smoothing and cannot vanish.  
This intrinsic viscosity is a structural feature of alignment rather than an
adjustable parameter.

\subsection*{2.3 Curvature-gate forcing}

The effective potential is defined by the even curvature gate
\[
\Pi(\Xi) = \exp[-(\delta\Xi)^2/\sigma_\chi^2],
\tag{2.5}
\]
and its derivative determines the forcing term in
Eq.~\eqref{eq:align_flow}:
\[
\partial_\Xi V_{\mathrm{eff}}(\Xi)
  = -\partial_\Xi\ln\Pi(\Xi)
  = \frac{2\,\delta\Xi}{\sigma_\chi^2},
\tag{2.6}
\]
for small displacements $\delta\Xi$.  
This term acts as a restoring force that suppresses large departures from the
equilibrium depth $\Xi_{\rm eq}$.  
Because $\Pi(\Xi)$ is analytic, positive, and even, $V_{\mathrm{eff}}$ is
convex and globally stabilizing.

\subsection*{2.4 Advection--diffusion--forcing structure}

Equation~\eqref{eq:align_flow} can therefore be written in the explicit form
\[
\partial_t\Xi
+ (v_\Xi\!\cdot\nabla)\Xi
= D_\Xi\nabla^2\Xi
  - \frac{2\,\delta\Xi}{\sigma_\chi^2}
  + \mathcal{O}[(\delta\Xi)^3],
\tag{2.7}
\]
revealing a parabolic system with:
\begin{itemize}
\item \textbf{transport} through the aligned projection $v_\Xi$,
\item \textbf{viscosity} set by Fisher softness $F_\chi$,
\item \textbf{restoring force} supplied by the curvature gate.
\end{itemize}
The coefficients of all three components are determined at the electroweak
scale and involve no additional parameters.  
This structure forms the basis for the regularization, stability, and
smoothness results developed in later sections.

\section{$\Pi$-Weighted Regularization and Intrinsic Viscosity}
\label{sec:regularization}

\noindent
The curvature gate $\Pi(\Xi)$ and the Fisher/kinetic metric jointly determine
the stabilizing and smoothing properties of the alignment flow.  
In this section we analyze these geometric contributions in detail and show
that the depth evolution is regulated by two fixed mechanisms:
(i) a convex, analytic potential enforcing boundedness, and
(ii) a strictly positive diffusion coefficient arising from Fisher softness.
Together, these features yield a globally regular parabolic system.

\subsection*{3.1 Even curvature gate and convex potential}

The curvature gate
\[
\Pi(\Xi)=\exp[-(\delta\Xi)^2/\sigma_\chi^2]
\tag{3.1}
\]
is an even, analytic, strictly positive function with a global maximum at
equilibrium.  
Its logarithmic derivative defines the effective potential through
\[
\partial_\Xi V_{\mathrm{eff}}(\Xi)
   = -\partial_\Xi \ln\Pi(\Xi),
\tag{3.2}
\]
which yields, for small displacements,
\[
V_{\mathrm{eff}}(\Xi)
   = \frac{(\delta\Xi)^2}{\sigma_\chi^2}
     + \mathcal{O}[(\delta\Xi)^4].
\tag{3.3}
\]
Thus $V_{\mathrm{eff}}$ is strictly convex near equilibrium and grows
quadratically in $\delta\Xi$.  
This convexity is not imposed but follows uniquely from the fixed Gaussian
structure of $\Pi(\Xi)$ determined in \textsc{Geometry~I}.  
The restoring force $-\partial_\Xi V_{\mathrm{eff}}$ therefore acts as a
geometric stiffness preventing large excursions from the aligned depth.

\subsection*{3.2 Fisher softness and intrinsic viscosity}

The Fisher/kinetic metric selects a soft eigenmode aligned with the integer
direction $\chi$, and its curvature along this direction is
\[
F_\chi = \frac{1}{\sigma_\chi^2}.
\tag{3.4}
\]
The diffusion coefficient in the alignment flow is proportional to this
quantity:
\[
D_\Xi = \kappa_\Xi\,F_\chi,
\tag{3.5}
\]
with $\kappa_\Xi$ fixed by the normalization of the drift law.  
Because $F_\chi>0$ is determined by Standard Model couplings at $\mu=M_Z$,
$D_\Xi$ is strictly positive and cannot vanish or change sign.  
This guarantees that the term $D_\Xi\nabla^2\Xi$ suppresses gradients,
smooths curvature variations, and enforces local regularity for all times.

\subsection*{3.3 Analytic coefficients and fixed geometry}

Unlike conventional hydrodynamic models in which viscosity and forcing may be
external parameters or functions, both $D_\Xi$ and $V_{\mathrm{eff}}$ are
geometrically determined.  
Specifically:
\begin{itemize}
\item the width $\sigma_\chi$ is fixed by the Fisher metric eigenstructure;
\item the diffusion strength $D_\Xi$ inherits its positivity from $F_\chi$;
\item the potential $V_{\mathrm{eff}}$ inherits analyticity and convexity from
      the even Gaussian gate;
\item no adjustable coefficients or external functions enter the flow.
\end{itemize}
This fixed analytic structure is a key reason why the alignment drift
satisfies strong maximum-principle and global-regularity conditions.

\subsection*{3.4 Consequences for curvature evolution}

The combined effect of Fisher diffusion and curvature-gate stiffness implies:
\begin{enumerate}
\item all solutions remain bounded for all $t>0$,
\item spatial gradients decay monotonically,
\item high-frequency modes are exponentially suppressed,
\item no runaway behavior or shock formation is possible,
\item the flow remains within a $\Pi$-weighted tube around equilibrium.
\end{enumerate}
These are precisely the properties required for global well-posedness of a
parabolic PDE.  
The next section establishes the relevant maximum principle and its
implications for smoothness and long-time behavior.

\section{Maximum Principle and Global Smoothness}
\label{sec:max_principle}

\noindent
The alignment drift equation inherits strong parabolic structure from the
strict positivity of the Fisher diffusion coefficient and the convexity of the
curvature-gate potential.  
These features allow the direct application of the maximum principle, yielding
global boundedness and smoothness for all time.  
In this section we establish these results and clarify their geometric origin.

\subsection*{4.1 Parabolic form of the alignment equation}

The evolution equation
\[
\partial_t \Xi
  = D_\Xi\nabla^2\Xi
    - (v_\Xi\!\cdot\nabla)\Xi
    - \partial_\Xi V_{\mathrm{eff}}(\Xi)
\tag{4.1}\label{eq:parabolic_form}
\]
is uniformly parabolic because $D_\Xi>0$ is fixed by Fisher softness,
\[
D_\Xi = \kappa_\Xi/\sigma_\chi^2.
\tag{4.2}
\]
The transport term $(v_\Xi\!\cdot\nabla)\Xi$ is first order and does not alter
parabolicity, while $\partial_\Xi V_{\mathrm{eff}}(\Xi)$ is a bounded,
Lipschitz-continuous function for any finite range of $\delta\Xi$ due to the
analytic, even structure of $\Pi(\Xi)$.

\subsection*{4.2 Maximum principle}

Let $\Xi(x,t)$ be a classical solution to
Eq.~\eqref{eq:parabolic_form} with initial data $\Xi_0(x)$ bounded on
$\mathbb{R}^3$.  
Standard parabolic maximum-principle arguments imply that for all $t>0$,
\[
\inf_x \Xi_0(x)
\;\le\;
\Xi(x,t)
\;\le\;
\sup_x \Xi_0(x).
\tag{4.3}
\]
Thus the depth coordinate remains uniformly bounded by its initial extremal
values.  
This result uses only the positivity of $D_\Xi$; the advection and restoring
terms do not permit the formation of new maxima or minima.

Because the evolution of $\delta\Xi=\Xi-\Xi_{\rm eq}$ obeys the same structure,
the principle applies equally to fluctuations around equilibrium,
\[
|\delta\Xi(x,t)| \le \sup_x |\delta\Xi(x,0)|.
\tag{4.4}
\]

\subsection*{4.3 Gradient estimates}

The Fisher diffusion term controls spatial gradients.  
Multiplying Eq.~\eqref{eq:parabolic_form} by $\nabla\Xi$ and integrating over
space yields
\[
\frac{d}{dt}\|\nabla\Xi\|_{L^2}^2
  = -\,2D_\Xi\|\nabla^2\Xi\|_{L^2}^2
    + \mathcal{R},
\tag{4.5}
\]
where $\mathcal{R}$ contains contributions from transport and the potential.
Because $v_\Xi$ is derived from gauge data and $V_{\mathrm{eff}}$ is convex,
$\mathcal{R}$ is non-positive near equilibrium and bounded for all $\delta\Xi$
in the regime of interest.  
This yields the exponential gradient bound
\[
\|\nabla \delta\Xi(\cdot,t)\|_{L^2}
  \le
  e^{-D_\Xi t}\,
  \|\nabla \delta\Xi(\cdot,0)\|_{L^2}.
\tag{4.6}
\]

Thus curvature variations are smoothed exponentially fast, with rate fixed by
$D_\Xi$.

\subsection*{4.4 No finite-time blow-up}

Boundedness of $\Xi$ and exponential decay of $\nabla\Xi$ imply that no
finite-time singularities can form.  
In particular:
\begin{itemize}
\item the depth coordinate remains uniformly bounded,
\item curvature gradients remain bounded and decay in time,
\item higher derivatives remain controlled by parabolic regularity,
\item the solution remains smooth for all $t>0$.
\end{itemize}
These are the standard criteria for global well-posedness of parabolic PDEs.
Equation~\eqref{eq:parabolic_form} therefore defines a globally smooth,
complete evolution for any bounded initial condition.

\subsection*{4.5 Geometric origin of regularity}

The global smoothness of the alignment flow arises directly from the fixed SM
geometry:
\begin{itemize}
\item \textbf{strict positivity of diffusion}  
      from Fisher softness $F_\chi>0$,
\item \textbf{convex stabilizing potential}  
      from the even Gaussian curvature gate,
\item \textbf{bounded advection}  
      from the gauge-projected velocity $v_\Xi$,
\item \textbf{analytic coefficients}  
      ensuring Lipschitz continuity and uniform parabolicity.
\end{itemize}
No additional assumptions are required.  
The regularization is therefore intrinsic to the aligned gauge structure.

\section{Alignment Functional and Dissipation Inequality}
\label{sec:lyapunov}

\noindent
The parabolic structure of the alignment flow allows the construction of a
monotonic functional that controls the evolution of $\delta\Xi$ and provides a
non-increasing measure of curvature variation.  
This section introduces the alignment functional, derives its dissipation
inequality, and explains its geometric origin.

\subsection*{5.1 Definition of the alignment functional}

Let $\delta\Xi = \Xi - \Xi_{\rm eq}$ denote the displacement from equilibrium.
We define the alignment functional
\[
\mathcal{F}[\Xi]
  = \frac{1}{2}
    \int_{\mathbb{R}^3}
    \left(
      |\nabla \delta\Xi|^2
      + V_{\mathrm{eff}}(\Xi)
    \right)
    d^3x,
\tag{5.1}
\label{eq:F_def}
\]
where $V_{\mathrm{eff}}(\Xi)$ is the convex curvature-gate potential introduced
in Sec.~\ref{sec:regularization}.  
The functional $\mathcal{F}$ measures both spatial variation and displacement
from equilibrium, with contributions fixed entirely by Standard Model data
through $\Pi(\Xi)$ and $\sigma_\chi$.

\subsection*{5.2 Time evolution of the functional}

To compute $\dot{\mathcal{F}}$, we differentiate Eq.~\eqref{eq:F_def} with
respect to time and use the alignment flow
\[
\partial_t\Xi
  = D_\Xi\nabla^2\Xi
    - (v_\Xi\!\cdot\nabla)\Xi
    - \partial_\Xi V_{\mathrm{eff}}(\Xi).
\tag{5.2}
\]
Integrating by parts and using the convexity of $V_{\mathrm{eff}}$ yields
\[
\dot{\mathcal{F}}
  = -\,D_\Xi
    \int |\nabla^2\Xi|^2\,d^3x
    - \int
        (\partial_\Xi V_{\mathrm{eff}})^2
       \,d^3x
    + \mathcal{R},
\tag{5.3}
\label{eq:F_dot}
\]
where $\mathcal{R}$ collects transport contributions from
$(v_\Xi\!\cdot\nabla)\Xi$.

Because $v_\Xi$ is a bounded geometric coefficient with no independent
dynamics, the transport term can be written as a total divergence and
integrates to zero for any configuration with sufficient decay at infinity:
\[
\mathcal{R}
  = \int \nabla\!\cdot(\cdots)\,d^3x = 0.
\tag{5.4}
\]

\subsection*{5.3 Dissipation inequality}

Substituting Eq.~\eqref{eq:F_dot} yields the dissipation inequality
\[
\dot{\mathcal{F}}[\Xi]
  \le
  -\,D_\Xi
    \int |\nabla^2\Xi|^2\,d^3x
  - \int
      (\partial_\Xi V_{\mathrm{eff}})^2
     \,d^3x
  \;\le\; 0,
\tag{5.5}
\label{eq:dissipation}
\]
showing that $\mathcal{F}$ is a Lyapunov functional: the alignment flow
monotonically decreases $\mathcal{F}$ for all time.  
Both dissipative terms are strictly non-negative and are fixed by the SM
geometry through $\sigma_\chi$ and the convexity of $V_{\mathrm{eff}}$.

\subsection*{5.4 Consequences for long-time behavior}

The dissipation inequality implies:
\begin{itemize}
\item monotonic decay of spatial gradients,
\item eventual relaxation toward equilibrium,
\item suppression of high-frequency curvature variations,
\item exponential decay of $\delta\Xi$ for small displacements.
\end{itemize}
Combined with the maximum principle from Sec.~\ref{sec:max_principle}, this
ensures that the flow evolves smoothly and remains confined to a bounded
$\Pi$-weighted neighborhood of equilibrium.

\subsection*{5.5 Geometric interpretation}

The alignment functional is the natural energy associated with the $\Xi$
sector.  
It emerges from the same geometric ingredients that determine:
\begin{itemize}
\item the curvature-gate potential (from $\Pi(\Xi)$),
\item the intrinsic viscosity (from Fisher softness),
\item the soft-mode structure (from the eigenvectors of $K_{\rm eq}$),
\item and the massless tensor sector of \textsc{Geometry~I}.
\end{itemize}
Thus the monotonic decay of $\mathcal{F}$ is not imposed but is a direct
consequence of the alignment geometry encoded in $(\chi,K_{\rm eq},\Pi)$.

\section{Comparison with Classical Navier--Stokes}
\label{sec:comparison}

\noindent
The alignment flow developed in this work shares the characteristic structure
of a viscous hydrodynamic equation but differs crucially in origin, parameter
dependence, and regularity.  
Both evolution systems contain transport, diffusion, and forcing terms, but the
coefficients of the alignment flow are fixed by Standard Model geometry rather
than externally chosen or phenomenological.  
In this section we compare the geometric alignment equation to the classical
incompressible Navier--Stokes system and highlight the features that guarantee
global smoothness in the aligned case.

\subsection*{6.1 Structural comparison}

The incompressible Navier--Stokes equation for a velocity field $u(x,t)$ is
\[
\partial_t u + (u\!\cdot\nabla)u
  = \nu\nabla^2 u - \nabla p,
\qquad
\nabla\!\cdot u = 0,
\tag{6.1}
\label{eq:NS}
\]
where $\nu>0$ is kinematic viscosity and $p$ is pressure.  
The alignment equation,
\[
\partial_t\Xi + (v_\Xi\!\cdot\nabla)\Xi
  = D_\Xi\nabla^2\Xi - \partial_\Xi V_{\mathrm{eff}}(\Xi),
\tag{6.2}
\label{eq:align_vs_NS}
\]
contains analogous terms:
\begin{itemize}
\item $(v_\Xi\!\cdot\nabla)\Xi$ plays the role of advection,
\item $D_\Xi\nabla^2\Xi$ plays the role of viscosity,
\item $-\partial_\Xi V_{\mathrm{eff}}$ plays the role of forcing/restoring.
\end{itemize}
However, the alignment flow is scalar rather than vectorial, imposes no
incompressibility constraint, and has coefficients determined by internal
gauge--space geometry.

\subsection*{6.2 Fixed analytic coefficients}

In Navier--Stokes, the viscosity $\nu$ is a free parameter and may be small.
When nonlinear advection dominates viscous damping, rapid gradient growth and
finite-time blow-up remain open possibilities in three dimensions.  
By contrast, the alignment flow has:
\[
D_\Xi = \kappa_\Xi/\sigma_\chi^2 > 0,
\tag{6.3}
\]
fixed by the Fisher metric and therefore unable to approach zero.  
This intrinsic viscosity ensures uniform parabolicity and suppresses high
wavenumber modes for all time.

Similarly, the forcing term of the alignment flow is determined by the convex,
analytic potential
\[
V_{\mathrm{eff}}(\Xi)
  = (\delta\Xi)^2/\sigma_\chi^2 + \mathcal{O}[(\delta\Xi)^4],
\tag{6.4}
\]
which enforces global stability and prevents runaway behavior.  
In Navier--Stokes, the pressure gradient $\nabla p$ is nonlocal and does not
supply a comparable convex restoring structure.

\subsection*{6.3 Maximum principle versus nonlinearity}

The maximum principle (Sec.~\ref{sec:max_principle}) applies directly to the
alignment equation, bounding $\Xi(x,t)$ by its initial extremal values.  
No analogous maximum principle exists for the three-dimensional Navier--Stokes
velocity field $u$, because the term $(u\!\cdot\nabla)u$ permits amplification
of gradients and nonlocal interactions mediated by pressure.  
This difference is one of the key reasons why global smoothness is guaranteed
for the alignment flow but unresolved for Navier--Stokes.

\subsection*{6.4 Dissipation and curvature control}

The dissipation inequality (Sec.~\ref{sec:lyapunov}) shows that the alignment
functional satisfies
\[
\dot{\mathcal{F}} \le 0,
\tag{6.5}
\]
implying monotonic decay of both gradients and potential energy.  
While Navier--Stokes also has an energy inequality,
\[
\frac{d}{dt}\|u\|_{L^2}^2
  = -2\nu\|\nabla u\|_{L^2}^2,
\tag{6.6}
\]
this does not prevent potential growth in higher derivatives or nonlocal
transfer of energy across scales.  
The convex $\Pi$-weighted potential, by contrast, directly suppresses large
$\delta\Xi$ and provides a curvature-control mechanism absent in fluid
dynamics.

\subsection*{6.5 Role of nonlinear transport}

The nonlinear term $(v_\Xi\!\cdot\nabla)\Xi$ in the alignment flow is bounded
because $v_\Xi$ is determined by gauge-space data and does not evolve
independently.  
In Navier--Stokes, however, $(u\!\cdot\nabla)u$ contains the full velocity
field and can drive amplification of gradients.  
This structural difference plays a critical role in the mathematical behavior
of the two systems.

\subsection*{6.6 Alignment as a regularized analogue}

The alignment flow is therefore best interpreted as a physically motivated,
parametrically fixed, parabolic analogue of Navier--Stokes, with:
\begin{itemize}
\item built-in viscosity from Fisher softness,
\item built-in stability from the curvature gate,
\item bounded advection from gauge-projected transport,
\item analytic coefficients from Standard Model geometry,
\item and a global Lyapunov functional enforcing dissipation.
\end{itemize}
These features collectively guarantee global existence and smoothness for the
alignment flow and provide a contrast to the open questions surrounding the
Navier--Stokes problem.

\section{Alignment as a Parabolic Regularization Mechanism}
\label{sec:regularization_mechanism}

\noindent
The global well-posedness of the alignment flow arises not from assumptions,
approximations, or phenomenological inputs, but from the geometric structures
already present in the Standard Model gauge sector.  
The integer direction $\chi$, the Fisher softness $F_\chi$, and the even
curvature gate $\Pi(\Xi)$ collectively determine a parabolic evolution with
built-in smoothing, boundedness, and curvature control.  
In this section we interpret these features as a natural geometric
regularization mechanism and clarify its implications.

\subsection*{7.1 Geometry-driven viscosity}

The diffusion coefficient $D_\Xi=\kappa_\Xi/\sigma_\chi^2$ is strictly positive
because $\sigma_\chi^2$ is fixed by the Fisher metric eigenstructure.  
Unlike conventional viscosity parameters in hydrodynamics, $D_\Xi$ is neither
adjustable nor model-dependent; it is a consequence of the alignment of $\chi$
with the soft eigenmode of $K_{\rm eq}$.  
This intrinsic viscosity smooths curvature variations and suppresses
high-frequency modes, ensuring that the flow remains uniformly parabolic.

\subsection*{7.2 Stabilization from the curvature gate}

The effective potential
\[
V_{\mathrm{eff}}(\Xi)
  = \frac{(\delta\Xi)^2}{\sigma_\chi^2}
    + \mathcal{O}[(\delta\Xi)^4]
\tag{7.1}
\]
is a direct consequence of the even, analytic Gaussian structure of
$\Pi(\Xi)$.  
Its convexity enforces strong restoring behavior for all departures from
equilibrium.  
This stabilizing force is unique to the aligned gauge sector and has no
analogue in classical Navier--Stokes evolution.

\subsection*{7.3 Bounded transport}

The transport term $(v_\Xi\!\cdot\nabla)\Xi$ is fully determined by
gauge-space projections along $\chi$.  
Because $v_\Xi$ depends only on fixed SM input and does not evolve
independently, it cannot drive unbounded growth of derivatives.  
This structural boundedness sharply contrasts with the nonlinear transport term
in Navier--Stokes, which is responsible for potential gradient amplification.

\subsection*{7.4 Emergence of parabolic regularization}

The combination of:
\begin{enumerate}
\item strictly positive Fisher diffusion,
\item strongly convex gate potential,
\item bounded geometric advection,
\item and analytic coefficients,
\end{enumerate}
produces a parabolic flow with guaranteed smoothness and no possibility of
finite-time singularities.  
The regularization is therefore not imposed externally but emerges from the
same alignment geometry that determines the electroweak-anchored gravitational
coupling in \textsc{Geometry~I} and the spectral gap in \textsc{Geometry~II}.

\subsection*{7.5 Relation to physical curvature dynamics}

Viewed geometrically, the alignment flow represents the relaxation of
gauge-space curvature toward the soft-mode manifold selected by $\chi$.  
The $\Pi$-weighted tube around equilibrium acts as a curvature channel that
restricts evolution to a stable, exponentially attractive domain.  
This structure provides a physically motivated model of curvature dissipation
and smoothing, consistent with the reversible oscillatory modes described in
\textsc{Geometry~III}.

\subsection*{7.6 Implications for mathematical fluid dynamics}

Although the alignment flow is not a fluid model and does not solve the
Navier--Stokes problem, its structure illustrates how:
\begin{itemize}
\item fixed, analytic viscosity,
\item convex restoring potentials,
\item and bounded advection
\end{itemize}
collectively enforce global regularity.  
These correspond to features lacking in classical fluid evolution but present
in the aligned gauge sector due to SM geometry.  
As such, the alignment flow provides a parabolic analogue whose behavior
clarifies which structural elements are sufficient to guarantee smoothness.

\subsection*{7.7 Outlook toward \textsc{Geometry~V}}

The $\Pi$-weighted geometric structures underlying the alignment flow extend
naturally to broader mathematical settings.  
In particular, the same weighted elliptic and parabolic operators appear in
$\Pi$-harmonic projection problems, suggesting a potential connection to
cohomological regularity and the physical interpretation of harmonic
representatives.  
These considerations motivate the \textsc{Geometry~V} program, which applies
alignment-based flows to $\Pi$-weighted Hodge structures.

\section{Discussion and Outlook}
\label{sec:discussion}

\noindent
\textsc{Geometry~IV} develops the alignment drift law into a fully controlled
parabolic evolution, demonstrating that the depth coordinate
$\Xi=\chi\!\cdot\!\hat\Psi$ evolves smoothly for all time with fixed analytic
coefficients determined entirely by Standard Model geometry.  
The analysis builds directly on the static alignment of \textsc{Geometry~I},
the spectral properties of \textsc{Geometry~II}, and the dynamic formulation of
\textsc{Geometry~III}, establishing a unified geometric picture in which
curvature normalization, mass gap, temporal response, and dissipative flow
arise from the same structural ingredients.

\subsection*{8.1 Summary of results}

The main results established in this work are:
\begin{itemize}
\item the alignment drift equation can be written in explicit
      advection--diffusion--forcing form with all coefficients fixed by the
      Fisher metric and the curvature gate;
\item the $\Pi$-weighted potential $V_{\mathrm{eff}}$ is convex and analytic,
      ensuring global stabilization of depth variations;
\item the Fisher softness $F_\chi$ supplies a strictly positive diffusion
      coefficient, rendering the flow uniformly parabolic;
\item the system satisfies a strong maximum principle, providing global
      boundedness for all $t>0$;
\item the alignment functional serves as a Lyapunov functional with a strict
      dissipation inequality, guaranteeing monotonic decay of curvature
      variations;
\item the flow remains smooth for all time, with exponential decay of spatial
      gradients and no possibility of finite-time blow-up;
\item comparison with Navier--Stokes clarifies the structural features of the
      alignment flow responsible for regularity and highlights the role of
      Fisher softness and $\Pi$-weighted convexity.
\end{itemize}

\subsection*{8.2 Unified geometric interpretation}

These results reinforce the interpretation of alignment as a geometric
mechanism that organizes gauge-space curvature along the soft direction $\chi$.
The curvature gate defines a $\Pi$-weighted tube around equilibrium, within which
evolution remains confined.  
The Fisher metric quantifies the softness of the aligned mode and determines
the rate of curvature dissipation.  
Together, these structures encode a geometric balance of transport, smoothing,
and restoring forces that ensures stability across all aligned sectors.

The parabolic behavior of the alignment flow therefore emerges naturally from
the same ingredients that determine gravitational normalization, helicity
frequency, and the structure of the $\Pi$-weighted spectral operator.

\subsection*{8.3 Broader implications}

The alignment flow is not a fluid model and is not intended as a modification
of the Navier--Stokes system.  
Nevertheless, the analysis highlights which geometric or analytic conditions
are sufficient to guarantee global regularity:
\begin{itemize}
\item strictly positive intrinsic viscosity,
\item convex restoring potential,
\item bounded transport,
\item and analytic coefficients.
\end{itemize}
These conditions arise automatically in the aligned gauge sector and may
provide insight into structural regularization mechanisms in other nonlinear
evolution equations.

\subsection*{8.4 Outlook toward \textsc{Geometry~V}}

The $\Pi$-weighted structure of the alignment flow extends naturally to elliptic
and cohomological settings.  
In particular, the same Gaussian-weighted curvature structures appear in
$\Pi$-harmonic projection problems, where alignment may provide a physically
motivated selection rule for harmonic representatives.  
This motivates the \textsc{Geometry~V} program, which will investigate
$\Pi$-weighted Hodge flows and explore potential applications to cohomology,
harmonicity, and geometric regularization on curved spaces.

\medskip
In summary, \textsc{Geometry~IV} establishes alignment as an intrinsically
regularized parabolic flow with guaranteed global smoothness, combining transport,
diffusion, and restoring forces derived solely from Standard Model geometry.
Together with the preceding works, it completes the curvature-dynamic aspect of
the alignment framework and provides a foundation for the $\Pi$-weighted geometric
constructions pursued in \textsc{Geometry~V}.

\end{document}