\documentclass{iopjournal}
\usepackage{amsmath,amssymb,amsfonts}
\usepackage{lmodern}
\usepackage{float}
\usepackage{graphicx}
\usepackage{tikz}
\usepackage{pgfplots}
\pgfplotsset{compat=1.18}
\usepgfplotslibrary{fillbetween,polar}
% --- Tables and spacing helpers ---
\usepackage{array}
\usepackage{tabularx}   % optional: flexible width
\usepackage{longtable}  % if any table spans multiple pages
\usepackage{booktabs}
\usepackage[T1]{fontenc}
\DeclareFontShape{T1}{lmr}{bx}{sc}{<-> ssub * lmr/m/sc}{}
\begin{document}

\articletype{Paper}%

\title{GEOMETRY I: A parameter-free prediction for the strong coupling 
$\alpha_s(M_Z)$ from Standard Model gauge geometry and gravitational closure}

\author{Michael DeMasi$^1$}
\affil{$^1$Independent Researcher, Milford, CT, USA}

\email{demasim90@gmail.com}

\begin{abstract}
Within the internal constraints of the Standard Model (SM)—no new fields,
no tunable functions, and fixed renormalization at $\mu=M_Z$ in the
$\overline{\mathrm{MS}}$ scheme—we show that the gravitational normalization can
be obtained from SM data alone.  
The aligned depth $\Xi=\chi\!\cdot\!\hat\Psi$ is an internal gauge–log
coordinate, not a propagating field, and $G(M_Z)$ denotes an electroweak-anchored
normalization rather than a varying-$G$ framework.

At one loop, the SM decoupling matrix possesses a unique primitive integer
left-kernel $\chi=(16,13,2)$ in Smith normal form.  
Independently, the Fisher/kinetic metric selects a soft eigenmode of maximal
responsiveness.  
Their alignment identifies $\Xi$ as the unique admissible depth coordinate.
Exponentiation then yields the internal anchor
\[
\Omega=e^\Xi=\hat\alpha_s^{16}\hat\alpha_2^{13}\hat\alpha^{2}.
\]

Dimensional consistency with baryonic matter gives
\[
G(M_Z) = \frac{\hbar c}{m_p^2}\,\Omega(M_Z),
\]
a normalization fixed entirely by SM input.  
Using pinned electroweak values and the measured proton–proton gravitational
coupling, we obtain a closure ratio $Z_G\simeq1.0937$ and a leave-one-out
prediction $\hat\alpha_s^\star(M_Z)\approx0.11734$, consistent with modern
lattice and PDF analyses but achieved here without hadronic tuning.

An even, Gaussian curvature gate $\Pi(\Xi)$ with Fisher-fixed width
$\sigma_\chi$ promotes $G(M_Z)$ to $G(x)=G(M_Z)\Pi(\Xi(x))$ while preserving the
massless GR tensor sector and enforcing the near-equilibrium lab-null
\(
\Delta G/G = (\delta\Xi/\sigma_\chi)^2.
\)

All numerical values and figures are generated through a public,
hash-verified workflow.  
The construction remains entirely equilibrium-based; $\Xi$ carries no dynamics
in this paper.  
\end{abstract}

\keywords{general relativity, Standard Model, gauge theory, gravity, information geometry, renormalization group}


\section{Introduction}
\label{sec:intro} %motivation
The Standard Model (SM) provides a precise, renormalizable account of the three gauge
interactions and their running couplings, yet it contains no internal mechanism that fixes
Newton’s gravitational constant \(G_N\). In General Relativity (GR) the Einstein–Hilbert term
\begin{equation}
  \mathcal{L}_{\rm EH} = \frac{1}{16\pi G_N}\,R
\end{equation}
carries an empirically measured normalization: GR specifies how curvature responds to
stress--energy but does not determine the numerical value of that response. This motivates
the central question:
\begin{quote}
\emph{Does the SM gauge sector at \(\mu=M_Z\) contain sufficient, basis-invariant structure
to fix a gravitational normalization without new degrees of freedom or modification of GR?}
\end{quote}

We restrict throughout to SM inputs at \(\mu=M_Z\) in the \(\overline{\mathrm{MS}}\) scheme, assume
no new fields, parameters, or tunable functions, and retain the massless, luminal
helicity-\(\pm2\) tensor sector of GR. The goal is to determine whether the SM’s internal
geometry—its one-loop integer structure and Fisher/kinetic metric—provides a parameter-free
gravitational normalization \(G(M_Z)\) consistent with observation. All numerical values and
figures are obtained directly from public SM inputs using a reproducible, hash-verified build
workflow archived under a public DOI (see Data Availability). %Scope

\paragraph*{Physical status and scope.}
Throughout this work, \(\hat\Psi\) and \(\Xi=\chi\!\cdot\!\hat\Psi\) are internal
gauge--log coordinates rather than propagating spacetime fields. The construction in
\textsc{Geometry~I} is static and equilibrium--restricted: the gravitational normalization is
fixed at \(\delta\Xi=0\), where the curvature gate satisfies \(\Pi(\Xi_{\rm eq})=1\) and the
Einstein--Hilbert sector is unchanged. No assumption is made regarding any time evolution of
\(\Xi\); dynamical alignment and drift-law extensions appear only in later papers of the
\textsc{Geometry} series. Predictions for the observable ratio \(\Delta G/G\) use only small
departures \(|\delta\Xi|\ll\sigma_\chi\) as a near-equilibrium expansion parameter.

A key distinction in what follows is that \(G(M_Z)\) denotes an electroweak-anchored
normalization obtained from SM data, not a modification of GR or a varying-\(G\) framework.
The normalization is defined at equilibrium and later encoded in the shorthand
\[
  G(x)=G(M_Z)\,\Pi(\Xi(x))
\]
as an \emph{internal curvature weighting} inside the Einstein–Hilbert term. In this paper
\(G(x)\) is a bookkeeping notation for how the SM-aligned curvature weight depends on
\(\Xi\); it is not a dynamical, spacetime-varying Newton constant, and the GR tensor sector
remains intact.

%definition
Two rigid SM structures, normally analyzed separately, play the key role:
(i) the integer lattice arising from one-loop decoupling, and (ii) the Fisher/kinetic metric
on log--coupling space. Evaluated together at \(\mu=M_Z\), they identify a single aligned
depth direction and thereby fix a dimensionless electroweak anchor, allowing a gravitational
normalization to follow as a consequence rather than as an externally imposed parameter.

%definition der 
At one loop, the SM decoupling matrix is an exact integer matrix whose Smith normal form (SNF)
admits a unique primitive left-kernel generator (up to overall sign), obtained using a standard
integer-normal-form algorithm implemented verbatim in the accompanying reproducibility scripts:
\begin{equation}
  \chi = (16,13,2).
  \label{eq:chi}
\end{equation}
This vector defines a depth coordinate
\begin{equation}
  \hat\Psi=(\ln\hat\alpha_s,\ln\hat\alpha_2,\ln\hat\alpha),
\label{eq:Psi_def}
\end{equation}
\begin{equation}
\Xi = \chi\cdot\hat\Psi
\label{eq:xidepth}
\end{equation}
in log--coupling space. Independently, the positive-definite (equilibrium) Fisher/kinetic
metric \(K\) (hereafter \(K\equiv K_{\rm eq}\)), constructed from one-loop sensitivity data,
possesses a soft eigenmode of maximal responsiveness. Using the same SM input pins and
renormalization scheme, we find that the integer direction \(\chi\) is numerically aligned
with the softest eigenvector of \(K\), with \(\cos\theta\simeq1\). Thus the aligned depth
\(\Xi\) is not a model assumption but the coordinate jointly selected by integer rigidity and
metric softness.

%def
Exponentiating the aligned depth yields the electroweak anchor
\begin{equation}
  \Omega \equiv e^{\Xi}
  = e^{\chi\cdot\hat\Psi}
  = \hat\alpha_s^{16}\,\hat\alpha_2^{13}\,\hat\alpha^{2}.
  \label{eq:omega}  
\end{equation}

so that \(\Omega\) is an internal, dimensionless SM anchor whose value is fixed entirely by
electroweak-scale couplings. No additional dynamical field is introduced, and no tuning is
applied; \(\Omega\) is simply the exponential of the aligned gauge--log depth.

This electroweak anchor defines an SM-derived gravitational normalization
\begin{equation}
  G(M_Z)=\frac{\hbar c}{m_p^{2}}\,\Omega(M_Z),
  \label{eq:gmz}
\end{equation}
with no adjustable parameters. The proton mass \(m_p\) is chosen as the dimensional anchor
because laboratory and astrophysical determinations of \(G_N\) predominantly probe baryonic
(proton-dominated) matter; alternative choices such as \(m_e\) or \(m_n\) simply rescale the
same dimensionless anchor \(\Omega\). Using the pinned electroweak inputs and the experimental
proton--proton gravitational coupling, this relation yields a preferred strong coupling
\begin{equation}
  \alpha_s^\star(M_Z) \approx 0.11734,
\end{equation}
which agrees with high-precision lattice and global-fit determinations while arising here
entirely from internal SM structure and gravitational closure, without hadronic fitting.

In this work we show that:
(i) gauge-log alignment in the Standard Model defines a parameter-free relation that fixes a
preferred value of \(\alpha_s(M_Z)\) from consistency with the proton--proton gravitational
coupling;
(ii) the same relation defines an SM-derived gravitational normalization \(G(M_Z)\) anchored
at \(\mu=M_Z\) in the \(\overline{\mathrm{MS}}\) scheme; and
(iii) an even, Gaussian curvature gate \(\Pi(\Xi)\) defines a near-equilibrium curvature
weight \(\Pi(\Xi(x))\) which we notationally package as
\(G(x)=G(M_Z)\Pi(\Xi(x))\).
In the static, equilibrium framework of \textsc{Geometry~I} this \(G(x)\) is an internal
weighting of the Einstein–Hilbert normalization rather than a varying-\(G\) theory, and the
massless, luminal tensor sector of GR is preserved.

\paragraph*{Internal displacement.}
In \textsc{Geometry~I}, $\delta\Xi$ denotes an internal displacement in gauge--log space
rather than a propagating scalar field. It represents the net response of the SM
gauge--log vector
$\hat\Psi=(\ln\hat\alpha_s,\ln\hat\alpha_2,\ln\hat\alpha)$
along the aligned integer direction $\chi=(16,13,2)$, with
\begin{equation}
\delta\Xi=\chi\cdot(\hat\Psi-\hat\Psi_{\rm eq}).
\end{equation}
No equation of motion for $\delta\Xi$ is introduced in this work. Matter and
stress--energy can in principle perturb the gauge--log coordinates and source
nonzero $\delta\Xi$, but the functional form of this sourcing belongs to the
dynamical extensions developed in \textsc{Geometry~III}. In the present static
framework, all observable results are evaluated at equilibrium ($\delta\Xi=0$),
where $\Pi(\Xi_{\rm eq})=1$ and the Einstein--Hilbert tensor sector is
unchanged.

\paragraph*{Summary of results and roadmap.}
Section~\ref{sec:lattice_depth} establishes the integer structure and the
unique primitive kernel $\chi$ of the one-loop decoupling matrix.
Section~\ref{sec:metric_softness} constructs the Fisher/kinetic metric and
verifies its numerical alignment with $\chi$, fixing the admissible depth
coordinate $\Xi$. Section~\ref{sec:curvature_gate} introduces the
parity-preserving curvature gate $\Pi(\Xi)$ and determines its fixed curvature
scale $\sigma_\chi$. Section~\ref{sec:SM_derived_G} derives the
electroweak-anchored normalization $G(M_Z)$ and encodes its near-equilibrium
response as $G(x)=G(M_Z)\Pi(\Xi(x))$ without modifying GR, with $G(x)$ serving
as an internal curvature weight rather than a varying Newton constant.
Section~\ref{sec:predictions} presents the quadratic lab-null, closure ratio,
and falsifiers. Section~\ref{sec:discussion} summarizes the implications and
outlines extensions to dynamical settings in future work within the broader
\textsc{Geometry} sequence.

\paragraph*{Program and provenance.}
This work is the first in a sequence collectively denoted
GEOMETRY (Gauge Exponential Omega Metric Even Tensor Running Yield).
\textsc{Geometry~I} is restricted to the static, equilibrium geometry and
derives an electroweak-anchored gravitational normalization from SM data alone.
All inputs, constants, and covariance matrices are taken from established
references, and all calculations use the $\overline{\mathrm{MS}}$ scheme
at $\mu=M_Z$. Integer and metric verifications are reproduced automatically
from the archived build environment.

Subsequent papers in the \textsc{Geometry} sequence extend this equilibrium
framework in distinct directions: the tensor-sector dynamics, mass gap, and
helicity frequency (\textsc{Geometry~II}); the time-dependent alignment,
drift-law evolution, and spectral structure (\textsc{Geometry~III}); the
alignment-regularized parabolic flow connecting to Navier--Stokes
(\textsc{Geometry~IV}); and the $\Pi$-weighted Hodge and cohomological
structures (\textsc{Geometry~V}). None of these dynamical elements are assumed
or required in \textsc{Geometry~I}.

Renormalization conventions follow
Weinberg~\cite{Weinberg1996_QFTv2},
Peskin and Schroeder~\cite{PeskinSchroeder1995_QFT},
and Langacker~\cite{Langacker2017_SMBeyond}.
Decoupling and integer-lattice methods follow
Appelquist and Carazzone~\cite{AppelquistCarazzone1975_Decoupling},
Kannan and Bachem~\cite{KannanBachem1979_SNF},
and Newman~\cite{Newman1997_SNF}.
Electroweak pins, covariance matrices, and physical constants are taken from
PDG and CODATA~\cite{PDG2024,PDG2024_EWReview,PDG2025_GaugeHiggs,CODATA2022_RMP}.
Two-loop RG coefficients follow
Machacek and Vaughn~\cite{Machacek1983_TwoLoopI,Machacek1984_TwoLoopII}
and Luo \textit{et al.}~\cite{Luo2003_TwoLoopSM},
and the running of $\hat\alpha$ follows
Jegerlehner~\cite{Jegerlehner2019_alphaRun}.
Gravitational and observational constraints follow
Carroll~\cite{Carroll2004_SG},
Will~\cite{Will2014_LRR_TestsGR},
Bertotti \textit{et al.}~\cite{Cassini2003_PPN},
and Abbott \textit{et al.} (LVK)~\cite{LVK2021_TestsGR}.
No additional data, fitting, or tuning is employed.

\begin{longtable}{l l p{0.52\textwidth}}
\caption{
Assumptions and first-principles spine of \textsc{Geometry~I}.
Entries A1--A11 specify the static, equilibrium, SM-internal framework.
Entries FP1--FP10 summarize the rigid integer, metric, parity, and anchor
structures used in \textsc{Geometry~I}, without invoking dynamical or spectral
extensions.
}
\label{tab:assumptions}\\[0.25em]
\hline\hline
\textbf{ID} & \textbf{Category} & \textbf{Assumption (\textsc{Geometry~I} scope)} \\
\hline
\endfirsthead

\multicolumn{3}{l}{\textit{Table \ref{tab:assumptions} (continued)}}\\[0.25em]
\hline\hline
\textbf{ID} & \textbf{Category} & \textbf{Assumption (\textsc{Geometry~I} scope)} \\
\hline
\endhead

\hline\hline
\multicolumn{3}{r}{\textit{Continued on next page}}\\
\endfoot

\hline\hline
\endlastfoot

A1 & Framework &
Static electroweak-scale equilibrium geometry at
$\mu=M_Z$ in the $\overline{\mathrm{MS}}$ scheme with
GUT-normalized hypercharge.
Only SM data and RG structure at this scale are used. \\[0.35em]

A2 & Fields / DoF &
$\hat\Psi=(\ln\hat\alpha_s,\ln\hat\alpha_2,\ln\hat\alpha)$ and
$\Xi=\chi\cdot\hat\Psi$ are internal gauge--log coordinates only.
\textsc{Geometry~I} introduces no new fields, potentials, or kinetic terms and does
not promote $\Xi$ to a dynamical scalar. \\[0.35em]

A3 & Equilibrium &
\textsc{Geometry~I} is static and equilibrium-restricted.
The gravitational normalization is defined at $\delta\Xi=0$, where
$\Pi(\Xi_{\rm eq})=1$ and the Einstein--Hilbert sector coincides with GR.
No time evolution or sourcing equation for $\Xi$ is assumed. \\[0.35em]

A4 & Metric &
The Fisher/kinetic metric $K\equiv K_{\rm eq}$ is defined locally at
$\mu=M_Z$ from one-loop SM sensitivity data.
It is positive definite ($K\succ0$) and used only as a local quadratic form on
log--coupling space. \\[0.35em]

A5 & Integer lattice &
The decoupling matrix $\Delta W$ is an exact integer matrix.
Its Smith normal form yields a unique primitive left kernel
$\chi=(16,13,2)$, invariant under unimodular integer transports.
All $\chi$-dependent quantities are treated as basis invariant. \\[0.35em]

A6 & Gate shape &
The curvature gate $\Pi(\Xi)$ is analytic and even about equilibrium, with
$\Pi(\Xi_{\rm eq})=1$ and $\Pi'(\Xi_{\rm eq})=0$.
Fisher matching fixes its curvature:
$-\tfrac12\Pi''(\Xi_{\rm eq})=F_\chi$, giving
$\sigma_\chi=F_\chi^{-1/2}$ as a derived quantity. \\[0.35em]

A7 & Tensor sector &
At $\delta\Xi=0$ the tensor kernel reduces to the GR Lichnerowicz operator with
$m_{\rm PF}=0$ and luminal propagation.
Even parity forbids any Brans--Dicke–like linear mixing between $\delta\Xi$ and
$h_{\mu\nu}$.
No extra scalar or vector propagating modes. \\[0.35em]

A8 & Dimensional anchor &
$G(M_Z)=(\hbar c/m_*^2)\Omega(M_Z)$ with
$m_*=m_p$ as the physical anchor.
The dimensionless anchor
$\Omega=\hat\alpha_s^{16}\hat\alpha_2^{13}\hat\alpha^{2}$
is SM-internal and independent of $m_*$. \\[0.35em]

A9 & Closure / data &
PDG/CODATA electroweak pins and two-loop running are taken as inputs.
The closure ratio $Z_G=\alpha_G^{(\rm pp)}/\Omega(M_Z)$ is used only as an
a posteriori diagnostic; $G_N$ never enters the construction of
$\Omega$, $\Pi(\Xi)$, or $\sigma_\chi$. \\[0.35em]

A10 & Perturbative stability &
The integer kernel, alignment angle ($\cos\theta_K\simeq1$), and Fisher
curvature $F_\chi$ are stable under permissible one-loop scheme and threshold
variations and are treated as robust SM features at $\mu=M_Z$. \\[0.35em]

A11 & Quasi-static regime &
Laboratory and weak-field environments satisfy $|\delta\Xi|\ll\sigma_\chi$.
Matter sources only small quasi-static displacements
$\delta\hat\Psi$, giving $\delta\Xi=\chi\cdot\delta\hat\Psi$.
The prediction $\Delta G/G=(\delta\Xi/\sigma_\chi)^2$ is strictly a
near-equilibrium statement. \\[0.6em]

\multicolumn{3}{l}{\textbf{First-Principles Spine (FP1--FP10) – \textsc{Geometry~I} compatibility}}\\
\hline

FP1 & SM scope &
No new fields, functions, or free parameters.
Construction is SM-internal at $\mu=M_Z$ and evaluated at $\delta\Xi=0$
where $\Pi=1$. \\[0.35em]

FP2 & Integer constraint &
$\chi=(16,13,2)$ is the unique primitive SNF kernel and the only admissible
depth direction.
No alternative scalar argument or deformation is allowed. \\[0.35em]

FP3 & Metric softness &
$K_{\rm eq}$ selects the same soft mode as $\chi$.
The Fisher curvature $F_\chi$ and width $\sigma_\chi$ are derived SM outputs,
not tunable parameters. \\[0.35em]

FP4 & Depth coordinate &
$\Xi=\chi\cdot\hat\Psi$ and $\delta\Xi$ are internal coordinates only and are
used algebraically inside $\Pi(\Xi)$, never as dynamical fields. \\[0.35em]

FP5 & Parity gate &
$\Pi(\Xi)$ is even and analytic with $\Pi=1$ and $\Pi'=0$ at equilibrium.
Fisher matching fixes the quadratic curvature coefficient. \\[0.35em]

FP6 & Dimensionless anchor &
$\Omega=e^{\Xi}=\hat\alpha_s^{16}\hat\alpha_2^{13}\hat\alpha^{2}$
is a purely SM-derived quantity and cannot be tuned. \\[0.35em]

FP7 & Dimensional lift &
$G(M_Z)=(\hbar c/m_*^2)\Omega(M_Z)$ with fixed $m_*$.
No new scale or degree of freedom is introduced. \\[0.35em]

FP8 & A posteriori closure &
$Z_G$ is an output-only diagnostic.
$G_N$ does not enter the construction of $\Omega$, $\Pi$, or $\sigma_\chi$. \\[0.35em]

FP9 & Equilibrium alignment structure &
The aligned contraction $\chi^{\mathsf T}K\,\partial^\mu\hat\Psi$ is an
equilibrium structure only.
Time-dependent conservation laws appear only in later work. \\[0.35em]

FP10 & Parity geometry &
Even $\Pi(\Xi)$ defines the curvature weight used in \textsc{Geometry~I}.
At $\delta\Xi=0$ this fixes all admissible near-equilibrium responses. \\[0.35em]

\end{longtable}

\paragraph*{Internal coordinates and degrees of freedom.}
Throughout this work the variables 
$\hat\Psi = (\ln\hat\alpha_s,\ln\hat\alpha_2,\ln\hat\alpha)$ 
and 
$\Xi = \chi\!\cdot\!\hat\Psi$ 
are treated as internal gauge--log coordinates, not as new propagating scalar fields. 
They summarize how the renormalized Standard Model couplings are organized along the 
integer direction $\chi=(16,13,2)$ in the $\overline{\mathrm{MS}}$ scheme at $\mu=M_Z$.  
The fundamental dynamical content of the theory remains that of the Standard Model plus 
General Relativity; no additional canonical coordinates or conjugate momenta are introduced.  
In particular, $\Xi$ has no kinetic term and no independent initial data in 
\textsc{Geometry~I}; it is a composite depth coordinate defined on the gauge--log manifold.

\paragraph*{Static displacements and sourcing.}
In the equilibrium setting considered here we allow for small, static displacements 
$\delta\Xi(x) \equiv \Xi(x)-\Xi_{\mathrm{eq}}$ of the internal depth coordinate, 
interpreted as the net aligned response of the gauge--log manifold to ordinary 
matter and energy.  
The detailed microscopic mapping from local stress--energy $T_{\mu\nu}(x)$ and 
Standard Model sources to $\delta\hat\Psi(x)$, and hence to $\delta\Xi(x)=\chi\!\cdot\!\delta\hat\Psi(x)$, 
is \emph{not} specified in \textsc{Geometry~I}.  
Instead, this work analyzes the consequences of such small displacements for the 
curvature response and for the SM-derived gravitational normalization $G(x)$.  
A dynamical description of how $\delta\Xi(x)$ is generated and relaxes, including an 
explicit sourcing relation, is developed in the time-dependent extension 
(\textsc{Geometry~III}).

\section{Integer lattice and the aligned depth coordinate}
\label{sec:lattice_depth}

We begin by identifying the structures that remain fixed once we restrict to
the Standard Model at $\mu=M_Z$ in the $\overline{\mathrm{MS}}$ scheme with no
new fields, tunable functions, or adjustable parameters. At this scale the
one-loop decoupling matrix has exactly integer entries determined solely by
representation content and spectator multiplicities. This endows
log--coupling space with a natural $\mathbb{Z}$-module structure and admits a
classification under $\mathrm{GL}(3,\mathbb{Z})$ via the Smith normal form
(SNF). Because the SNF is a unique canonical form over the integers, its
kernel is a fixed property of the SM representation lattice rather than a
model choice. The SNF is obtained using a standard integer-normal-form
algorithm (e.g. Kannan–Bachem), as implemented in \textsc{SymPy} and reproduced
verbatim in the public reproducibility repository. Since the SNF kernel is
invariant under all unimodular row and column operations, this structure is
basis-independent and scheme-consistent at one loop.

\paragraph*{Fixed structure.}
The integer kernel identified here is fully determined once the SM
representation content is specified. No phenomenological inputs, threshold
choices, or adjustable coefficients can alter the kernel or introduce
additional integer directions. In this sense the integer lattice provides a
representation-level invariant of the SM, and the existence of a unique
primitive kernel vector is a structural property rather than a modeling
decision.

Applying SNF to the SM one-loop decoupling matrix yields a unique primitive
left-kernel generator (up to overall sign),
\begin{equation*}
  \chi = (16,13,2),
  \label{eq:chi_def}
\end{equation*}
which is the sole integer direction annihilating the decoupling matrix. No
additional integer kernel vectors exist, and unimodular transformations
(integer determinant $\pm1$) cannot change the kernel rank or its primitive
representative. Thus $\chi$ is fixed by the SM’s integer structure and does
not depend on renormalization schemes, higher-loop corrections, numerical
fitting, or phenomenological input.

This uniqueness is crucial for the construction that follows: it implies that
any $\mathbb{Z}$-invariant combination of the three log--couplings must be a
function of $\Xi=\chi\cdot\hat\Psi$ alone. No second independent integer
constraint exists, and therefore no alternative depth coordinate compatible
with the SM integer structure can be defined.

Let the renormalized gauge couplings at $\mu=M_Z$ be
$\hat\alpha_s$, $\hat\alpha_2$, and $\hat\alpha$, and define
log--coupling coordinates
\begin{equation*}
  \hat\Psi = (\ln\hat\alpha_s,\,\ln\hat\alpha_2,\,\ln\hat\alpha),
\end{equation*}
together with the associated depth coordinate
\begin{equation*}
  \Xi = \chi\cdot\hat\Psi
  = 16\ln\hat\alpha_s + 13\ln\hat\alpha_2 + 2\ln\hat\alpha.
\end{equation*}

The coordinate vector $\hat\Psi$ is purely internal and is not associated with
any spacetime degree of freedom. Its role is to provide a basis-independent
description of how the gauge sector occupies the three-dimensional
log--coupling manifold at $\mu=M_Z$. All physical statements in this section
refer exclusively to this internal geometry; no dynamical assumptions are
introduced.


\paragraph*{Interpretation of $\Xi$.}
The scalar $\Xi$ defined in Eq.~\eqref{eq:xidepth} is an internal coordinate on
log--coupling space and is not introduced as a propagating, canonical, or
dynamical scalar field. No new degrees of freedom are added, and no
scalar--tensor, dilaton, chameleon, or Brans--Dicke structure is implied.
Throughout \textsc{Geometry~I}, $\Xi$ functions solely as an aligned internal
coordinate selected jointly by the integer lattice and the Fisher/kinetic
softness.

In particular, no kinetic term, potential, or equation of motion is assigned
to $\Xi$ in \textsc{Geometry~I}. Its sole purpose is to identify the unique
direction in gauge--log space that remains invariant under all unimodular
basis transforms and that later controls the curvature response through the
even gate $\Pi(\Xi)$ in a near-equilibrium expansion. Any appearance of
$\delta\Xi$ in near-equilibrium formulas (e.g.\ the quadratic lab-null) is
therefore an internal displacement, not a propagating field or a modification
of the gravitational sector.

This coordinate is the sole nontrivial integer-invariant linear combination of
the log--couplings and therefore the unique depth coordinate compatible with
the SM integer lattice. Under the stated constraints, any function of the
three couplings that respects integer invariance must reduce to a function of
$\Xi$ alone; this is a direct consequence of $\mathrm{GL}(3,\mathbb{Z})$
rigidity and does not depend on fitting, phenomenology, or model-specific
choices.

Exponentiating transports $\Xi$ back into the coupling manifold and defines
the dimensionless electroweak anchor
\begin{equation*}
\Omega \equiv e^{\Xi}
 = e^{\chi\cdot\hat\Psi}
 = e^{\,16\ln\hat\alpha_s + 13\ln\hat\alpha_2 + 2\ln\hat\alpha}
 = e^{\,\ln\big(\hat\alpha_s^{16}\hat\alpha_2^{13}\hat\alpha^{2}\big)}
 = \hat\alpha_s^{16}\,\hat\alpha_2^{13}\,\hat\alpha^{2},
\label{eq:Omega_def}
\end{equation*}

Because the log--couplings appear additively in $\Xi$, the anchor
$\Omega=e^\Xi$ is the unique multiplicative quantity that respects both the
integer lattice and the RG scaling properties of the SM couplings. No
additional powers, coefficients, or normalization choices are introduced or
required. Exponentiation is the natural map from log--coupling space to the
multiplicative coupling manifold, making $\Omega$ the uniquely associated
dimensionless quantity determined by the integer depth coordinate.

Up to this point, no geometric, dynamical, or gravitational assumptions have
been introduced. Equation~\eqref{eq:Omega_def} follows directly from the SM
representation content and the existence of a unique primitive integer kernel,
verified in a scheme-consistent, integer-preserving manner. The next section
shows that the same direction $\Xi$ is independently selected by the soft
eigenmode of the Fisher/kinetic metric, establishing that $\Xi$ is not only
algebraically admissible but also physically responsive and maximally
sensitive.


\paragraph*{Summary.}
Section~\ref{sec:lattice_depth} therefore identifies a single admissible
internal coordinate $\Xi$ dictated entirely by SM representation content and
basis-invariant integer structure. No gravitational, geometric, or dynamical
assumptions enter here. The next section shows that the same direction is
selected independently by the Fisher/kinetic metric, establishing that $\Xi$
is not only algebraically fixed but also physically privileged by the SM’s
sensitivity structure.



\section{Metric softness and alignment}
\label{sec:metric_softness}

The integer-aligned depth coordinate $\Xi$ identified in
Section~\ref{sec:lattice_depth} is fixed by the SM representation lattice and
is unique under $\mathrm{GL}(3,\mathbb{Z})$ invariance. We now show that the
same coordinate is independently selected by the geometric softness of the SM
gauge sector, quantified by a Fisher/kinetic metric constructed from the
one-loop sensitivity of the renormalization-group (RG) flow at $\mu=M_Z$ in
the $\overline{\mathrm{MS}}$ scheme, using the same inputs and electroweak pins
as in Section~\ref{sec:intro}. The metric is not an additional structure; it is
derived directly from the SM $\beta$ functions and introduces no new degrees of
freedom or tunable functions.

\paragraph*{Status of the Fisher/kinetic metric.}
The metric $K$ used here is a derived geometric object obtained entirely from
the SM RG flow. It is not a free ansatz or phenomenological fit. The entries of
$K$ are fixed once the $\beta$ functions and the renormalized couplings at
$\mu=M_Z$ are specified. No new fields, potentials, or dynamical assumptions
enter this construction.

Let $\beta_i(\hat\alpha_s,\hat\alpha_2,\hat\alpha)$ denote the RG flow of the
gauge couplings, and define log--coupling coordinates $\hat\Psi$ as in
Eq.~\eqref{eq:Psi_def}. Following standard practice, the equilibrium
Fisher/kinetic metric on log--coupling space is defined by
\begin{equation}
  K_{ij}
  =
  \frac{\partial}{\partial\hat\Psi_j}
  \!\left(\frac{\beta_i}{\hat\alpha_i}\right)_{\!\rm eq},
  \label{eq:metric_def}
\end{equation}
with all quantities evaluated at $\mu=M_Z$. The symmetric matrix $K$ is
positive definite ($K\succ0$), so it admits three orthonormal eigenvectors with
strictly positive eigenvalues. Large eigenvalues correspond to stiff RG
response, and small eigenvalues correspond to soft response. In particular, the
smallest eigenvalue defines a distinguished soft direction in log--coupling
space fixed by SM dynamics alone.


\paragraph*{Internal geometric role.}
The metric $K$ acts only on the internal log--coupling coordinates $\hat\Psi$,
and therefore its eigenvectors describe relative RG responsiveness, not
spacetime propagation or physical particles. In \textsc{Geometry~I} this
internal geometry serves only to identify the softest internal direction. No
kinetic terms, dynamical fields, or equations of motion are associated with $K$
or its eigenvectors.

Let $e_\chi$ denote the unit eigenvector associated with the smallest
eigenvalue of $K$. By direct computation, the primitive integer kernel vector
$\chi$ from Eq.~\eqref{eq:chi_def} is aligned with $e_\chi$ to numerical
precision. Writing
\begin{equation}
  \hat\chi = \frac{\chi}{\|\chi\|}, \qquad
  \cos\theta \equiv \hat\chi\cdot e_\chi,
  \label{eq:alignment_def}
\end{equation}
we obtain
\begin{equation}
  \cos\theta = 1 - \varepsilon_\chi,
  \qquad
  \varepsilon_\chi \lesssim 10^{-8},
\end{equation}
in our numerical evaluation.\footnote{This computation uses the one-loop
$\beta$ functions, electroweak pins, and physical constants cited in
Section~\ref{sec:intro}, with all intermediate values and eigenvectors
generated reproducibly in the archived build workflow.}


\paragraph*{Interpretation of alignment.}
This agreement is not imposed but follows from two independent SM structures:
(i) the integer lattice determined by representation content, and (ii) the
Fisher/kinetic response geometry of the RG flow. Their numerical concurrence
identifies $\Xi$ as both the unique integer-invariant depth coordinate and the
maximally responsive (softest) internal direction. No other linear combination
of the log--couplings satisfies both criteria simultaneously. Alignment is
therefore not a tuned property but a structural consequence of combining two
fixed SM ingredients. Because these ingredients originate from unrelated parts
of the SM (representation theory versus differential response), their
concurrence constitutes a geometric rigidity.


\paragraph*{Aligned displacement.}
To parameterize displacements along the soft direction, define the
Euclidean-normalized aligned vector $\hat\chi=\chi/\|\chi\|$ and write
\begin{equation}
  \delta\Xi
  =
  \hat\chi\cdot(\hat\Psi - \hat\Psi_{\rm eq}),
  \label{eq:deltaXi_def}
\end{equation}
where $\hat\Psi_{\rm eq}$ is evaluated at $\mu=M_Z$. The scalar $\delta\Xi$
therefore measures displacement strictly along the soft direction of $K$, while
transverse components are suppressed by larger eigenvalues. In \textsc{Geometry~I} this
quantity has no dynamical interpretation: it is an internal displacement
variable used to label departures from equilibrium, not a propagating field or
a modification of GR. Its sole appearance is in near-equilibrium response
formulas such as the quadratic lab-null.

Because $\Xi$ is simultaneously the unique integer-invariant depth coordinate,
any admissible curvature or response function consistent with both integer
rigidity and metric softness must depend only on $\delta\Xi$ under the stated
SM-only assumptions. No transverse combination can contribute without violating
either $\mathrm{GL}(3,\mathbb{Z})$ rigidity or the eigenvalue ordering of $K$.


\paragraph*{Irreversibility.}
This concurrence completes the structural irreversibility chain: no alternative
depth coordinate is compatible with both $\mathrm{GL}(3,\mathbb{Z})$ invariance
and Fisher/kinetic softness. Any attempt to modify the direction would either
contradict the integer lattice or select a stiffer direction forbidden by the
eigenvalue ordering. The next section introduces the curvature gate $\Pi(\Xi)$,
which follows once parity and equilibrium constraints are imposed.









\section{Curvature gate and parity constraint}
\label{sec:curvature_gate}

Since the aligned depth coordinate $\Xi$ is simultaneously the unique
$\mathrm{GL}(3,\mathbb{Z})$-invariant depth direction
(Section~\ref{sec:lattice_depth}) and the Fisher/kinetic soft eigenmode
(Section~\ref{sec:metric_softness}), any admissible curvature response
consistent with the stated SM-only internal constraints must depend only on the
scalar displacement $\delta\Xi$ of Eq.~\eqref{eq:deltaXi_def}. In this section
we show that the curvature response function $\Pi(\Xi)$ is fixed, up to overall
normalization, by equilibrium, parity, and Fisher curvature, and that these
conditions select an even Gaussian with no tunable parameters under the stated
assumptions. Throughout, $\Pi$ is a function of the internal coordinate $\Xi$
alone and does not represent a propagating scalar, auxiliary field, or
additional dynamical degree of freedom.

\paragraph*{Internal role of the gate.}
The role of $\Pi(\Xi)$ in \textsc{Geometry~I} is purely that of an internal curvature
weighting factor applied to the Einstein--Hilbert term. It does not carry its
own kinetic term, potential, or source, and it is never promoted to a dynamical
scalar--tensor degree of freedom. All departures from equilibrium are described
by the internal displacement $\delta\Xi$ and the induced change in the weight
$\Pi(\Xi)$; the underlying GR tensor sector remains that of a massless
helicity-$\pm2$ field.

We consider a multiplicative scalar gate applied to the Einstein--Hilbert term:
\begin{equation}
  \mathcal{L}_{\rm eff}
  =
  \frac{1}{16\pi G(M_Z)}\,\Pi(\Xi)\,R,
  \label{eq:Leff_gate}
\end{equation}
where $G(M_Z)$ is defined in Eq.~\eqref{eq:gmz}, and no new fields, mass
terms, or independent kinetic terms are introduced. The gate must satisfy:
\begin{itemize}
  \item[(i)] \textbf{Equilibrium normalization:}
             $\Pi(\Xi_{\rm eq})=1$ so that GR is recovered at equilibrium.
  \item[(ii)] \textbf{Parity preservation:}
             $\Pi(\Xi_{\rm eq}+\delta\Xi)=\Pi(\Xi_{\rm eq}-\delta\Xi)$,
             forbidding odd powers of $\delta\Xi$ and ensuring a massless
             helicity-$\pm2$ tensor sector. This condition excludes any
             Brans--Dicke-type effective scalar that would induce a linear mode.
  \item[(iii)] \textbf{Curvature matching:}
             the second derivative $\Pi''(\Xi_{\rm eq})$ must reproduce the
             Fisher curvature along the aligned direction, ensuring that
             departures from equilibrium are penalized with the same softness
             scale as the RG geometry.
  \item[(iv)] \textbf{Analytic minimality:}
             no additional coefficients, tunable parameters, or non-analytic
             completions are permitted.
\end{itemize}

These conditions encode the requirement that the curvature response is
completely determined by SM structure and the Fisher softness along the aligned
direction. In particular, (ii) and (iii) ensure that no linear scalar mode
appears and that the unique softness scale $\sigma_\chi$ extracted from $K$
also fixes the curvature of $\Pi(\Xi)$, preventing any ad hoc adjustment of the
response.

\subsection*{Lemma 1 (Parity restriction)}
Under (i)–(ii), the Taylor expansion of $\Pi$ about $\Xi_{\rm eq}$ is
\begin{equation}
  \Pi(\Xi)
  =
  1
  + \frac{1}{2}\,\Pi''(\Xi_{\rm eq})\,(\delta\Xi)^2
  + \mathcal{O}\!\big((\delta\Xi)^4\big),
  \label{eq:Pi_expansion}
\end{equation}
with all odd powers forbidden, so any departure from equilibrium begins at
quadratic order.

\subsection*{Lemma 2 (Fisher curvature matching)}
Along the aligned direction $\hat\chi$,
\begin{equation}
  \sigma_\chi^2 \equiv \frac{1}{\hat\chi^\top K\,\hat\chi},
  \label{eq:sigma_chi_def}
\end{equation}
and consistency with Fisher softness requires
\begin{equation}
  \Pi''(\Xi_{\rm eq})
  =
  -\frac{2}{\sigma_\chi^2}.
  \label{eq:Pi_curvature}
\end{equation}

\smallskip
\noindent\emph{Proof sketch.}
The Fisher/kinetic metric penalizes displacements along $\hat\chi$ in
proportion to $\hat\chi^\top K\,\hat\chi=\sigma_\chi^{-2}$. Matching this
penalty to the quadratic response term in Eq.~\eqref{eq:Pi_expansion} and
enforcing stability fixes both the curvature scale and sign, ensuring that
$\Pi$ decreases away from equilibrium and thus preserves the GR tensor limit.
The quantity $\sigma_\chi$ is therefore derived directly from the SM Fisher
geometry and is not an adjustable softness parameter.

\subsection*{Theorem (Uniqueness of the curvature gate)}
Under assumptions (i)–(iv), the unique analytic, parity-even,
curvature-matched response consistent with the stated SM-only constraints is
\begin{equation}
  \boxed{
  \Pi(\Xi)
  =
  \exp\!\left[-\,\frac{(\delta\Xi)^2}{\sigma_\chi^2}\right]
  }
  \label{eq:Pi_gaussian}
\end{equation}
with no tunable parameters and no dependence on additional fields or auxiliary
potentials.

\paragraph*{Remark on analytic completion.}
Polynomial completions of Eq.~\eqref{eq:Pi_expansion} introduce undetermined
higher-order coefficients that are neither fixed by parity nor by Fisher
curvature. Exponentiation provides an analytic completion with a single
dimensionless scale $\sigma_\chi$ fixed by Eq.~\eqref{eq:sigma_chi_def},
consistent with equilibrium normalization, even parity, curvature matching, and
the absence of tunable parameters. Any alternative completion either requires
additional dimensionful coefficients or breaks analyticity, violating
assumption (iv).

\smallskip
\noindent\emph{Proof sketch.}
Equation~\eqref{eq:Pi_expansion} fixes the quadratic coefficient; parity
enforces evenness, equilibrium fixes normalization, and analytic minimality
completes the response via exponentiation of the quadratic form.
Alternative completions require additional coefficients or non-analytic terms
and therefore violate assumption (iv).

\paragraph*{Status within \textsc{Geometry~I}.}
Within \textsc{Geometry~I}, $\Pi(\Xi)$ should thus be interpreted solely as an internal
curvature gate multiplying the Einstein--Hilbert term. It does not alter the
tensorial structure of the field equations, and at equilibrium $\delta\Xi=0$
one has $\Pi(\Xi_{\rm eq})=1$ so that the Einstein--Hilbert sector coincides
exactly with GR. Departures from equilibrium are encoded only through the
internal displacement $\delta\Xi$ and the corresponding weighting of $G(M_Z)$.

\subsection*{Internal curvature weighting}
Substituting Eq.~\eqref{eq:Pi_gaussian} into Eq.~\eqref{eq:Leff_gate} yields
a convenient notation for the curvature-gated normalization,
\begin{equation}
  G(x)
  =
  G(M_Z)\,\Pi(\Xi(x))
  =
  \frac{\hbar c}{m_p^2}\,\Omega(M_Z)\,
  \exp\!\left[-\frac{(\delta\Xi(x))^2}{\sigma_\chi^2}\right],
  \label{eq:running_G}
\end{equation}
which preserves the equilibrium tensor sector and introduces no new fields or
adjustable parameters. Here $G(x)$ is a shorthand for the internal,
SM-determined curvature weight $G(M_Z)\Pi(\Xi(x))$; it is not a free function
of spacetime, not a dynamical scalar field, and not a varying-$G$ theory. Near
equilibrium, the strict quadratic prediction
\begin{equation}
  \frac{\Delta G}{G}
  =
  \left(\frac{\delta\Xi}{\sigma_\chi}\right)^{2}
  + \mathcal{O}\!\big((\delta\Xi)^4\big)
\end{equation}
follows immediately.

\paragraph*{Interpretation of $G(x)$.}
The notation $G(x)$ is used here as a compact way to encode the internal,
curvature-gated normalization $G(M_Z)\Pi(\Xi(x))$ induced by the SM gauge
sector. It does not represent an independent dynamical degree of freedom in
the gravitational sector. At $\delta\Xi=0$ one has $G(x)=G(M_Z)$ and the
Einstein--Hilbert action reduces to its standard GR form. The prediction
$\Delta G/G=(\delta\Xi/\sigma_\chi)^2$ should therefore be interpreted as a
near-equilibrium laboratory null relation controlled by internal SM geometry,
rather than as a cosmological varying-$G$ scenario.





\section{Electroweak anchor and SM-derived gravitational coupling}
\label{sec:SM_derived_G}

With $\Xi$ uniquely fixed by the integer lattice and Fisher/kinetic softness,
and with $\Pi(\Xi)$ determined by equilibrium, parity, and curvature matching,
we now connect these internal structures to the gravitational normalization
multiplying the Einstein--Hilbert term. No new inputs, parameters, or external
assumptions are introduced in this section; all quantities are Standard Model
objects evaluated at $\mu=M_Z$ in the $\overline{\mathrm{MS}}$ scheme. The
construction defines an electroweak-anchored normalization, not a time- or
space-varying gravitational constant and not a modification of GR.

\paragraph*{Internal role of the anchor.}
The scalar $\Omega$ functions as an electroweak-scale anchor relating a
dimensionless SM quantity to a dimensionful normalization of the Einstein--Hilbert term.
It is not a dynamical scalar, carries no equation of motion, and introduces no
extra degrees of freedom. Its significance stems entirely from the uniquely
admissible exponentiation of the aligned depth $\Xi$, and all curvature
responses arise through the gate $\Pi(\Xi)$ rather than through any propagation
or dynamics associated with $\Omega$.

Exponentiating the aligned depth coordinate transports $\Xi$ back into
gauge--coupling space and defines the dimensionless electroweak anchor
\begin{equation*}
  \Omega \equiv e^{\Xi}
  = \hat\alpha_s^{16}\,\hat\alpha_2^{13}\,\hat\alpha^{2},
  \label{eq:Omega_recall}
\end{equation*}
where hatted quantities denote $\overline{\mathrm{MS}}$ couplings at
$\mu=M_Z$. Equation~\eqref{eq:Omega_recall} is the unique exponentiation of the
integer-invariant depth scalar $\Xi$ and introduces no free coefficients or
additional scales. $\Omega$ is therefore a pure, dimensionless SM construct
fixed entirely by measured electroweak-scale couplings.


\paragraph*{Dimensional consistency and the role of $m_p$.}
Because $\Omega$ is dimensionless, the only admissible conversion to a
gravitational normalization without introducing new parameters is supplied by
dimensional analysis in Planck units:
\begin{equation*}
  G(M_Z)
  \equiv
  \frac{\hbar c}{m_p^{2}}\,\Omega(M_Z),
  \label{eq:G_MZ_def}
\end{equation*}
which defines the normalization appearing in Eq.~\eqref{eq:Leff_gate}. No
phenomenological value of $G_N$ is used here, and no tuning or matching step is
performed. Under the stated internal constraints, Eq.~\eqref{eq:G_MZ_def}
follows directly from dimensional consistency, integer invariance, and the
absence of additional scales. The choice of $m_p$ does not introduce a free
parameter: any Standard Model mass scale can serve as the dimensional anchor,
and all choices differ only by fixed SM mass ratios. The proton mass is chosen
for phenomenological relevance, as laboratory and astrophysical determinations
of $G_N$ predominantly involve baryonic matter.

\medskip
\noindent\textbf{Theorem (Electroweak anchoring of gravitational normalization).}
\emph{Under SM-only constraints at $\mu=M_Z$, with no new fields, tunable
functions, or adjustable parameters, the effective gravitational normalization
is fixed by Eq.~\eqref{eq:G_MZ_def}. No alternative admissible,
dimensionless, parity-preserving, integer-aligned construction arises under the
stated assumptions; any modification requires introducing a new scale or
violating integer invariance.}

\medskip
\noindent\emph{Proof.}
Equation~\eqref{eq:Omega_recall} is the unique exponentiation of the primitive
integer-invariant scalar $\Xi$. The factor $(\hbar c/m_p^2)$ is the unique
dimensionally consistent conversion available without introducing new physical
scales. Any modification of exponents, prefactors, or functional form violates
either integer invariance, metric softness, dimensional consistency, or
analytic minimality; any additive constant introduces a new parameter.
\hfill$\square$

\bigskip

\paragraph*{Notation for expansion coefficients.}
For later empirical interpretation it is useful to write the fractional
curvature-weight response as a Taylor expansion about equilibrium,
\begin{equation}
  \frac{\Delta G}{G}
  =
  A\,\delta\Xi
  +
  B\,(\delta\Xi)^2
  +
  \mathcal{O}\!\big((\delta\Xi)^3\big),
  \label{eq:AB_def}
\end{equation}
where $A$ and $B$ are not tunable parameters but the fixed Taylor coefficients
implied by the curvature gate $\Pi(\Xi)$ under the assumptions of
Section~\ref{sec:curvature_gate}. Parity and equilibrium normalization require
\begin{equation}
  A = 0,
  \qquad
  B = \frac{1}{\sigma_\chi^2},
  \label{eq:AB_values}
\end{equation}
so the first nonzero deviation from equilibrium is strictly quadratic with a
unit-normalized curvature scale when expressed in the dimensionless coordinate
$s \equiv \delta\Xi/\sigma_\chi$. Therefore
\begin{equation}
  \frac{\Delta G}{G}
  =
  s^{2}
  +
  \mathcal{O}(s^{4})
  ,
  \qquad
  s \equiv \frac{\delta\Xi}{\sigma_\chi},
  \label{eq:lab_null_s}
\end{equation}


\paragraph*{Status of the quadratic null.}
The vanishing of the linear term is a direct consequence of parity and cannot
be altered without introducing a new field or violating the SM-aligned
assumptions. The quadratic coefficient $B=1/\sigma_\chi^2$ is fixed by Fisher
curvature and is therefore experimentally meaningful: any observed linear
response or deviation from quadratic scaling immediately falsifies the
framework.

\bigskip
\noindent\textbf{Corollary (Internal curvature weighting).}
\emph{With the curvature gate of Eq.~\eqref{eq:Pi_gaussian}, the internal
curvature-weighted normalization takes the form}
\begin{equation*}
  G(x)
  =
  G(M_Z)\,\Pi(\Xi(x))
  =
  \frac{\hbar c}{m_p^2}\,
  \Omega(M_Z)\,
  \exp\!\left[-\frac{(\delta\Xi(x))^2}{\sigma_\chi^2}\right].
  \label{eq:G_running_final}
\end{equation*}

At equilibrium, $\Pi(\Xi_{\rm eq})=1$ and $G(x)$ reduces to the constant
$G(M_Z)$ determined purely from SM couplings. Away from equilibrium the
response is fixed and strictly quadratic:
\begin{equation}
  \frac{\Delta G}{G}
  =
  \left(\frac{\delta\Xi}{\sigma_\chi}\right)^{2}
  + \mathcal{O}\!\big((\delta\Xi)^4\big),
  \label{eq:quadratic_prediction}
\end{equation}
with no linear term, no tunable scale, and no phenomenological matching
coefficients. Equation~\eqref{eq:quadratic_prediction} is an immediate,
laboratory-accessible falsifier rather than a fitting ansatz.


\paragraph*{GR compatibility at equilibrium.}
At $\Xi=\Xi_{\rm eq}$ one has $\Pi(\Xi_{\rm eq})=1$, so the Einstein--Hilbert
action, field equations, diffeomorphism symmetry, and the massless luminal
helicity-$\pm2$ tensor sector are identical to those of GR. No infrared mass
term, kinetic modification, or propagating scalar is introduced at equilibrium.
The normalization $G(M_Z)$ is thus an internally derived electroweak anchor and
not a modification of GR; it parameterizes the SM-aligned curvature weight
multiplying the Einstein--Hilbert term, with GR fully recovered at
$\delta\Xi=0$.







\section{Predictions, closure, and falsifiers}
\label{sec:predictions}

All empirical consequences follow directly from the fixed Standard Model
constraints established in
Sections~\ref{sec:lattice_depth}--\ref{sec:SM_derived_G}.
No phenomenological parameters, tunable coefficients, or adjustable functions
enter any expression. The empirical status of the construction is therefore
decided by direct tests of the statements below; violation of \emph{any} item
falsifies the framework under its stated assumptions. Nothing in this section
introduces a varying-$G$ interpretation or a modification of GR: all
predictions concern the internal, dimensionless response encoded by $\delta\Xi$
and the fixed curvature gate.

\subsection*{Fixed predictions}

\begin{enumerate}
\item \textbf{Electroweak anchoring of gravitational normalization}
\begin{equation*}
  G(M_Z)
  =
  \frac{\hbar c}{m_p^2}\,\Omega(M_Z),
  \qquad
  \Omega(M_Z)=\hat\alpha_s^{16}\hat\alpha_2^{13}\hat\alpha^{2}.
\end{equation*}
No external $G_N$ value is inserted; $G(M_Z)$ is fixed entirely from SM inputs.
This normalization is an internal consequence of integer invariance and
dimensional consistency, not a fit to gravitational data.
\vspace{4pt}

\item \textbf{Even curvature gate and quadratic response}
\begin{equation*}
  \Pi(\Xi)
  =
  \exp\!\left[-\,\frac{(\delta\Xi)^2}{\sigma_\chi^2}\right],
  \qquad
  \frac{\Delta G}{G}
  =
  \left(\frac{\delta\Xi}{\sigma_\chi}\right)^{2}
  =
  \left(\frac{\phi_\chi}{\Lambda_\chi}\right)^{2},
\end{equation*}
so the first nonzero departure from equilibrium is strictly quadratic. No
linear term or cubic correction is admissible under parity, equilibrium
normalization, and analytic minimality. The quadratic coefficient is fixed by
Fisher curvature and cannot be adjusted without violating the assumptions of
Section~\ref{sec:curvature_gate}.
\vspace{4pt}

\begin{figure}[t]
  \centering
  \includegraphics[width=\columnwidth]{tabs_and_figs/fig_gate.pdf}
  \caption{Even curvature gate $\Pi(\Xi)$ and quadratic parity-null prediction
  $\Delta G/G = s^{2}$ on the normalized depth coordinate
  $s=\delta\Xi/\sigma_\chi=\phi_\chi/\Lambda_\chi$.
  Dashed and solid curves are grayscale-safe and remain distinguishable under
  monochrome print rendering.}
  \label{fig:gate}
\end{figure}

\item \textbf{Fixed curvature scale}
\begin{equation}
  \sigma_\chi^2 = (\hat\chi^\top K \hat\chi)^{-1},
  \qquad
  \sigma_\chi \simeq 247.683,
  \qquad
  \|\chi\|_K \simeq 17.6278,
  \qquad
  \Lambda_\chi \equiv \frac{\sigma_\chi}{\|\chi\|_K} \simeq 14.0507.
\end{equation}
The curvature scale is determined solely by the Fisher/kinetic metric along
$\hat\chi$; no adjustable scale enters $\Pi(\Xi)$, and no phenomenological
normalization is allowed.
\end{enumerate}

\subsection*{Numerical illustration (PDG/CODATA inputs)}

Using current PDG and CODATA pins at $\mu=M_Z$ in the
$\overline{\mathrm{MS}}$ scheme, the aligned Fisher curvature and gate width are
\begin{equation}
  F_\chi \equiv \frac{1}{\sigma_\chi^2} \simeq 1.629\times10^{-5},
  \qquad
  \sigma_\chi \simeq 247.683.
\end{equation}
The electroweak anchor and proton--proton gravitational coupling are
\begin{equation}
  \Omega(M_Z) \simeq 6.4597\times10^{-39},
  \qquad
  \alpha_G^{(\mathrm{pp})} \simeq 5.9061\times10^{-39},
\end{equation}
yielding a closure ratio
\begin{equation}
  Z_G \equiv \frac{\alpha_G^{(\mathrm{pp})}}{\Omega(M_Z)} \simeq 0.9143,
\end{equation}
which represents an order-$10\%$ deviation without any form of tuning and is
interpreted solely as an \emph{a posteriori} consistency check arising from a
fixed, parameter-free construction rather than an input, matching criterion, or
fitted quantity. The closure ratio is not used to calibrate or adjust the
framework.

\medskip
\noindent\textbf{Leave-one-out (LOO) forecast.}
Holding $\hat\alpha_2$ and $\hat\alpha$ fixed, the implied strong coupling is
\begin{equation}
  \hat\alpha_s^{\star}(M_Z)
  =
  \left[
    \frac{\alpha_G^{(\mathrm{pp})}}{\hat\alpha_2^{13}\hat\alpha^{2}}
  \right]^{1/16}
  =
  0.1173411 \pm 1.86\times10^{-5},
\end{equation}
which lies in $\simeq 0.7\sigma$ agreement with the PDG world average. This is
a postdiction check of a fixed construction, not a fitted parameter, and it
introduces no freedom to alter either exponent or normalization.

\medskip
With current pinned inputs, both the $\sim9\%$ closure deviation and the
$\sim0.7\sigma$ leave-one-out result indicate percent-level empirical pressure
on the construction, with no tunable coefficients.

\subsection*{Quasi-static sourcing of \texorpdfstring{$\delta\Xi$}{δΞ}}

In \textsc{Geometry~I} no new propagating scalar is introduced and $\Xi$ is not a
dynamical field. Nevertheless, ordinary matter perturbs the renormalized SM
gauge couplings through their standard quasi-static dependence on local
stress--energy. These perturbations appear as small displacements in
log--coupling space,
\[
  \delta\hat\Psi \equiv
  (\delta\ln\hat\alpha_s,\;\delta\ln\hat\alpha_2,\;\delta\ln\hat\alpha),
\]
conceptually generated by the response of the SM sector to local energy density
and pressure. Since the aligned depth is defined algebraically by
$\Xi=\chi\cdot\hat\Psi$, these perturbations induce
\[
  \delta\Xi = \chi\cdot\delta\hat\Psi .
\]

The quasi-static regime relevant for laboratory and weak-field tests corresponds
to $|\delta\Xi|\ll\sigma_\chi$, where the curvature gate may be expanded about
equilibrium. With
\[
  \Pi(\Xi)=\exp\!\left[-\left(\frac{\delta\Xi}{\sigma_\chi}\right)^2\right],
  \qquad
  \Pi(\Xi_{\rm eq})=1,\quad \Pi'(\Xi_{\rm eq})=0,
\]
the first nonvanishing departure from equilibrium is fixed to be quadratic in
$\delta\Xi$. Because $\Pi$ is even and $\Pi'(\Xi_{\rm eq})=0$, no linear term
appears, and the leading observable deviation of the SM-aligned curvature
weight scales as $(\delta\Xi/\sigma_\chi)^2$.

This yields the \textsc{Geometry~I} laboratory prediction
\[
  \frac{\Delta G}{G}
  = \frac{\Pi(\Xi)-1}{1}
  = \left(\frac{\delta\Xi}{\sigma_\chi}\right)^2
  = \left(\frac{\phi_\chi}{\Lambda_\chi}\right)^2,
\]
where $\phi_\chi$ denotes the metric-projected displacement along $\chi$ and
$\Lambda_\chi$ its fixed curvature scale. The quadratic form is interpreted
purely as the response of the SM-aligned curvature weight to quasi-static,
matter-induced displacements. A dynamical law for $\Xi(x)$ and explicit
stress--energy sourcing of $\delta\Xi$ are developed in \textsc{Geometry~III};
\textsc{Geometry~I} remains strictly equilibrium and quasi-static.


\subsection*{Empirical falsifiers}

\begin{enumerate}
\item \textbf{No linear term}
\begin{equation}
  \left.\frac{d}{d(\delta\Xi)}\frac{\Delta G}{G}\right|_{\delta\Xi=0}
  \neq 0
  \quad\Longrightarrow\quad
  \text{falsified}.
\end{equation}
Any detectable linear dependence violates parity, equilibrium normalization,
and the absence of propagating scalars under the assumptions of
Sections~\ref{sec:curvature_gate}--\ref{sec:SM_derived_G}.

\item \textbf{Quadratic coefficient fixed ($B=1$ in $s$-units)}
\begin{equation}
  \frac{\Delta G}{G}
  =
  1\cdot\left(\frac{\delta\Xi}{\sigma_\chi}\right)^{2}
  + \mathcal{O}\!\big((\delta\Xi)^4\big),
\end{equation}
so in the dimensionless coordinate $s=\delta\Xi/\sigma_\chi$ the leading
coefficient is fixed to $1$. Any measurable deviation in this leading quadratic
coefficient falsifies. No alternative curvature scale or normalization can be
introduced without violating Section~\ref{sec:curvature_gate}.

\item \textbf{Closure ratio consistency}
\begin{equation}
  Z_G \equiv \frac{\alpha_G^{(\mathrm{pp})}}{\Omega(M_Z)}
  \quad\text{must be statistically consistent with } 1 \text{ (within uncertainties).}
\end{equation}
Persistent, statistically significant disagreement falsifies the proposed
SM-internal normalization mechanism; this is a consistency test of a fixed
prediction, not a calibration step.

\item \textbf{Tensor-sector preservation (equilibrium)}
Helicity-$\pm2$ modes must remain massless and luminal at equilibrium. Any
effective mass or kinetic deformation at $\Xi_{\rm eq}$ falsifies the gate
construction under the stated GR-compatibility assumptions.

\item \textbf{Alignment stability}
\begin{equation}
  \cos\theta \simeq 1,
  \qquad
  \text{(integer projector $\chi$ aligned with soft eigenmode $e_\chi$).}
\end{equation}
Sustained misalignment falsifies. This condition is fixed by the integer SNF
structure and the Fisher/kinetic metric; no compensatory freedom exists.
\end{enumerate}

\subsection*{Sufficiency}

The construction is falsified if \emph{any one} of items (1)–(5) above fails;
no auxiliary assumptions, retuning, replacement coefficients, or post hoc
adjustments are permitted. These falsifiers exhaust all degrees of freedom
available under the SM-only constraints and apply to the SM-internal
normalization mechanism without modifying GR at equilibrium.



\section{Discussion and consequences}
\label{sec:discussion}

Sections~\ref{sec:lattice_depth}--\ref{sec:predictions} show that, under
Standard-Model-only constraints at $\mu=M_Z$ in the $\overline{\mathrm{MS}}$
scheme, a gravitational normalization can be defined without introducing new
fields, tunable functions, or free parameters. The construction does
\emph{not} modify General Relativity (GR) or propose an alternative
gravitational theory; rather, it identifies an internally determined
normalization for the Einstein--Hilbert term arising from fixed gauge-sector
structure. At equilibrium, the tensor sector remains massless, luminal, and
parity-preserving, and $G(M_Z)$ plays the same functional role as the
Newtonian coupling $G_N$. No effective modification of the Einstein field
equations occurs at $\Xi=\Xi_{\rm eq}$.

\paragraph*{Scope of the claim.}
The results are existence and consistency statements. The framework does not
claim uniqueness of all possible gravity theories; it shows only that the
Standard Model gauge sector \emph{already contains} sufficient fixed structure
to provide an electroweak-anchored normalization for the Einstein--Hilbert
term, together with a specific parity-even curvature response, all without
enlarging the field content or altering the tensor equations at equilibrium.

This reframes the gauge--gravity interface: instead of treating $G_N$ as an
externally supplied empirical parameter, the Standard Model supplies an
internally fixed anchor. The mechanism rests on three independently determined
SM structures:
(i)~the unique primitive integer kernel $\chi=(16,13,2)$ from the one-loop
decoupling lattice,
(ii)~the soft eigenmode of the positive-definite Fisher/kinetic metric $K$, and
(iii)~the uniquely determined, parity-even curvature gate $\Pi(\Xi)$.
None introduce model freedom; each follows from established SM representation
content and one-loop RG sensitivity data, with all quantities generated
reproducibly from public inputs. Together these form a closed alignment chain:
\[
\text{integer rigidity}
\;\rightarrow\;
\text{metric softness}
\;\rightarrow\;
\text{even curvature response}.
\]

To avoid misinterpretation, we emphasize that $\Xi$ is an internal aligned
coordinate in log--coupling space and is \emph{not} introduced as a dynamical,
canonical, or propagating scalar field. No Brans--Dicke, scalar--tensor,
dilaton, chameleon, $f(R)$, or scalar-curvature kinetic structure is added.
Likewise, $\Pi(\Xi)$ is not a potential, not a Lagrangian degree of freedom,
and not a new mediator field; it is an internal curvature-response gate fixed
by SM integer and metric constraints. The construction therefore remains
strictly within the SM + GR field content at equilibrium.

\paragraph*{Experimental interpretation.}
At equilibrium the curvature gate and depth coordinate imply the local response
\begin{equation}
  \frac{\Delta G}{G}
  =
  \left(\frac{\delta\Xi}{\sigma_\chi}\right)^{2}
  =
  \left(\frac{\phi_\chi}{\Lambda_\chi}\right)^{2},
\end{equation}
so the first nonzero departure from equilibrium is strictly quadratic in the
dimensionless displacement $\delta\Xi$ (or equivalently in the aligned
coordinate $\phi_\chi$).
Parity, equilibrium normalization, and analytic minimality forbid any linear
term or tunable coefficient in this expansion.
Experimentally, any measurable nonzero linear dependence of $\Delta G/G$ on a
control parameter $s$ that probes $\delta\Xi$ would therefore falsify the
construction, as would any attempt to restore agreement by introducing
adjustable functions in $\Pi(\Xi)$ or additional scalar degrees of freedom.

Comparison of the SM-derived normalization $G(M_Z)$ with the proton--proton
gravitational coupling is summarized by the closure ratio
\begin{equation}
  Z_G \equiv \frac{\alpha_G^{(\mathrm{pp})}}{\Omega(M_Z)} \simeq 0.9143.
\end{equation}
This represents an $\mathcal{O}(10\%)$ deviation obtained without any parameter
tuning and is interpreted solely as an \emph{a posteriori} consistency test of
a fixed, SM-internal construction.
The closure ratio is not used to calibrate or fit the theory: persistent
disagreement in $Z_G$, or the appearance of a linear term in $\Delta G/G$,
would falsify the framework rather than determine new parameters.

The leave–one–out forecast
\begin{equation}
  \hat\alpha_s^\star(M_Z)
  =
  0.1173411 \pm 1.86\times10^{-5}
\end{equation}
lies within $\approx 0.7\sigma$ of the PDG world average. The quoted
uncertainty is obtained by propagating the PDG electroweak input
uncertainties and the experimental proton–proton gravitational coupling,
including their covariance, through the deterministic closure relation; no
parameters are fit. These numerical statements are postdictions of a fixed,
parameter-free construction rather than fitted results, and the error budget
is entirely experimental. The static construction is robust under known
higher–loop and scheme variations: the integer kernel and metric softness are
one-loop structures, but their alignment and the resulting electroweak anchor
remain numerically stable across the currently quoted ranges of multi-loop
and threshold corrections.

As an informal robustness check, we also evaluate the aligned monomial
$\Omega$ using a mixed, physically motivated set of couplings:
$\hat\alpha_s$ in the $\overline{\mathrm{MS}}$ scheme at $\mu=M_Z$,
$\hat\alpha_2$ defined from $G_F$ and $m_W$ near the $W$--boson mass,
and $\hat\alpha$ in the Thomson limit.  In this “natural anchor”
configuration the closure ratio $Z_G$ moves from the ${\sim}10\%$
deviation found with fully $\overline{\mathrm{MS}}$ electroweak pins to
a value within ${\cal O}(10^{-3})$ of unity.  This mixed configuration
is \emph{not} used in the construction itself, but it indicates that the
integer-aligned anchor is not finely tuned to a particular
renormalization prescription and remains numerically stable under
deliberate, scheme-level deformations of the inputs.


\paragraph*{Equilibrium restriction and open questions.}
The present work is restricted to equilibrium or quasi-static configurations
of $\Xi(x)$ and does not attempt to specify a dynamical evolution law, derive
stress--energy sources for $\delta\Xi$, or characterize non-equilibrium
propagation, transport, or causal structure. These questions require
extensions beyond the static framework but leave its fixed internal ingredients
unchanged. Natural next steps include:
(i)~a dynamical evolution equation for $\Xi(x)$ away from equilibrium,
(ii)~identifying physical generators of $\delta\Xi$, and
(iii)~relating curvature response to stress--energy transport.
These are developed in \textsc{Geometry~II} (Euclidean tensor sector and
aligned mass gap) and \textsc{Geometry~III} (dynamic alignment, drift law, and
stress--energy response). Both preserve all equilibrium pins and all integer
and metric structures established here.

Taken together, the results indicate that the Standard Model contains
sufficient internal algebraic and geometric structure to define a gravitational
normalization and parity-even curvature response without new degrees of
freedom at equilibrium. The framework is therefore best interpreted as a
Standard-Model-anchored mechanism consistent with GR, with empirical
validation depending solely on the experimental tests in
Section~\ref{sec:predictions}. No auxiliary assumptions or tunable extensions
are available within the static sector.

\paragraph*{Empirical falsifiers.}
The equilibrium construction leads to several sharp empirical conditions:

\begin{enumerate}
  \item \textbf{No linear term in $\Delta G/G$.}
  Near equilibrium the framework predicts a strictly quadratic response,
  \[
    \frac{\Delta G}{G}
    =
    \left(\frac{\delta\Xi}{\sigma_\chi}\right)^{2}
    =
    \left(\frac{\phi_\chi}{\Lambda_\chi}\right)^{2}
    + O\!\left((\delta\Xi)^{4}\right),
  \]
  with no linear contribution.
  In experimental terms, if $s$ denotes any control parameter that is linearly
  proportional to $\delta\Xi/\sigma_\chi$ or $\phi_\chi/\Lambda_\chi$, then
  \[
    \left.\frac{d}{ds}\frac{\Delta G}{G}\right|_{s=0} \neq 0
    \;\;\Longrightarrow\;\;
    \text{framework falsified}.
  \]
  Any statistically significant linear dependence of $\Delta G/G$ on $s$
  violates parity, equilibrium normalization, and the absence of a propagating
  scalar mode under the assumptions of GEOMETRY~I.

  \item \textbf{Fixed quadratic coefficient ($B = 1$ in $s$--units).}
  In the same $s$--parameterization, the leading response must take the form
  \[
    \frac{\Delta G}{G} = 1 \cdot s^{2} + O(s^{4}),
  \]
  with no freedom to adjust the quadratic coefficient.
  The value $B = 1$ is fixed by the Fisher softness and curvature width
  $\sigma_\chi$. Any need to fit $B \neq 1$ in order to match data falsifies
  the construction or the assumptions underlying the curvature gate
  $\Pi(\Xi)$.

  \item \textbf{No auxiliary fields or tunable curvature functions.}
  The analysis is performed under the hypothesis that the field content is
  ``SM + GR only'' and that the curvature response is encoded by a fixed,
  analytic, parity-even gate $\Pi(\Xi)$ with width $\sigma_\chi$ derived from
  the Fisher metric. Any attempt to restore agreement with experiment by
  \emph{adding} a propagating scalar degree of freedom, by promoting $\Pi(\Xi)$
  to a tunable potential with free coefficients, or by introducing additional
  dimensionful scales beyond $\sigma_\chi$ and $m_p$ is counted as a failure of
  the framework rather than as a viable modification.

  \item \textbf{Closure ratio and leave-one-out consistency.}
  The comparison between the SM-derived normalization and the measured
  proton--proton coupling is encoded in the closure factors
  \[
    Z_G \equiv \frac{\alpha_G^{(\mathrm{pp})}}{\Omega(M_Z)}
          = \frac{G_N}{G(M_Z)},
    \qquad
    Z_G^{-1} = \frac{\Omega(M_Z)}{\alpha_G^{(\mathrm{pp})}}
             = \frac{G(M_Z)}{G_N}.
  \]
  With current inputs
  \[
      Z_G \simeq 0.914355\;,
      \qquad
      Z_G^{-1} \simeq 1.09372878,
  \]
  the SM-derived normalization is
  $\approx 9.37\%$ \emph{above} the Newtonian value, and equivalently
  the Newtonian value is $\approx 8.56\%$ \emph{below} the SM-derived
  normalization.

  The associated leave-one-out strong coupling,
  \[
  \hat\alpha_s^{\star}(M_Z)
  =
  \left[
    \frac{\alpha_G^{(\mathrm{pp})}}{\hat\alpha_2^{13}\hat\alpha^{2}}
  \right]^{1/16},
  \]
  must remain statistically consistent with evolving PDG/CODATA inputs.
  Because no exponents or normalizations are adjustable, any persistent,
  high-significance disagreement once the input uncertainties stabilize
  would falsify the construction.

\end{enumerate}

\paragraph*{Summary of the static picture.}
GEOMETRY~I provides a closed, equilibrium-only account of how three fixed SM
ingredients---the SNF integer lattice, Fisher/kinetic metric softness, and an
even, curvature-matched gate---combine to determine a gravitational
normalization and a specific quadratic lab-null. All dynamical questions are
deliberately deferred, and any future extension must retain the integer,
metric, and gate structures established here to remain consistent with the
equilibrium result.

The framework does not
claim uniqueness of all possible gravity theories; it shows only that the
Standard Model gauge sector already contains sufficient fixed structure
to provide an electroweak-anchored normalization for the Einstein--Hilbert
term, together with a specific parity-even curvature response, all without
enlarging the field content or altering the tensor equations at equilibrium.
In this view, gravity is the curvature response of spacetime that appears
\emph{when} the gauge sector is aligned: the alignment mechanism provides
the internal structure through which the Einstein--Hilbert term is
normalized, and the resulting curvature response is what we ordinarily
describe as gravitational interaction.




\section{Conclusion}
\label{sec:conclusion}

This work identifies a Standard Model mechanism that fixes the gravitational
normalization at $\mu=M_Z$ using established gauge-sector structure, with no
new fields, tunable functions, or free parameters. A unique primitive integer
left-kernel of the one-loop decoupling matrix selects the depth direction
$\chi=(16,13,2)$ in log--coupling space, and the positive-definite
Fisher/kinetic metric independently selects the same soft eigenmode. Their
alignment defines the depth coordinate $\Xi=\chi\cdot\hat\Psi$ and the
dimensionless electroweak anchor
$\Omega=\hat\alpha_s^{16}\hat\alpha_2^{13}\hat\alpha^{2}$, yielding an
SM-derived gravitational normalization
\[
  G(M_Z)=\frac{\hbar c}{m_p^2}\,\Omega(M_Z).
\]
This normalization is therefore a consequence of internal SM structure and
dimensional consistency, not a fitted parameter or a modification of GR.

\paragraph*{Scope of the result.}
The construction provides a parameter-free consistency relation that fixes both
a preferred strong coupling $\alpha_s(M_Z)$ and an SM-derived gravitational
normalization $G(M_Z)$ from internal Standard Model gauge geometry. Consistency
with the measured proton--proton gravitational coupling then determines both
$G(M_Z)$ and $\alpha_s(M_Z)$ without new fields or tunable parameters.

The mechanism established here operates solely in the static, equilibrium
sector of the SM+GR framework. No scalar degree of freedom, kinetic term,
potential, or additional curvature invariant is introduced. The alignment
$\chi\parallel e_\chi$ and the function $\Pi(\Xi)$ arise entirely from
structures already present in the SM: integer rigidity of the decoupling
lattice and metric softness of the RG flow. The role of $G(M_Z)$ is therefore
that of a theoretically determined normalization for the Einstein--Hilbert term
at equilibrium, not a proposal for a dynamical or varying-$G$ theory.

An even, parity-preserving curvature gate $\Pi(\Xi)$ promotes this equilibrium
normalization to the internal curvature weighting
\begin{equation*}
  G(x)=G(M_Z)\,\Pi(\Xi(x)),
\end{equation*}
while preserving the massless, luminal tensor sector of General Relativity.
Near equilibrium, the curvature response is fixed and strictly quadratic,
\begin{equation*}
  \frac{\Delta G}{G}
  =
  \left(\frac{\delta\Xi}{\sigma_\chi}\right)^2
  + \mathcal{O}\!\big((\delta\Xi)^4\big),
\end{equation*}
with a provably absent linear term. This absence provides a direct laboratory
falsifier requiring no parameter adjustment, renormalization choice, or model
tuning. Consistency with the measured Newtonian coupling $G_N$ enters only as
an \emph{a posteriori} closure test, not as an input or calibration.

With current PDG/CODATA pins, the closure factors
\[
  Z_G \equiv \frac{\alpha_G^{(\mathrm{pp})}}{\Omega(M_Z)} \simeq 0.9143,
  \qquad
  Z_G^{-1} \equiv \frac{\Omega(M_Z)}{\alpha_G^{(\mathrm{pp})}} \simeq 1.0937,
\]
and the leave-one-out determination of the strong coupling
\[
  \hat\alpha_s^\star(M_Z)=0.1173411 \pm 1.86\times10^{-5}
\]
show percent-to-few-percent sensitivity with no free parameters. All
uncertainties derive solely from experimental pins, not from theoretical
degrees of freedom.

\paragraph*{Interpretation of the closure comparison.}
The numerical proximity of $G(M_Z)$ and $G_N$ is not built in, matched, or
enforced; it arises from SM couplings already fixed by independent
electroweak-scale measurements. Agreement or disagreement with $G_N$ therefore
constitutes a clean, non-adjustable observational test. No tuning of $\chi$,
$\sigma_\chi$, $\Omega$, or the gate shape is available under the stated
assumptions: all quantities are fixed by the SM representation lattice,
electroweak pins, and Fisher/kinetic geometry.

The present analysis applies to equilibrium or quasi-static configurations and
does not address non-equilibrium dynamics, sourcing of $\delta\Xi$, or
stress--energy evolution. These questions lie beyond the static framework but
can be pursued without altering the fixed equilibrium ingredients established
here. The integer structure, metric softness, and parity-even curvature gate
are rigid at equilibrium and provide the foundation on which any dynamical
extension must be built.

\paragraph*{Conceptual significance.}
Taken together, the results indicate that the Standard Model contains sufficient
internal algebraic and geometric structure to define a gravitational
normalization compatible with GR, reframing the role of $G_N$ from a purely
external parameter to a quantity that can be tested against a theoretically
derived electroweak-scale value. The construction highlights a nontrivial
alignment between integer rigidity and RG-response softness in the SM, and
demonstrates that this alignment is strong enough to determine both the
dimensionless anchor $\Omega$ and its curvature response under parity
constraints.

\paragraph*{Acknowledgments}
This work was conducted independently with no external funding. The author
thanks PDG, CODATA, Overleaf, and the open-source Python ecosystem for publicly
accessible tools and data. An AI-assisted writing tool (OpenAI ChatGPT) was
used for language optimization and workflow organization only; all scientific
content, calculations, and claims are the sole responsibility of the author.
The author declares no competing interests.

\paragraph*{Data availability}
All reproducibility materials are archived on Zenodo
\cite{DeMasi_GEOMETRY_repo_v2_0_0_2025}
(\texttt{GEOMETRY\_repo v2.0.0}, DOI:
\href{https://doi.org/10.5281/zenodo.17691514}{10.5281/zenodo.17691514}),
including pinned PDG/CODATA inputs, scripts, figure data, and deterministic
build manifests. 
\emph{Build artifacts (SHA--256):}
\texttt{results.json} = 08f0371b31de\ldots c7cd5edc;
\texttt{metric\_results.json} = e0e3bee8a70c\ldots b9b251b6451;
\texttt{stdout.txt} = 0f232a0be6f8\ldots 6c7cd5edc.
Additional materials are available from the author upon reasonable request.

\paragraph*{Outlook}
Future work will examine how the even-gate symmetry extends to dynamical and
spectral sectors, including the time-evolution operator, alignment-driven
transport, and curvature spectrum. If experimentally validated, the GEOMETRY
program may provide a continuous link from Standard-Model information geometry
to the equilibrium, dynamical, and spectral structure of gravitation.

\bibliographystyle{iopart-num}
\bibliography{geo_cqg_refs}




\end{document}








