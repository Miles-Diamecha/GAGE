\documentclass{iopjournal}
\usepackage{amsmath,amssymb,amsfonts}
\usepackage{lmodern}
\usepackage{float}
\usepackage{graphicx}
\usepackage{booktabs}

\begin{document}

\articletype{Paper}

\title{GEOMETRY V: $\Pi$-Weighted Hodge Flow and Harmonic Projection in Aligned Gauge Geometry}

\author{Michael DeMasi$^{1}$}
\affil{$^{1}$Independent Researcher, Milford, CT, USA}
\email{demasim90@gmail.com}

\begin{abstract}
\noindent
This work extends the alignment framework developed in \textsc{Geometry~I--IV}
to the differential-form sector, introducing a $\Pi$-weighted Hodge flow that
selects harmonic representatives of gauge-curvature cohomology classes.  
The same geometric structures that determine gravitational normalization,
spectral gaps, and curvature-dynamic regularization---the integer direction
$\chi$, the Fisher/kinetic metric, and the even curvature gate $\Pi(\Xi)$---are
shown to define a Gaussian-weighted Hodge operator and a corresponding
elliptic flow,
\[
\partial_t \omega
  = -\,d^\dagger_\Pi d\,\omega
    - d\,d^\dagger_\Pi \omega,
\]
where $d^\dagger_\Pi = \Pi^{-1} d^\dagger \Pi$ is the $\Pi$-weighted codifferential.  
This flow decreases a $\Pi$-weighted energy functional monotonically and converges
to $\Pi$-harmonic forms satisfying $d\omega=0$ and $d^\dagger_\Pi\omega=0$.  No new
fields or tunable functions are introduced; all weighting arises from the
Gaussian gate and the Fisher softness of the aligned depth coordinate.

We prove that the $\Pi$-weighted Hodge Laplacian is self-adjoint, positive, and
has discrete spectrum on appropriate weighted function spaces, establishing the
existence and uniqueness of $\Pi$-harmonic representatives.  
The $\Pi$-weighted flow is globally well-posed, smooth for all time, and possesses
a convex Lyapunov functional that enforces decay of non-harmonic components.  
These properties illustrate how alignment geometry provides a natural
regularization mechanism for differential-form evolution and suggest a physical
interpretation for $\Pi$-harmonic projection in gauge-curvature cohomology.

The results of this work complete the geometric development of the $\Pi$-weighted
alignment framework and open a connection between gauge-space geometry,
weighted Hodge theory, and physical curvature flows.
\end{abstract}

\section{Introduction}
\label{sec:intro}

\noindent
The \textsc{Geometry} series develops a unified gauge--space framework in which
the Standard Model (SM) determines gravitational normalization, curvature
response, spectral structure, and dynamic evolution through the alignment of
the integer direction $\chi=(16,13,2)$ with the soft eigenmode of the
Fisher/kinetic metric.  
In \textsc{Geometry~I}, this alignment fixed the electroweak--anchored
gravitational coupling $G(M_Z)$.  
\textsc{Geometry~II} established a finite spectral gap and reflection-positive
Euclidean functional for aligned gauge curvature.  
\textsc{Geometry~III} introduced time dependence, deriving a covariant drift law
and identifying the $\Pi$-weighted alignment operator as the generator of temporal
response.  
\textsc{Geometry~IV} analyzed the drift equation as a parabolic evolution,
demonstrating global smoothness, boundedness, and intrinsic geometric
regularization.

\medskip
In this fifth work, we extend the alignment framework to the differential-form
sector and introduce a $\Pi$-weighted Hodge flow acting on curvature-induced forms.
The key observation is that the same structures appearing in
\textsc{Geometry~I--IV}---the Gaussian curvature gate $\Pi(\Xi)$, the Fisher
softness $F_\chi$, and the aligned direction $\chi$---define a natural
weighted codifferential
\[
d^\dagger_\Pi = \Pi^{-1} d^\dagger \Pi,
\tag{1.1}
\]
and therefore a $\Pi$-weighted Hodge Laplacian
\[
\Delta_\Pi = d^\dagger_\Pi d + d\,d^\dagger_\Pi.
\tag{1.2}
\]
No new fields or tunable functions are introduced; the weighting arises solely
from the curvature gate and the Fisher metric established in earlier work.

\medskip
We show that $\Delta_\Pi$ is an elliptic, self-adjoint operator on the
$\Pi$-weighted space of differential forms, with discrete spectrum and positive
semidefinite quadratic form.  
This structure allows the construction of a $\Pi$-weighted Hodge flow,
\[
\partial_t \omega = -\Delta_\Pi \omega,
\tag{1.3}
\]
which monotonically decreases a $\Pi$-weighted energy functional and converges to a
$\Pi$-harmonic representative satisfying
\[
d\omega = 0,
\qquad
d^\dagger_\Pi \omega = 0.
\tag{1.4}
\]
These fixed points define harmonic representatives of gauge-curvature
cohomology classes in the $\Pi$-weighted geometry selected by alignment.

\medskip
The $\Pi$-weighted Hodge flow provides a natural geometric mechanism for smoothing
differential forms, analogous to the parabolic regularization identified in
\textsc{Geometry~IV} for the scalar depth coordinate.  
In particular:
\begin{itemize}
\item the Fisher softness supplies intrinsic viscosity for form evolution,
\item the Gaussian weighting provides a stabilizing potential,
\item the flow remains uniformly parabolic, smooth, and globally well-posed,
\item and no new dynamical degrees of freedom are required.
\end{itemize}
These properties suggest a physically motivated interpretation of $\Pi$-harmonic
projection and link the alignment framework to weighted Hodge theory.

\medskip
The remainder of the paper is organized as follows.  
Section~\ref{sec:weighted} introduces the $\Pi$-weighted inner product and the
associated codifferential.  
Section~\ref{sec:Laplacian} establishes self-adjointness and the spectral
structure of the $\Pi$-weighted Hodge Laplacian.  
Section~\ref{sec:flow} defines the $\Pi$-weighted Hodge flow and proves global
well-posedness and monotonic decay of the $\Pi$-energy functional.  
Section~\ref{sec:harmonic} analyzes convergence to $\Pi$-harmonic representatives.
Section~\ref{sec:discussion} discusses implications for gauge geometry,
cohomology, and the broader alignment framework.

\section{$\Pi$-Weighted Inner Product and Codifferential}
\label{sec:weighted}

\noindent
The $\Pi$-weighted Hodge theory introduced in this work builds directly on the
structures established in \textsc{Geometry~I--IV}.  
The Gaussian curvature gate,
\[
\Pi(\Xi)=\exp\!\left[-\frac{(\delta\Xi)^2}{\sigma_\chi^2}\right],
\tag{2.1}
\]
originates solely from the aligned depth coordinate
$\Xi=\chi\cdot\hat\Psi$ and the Fisher softness $F_\chi=1/\sigma_\chi^2$.
No new parameters or functions are introduced; $\Pi(\Xi)$ is fixed entirely by
the alignment geometry and was already required to preserve parity and the
massless helicity-$\pm2$ sector in earlier work.

\medskip
To incorporate $\Pi$ into the differential-form sector, we equip the space of
$k$-forms with the $\Pi$-weighted inner product
\[
\langle\omega,\eta\rangle_\Pi
   = \int_{\mathcal{M}} \Pi(\Xi)\, \omega \wedge \star \eta,
\tag{2.2}
\]
where $\mathcal{M}$ is the spacetime manifold and $\star$ is the standard Hodge
dual defined by the Lorentzian or Euclidean metric, as appropriate.  
The factor $\Pi(\Xi)$ acts as a strictly positive, smooth weight, preserving
orientation and ensuring ellipticity of all associated operators.

\medskip
The $\Pi$-weighted codifferential $d_\Pi^\dagger$ is defined as the adjoint of the
exterior derivative $d$ with respect to the inner product \eqref{2.2}.  
Explicitly, for compactly supported forms,
\[
\langle d\omega,\eta\rangle_\Pi
   = \langle\omega,d_\Pi^\dagger\eta\rangle_\Pi,
\tag{2.3}
\]
which yields the closed-form expression
\[
d_\Pi^\dagger
   = \Pi^{-1} d^\dagger \Pi,
\tag{2.4}
\]
with $d^\dagger$ the standard codifferential.  
Equation \eqref{2.4} makes manifest that weighting introduces no new geometric
structure beyond multiplication by $\Pi(\Xi)$.  
In particular,
\begin{itemize}
\item the aligned coordinate $\Xi$ governs all departures from equilibrium,
\item the Fisher softness determines the width $\sigma_\chi$,
\item the Gaussian weight provides the unique even function with vanishing
      first derivative at equilibrium,
\item and no additional fields or couplings enter the differential-form sector.
\end{itemize}

\medskip
Two immediate consequences follow.

\textbf{(i) Positivity.}
Since $\Pi>0$, the weighted inner product defines a Hilbert norm, and the
operator $d_\Pi^\dagger$ preserves the adjoint relation without altering the
sign of the quadratic form.

\textbf{(ii) Compatibility with curvature alignment.}
If $\omega$ arises from aligned gauge curvature (e.g.\ via contraction with
$\chi$ or the drift-law evolution of $\Xi$), the $\Pi$-weighted adjoint is
automatically adapted to the same aligned geometry.  
This ensures that the $\Pi$-weighted Hodge theory is not an independent structure
but a direct extension of the alignment principle.

\medskip
These results prepare the ground for the $\Pi$-weighted Hodge Laplacian discussed
in Section~\ref{sec:Laplacian}.

\section{$\Pi$-Weighted Hodge Laplacian: Self-Adjointness, Positivity, and Spectrum}
\label{sec:Laplacian}

\noindent
The $\Pi$-weighted codifferential introduced in Section~\ref{sec:weighted} defines a
natural Laplacian acting on differential forms,
\[
\Delta_\Pi = d_\Pi^\dagger d + d\,d_\Pi^\dagger,
\tag{3.1}
\]
which inherits all of its geometric structure from the alignment framework.
Because $\Pi(\Xi)$ is strictly positive, smooth, and even, the weighted
Laplacian preserves the ellipticity and symmetry properties of the standard
Hodge Laplacian, while introducing no new degrees of freedom.

\medskip
To study $\Delta_\Pi$, we work on the $\Pi$-weighted Hilbert space of $k$-forms,
\[
\mathcal{H}_\Pi^k(\mathcal{M})
  = L^2_\Pi\!\big(\Lambda^kT^\ast\mathcal{M}\big),
\qquad
\|\omega\|_\Pi^2 = \langle\omega,\omega\rangle_\Pi.
\tag{3.2}
\]
The weight $\Pi(\Xi)$ factors multiplicatively in this inner product, allowing
all analysis to proceed using standard elliptic-operator theory with
coefficients determined by the aligned curvature gate.

\subsection*{3.1 Symmetry and quadratic form}
For compactly supported forms $\omega$ and $\eta$,
\[
\langle \Delta_\Pi \omega, \eta \rangle_\Pi
    = \langle d\omega, d\eta\rangle_\Pi
    + \langle d_\Pi^\dagger\omega, d_\Pi^\dagger\eta\rangle_\Pi,
\tag{3.3}
\]
which is manifestly symmetric.  
The associated quadratic form,
\[
\mathfrak{q}[\omega]
  = \|d\omega\|_\Pi^2 + \|d_\Pi^\dagger\omega\|_\Pi^2,
\tag{3.4}
\]
is nonnegative and vanishes only for $\Pi$-harmonic forms.  
Since $\Pi(\Xi)$ is bounded above and below on compact sets and falls
Gaussianly in the depth direction, the form domain is complete, and
$\mathfrak{q}$ is closed.

\medskip
Closedness follows from the same $\Pi$-weighted coercivity used in
\textsc{Geometry~III} for the scalar alignment operator:
the Gaussian weight ensures compactness of the embedding of the form domain
into $\mathcal{H}_\Pi^k$, paralleling the Rellich--Kondrachov lemma in the
$\Pi$-weighted setting.

\subsection*{3.2 Essential self-adjointness}
The symmetry and closedness of $\mathfrak{q}$ imply that $\Delta_\Pi$ admits a
unique self-adjoint Friedrichs extension acting on $\mathcal{H}_\Pi^k$.  
Thus,
\[
\Delta_\Pi = \Delta_\Pi^\dagger,
\qquad
\mathrm{Dom}(\Delta_\Pi) = \mathrm{Dom}(\Delta_\Pi^{\mathrm{F}}),
\tag{3.5}
\]
with no additional boundary conditions required beyond those inherited from the
underlying manifold.

\medskip
This result mirrors the self-adjointness of the $\Pi$-weighted alignment operator
in \textsc{Geometry~III}; the $\Pi$-weighted Laplacian is simply the differential-form
version of the same geometric mechanism.

\subsection*{3.3 Positivity and ellipticity}
Because $d$ and $d_\Pi^\dagger$ are first-order operators, $\Delta_\Pi$ is
second-order elliptic with smooth coefficients determined by derivatives of
$\Pi(\Xi)$.  
The positivity of the quadratic form \eqref{3.4} ensures the operator is
nonnegative,
\[
\langle\omega,\Delta_\Pi\omega\rangle_\Pi = \mathfrak{q}[\omega] \ge 0.
\tag{3.6}
\]
The kernel consists exactly of $\Pi$-harmonic forms satisfying
\[
d\omega = 0,
\qquad
d_\Pi^\dagger \omega = 0.
\tag{3.7}
\]

\subsection*{3.4 Discrete spectrum}
The Gaussian weight in $\Pi(\Xi)$ ensures that the $\Pi$-weighted embedding of the
form domain into $\mathcal{H}_\Pi^k$ is compact, implying compact resolvent for
$\Delta_\Pi$.  
Thus the spectrum is discrete,
\[
0 = \lambda_1 \le \lambda_2 \le \lambda_3 \le \cdots \to \infty,
\tag{3.8}
\]
with $\lambda_1$ corresponding to $\Pi$-harmonic forms.  
The $\Pi$-weighted Hodge decomposition
\[
\mathcal{H}_\Pi^k
  = \ker(\Delta_\Pi)
  \oplus \overline{\mathrm{Im}(d)}
  \oplus \overline{\mathrm{Im}(d_\Pi^\dagger)}
\tag{3.9}
\]
follows by standard elliptic theory adapted to the weighted inner product.

\medskip
The spectral properties established in this section provide the analytic
foundation for the $\Pi$-weighted Hodge flow, defined in
Section~\ref{sec:flow}.

\section{$\Pi$-Weighted Hodge Flow: Well-Posedness and Energy Decay}
\label{sec:flow}

\noindent
The $\Pi$-weighted Hodge Laplacian introduced in Section~\ref{sec:Laplacian} defines
a natural parabolic evolution equation for differential forms,
\[
\partial_t \omega = -\Delta_\Pi \omega,
\tag{4.1}
\]
which we refer to as the $\Pi$-weighted Hodge flow.  
This flow is the differential-form analogue of the scalar alignment-drift law
in \textsc{Geometry~III--IV}.  
It requires no new fields or parameters: all weighting arises solely from the
Gaussian curvature gate $\Pi(\Xi)$ and the Fisher softness $F_\chi$ inherited
from the aligned depth coordinate.

\medskip
Equation \eqref{4.1} is uniformly parabolic on the $\Pi$-weighted Hilbert space
$\mathcal{H}_\Pi^k$ for all form degrees $k$, owing to the strict positivity and
smoothness of the weight.  
Its analytic structure is governed by the self-adjoint, nonnegative operator
$\Delta_\Pi$, ensuring the same semigroup properties that characterize heat
flows in standard Hodge theory.

\subsection*{4.1 Existence and uniqueness}
Because $\Delta_\Pi$ is self-adjoint with compact resolvent, the operator
$-\Delta_\Pi$ generates a strongly continuous contraction semigroup
$e^{-t\Delta_\Pi}$ on $\mathcal{H}_\Pi^k$.  
Thus, for any initial form
$\omega_0 \in \mathcal{H}_\Pi^k$, the flow
\eqref{4.1} admits a unique global solution,
\[
\omega(t) = e^{-t\Delta_\Pi}\,\omega_0,
\tag{4.2}
\]
which is smooth for all $t>0$ and depends continuously on the initial data.
The $\Pi$-weighted Hodge flow is therefore globally well-posed.

\medskip
This property parallels the global well-posedness of the scalar drift law in
\textsc{Geometry~IV}, where the Gaussian weight similarly ensured uniform
parabolicity and excluded singular behavior.

\subsection*{4.2 $\Pi$-weighted energy functional}
Define the $\Pi$-weighted Hodge energy
\[
\mathcal{E}_\Pi[\omega]
  = \tfrac12\,\mathfrak{q}[\omega]
  = \tfrac12\!\left(
      \|d\omega\|_\Pi^2 + \|d_\Pi^\dagger\omega\|_\Pi^2
    \right),
\tag{4.3}
\]
with $\mathfrak{q}$ the quadratic form of $\Delta_\Pi$.  
Differentiating \eqref{4.3} along the flow \eqref{4.1} yields
\[
\frac{d}{dt}\mathcal{E}_\Pi[\omega(t)]
   = -\,\langle \Delta_\Pi\omega,\,\Delta_\Pi\omega\rangle_\Pi
   = -\,\|\Delta_\Pi\omega\|_\Pi^2
   \le 0,
\tag{4.4}
\]
demonstrating that $\mathcal{E}_\Pi$ is a strict Lyapunov functional.
Therefore:
\begin{itemize}
\item the $\Pi$-weighted energy decreases monotonically,
\item the decrease is strict unless $\Delta_\Pi\omega=0$,
\item and only $\Pi$-harmonic forms are fixed points of the flow.
\end{itemize}

\medskip
Monotonic decay of a convex functional is the weighted analogue of the
dissipation principle identified in \textsc{Geometry~IV}; both arise from the
Gaussian stabilization encoded in $\Pi(\Xi)$.

\subsection*{4.3 Uniform parabolicity and smoothing}
The coefficients of $\Delta_\Pi$ involve at most second derivatives of
$\Pi(\Xi)$, all of which are smooth, bounded, and even in the depth direction.
This ensures that \eqref{4.1} is uniformly parabolic on any compact coordinate
patch and globally parabolic in the $\Pi$-weighted sense.  
Standard parabolic regularity theory implies:
\[
\omega_0 \in \mathcal{H}_\Pi^k
\quad\Longrightarrow\quad
\omega(t)\in C^\infty(\mathcal{M})
\quad\text{for all } t>0.
\tag{4.5}
\]
Thus, the flow possesses instantaneous smoothing: non-harmonic components are
immediately regularized by the $\Pi$-weighted Laplacian.

\medskip
This behavior reflects the same regularizing mechanism identified for the depth
coordinate in \textsc{Geometry~IV}.  
The Fisher softness $F_\chi$ again acts as a source of intrinsic viscosity in
gauge-space directions aligned with $\chi$.

\subsection*{4.4 Long-time behavior}
Since $\Delta_\Pi$ has discrete spectrum $0=\lambda_1<\lambda_2\le\cdots$,
expansion of the initial data in the $\Pi$-weighted eigenbasis gives
\[
\omega(t)
 = \sum_{n=1}^\infty e^{-\lambda_n t}\,\omega_n,
\tag{4.6}
\]
where $\omega_1$ spans the $\Pi$-harmonic subspace.  
Modes with $\lambda_n>0$ decay exponentially, and only the $\Pi$-harmonic
component survives at long times.  
Consequently,
\[
\lim_{t\to\infty}\omega(t)
   = \omega_{\mathrm{harm}},
\tag{4.7}
\]
where $\omega_{\mathrm{harm}}$ is the unique $\Pi$-harmonic representative in the
cohomology class of $\omega_0$.

\medskip
This establishes the $\Pi$-weighted Hodge flow as a natural projection mechanism
onto $\Pi$-harmonic forms.  
The convergence properties in \eqref{4.7} set the stage for the structural
analysis of $\Pi$-harmonic representatives in Section~\ref{sec:harmonic}.

\section{Convergence to $\Pi$-Harmonic Representatives}
\label{sec:harmonic}

\noindent
The $\Pi$-weighted Hodge flow defined in Section~\ref{sec:flow},
\[
\partial_t \omega = -\,\Delta_\Pi \omega,
\tag{5.1}
\]
acts as a gradient flow for the $\Pi$-weighted energy functional
\[
\mathcal{E}_\Pi[\omega]
   = \frac12 \big( \|d\omega\|_\Pi^2
                   + \|d_\Pi^\dagger\omega\|_\Pi^2 \big),
\tag{5.2}
\]
and the monotonic decay of $\mathcal{E}_\Pi$ implies that the long-time
behaviour of $\omega(t)$ is governed entirely by the structure of the
$\Pi$-weighted Hodge decomposition.  
In this section we show that the flow converges to the unique $\Pi$-harmonic
representative in the cohomology class of $\omega_0$.

\subsection*{5.1 $\Pi$-weighted Hodge decomposition}
Since $\Delta_\Pi$ is self-adjoint with discrete spectrum and nonnegative
quadratic form (Section~\ref{sec:Laplacian}), the $\Pi$-weighted Hilbert space of
$k$-forms admits the orthogonal decomposition
\[
\mathcal{H}^k_\Pi
  = \mathcal{H}^k_{\Pi,\mathrm{harm}}
  \oplus \overline{\mathrm{Im}(d)}
  \oplus \overline{\mathrm{Im}(d_\Pi^\dagger)},
\tag{5.3}
\]
where
\[
\mathcal{H}^k_{\Pi,\mathrm{harm}}
  = \ker d \;\cap\; \ker d_\Pi^\dagger
  = \ker \Delta_\Pi.
\tag{5.4}
\]
For any initial form $\omega_0$, we may therefore write
\[
\omega_0
  = \omega_{\mathrm{harm}}
  \,+\, d\alpha
  \,+\, d_\Pi^\dagger\beta,
\tag{5.5}
\]
with each term orthogonal in the $\Pi$-weighted inner product.  
The flow acts trivially on $\omega_{\mathrm{harm}}$ and exponentially damps
the remaining components.

\subsection*{5.2 Spectral-mode analysis}
Expanding $\omega(t)$ in the eigenbasis $\{\phi_n\}$ of $\Delta_\Pi$,
\[
\omega(t) = \sum_{n} c_n\, e^{-\lambda_n t}\, \phi_n,
\tag{5.6}
\]
yields explicit mode-by-mode control.  
Eigenmodes with $\lambda_n>0$ decay exponentially with rate $\lambda_n$, while
$\Pi$-harmonic modes ($\lambda_n=0$) remain constant.  
Thus
\[
\lim_{t\to\infty} \omega(t)
  = \sum_{\lambda_n=0} c_n\,\phi_n
  = \omega_{\mathrm{harm}},
\tag{5.7}
\]
with convergence in the $\Pi$-weighted Hilbert norm.

\subsection*{5.3 Energy convergence}
From \eqref{4.5} we have $\mathcal{E}_\Pi[\omega(t)]$ strictly decreasing,
bounded below by $0$, and differentiable for all $t>0$.  
Since the energy is exactly the squared $\Pi$-norm of the non-harmonic component,
\[
\mathcal{E}_\Pi[\omega(t)]
   = \frac12 \sum_{\lambda_n>0} \lambda_n\, |c_n|^2 e^{-2\lambda_n t},
\tag{5.8}
\]
it follows that
\[
\lim_{t\to\infty}\mathcal{E}_\Pi[\omega(t)]
  = 0.
\tag{5.9}
\]
Thus all curvature away from the $\Pi$-harmonic sector dissipates under the flow.

\subsection*{5.4 Uniqueness of the $\Pi$-harmonic representative}
Because $\Delta_\Pi$ has discrete spectrum with finite-dimensional kernel,
each cohomology class contains exactly one $\Pi$-harmonic representative.
The limiting form $\omega_{\mathrm{harm}}$ appearing in \eqref{5.7} is therefore
uniquely determined by $\omega_0$ and independent of any choices made in the
definition of the flow or in decomposition \eqref{5.5}.  
In particular,
\[
d\omega_{\mathrm{harm}} = 0,
\qquad
d_\Pi^\dagger \omega_{\mathrm{harm}} = 0,
\tag{5.10}
\]
with convergence $\omega(t)\to\omega_{\mathrm{harm}}$ in $L^2_\Pi$ and,
by elliptic regularity, in $C^\infty$ on compact subsets.

\medskip
\noindent
These results complete the analysis of the $\Pi$-weighted Hodge flow: 
the evolution is globally smooth for all $t>0$, decreases $\Pi$-energy
monotonically, and converges to the unique $\Pi$-harmonic representative of the
initial data.  
The geometric and analytic foundations established in this section support the
broader interpretation of $\Pi$-weighted harmonic projection discussed in
Section~\ref{sec:discussion}.

\section{Implications and Outlook}
\label{sec:discussion}

\noindent
The analysis presented in this work extends the alignment framework to the
differential–form sector, showing that the same χ/K/$\Pi$ structure that determines
gravitational normalization, spectral gaps, and dynamic smoothness also defines
a $\Pi$-weighted Hodge theory.  
No new fields or parameters were introduced at any stage.  
The $\Pi$-weighted codifferential and Laplacian arise entirely from multiplication
by the Gaussian curvature gate $\Pi(\Xi)$, whose form was fixed in
\textsc{Geometry~I} by parity, softness, and the requirement that the tensor
sector of GR remain unmodified at equilibrium.

\medskip
The $\Pi$-weighted Hodge flow derived in Section~\ref{sec:flow} provides a geometric
regularization mechanism for differential forms parallel to the drift-law
regularization of the scalar depth coordinate.  
The flow is strictly parabolic, admits a convex Lyapunov functional, and
converges smoothly to $\Pi$-harmonic representatives.  
This construction demonstrates that alignment naturally extends to cohomology:
the $\Pi$-harmonic condition $d\omega=0$ and $d^\dagger_\Pi\omega=0$ selects
preferred representatives in each class, determined by the same geometric
quantities that fixed $G(M_Z)$ in \textsc{Geometry~I}.

\medskip
From a mathematical perspective, the $\Pi$-weighted Laplacian provides a new
example of a Gaussian-weighted Hodge operator with compact resolvent and a
fully discrete spectrum.  
Its structure is reminiscent of Bakry–Émery modifications of the Laplacian,
but with coefficients determined directly by the alignment geometry and fixed
without tuning.  
The resulting Hodge flow supplies a well-defined, globally smooth evolution with
potential applications to weighted cohomology, gauge curvature flows, and
Euclidean functional positivity.

\medskip
From a physical perspective, the existence of $\Pi$-harmonic representatives
suggests a natural decomposition of gauge curvature into aligned and
transverse components, with the $\Pi$-weight enforcing suppression of departures
from equilibrium depth.  
The $\Pi$-weighted flow may offer a route to understanding how curvature
perturbations relax toward aligned configurations, complementing the drift-law
evolution and helicity-frequency structure developed in earlier work.

\medskip
Finally, the results of this paper complete the geometric extension of the
alignment framework to differential forms.  
\textsc{Geometry~VI} will examine potential connections between $\Pi$-weighted Hodge
flows, curvature quantization, and possible extensions to phenomenological
contexts.  
Together, \textsc{Geometry~I--V} establish a coherent structure in which the
SM-derived alignment geometry governs curvature response, spectral structure,
dynamic evolution, and cohomological projection.

\section{Conclusion}
\label{sec:conclusion}

\noindent
This work extends the alignment framework developed in \textsc{Geometry~I--IV}
to the differential-form sector by introducing a $\Pi$-weighted Hodge theory
derived entirely from the aligned depth coordinate $\Xi$, the Fisher/kinetic
metric, and the Gaussian curvature gate $\Pi(\Xi)$.  
No new fields, parameters, or tunable functions were introduced; all weighting
arises from the same geometric structures that fix the gravitational
normalization, determine the mass gap, and govern dynamic alignment.

\medskip
We showed that the $\Pi$-weighted codifferential $d^\dagger_\Pi=\Pi^{-1}d^\dagger\Pi$
defines a self-adjoint, positive $\Pi$-weighted Hodge Laplacian
$\Delta_\Pi=d^\dagger_\Pi d + d d^\dagger_\Pi$ with discrete spectrum on the
$\Pi$-weighted Hilbert space of differential forms.  
The corresponding $\Pi$-weighted Hodge flow,
\[
\partial_t\omega = -\Delta_\Pi \omega,
\]
is globally well-posed, smooth for all time, and strictly decreases the
$\Pi$-weighted energy functional.  
The flow converges to $\Pi$-harmonic representatives satisfying
$d\omega=0$ and $d^\dagger_\Pi\omega=0$, establishing a canonical selection
principle for harmonic forms in the $\Pi$-weighted geometry.

\medskip
These results demonstrate that alignment geometry supplies a natural
regularization mechanism for differential-form evolution, paralleling the
scalar drift-law regularization of \textsc{Geometry~IV}.  
In both cases, the Gaussian curvature gate provides stabilizing weight and the
Fisher softness ensures parabolicity, enabling the construction of smooth,
globally stable flows without introducing additional dynamical degrees of
freedom.

\medskip
The $\Pi$-weighted Hodge framework developed here completes the geometric core of
the alignment program and highlights a deeper connection between gauge-space
structure, weighted Hodge theory, and physical curvature flows.  
Future work may explore the extension of $\Pi$-harmonic projection to broader
cohomological settings and its potential implications for gauge-gravity
interplay within aligned curvature geometry.

\end{document}