\documentclass{iopjournal}
\usepackage{amsmath,amssymb,amsfonts}
\usepackage{lmodern}
\usepackage{float}

\begin{document}

\articletype{Paper}%

\title{GEOMETRY I: SM-derived gravitational coupling $G(M_Z)$ anchored at the electroweak scale}

\author{Michael DeMasi$^1$}
\affil{$^1$Independent Researcher, Milford, CT, USA}

\email{demasim90@gmail.com}


\begin{abstract}
At the electroweak scale $\mu = M_Z$ in the $\overline{\mathrm{MS}}$ scheme, the
one-loop Standard Model (SM) decoupling matrix admits a unique primitive integer
left-kernel $\chi=(16,13,2)$ (Smith normal form).  This identifies an aligned
depth coordinate $\Xi=\chi\!\cdot\!\hat\Psi$ in log--coupling space, with
$\hat\Psi=(\ln\hat\alpha_s,\ln\hat\alpha_2,\ln\hat\alpha)$.  The same direction
numerically coincides with the soft eigenmode of the positive-definite
Fisher/kinetic metric $K_{\rm eq}$, fixing the electroweak-scale dimensionless
anchor
\[
\Omega=e^{\Xi}
=\hat\alpha_s^{16}\hat\alpha_2^{13}\hat\alpha^{2}.
\]

This anchor defines an SM-derived gravitational normalization,
\[
G(M_Z)=(\hbar c/m_p^2)\,\Omega(M_Z),
\]
with no new fields, tunable parameters, or functional freedoms.  An even,
parity-preserving curvature gate $\Pi(\Xi)$ satisfying $\Pi'(\Xi_{\rm eq})=0$
extends this equilibrium value to an effective coupling
$G(x)=G(M_Z)\Pi(\Xi(x))$ while maintaining a massless, luminal helicity~$\pm2$
tensor sector.  Near equilibrium the laboratory response is fixed and strictly
quadratic,
\[
\Delta G/G=(\delta\Xi/\sigma_\chi)^2,
\]
so any detectable linear term directly falsifies the construction.

All quantities are determined solely from SM inputs at $\mu=M_Z$; comparison with
measured gravity enters only through an \emph{a posteriori} dimensionless closure
ratio.  The framework is therefore parameter-free, basis-invariant,
reproducible, and experimentally testable.
\end{abstract}


\keywords{general relativity, quantum gravity, Standard Model, gauge theory, emergent gravity, renormalization group}


\section{Introduction, premise, and overview}
\label{sec:premise}

We restrict throughout to Standard Model (SM) inputs at $\mu=M_Z$ in the
$\overline{\mathrm{MS}}$ scheme, introduce no new fields, parameters, or
tunable functions, and retain the massless, luminal helicity-$\pm2$ tensor
sector of General Relativity (GR).  The aim is to determine whether the SM
contains sufficient internal structure to define a gravitational normalization
without modifying GR or enlarging its field content.  All numerical values and
figures follow directly from public SM inputs using a reproducible,
hash-verified build workflow archived under a public DOI (see Data
Availability).

The SM provides precise descriptions of the three gauge interactions and their
renormalization-group (RG) evolution, but it does not internally determine
Newton's gravitational constant $G_N$.  In GR, the Einstein--Hilbert term
\[
\mathcal{L}_{\rm EH}=\frac{1}{16\pi G_N}R
\]
contains an empirically measured normalization: GR specifies how curvature
responds to stress--energy but does not fix the strength of that response.
By contrast, the electroweak-scale couplings
$(\hat\alpha_s,\hat\alpha_2,\hat\alpha)$ are scheme-consistent,
experimentally constrained, and fully determined at $\mu=M_Z$.  This motivates
the central question:
\begin{quote}
\emph{Does the SM gauge sector at $\mu=M_Z$ contain sufficient, basis-invariant
structure to define a gravitational normalization without additional degrees of
freedom or modifications of GR?}
\end{quote}

Two rigid SM ingredients---(i) the integer lattice structure of one-loop
decoupling and (ii) the Fisher/kinetic metric on log--coupling space---play the
essential role.  Evaluated together at $\mu=M_Z$, they select a single aligned
depth direction and thereby a unique dimensionless electroweak anchor.  The
gravitational normalization then follows as a consequence rather than an
externally imposed parameter.

\subsection*{Conventions and summary}

We work in $\overline{\mathrm{MS}}$ at $\mu=M_Z$ with GUT-normalized
hypercharge ($\hat\alpha_1=\tfrac{5}{3}\hat\alpha_Y$) and set $c=\hbar=1$
unless displayed explicitly.

At one loop, the SM decoupling matrix is an exact integer matrix whose Smith
normal form (SNF) has a unique primitive left-kernel generator (up to sign):
\begin{equation}
  \chi = (16,13,2). \label{eq:chi}
\end{equation}
Let $\hat\alpha_s$, $\hat\alpha_2$, and $\hat\alpha$ be the renormalized gauge
couplings at $\mu=M_Z$, and introduce the log--coupling coordinates
\begin{equation}
  \hat\Psi = (\ln\hat\alpha_s,\,\ln\hat\alpha_2,\,\ln\hat\alpha).
  \label{eq:Psi_def}
\end{equation}
The associated depth coordinate is
\begin{equation}
  \Xi = \chi \cdot \hat\Psi
      = 16\ln\hat\alpha_s + 13\ln\hat\alpha_2 + 2\ln\hat\alpha.
  \label{eq:Xi_def}
\end{equation}
In \textsc{Geometry~I}, both $\hat\Psi$ and $\Xi$ are internal coordinates on
gauge--log space; they are \emph{not} additional spacetime fields and possess
no independent dynamics.

Exponentiation defines the dimensionless electroweak anchor
\begin{equation}
\Omega \equiv e^{\Xi}
  = e^{\chi\cdot\hat\Psi}
  = \prod_i e^{\chi_i \ln\hat\alpha_i}
  = \hat\alpha_s^{16}\hat\alpha_2^{13}\hat\alpha^{2}.
  \label{eq:omega}
\end{equation}
Once $\chi$ is fixed, no free parameters or functional freedoms enter the
definitions of $\Xi$ or $\Omega$.

\paragraph{Alignment chain.}
Two independent SM structures select the same direction in log--coupling
space: (i) the primitive integer kernel of the one-loop decoupling matrix and
(ii) the soft eigenmode of the positive-definite Fisher/kinetic metric
$K_{\rm eq}$.  Their numerical alignment identifies a unique depth coordinate
$\Xi=\chi\cdot\hat\Psi$.  Even parity of the curvature response along this axis
imposes $\Pi'(\Xi_{\rm eq})=0$, yielding the strictly quadratic prediction
\[
\Delta G/G=(\delta\Xi/\sigma_\chi)^2,
\]
and preserving the massless, luminal helicity~$\pm2$ tensor sector of GR at
equilibrium.  This integer--metric alignment chain is the geometric core of
\textsc{Geometry~I}.

\paragraph{Connection to GR normalization.}
The Einstein--Hilbert term
\[
\mathcal{L}_{\rm EH} = \frac{1}{16\pi G_N}\,R
\]
is conventionally normalized using the measured Newtonian coupling $G_N$.
The SM electroweak anchor instead defines an SM-derived normalization via a
reference mass $m_\ast$,
\begin{equation}
  G(M_Z) \equiv \frac{\hbar c}{m_\ast^2}\,\Omega(M_Z).
  \label{eq:gmz}
\end{equation}
In this work we take $m_\ast = m_p$, enabling comparison to the dimensionless
proton--proton reference
$\alpha_G^{(\mathrm{pp})}=G_N m_p^2/(\hbar c)$.  This choice affects only the
interpretation of $G(M_Z)$, not the derivation of $\Omega(M_Z)$, which is
purely SM-internal and parameter-free.  Matching the Einstein--Hilbert
coefficient to this anchor yields
\begin{equation}
  G(M_Z) = \frac{1}{8\pi M_P^2},
\end{equation}
demonstrating that the GR normalization is consistent with the SM anchor.
Here $G_N$ enters solely as an \emph{a posteriori} comparison.

\paragraph{Electroweak-anchored vs.\ Newtonian coupling.}
This work derives an electroweak-anchored $G(M_Z)$; it does not predict $G_N$
directly.  Consistency is tested using the closure ratio
\[
  Z_G=\frac{\alpha_G^{(pp)}}{\widehat{\Omega}(M_Z)}.
\]

\paragraph*{\textbf{G--Match stage.}}
We refer to the comparison between the SM-derived normalization $G(M_Z)$ and
the experimentally inferred Newtonian normalization $G_N$ as the
\emph{G--Match} stage.  This step is neither a fit nor an input; it is an
\emph{a posteriori} closure check performed only after $G(M_Z)$ has been
obtained from $\Omega(M_Z)$.

\paragraph{Even curvature gate.}
At equilibrium $\delta\Xi=0$, the curvature response along the aligned axis is
encoded in an even gate
\begin{equation}
  \Pi(\Xi)=\exp[-(\delta\Xi)^2/\sigma_\chi^2],
\end{equation}
with width fixed by the Fisher curvature,
\[
  \sigma_\chi^{-2}=F_\chi=\hat\chi_K^\top K \hat\chi_K,
\]
where $\hat\chi_K=\chi/\|\chi\|_K$ and $\|\chi\|_K=\sqrt{\chi^\top K\chi}$.

\paragraph{Interpretation of $\Xi$ and $\delta\Xi$.}
The coordinate $\Xi$ is an internal SM-defined variable at $\mu=M_Z$, not a
propagating spacetime field.  \textsc{Geometry~I} is strictly static and
equilibrium: no time dependence of $\Xi$ or dynamical alignment is introduced
in this work.  Possible dynamical evolution or sourcing of $\Xi$ is deferred to
later papers.  Even parity enforces $\Pi(\Xi_{\rm eq})=1$ and
$\Pi'(\Xi_{\rm eq})=0$, forbidding any Brans--Dicke--like linear response and
yielding
\begin{equation}
  G(Q)=G(M_Z)\Pi(\Xi(Q)).
\end{equation}

\paragraph*{Closure and quadratic lab-null.}
Define
\[
  \alpha_G^{(pp)}=\frac{G_N m_p^2}{\hbar c}, \qquad
  Z_G=\frac{\alpha_G^{(pp)}}{\widehat{\Omega}(M_Z)}.
\]
Near equilibrium introduce the aligned variables
\[
  s=\delta\Xi/\sigma_\chi,\qquad
  \Lambda_\chi=\sigma_\chi/\|\chi\|_K,\qquad
  \phi_\chi=\chi^\top K\,\delta\hat\Psi/\|\chi\|_K,
\]
with $\sigma_\chi=247.683$ fixed by $F_\chi$.  The framework predicts the
quadratic lab-null
\[
  \frac{\Delta G}{G}=s^2
  =\left(\frac{\phi_\chi}{\Lambda_\chi}\right)^2,
\]
with no linear term.  The absence of any odd response provides a direct
empirical falsifier of the aligned-depth mechanism.

\paragraph*{Program and scope.}
This paper is the first in a sequence, \textsc{Geometry}
(Gauge Exponential Omega Metric Even Tensor Running Yield).
\textsc{Geometry~I} is restricted to the static, equilibrium setting and asks
whether the SM contains sufficient internal structure to define an
electroweak-anchored gravitational coupling without new degrees of freedom.
The GR tensor sector remains massless and luminal at equilibrium, and no
dynamics of $\Xi$ is introduced here.  All inputs use the $\overline{\mathrm{MS}}$
scheme at $\mu=M_Z$, with values, uncertainties, and covariance matrices taken
from established sources.

\paragraph*{Conventions and references.}
Renormalization conventions follow
Weinberg~\cite{Weinberg1996_QFTv2},
Peskin and Schroeder~\cite{PeskinSchroeder1995_QFT},
and Langacker~\cite{Langacker2017_SMBeyond}.
Decoupling and integer-lattice methods follow
Appelquist and Carazzone~\cite{AppelquistCarazzone1975_Decoupling},
Kannan and Bachem~\cite{KannanBachem1979_SNF},
and Newman~\cite{Newman1997_SNF}.
Electroweak parameters and covariance matrices are from PDG and CODATA
\cite{PDG2024,PDG2024_EWReview,PDG2025_GaugeHiggs,CODATA2022_RMP}.
Two-loop running follows
Machacek and Vaughn~\cite{Machacek1983_TwoLoopI,Machacek1984_TwoLoopII} and Luo \textit{et al.}~\cite{Luo2003_TwoLoopSM},
while electromagnetic running follows
Jegerlehner~\cite{Jegerlehner2019_alphaRun}.
Tests of GR used for comparison follow
Carroll~\cite{Carroll2004_SG},
Will~\cite{Will2014_LRR_TestsGR},
Bertotti \textit{et al.}~\cite{Cassini2003_PPN},
and LVK~\cite{LVK2021_TestsGR}.
No scalar--tensor, Brans--Dicke, or dilaton fields are introduced, and no
phenomenological potentials, free functions, or tunable parameters are added.

\paragraph*{Assumptions and scope.}
GEOMETRY~I is strictly static and equilibrium, uses only Standard Model
inputs at $\mu=M_Z$ in the $\overline{\mathrm{MS}}$ scheme, and introduces
no new fields, potentials, or tunable functions.  For clarity, the core
assumptions are summarized in Table~\ref{tab:assumptions}; all subsequent
constructions and falsifiers are derived under this fixed scope.

\begin{table}[H]
\centering
\includegraphics[width=\linewidth]{fig_ass.pdf}
\caption{
Assumptions and scope of \textsc{Geometry~I}.  Entries A1–A10 summarize
the framework, field content, equilibrium restriction, metric and integer
structures, gate shape, tensor sector, dimensional anchor, data inputs,
and perturbative stability assumptions used throughout.
}
\label{tab:assumptions}
\end{table}

\begin{table}[t]
\centering
\includegraphics[width=\linewidth]{tab1_key.pdf}
\caption{
Key definitions in \textsc{Geometry~I}. Shown are the integer generator
$\chi$, gauge--log coordinate $\hat\Psi$, depth displacement
$\delta\Xi=\Xi-\Xi_{\rm eq}$, the curvature gate $\Pi(\Xi)$, and the aligned
quantities $\phi_\chi$ and $\Lambda_\chi$ used in the lab--null prediction.
}
\label{tab:key}
\end{table}

\section{Integer certificate, one-loop weights, and the EM basis}
\label{sec:integer_certificate}

We work in log--coupling space, where multiplicative renormalization becomes
additive and changes of basis act linearly on weight vectors
\cite{Weinberg1996_QFTv2,PeskinSchroeder1995_QFT,Langacker2017_SMBeyond}.
Standard Dynkin indices and spectator multiplicities are used to form
integer-valued one-loop weight vectors, making them suitable for Smith normal
form (SNF) analysis.

\subsection*{Integerized one-loop weights}

For a Weyl fermion $f$,
\[
w_3(f)=4\,T_{\mathrm{SU(3)}}(f)\,d_{\mathrm{spect}}(f),\qquad
w_2(f)=4\,T_{\mathrm{SU(2)}}(f)\,d_{\mathrm{spect}}(f),
\]
and for scalars the same expressions appear without the factor of $4$.
The hypercharge column is integerized using the GUT-normalized definition
\[
w_1^{\rm (Weyl)}=\tfrac12 \sum_{\rm Weyl} Y^2,\qquad
w_1^{\rm (scalar)}=\tfrac13 \sum_{\rm scalars} Y^2.
\]
These choices ensure that each one-loop weight vector lies in $\mathbb{Z}^3$,
so the SNF is applied to an \emph{exact} integer matrix with no rounding or
floating-point ambiguity.

For any momentum window $W$ with light particle set $S_W$, define
\[
b(W)=
\begin{pmatrix}
\sum w_3\\[2pt]
\sum w_2\\[2pt]
\sum w_1
\end{pmatrix}
\in\mathbb{Z}^3,
\qquad
\Delta b(ij)=b(W_i)-b(W_j).
\]
Stacking such differences produces a rank-two integer matrix
\[
\Delta W=
\begin{pmatrix}
(\Delta b(i_1 j_1))^\top\\
(\Delta b(i_2 j_2))^\top\\
\vdots
\end{pmatrix}
\in\mathbb{Z}^{m\times 3}.
\]
Adjoint self-terms cancel identically in $\Delta b$, isolating the
two-dimensional integer lattice appropriate for SNF analysis.

\subsection*{Electromagnetic basis}

After electroweak symmetry breaking,
\[
w_{\mathrm{EM}}=w_2+\tfrac53\,w_1,
\]
so
\[
3\,w_{\mathrm{EM}} = 3\,w_2 + 5\,w_1 \in \mathbb{Z}.
\]
Thus, in the $(\mathrm{SU(3)},\mathrm{SU(2)},\mathrm{EM})$ basis, the integer
difference stack becomes
\[
\Delta W_{\mathrm{EM}}
=
\begin{pmatrix}
8 & 8 & 224\\
0 & 1 & 18
\end{pmatrix}
\in\mathbb{Z}^{2\times 3}.
\]

\subsection*{SNF and primitive integer kernel}

The Smith normal form
\[
U\,\Delta W_{\mathrm{EM}}\,V = \mathrm{diag}(1,8,0),
\qquad
U\in GL(2,\mathbb{Z}),\;\; V\in GL(3,\mathbb{Z}),
\]
implies $\mathrm{rank}(\Delta W_{\mathrm{EM}})=2$ and therefore a
one-dimensional integer left kernel.  Solving
$\Delta W_{\mathrm{EM}}\chi_{\mathrm{EM}}=0$ over $\mathbb{Z}$ yields the
primitive generator
\[
\chi_{\mathrm{EM}}=(-10,-18,1),\qquad \gcd(10,18,1)=1.
\]
Transporting back to the $(w_3,w_2,w_1)$ basis via a unimodular matrix
$M\in GL(3,\mathbb{Z})$ gives
\begin{equation}
  \chi = M^\top \chi_{\rm EM} = (16,13,2).
\end{equation}
The generator is unique up to overall sign, so the integer kernel is
\[
\ker_{\mathbb Z}(\Delta W^{\!\top})=\operatorname{span}_{\mathbb Z}\{\pm\chi\}.
\]

Because $\Delta W$ is an \emph{exact} integer matrix, all admissible row and
column transports are unimodular:
\[
\Delta W \;\longrightarrow\; U_{\rm row}\,\Delta W\,V_{\rm col},
\qquad
U_{\rm row},V_{\rm col}\in GL(\mathbb{Z}),
\]
which preserve the SNF invariants and therefore preserve the integer kernel.
Thus the certificate $\chi$ is basis-invariant under all integer transports,
and any derived quantities that depend solely on $\chi$ are likewise invariant.

\subsection*{Verification and reproducibility}

The SNF was computed with exact integer arithmetic using
\texttt{sympy.smith\_normal\_form} (Kannan--Bachem algorithm) in
\texttt{snf\_check.py}, yielding $\mathrm{diag}(1,8,0)$.  The primitive kernel
is $\chi_{\rm EM}=(-10,-18,1)$ and unimodular transport gives $\chi=(16,13,2)$.
All artifacts and SHA-256 checksums match the reproducibility archive
\cite{demasi_gage_repo_v1_0_0_2025}.

\section{Alignment: metric certificate}
\label{sec:metric_alignment}

The equilibrium Fisher/kinetic metric $K$ governs curvature on gauge--log
space and defines the local information geometry of the gauge couplings.  It
is constructed from the one-loop sensitivities of the SM $\beta$-functions
\cite{Machacek1983_TwoLoopI,Machacek1984_TwoLoopII,Luo2003_TwoLoopSM,Langacker2017_SMBeyond}:
\[
\beta_i=\frac{d\hat\alpha_i}{d\ln\mu}
=-\frac{b_i}{2\pi}\,\hat\alpha_i^2+\cdots,
\qquad
K_{ij}
=\frac{\partial(\beta_i/\hat\alpha_i)}{\partial\ln\hat\alpha_j}\Big|_{\rm eq},
\]
with the standard $2\pi$ normalization stripped off and all quantities
evaluated at $\mu=M_Z$.  We denote this equilibrium metric by $K\equiv K_{\rm eq}$.
As a quadratic form, $K$ acts as a Riemannian metric on the space of
log--couplings.

Using PDG/CODATA inputs for $(\hat\alpha_s,\hat\alpha_2,\hat\alpha)$ at
$\mu=M_Z$ and the one-loop SM coefficients $b_i$, one obtains
\cite{PDG2024,CODATA2022_RMP}
\[
K = \begin{pmatrix}
1.2509 & -0.6202 & -0.1813\\
-0.6202 & 1.5128 & -0.1633\\
-0.1813 & -0.1633 & 3.2362
\end{pmatrix},
\qquad
K \succ 0 .
\]

Its eigen-decomposition,
\[
K\,e_i=\lambda_i e_i,
\qquad
\{\lambda_i\}=\{0.7243,\,2.0156,\,3.2599\},
\]
identifies the soft (minimum-curvature) eigenmode
\[
e_{\chi}=(0.77249,\,0.62764,\,0.09656),
\]
with $\{e_2,e_3\}$ completing an orthonormal frame.

The SM-derived integer certificate $\chi=(16,13,2)$ from
Sec.~\ref{sec:integer_certificate} singles out a distinguished direction in
log--coupling space.  We use
\[
\hat\chi = \frac{\chi}{\|\chi\|}, \qquad
\hat\chi_K=\frac{\chi}{\|\chi\|_K},\qquad
\|\chi\|_{K}=\sqrt{\chi^\top K\chi}=17.6278,
\]
for the Euclidean-unit and metric-unit versions of $\chi$, respectively.  The
alignment between the integer and soft modes is measured by
\[
\cos\theta_K \equiv \hat\chi_K\!\cdot e_\chi
= 1 - \varepsilon_\chi,
\qquad
\varepsilon_\chi \lesssim 10^{-8},
\]
consistent with the reproducible result $\cos\theta_K = 1.0000000$ in the
repository.

Thus the integer direction selected by the SNF analysis coincides, to numerical
precision, with the minimal-curvature eigenvector of $K$.  No parameters are
adjusted: the integer lattice of $\Delta W$ and the analytic curvature of $K$
independently select the same soft mode.  We refer to this empirical
identification as the \emph{alignment principle}: at $\mu=M_Z$ the SM gauge
sector singles out a unique soft depth
\[
\Xi=\chi\!\cdot\!\hat\Psi .
\]

\subsection{Fisher curvature and curvature--gate matching}
\label{sec:fisher_matching}

The Fisher curvature along the aligned direction is
\[
F_\chi \equiv \hat\chi_K^\top K\,\hat\chi_K ,
\]
where $K$ is the Fisher/kinetic metric at $\mu = M_Z$ and $\hat\chi_K$ is the
metric-unit representative of $\chi$.

The even curvature gate is
\begin{equation}
  \Pi(\Xi) = \exp\!\left[-(\delta\Xi)^2/\sigma_\chi^2\right],
  \qquad
  \Pi'(\Xi_{\rm eq}) = 0 ,
\end{equation}
with $\delta\Xi=\Xi-\Xi_{\rm eq}$.  Matching the intrinsic Fisher curvature of
$K$ to the curvature of the gate at equilibrium,
\begin{equation}
  F_\chi = -\tfrac12\,\Pi''(\Xi_{\rm eq}),
\end{equation}
fixes the width uniquely:
\begin{equation}
  \sigma_\chi = F_\chi^{-1/2}.
\end{equation}

Using the SM pins at $\mu=M_Z$,
\begin{equation}
  F_\chi = \frac{1}{\sigma_\chi^2}
  \approx 1.629\times 10^{-5},
  \qquad
  \sigma_\chi = 247.683 .
\end{equation}
Here $F_\chi$ is evaluated directly from $K$ and $\hat\chi_K$ in
\texttt{metric\_eigs.py}, with $\sigma_\chi$ then following from the matching
relation, with no additional inputs.

The associated aligned depth scale is
\begin{equation}
  \Lambda_\chi \equiv \frac{\sigma_\chi}{\|\chi\|_K} = 14.0507 ,
\end{equation}
which sets the canonical curvature scale along the aligned direction.
Together with $\delta\Xi=\Xi-\Xi_{\rm eq}$ this yields the quadratic lab-null,
\begin{equation}
  \frac{\Delta G}{G}
  = \left(\frac{\delta\Xi}{\sigma_\chi}\right)^2
  = F_\chi\,(\delta\Xi)^2 ,
\end{equation}
with no tunable parameters.

\medskip\noindent\textbf{Verification.}
All numerical values
$K,\ \{\lambda_i\}=\{0.7243,2.0156,3.2599\},\
e_{\chi}=(0.77249,0.62764,0.09656),\
\|\chi\|_K=17.6278,$ and $\cos\theta_K=1.0000000\pm10^{-8}$
were reproduced by \texttt{metric\_eigs.py}, with SHA-256 checksums matching
the reproducibility archive~\cite{demasi_gage_repo_v1_0_0_2025}.

\section{Even gate and the quadratic lab-null}
\label{sec:even_gate}

The curvature response along the aligned depth $\Xi$ is encoded by an even
scalar gate $\Pi(\Xi)$ multiplying the Einstein--Hilbert term:
\begin{equation}
\mathcal{L}^{\rm eff}
  = \frac{1}{16\pi G(M_Z)}\,\Pi(\Xi)\,R ,
  \label{eq:leff}
\end{equation}
where $\Pi$ is treated as an emergent Standard Model form factor rather than a
free function.  Its local structure follows from (i) even parity along the
aligned direction, (ii) Fisher curvature matching, and (iii) the absence of any
tunable parameters.

The Fisher/kinetic metric $K$ induces a one-dimensional curvature along the
aligned direction $\chi=(16,13,2)$:
\[
F_\chi \equiv \hat\chi_K^{\!\top} K\,\hat\chi_K,
\qquad
\hat\chi_K \equiv \chi/\|\chi\|_K ,
\]
so infinitesimal displacements obey $ds^2 = F_\chi (d\Xi)^2$.

\paragraph*{Local expansion and curvature matching.}
Expanding an \emph{a priori} unknown even gate about equilibrium,
\begin{equation}
\Pi(\Xi)
  = 1
    + \tfrac12\,\Pi''(\Xi_{\rm eq})\,(\delta\Xi)^2
    + \mathcal{O}\!\big((\delta\Xi)^4\big),
\qquad
\delta\Xi \equiv \Xi-\Xi_{\rm eq},
\end{equation}
even parity requires $\Pi'(\Xi_{\rm eq}) = 0$.  Matching the intrinsic Fisher
curvature to the curvature of the gate,
\begin{equation}
-\Pi''(\Xi_{\rm eq})
  = \frac{2}{\sigma_\chi^2}
  = 2 F_\chi ,
\qquad
\sigma_\chi \equiv F_\chi^{-1/2} = 247.683 ,
\end{equation}
fixes the width $\sigma_\chi$ uniquely from $(K,\chi)$ with no tunable
quantities.

\paragraph*{Minimal analytic even completion.}
Any analytic even function with $\Pi(\Xi_{\rm eq})=1$,
$\Pi'(\Xi_{\rm eq})=0$, and $\Pi''(\Xi_{\rm eq})=-2/\sigma_\chi^2$ yields
identical local physics at equilibrium.  To provide a closed-form global model,
we adopt the minimal even analytic completion with exponential decay,
\begin{equation}
\Pi(\Xi)
  = \exp\!\left[-\frac{(\delta\Xi)^2}{\sigma_\chi^2}\right].
\end{equation}
With $\Pi(\Xi_{\rm eq}) = 1$ and the curvature $\Pi''(\Xi_{\rm eq})$ fixed by
the Fisher curvature $F_\chi$, the width $\sigma_\chi$ is determined uniquely
by the metric, so the Gaussian in eq.~(20) is the minimal even profile
consistent with these conditions and introduces no tunable form factor.
Higher even deformations correspond only to higher-order terms
$\mathcal{O}\big((\delta\Xi)^4\big)$ and do not alter the equilibrium quadratic
prediction.

\subsection*{Quadratic lab-null prediction}

Because $G(x)=G(M_Z)\,\Pi(\Xi(x))$, expanding near equilibrium gives
\begin{equation}
\frac{\Delta G}{G}
  \equiv \frac{G(x)}{G(M_Z)} - 1
  = \Pi(\Xi) - 1
  \simeq \frac{(\delta\Xi)^2}{\sigma_\chi^2}
  = \frac{\phi_\chi^2}{\Lambda_\chi^2},
\end{equation}
where
\begin{equation}
\phi_\chi = \frac{\chi^\top K\,\delta\hat\Psi}{\|\chi\|_{K}},
\qquad
\Lambda_\chi = \frac{\sigma_\chi}{\|\chi\|_{K}} = 14.0507 .
\end{equation}
Using
\begin{equation}
\delta\Xi = \chi^\top\delta\hat\Psi = \|\chi\|_{K}\,\phi_\chi,
\qquad
F_\chi = \frac{1}{\sigma_\chi^2},
\end{equation}
one obtains the equivalent forms
\begin{equation}
\frac{\Delta G}{G}
  = F_\chi\,(\delta\Xi)^2
  = \left(\frac{\delta\Xi}{\sigma_\chi}\right)^2
  = \frac{\phi_\chi^2}{\Lambda_\chi^2}.
\end{equation}
All quantities $(F_\chi,\sigma_\chi,\Lambda_\chi,\|\chi\|_K)$ are fixed
entirely by Standard Model data at $\mu=M_Z$.

\subsection*{Empirical falsifier}

A general analytic response may be written locally as
\begin{equation}
\frac{\Delta G}{G}
  = a_1\,\delta\Xi
    + a_2\,(\delta\Xi)^2
    + a_4\,(\delta\Xi)^4 + \cdots ,
\end{equation}
so any nonzero \emph{linear} term ($a_1\neq 0$) would directly falsify the
even, parity-preserving construction, and the quadratic coefficient is fixed:
\begin{equation}
a_2 = F_\chi = \frac{1}{\sigma_\chi^2}\,.
\end{equation}
A measured linear response or a statistically significant deviation of $a_2$
from $F_\chi$ constitutes empirical refutation.

\subsection*{Invariance and parameter independence}

The prediction is invariant under all integer basis transports
\[
\Delta W \rightarrow
  U_{\rm row}\,\Delta W\,V_{\rm col},\qquad
U_{\rm row},V_{\rm col}\in GL(\mathbb{Z}),
\]
which preserve $\ker_{\mathbb{Z}}(\Delta W^\top)=\mathrm{span}_{\mathbb{Z}}\{\pm\chi\}$.
No free parameters enter: $\sigma_\chi$, $\Lambda_\chi$, $F_\chi$, and
$\|\chi\|_K$ are determined solely by $(K,\chi)$ at $\mu=M_Z$.

\medskip\noindent\textbf{Verification.}
$\sigma_\chi=247.683$ and $\Lambda_\chi=14.0507$ were reproduced by
\texttt{gate\_null.py} using $K$ from \texttt{metric\_eigs.py}.  SHA--256
checksums match the reproducibility archive
\cite{demasi_gage_repo_v1_0_0_2025}.

\section{Tensor/helicity certificate (GR limit)}
\label{sec:tensor_certificate}

The curvature gate $\Pi(\Xi)$ multiplies the Einstein--Hilbert term in the
effective action,
\[
S
=\!\int\! d^4x\,\sqrt{-g}\Big[
\tfrac{M_P^2}{2}\,\Pi(\Xi)\,R
-\tfrac12\,\partial_\mu\hat\Psi^\top K\,\partial^\mu\hat\Psi
-V(\hat\Psi)\Big],
\]
where $\hat\Psi$ denotes the log--coupling coordinates,
$\Xi=\chi\!\cdot\!\hat\Psi$ is the aligned depth, and $V(\hat\Psi)$ collects
subdominant interactions stabilizing the internal depth and metric sectors.
GEOMETRY~I is strictly static: $\Xi$ is an internal gauge--log coordinate, not
a propagating spacetime field.

Expansions are performed about the equilibrium point,
\[
\Pi'(\Xi_{\rm eq})=0,\qquad
g_{\mu\nu}=\eta_{\mu\nu}+h_{\mu\nu},\qquad
\hat\Psi=\hat\Psi_{\rm eq}+\delta\hat\Psi,
\]
with
\[
\Pi(\Xi)
 = 1
 + \tfrac12\,\Pi''(\Xi_{\rm eq})(\delta\Xi)^2
 + \cdots,
\qquad
\delta\Xi\equiv\Xi-\Xi_{\rm eq}.
\]
Even parity removes any linear displacement in $\delta\Xi$.

Because $\Pi(\Xi)$ multiplies $R$, the variation of the gravitational term
factorizes:
\[
\delta\!\bigl(\Pi(\Xi)R\bigr)
  = \Pi'(\Xi_{\rm eq})\,\delta\Xi\,R
    + \Pi(\Xi_{\rm eq})\,\delta R.
\]
At equilibrium, $\Pi(\Xi_{\rm eq})=1$ and even parity enforces
$\Pi'(\Xi_{\rm eq})=0$, eliminating all $\delta\Xi\,R$ mixing.  Thus
$\delta\Xi$ does not couple to $h_{\mu\nu}$ at quadratic order, and the
graviton dynamics reduce to the standard GR form.

\subsection*{Quadratic tensor kernel}

The linearized Ricci tensor and scalar curvature are
\begin{align*}
R^{(1)}_{\mu\nu}
&=\tfrac12(\partial_\rho\partial_\mu h^\rho_{\ \nu}
+\partial_\rho\partial_\nu h^\rho_{\ \mu}
-\Box h_{\mu\nu}
-\partial_\mu\partial_\nu h),\\
R^{(1)}&=\partial_\mu\partial_\nu h^{\mu\nu}-\Box h,
\end{align*}
with $h=h^\mu_{\ \mu}$.  Inserting these into the action and integrating by
parts yields the quadratic tensor Lagrangian
\[
\mathcal{L}^{(2)}_{\rm tens}
=\frac{M_P^2}{8}\,
h_{\mu\nu}\,
E^{\mu\nu,\rho\sigma}
h_{\rho\sigma},
\qquad
E^{\mu\nu,\rho\sigma}
=-\Box\,P^{(2)}_{\mu\nu,\rho\sigma},
\]
where $P^{(2)}_{\mu\nu,\rho\sigma}$ is the Barnes--Rivers spin--2
projector~\cite{FierzPauli1939}.

Even parity also forbids any Pauli--Fierz mass term.  The second derivative
$\Pi''(\Xi_{\rm eq})$ multiplies only $(\delta\Xi)^2$ in the expansion,
\[
\Pi(\Xi)
 = 1
 + \tfrac12\Pi''(\Xi_{\rm eq})(\delta\Xi)^2
 + \cdots,
\]
so it cannot generate the structure
$h_{\mu\nu}h^{\mu\nu}-h^2$ required for a spin--2 mass.  Consequently
$m_{\rm PF}=0$ follows from parity alone, not from tuning, and the helicity
$\pm2$ sector remains strictly massless in the equilibrium limit.

\subsection*{Helicity decomposition and propagation}

Working in de Donder gauge $\partial^\mu h_{\mu\nu}=\tfrac12\partial_\nu h$
removes spin--1 components and isolates the pure spin--2 projection:
\[
P^{(2)}_{\mu\nu,\rho\sigma}h^{\rho\sigma}=h_{\mu\nu},
\qquad
E^{\mu\nu,\rho\sigma}
  = -\Box\,P^{(2)}_{\mu\nu,\rho\sigma}.
\]
The linearized field equation,
\[
E^{\mu\nu,\rho\sigma}h_{\rho\sigma}=0,
\]
implies the GR dispersion relation
\[
\omega^2 = k^2,\qquad c_T = 1,
\]
so the helicity eigenstates are $\pm2$, massless, and luminal.  Standard
post-Newtonian and gravitational-wave bounds are therefore satisfied
identically~\cite{Will2014_LRR_TestsGR,Abbott2017_PRL_GW170817,
Abbott2017_ApJL_GRB170817A,LVK2021_TestsGR}.

\subsection*{Propagator and soft limit}

In harmonic gauge the graviton propagator takes the usual GR form
\[
D_{\mu\nu,\rho\sigma}(k)
 = i\,16\pi G_N\,
   \frac{P^{(2)}_{\mu\nu,\rho\sigma}}{k^2+i\epsilon},
\]
so the soft-graviton theorem~\cite{Weinberg1996_QFTv2} and the universality of
soft emission are unchanged.  The helicity-$\pm2$ sector is therefore identical
to GR in the equilibrium limit.

\subsection*{Distinction from scalar--tensor and modified-gravity models}

Although $\Pi(\Xi)$ multiplies the Ricci scalar, GEOMETRY~I is not a
scalar--tensor theory in the Brans--Dicke sense.  No new spacetime scalar degree
of freedom is introduced, and $\Xi$ is not a dynamical field.  Instead, $\Xi$
is an internal gauge--log coordinate fixed by the alignment between the SNF
certificate $\chi$ and the soft eigenmode of the Fisher metric.  The
even-parity condition $\Pi'(\Xi_{\rm eq})=0$ removes the linear coupling that
would mix depth displacements with $h_{\mu\nu}$ and forbids any
Brans--Dicke-like admixture at equilibrium.  Only quadratic, SM-determined
curvature response remains, and the tensor kernel reduces exactly to the GR
Lichnerowicz operator.

This contrasts with $f(R)$ and screening models, where additional fields or
potentials are introduced and tuned to satisfy gravitational tests.  Here the
response follows from the geometry of log--coupling space and its aligned soft
mode, with no new parameters.

\subsection*{Summary and falsifier link}

The combined conditions
\[
\Pi'(\Xi_{\rm eq})=0,
\qquad
K\succ0,
\]
guarantee:
\begin{enumerate}
\item no scalar--tensor mixing at quadratic order,
\item no Pauli--Fierz mass term ($m_{\rm PF}=0$),
\item luminal propagation ($c_T=1$),
\item a GR-normalized helicity $\pm2$ sector.
\end{enumerate}
These properties constitute the \emph{tensor/helicity certificate} of
\textsc{Geometry I}.  Any observed deviation in $c_T$ or an inferred
$m_{\rm PF}\neq0$ would violate aligned-depth symmetry and falsify the
mechanism.

\medskip\noindent\textbf{Verification.}
The equilibrium quantities
$\|\chi\|_K=17.6278$ and $\sigma_\chi=247.683$ (from
\texttt{metric\_eigs.py} and \texttt{gate\_null.py})
match the reproducibility archive~\cite{demasi_gage_repo_v1_0_0_2025}.

\section{Closure (a posteriori) and pins}
\label{sec:closure}

\paragraph*{Closure concept}
At $\mu=M_Z$ in the $\overline{\mathrm{MS}}$ scheme, the SM determines the
dimensionless electroweak-anchored invariant
\begin{equation}
\hat\Omega(M_Z)=\hat\alpha_s^{16}\,\hat\alpha_2^{13}\,\hat\alpha^{2}. 
\label{eq:omega_MZ}
\end{equation}
Using the SM-derived coupling $G(M_Z)$ from Eq.~\eqref{eq:gmz}, the
\emph{closure test} compares the purely SM quantity $\hat\Omega(M_Z)$ with the
metrological proton--proton value
\begin{equation}
\hat\Omega(M_Z)\stackrel{?}{=}\alpha_G^{(\mathrm{pp})}
\equiv\frac{G_N m_p^{2}}{\hbar c}.
\label{eq:closure}
\end{equation}
At equilibrium the gate satisfies $\Pi(\Xi_{\rm eq})=1$, so
\begin{equation}
G(\Xi_{\rm eq})=G(M_Z).
\label{eq:Geq}
\end{equation}
Thus $G_N$ appears only through the dimensionless reference
$\alpha_G^{(\mathrm{pp})}$ and plays no role in constructing $\hat\Omega$ or
$\Pi$.

\paragraph*{Leave-one-out (LOO) prediction}
Equation~\eqref{eq:closure} can be inverted to predict any one gauge coupling
from the other two.  For the strong coupling,
\begin{equation}
\hat\alpha_s^{\star}(M_Z)
=\!\left[\frac{\alpha_G^{(\mathrm{pp})}}{\hat\alpha_2^{13}\hat\alpha^{2}}\right]^{1/16}
=0.1173411\pm1.86\times10^{-5},
\label{eq:LOO}
\end{equation}
which lies within $0.73\sigma$ of the PDG~2024 value.  The associated closure
ratio is
\begin{equation}
\frac{\hat\Omega(M_Z)}{\alpha_G^{(\mathrm{pp})}}
=1.09373393\quad(+9.37\%).
\label{eq:closure_ratio}
\end{equation}
This is an \emph{a posteriori} consistency measure; no parameters are adjusted
to improve agreement.

\paragraph*{Matching \( \mathrm{UV}\!\rightarrow\!\mathrm{IR} \)}
To relate $G(M_Z)$ to the measured $G_N$, define the empirical matching factor
\begin{equation}
Z_G^{-1}\equiv\frac{G_N}{G(M_Z)}
=\frac{\alpha_G^{(\mathrm{pp})}}{\hat\Omega(M_Z)}
=0.91430\quad(-8.57\%),
\label{eq:matching}
\end{equation}
interpreted as capturing scheme, threshold, and higher-order effects.  No
parameter is fitted or tuned; $Z_G$ is inferred only after $G(M_Z)$ and
$\hat\Omega(M_Z)$ are fixed by SM data at $\mu=M_Z$.

All quantities used here are fixed at $\mu=M_Z$ from PDG and CODATA
\cite{PDG2024,PDG2024_PhysConstants,PDG2024_EWReview,CODATA2022_RMP}, with
two-loop running following
\cite{Jegerlehner2019_alphaRun,Awramik2004_mW,Sirlin1980_DeltaR,
Machacek1983_TwoLoopI,Machacek1984_TwoLoopII,Luo2003_TwoLoopSM}.
Metrology enters only \emph{a posteriori} through $\alpha_G^{(\mathrm{pp})}$.

\medskip\noindent\textbf{Uncertainty propagation.}
Uncertainties are propagated in log-space:
\begin{equation}
\sigma^2(\ln Z_G)
=\sigma^2(\ln \alpha_G^{(\mathrm{pp})})
+\sum_{k\in\{s,2,\mathrm{em}\}}\chi_k^2\,\sigma^2(\ln\hat\alpha_k).
\label{eq:uncertainty}
\end{equation}

\begin{table}[t]
\centering
\includegraphics[width=\columnwidth]{tab2_pins.pdf}
\caption{
Canonical SM pins and \textsc{Geometry~I} reproducibility values.
Left: fixed physical inputs. Right: deterministic outputs from the build.
Includes electroweak couplings, proton mass, closure ratio $Z_G$,
aligned-direction invariants $(\|\chi\|_K,\sigma_\chi,\Lambda_\chi)$,
and Fisher-metric eigenstructure.
}
\label{tab:pins}
\end{table}

\noindent\textit{Build artifacts (SHA-256):}\,
\texttt{results.json}=08f0371b31de…c7cd5edc;\,
\texttt{metric\_results.json}=e0e3bee8a70c…b9b251b6451;\,
\texttt{stdout.txt}=0f232a0be6f8…6c7cd5edc.

\subsection*{Leave-one-out (LOO) forecast as falsifier}
Because $\chi=(16,13,2)$ couples $(\hat\alpha_s,\hat\alpha_2,\hat\alpha)$,
any one coupling is predicted from the other two:
\begin{equation}
\hat\alpha_i^{(\mathrm{LOO})}
=\exp\!\Bigg[
\frac{\Xi_{\rm eq}-\sum_{j\neq i}\chi_j\ln\hat\alpha_j}{\chi_i}
\Bigg].
\label{eq:LOO_general}
\end{equation}
Thus LOO acts as a \emph{direct gauge-sector falsifier}:
\begin{equation}
\sigma^2(\ln \hat\alpha_i^{(\mathrm{LOO})})
=\chi_i^{-2}\sum_{j\neq i}\chi_j^2\,\sigma^2(\ln\hat\alpha_j),
\label{eq:loo_uncertainty}
\end{equation}
and for PDG~2024 inputs
\[
\frac{\Delta\hat\alpha_s}{\hat\alpha_s}
=1.6\times10^{-4},
\]
which remains within present experimental precision.

\subsection*{Interpretation and falsifier set}
Three independent empirical tests follow:
\begin{center}
(i) gravitational closure via $Z_G$,\\
(ii) gauge-sector self-consistency via LOO,\\
(iii) parity/response via the quadratic lab-null.
\end{center}
All use only SM data at $\mu=M_Z$ with no tunable parameters.  Any
statistically significant deviation falsifies the mechanism.

\medskip\noindent\textbf{Verification.}
$Z_G$ and LOO values were regenerated by \texttt{omega\_chi.py} using the pins in
Table~\ref{tab:pins}; SHA-256 hashes match the reproducibility
archive~\cite{demasi_gage_repo_v1_0_0_2025}.

\section{Falsifiers and consistency}
\label{sec:falsifiers}

The framework contains no free parameters: every quantity is fixed by
Standard Model pins at $\mu=M_Z$.  
Each falsifier probes a distinct structural layer of the construction.

\paragraph*{\textbf{(1) Parity/response (quadratic lab--null).}}
Near equilibrium,
\[
\frac{\Delta G}{G}
= A\,s + B\,s^2 + \mathcal{O}(s^3),
\qquad s=\delta\Xi/\sigma_\chi .
\]
Even parity and alignment require
\[
A=0,\qquad B=1.
\]
Any statistically significant $A\neq0$ falsifies the mechanism.  
Parity forbids Brans--Dicke–type linear responses at equilibrium
\cite{BransDicke1961,Faraoni2004_STG,FujiiMaeda2003_STG,Carroll2004_SG}.

% ===== Figure: Even gate and quadratic parity-null =====
\begin{figure}[t]
\centering
\includegraphics[width=\columnwidth]{fig_gate.pdf}
\caption{
Even curvature gate $\Pi(\Xi)$ and quadratic parity-null relation on the
normalized depth coordinate $s=\delta\Xi/\sigma_\chi=\phi_\chi/\Lambda_\chi$.
The gate satisfies $\Pi'(\Xi_{\rm eq})=0$ (parity-even), and the lab-null
prediction $\Delta G/G=s^2$ has no odd term ($A=0$), providing a direct
falsifier of the aligned-depth mechanism.
}
\label{fig:gate}
\end{figure}

\paragraph*{\textbf{(2) Closure ratio $Z_G$.}}
At $\mu=M_Z$,
\[
Z_G=\frac{\alpha_G^{(\mathrm{pp})}}{\hat\Omega(M_Z)} ,
\]
tests consistency between the SM-derived $G(M_Z)$ and the measured $G_N$.
A deviation of $Z_G$ beyond PDG/CODATA uncertainties falsifies the
SM-anchored gravitational coupling.

\paragraph*{\textbf{(3) Leave--one--out (LOO) forecast.}}
Because $\Xi=\sum_i\chi_i\ln\hat\alpha_i$, any one coupling is predicted from
the other two:
\[
\hat\alpha_i^{(\mathrm{LOO})}
=\exp\!\Bigg[
\frac{\Xi_{\rm eq}-\sum_{j\neq i}\chi_j\ln\hat\alpha_j}{\chi_i}
\Bigg].
\]
A statistically significant deviation of
$\hat\alpha_i^{(\mathrm{LOO})}$ from measurement falsifies either the integer
certificate or the metric alignment.  
This test is predictive, not fitted.

\paragraph*{\textbf{(4) Tensor/helicity constraints.}}
With $\Pi'(\Xi_{\rm eq})=0$, the quadratic kernel reduces to the GR
Lichnerowicz operator.  Falsify if any of
\[
m_{\rm PF}\neq0,\qquad c_T\neq 1,
\]
or if scalar/spin--1 admixtures propagate.  
Current bounds from GW170817/GRB170817A enforce $c_T\simeq1$, and GWTC--3
limits $m_g\le1.27\times10^{-23}\,\mathrm{eV}/c^2$ (90\% C.L.)
\cite{Abbott2017_PRL_GW170817,Abbott2017_ApJL_GRB170817A,
LVK2021_TestsGR,Will2014_LRR_TestsGR}, both satisfied here.

\paragraph*{\textbf{(5) Metric alignment.}}
The equilibrium metric must satisfy
\[
\hat\chi_K = \chi/\|\chi\|_{K},\qquad
\cos\theta_K = \hat\chi_K\!\cdot\!e_\chi
= 1 \pm \varepsilon_\chi,\qquad
\varepsilon_\chi \lesssim 10^{-8}.
\]
Failure of positive definiteness $K\succ0$ or any measurable misalignment,
$\cos\theta_K<1-\varepsilon_\chi$, falsifies the alignment principle.

\paragraph*{\textbf{(6) Basis/invariance checks.}}
Integer transports
\[
\Delta W\to U_{\rm row}\,\Delta W\,V_{\rm col},
\qquad
U_{\rm row},V_{\rm col}\in GL(\mathbb{Z}),
\]
preserve the integer kernel
$\ker_{\mathbb{Z}}(\Delta W^\top)=\operatorname{span}_{\mathbb{Z}}\{\pm\chi\}$.
Therefore $\Xi=\chi\!\cdot\!\hat\Psi$, $\Pi(\Xi)$, and the lab-null are
basis invariant.  
Any gauge-weight basis under which these quantities change falsifies the
certificate.

\subsection*{Reporting protocol (reproducibility)}
For any dataset or update, report:
\begin{enumerate}
\item $K$ with eigenpairs and $\cos\theta_K$;
\item fitted $(A,B)$ in $\Delta G/G$ vs.\ $s$ with uncertainties;
\item LOO values $\hat\alpha_i^{(\mathrm{LOO})}$ with propagated errors;
\item $Z_G$ from current PDG/CODATA pins;
\item GW and PPN consistency (bounds on $c_T$, $m_g$).
\end{enumerate}
All quantities and artifacts are reproducible from the Zenodo archive
\cite{demasi_gage_repo_v1_0_0_2025}.

\section{Discussion}
\label{sec:discussion}

The construction reduces to a minimal, basis--invariant causal chain:
\begin{align*}
    \text{SNF certificate }\chi
\;\Rightarrow\;&\;
\text{metric alignment }(K\succ0,\ \chi\parallel e_\chi) \\
&\Rightarrow\;
\text{even curvature gate }(\Pi'(\Xi_{\rm eq})=0) \\
&\Rightarrow\;
\text{GR tensor sector + quadratic lab--null}.
\end{align*}

Two independent Standard--Model structures---the primitive integer kernel of
the one–loop decoupling matrix and the soft eigenmode of the Fisher/kinetic
metric---select the same direction in log–coupling space.  
This fixes the aligned depth $\Xi=\chi\cdot\hat\Psi$ and the electroweak anchor
$\Omega=\hat\alpha_s^{16}\hat\alpha_2^{13}\hat\alpha^{2}$.  
Because $\Omega=e^{\Xi}$, once $\Xi$ is fixed by the integer certificate, the
electroweak--scale coupling
$G(M_Z)=(\hbar c/m_p^2)\,\Omega(M_Z)$ follows with no adjustable parameters.  
The even curvature gate then promotes this anchor to a spacetime--dependent
coupling $G(Q)=G(M_Z)\Pi(\Xi(Q))$ while preserving the GR equilibrium limit:
$\Pi(\Xi_{\rm eq})=1$, $\Pi'(\Xi_{\rm eq})=0$, and a massless, luminal,
helicity~$\pm2$ tensor sector.

\subsection*{GR normalization from the electroweak anchor}

The Einstein--Hilbert normalization is not imposed but emerges as a
consistency relation of the aligned geometry.  
GR specifies the tensor dynamics,
\[
\mathcal{L}_{\rm EH}
=\frac{1}{16\pi G_N}R,
\]
but not the origin of $G_N$.  
Here the SM certificate and Fisher–metric alignment determine $\Xi$, and
therefore $\Omega(M_Z)$, entirely from SM inputs.  
The relation
\[
\frac{M_P^2}{2}
=\frac{1}{16\pi G(M_Z)}
\quad\Longleftrightarrow\quad
G(M_Z)=\frac{1}{8\pi M_P^2}
\]
shows compatibility between the SM--derived anchor and GR normalization.  
GR supplies the tensor operator; the SM supplies the magnitude of the coupling.

There is no circularity: $\Omega(M_Z)$ and $\Pi(\Xi)$ are computed from
SM data alone.  
Dimensional uplift uses a chosen mass reference $m_\ast$ (here $m_p$), and only
afterward is the dimensionless gravitational coupling $\alpha_G^{(\mathrm{pp})}$
compared to experiment.  
The measured value of $G_N$ therefore plays no role in determining
$G(M_Z)$, $\Omega(M_Z)$, or the curvature gate.

\subsection*{Physical status and scope}

In GEOMETRY~I, $\Xi$ and $\delta\Xi$ are internal gauge--log coordinates, not
independent spacetime fields.  
The construction is therefore static and equilibrium--restricted.
The G--Match comparison 
$Z_G = \alpha_G^{(\mathrm{pp})}/\Omega(M_Z)$  
is performed at $\delta\Xi=0$, where $\Pi(\Xi_{\rm eq})=1$ and the
Einstein--Hilbert sector is unchanged.

No statement is made here about how stress--energy might displace $\Xi$;
all such effects enter through $\delta\Xi$ and belong to GEOMETRY~III.  
The natural physical comparator is the dimensionless proton--proton coupling
$\alpha_G^{(\mathrm{pp})}$, with $G(M_Z)=(\hbar c/m_p^2)\,\Omega(M_Z)$ supplying
the dimensional scale.

\subsection*{Fixed versus environmental quantities}

The certificate $\chi=(16,13,2)$ and the alignment scales
$(\sigma_\chi,\|\chi\|_K,\Lambda_\chi)$ are fixed by the SM spectrum and the
renormalization geometry at $\mu=M_Z$.  
They introduce no tunable parameters.  
The gate width $\sigma_\chi$ follows from matching the gate curvature to the
Fisher curvature $F_\chi$, and
$\Lambda_\chi=\sigma_\chi/\|\chi\|_K$ sets the aligned response scale.  
By contrast, $\Xi_{\rm eq}$ and $\delta\Xi$ are environmental boundary data,
not theory parameters.

\subsection*{Relation to modified--gravity scenarios}

No new fields, potentials, or interactions are introduced.  
The curvature response is fixed entirely by the geometry of log--coupling space
and its aligned soft mode.  
Even parity enforces $\Pi'(\Xi_{\rm eq})=0$, removing all Brans--Dicke--type
linear couplings without tuning.  
At equilibrium the effective Lagrangian reduces exactly to GR while retaining a
parameter--free SM–determined response.

\subsection*{Closure and falsifiability}

The dimensionless proton--proton coupling enters only through the
a posteriori closure ratio  
$Z_G = \alpha_G^{(\mathrm{pp})}/\Omega(M_Z)$.  
Together with LOO self--consistency and the quadratic lab--null, this defines a
sharp falsifier set:  
any odd laboratory response, inconsistent closure, or violation of massless
luminal tensor propagation at $\Xi_{\rm eq}$ invalidates the mechanism.

\subsection*{Scope and extensions}

GEOMETRY~I is equilibrium and static.  
GEOMETRY~II extends the tensor analysis and curvature gate beyond equilibrium.  
GEOMETRY~III introduces dynamical alignment and describes off--equilibrium
evolution of $\Xi$.  
These developments build on, but do not modify, the equilibrium structure
established here.

\subsection*{Summary}

Electroweak--scale alignment between the integer certificate and the soft
eigenmode of the Fisher metric selects a unique internal coordinate $\Xi$.  
An even curvature gate along this axis yields a parameter--free,
SM--anchored gravitational coupling whose equilibrium tensor sector matches GR
exactly.  
The resulting framework is directly falsifiable through laboratory response,
closure, leave--one--out forecasts, and tensor propagation.  
This equilibrium geometry forms the foundation for the dynamical and spectral
components of the GEOMETRY program.

\section{Conclusion}
\label{sec:conclusion}

We have shown that the Standard Model gauge sector at $\mu=M_Z$ contains
sufficient internal structure to determine a gravitational coupling without
introducing new fields or modifying the tensor sector of General Relativity.
Two independent elements—the primitive integer kernel of one-loop decoupling
and the soft eigenmode of the Fisher/kinetic metric—select the same direction
in log–coupling space.  
This alignment fixes a unique depth coordinate $\Xi=\chi\!\cdot\!\hat\Psi$ and
its electroweak anchor
$\Omega=\hat\alpha_s^{16}\hat\alpha_2^{13}\hat\alpha^{2}$.  
Because $\Omega=e^{\Xi}$, the SM-derived coupling
$G(M_Z)=(\hbar c/m_p^2)\,\Omega(M_Z)$ follows directly from the integer
certificate with no additional parameters.

An even curvature gate $\Pi(\Xi)$, whose width is fixed by matching its curvature
to the Fisher curvature along the aligned axis, promotes this anchor to a
running coupling $G(Q)=G(M_Z)\Pi(\Xi(Q))$ while preserving the equilibrium GR
limit.  
At $\Xi_{\rm eq}$ the theory reduces to massless, luminal helicity~$\pm2$
propagation with no scalar–tensor mixing and no Pauli–Fierz mass term, without
tuning.

The resulting predictions are quantitative and parameter free.  
The laboratory response obeys the strictly quadratic relation
\[
\frac{\Delta G}{G}=(\delta\Xi/\sigma_\chi)^2,
\]
with the linear term forbidden by even parity.  
Closure is encoded by
$Z_G=\alpha_G^{(pp)}/\Omega(M_Z)$, providing a direct comparison between the
SM-anchored $G(M_Z)$ and the measured Newtonian coupling $G_N$.  
Leave-one-out consistency among the three gauge couplings supplies an
additional internal test.  
Any statistically significant deviation—odd laboratory response, closure
mismatch, LOO inconsistency, or departure from the GR tensor limit—falsifies
the aligned-depth mechanism.

This analysis is restricted to the static, equilibrium geometry in which
$\Pi(\Xi_{\rm eq})=1$ and the tensor kernel reduces to the Lichnerowicz operator
of GR.  
These results establish the electroweak alignment framework underlying the
GEOMETRY program.  
A companion paper (GEOMETRY~II) develops the tensor sector beyond equilibrium
and demonstrates a finite spectral gap, while GEOMETRY~III introduces
dynamical alignment and analyzes off-equilibrium evolution of the depth
coordinate~$\Xi$.  
These extensions build on, but do not modify, the equilibrium structure
established here.

\medskip
\noindent\textbf{Physical interpretation and scope.}
In this work, $\Xi$ and $\delta\Xi$ are internal gauge–log coordinates and not
propagating spacetime scalars; the construction is therefore equilibrium based
and non-dynamical.  
The condition $\delta\Xi=0$ corresponds to electroweak-scale equilibrium, where
$\Pi(\Xi_{\rm eq})=1$ and the Einstein–Hilbert sector retains its standard
GR form.  
Possible physical sources of $\delta\Xi$ (e.g.\ environmental, boundary, or
stress–energy dependence) are not assumed here and are deferred to
GEOMETRY~III.  
Accordingly, GEOMETRY~I identifies the gravitational normalization from
Standard-Model data rather than proposing a modification of GR or a varying-$G$
model.

\paragraph{Acknowledgments.}
This work was conducted independently with no external funding.  
The author thanks the Particle Data Group (PDG), CODATA, Overleaf, and the
open-source Python ecosystem for publicly accessible tools and data.
An AI-assisted language-editing tool (OpenAI ChatGPT) was used for
organizational and stylistic assistance; all scientific content and
responsibility rest with the author.  
The author declares no competing interests.

\paragraph*{Data availability.}
All reproducibility materials are archived in the Zenodo repository
\cite{demasi_gage_repo_v1_0_0_2025} (\texttt{GAGE\_repo v1.0.0}, DOI
\href{https://doi.org/10.5281/zenodo.17537647}{10.5281/zenodo.17537647}),
including pins, scripts, figure data, and build manifests.  
\textit{Build artifacts (SHA--256):}
\texttt{results.json}=08f0371b31de\ldots c7cd5edc;
\texttt{metric\_results.json}=e0e3bee8a70c\ldots b9b251b6451;
\texttt{stdout.txt}=0f232a0be6f8\ldots 6c7cd5edc.  
All scripts were tested with Python~3.11, NumPy~1.26, and SymPy~1.12 and
reproduce the numerical values and figures reported in this paper.  
Additional materials are available from the author upon reasonable request.

\paragraph*{Outlook.}
Future work will examine the extension of the even-gate symmetry to dynamical
and spectral sectors, including the time-evolution operator and curvature
spectrum.  
If empirically validated, the GEOMETRY program would provide a continuous link
from Standard-Model information geometry to the equilibrium, dynamical, and
spectral structure of gravitation.



\bibliographystyle{iopart-num}
\bibliography{geo_cqg_refs}

\end{document}



