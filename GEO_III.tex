\documentclass{iopjournal}
\usepackage{amsmath,amssymb,amsfonts}
\usepackage{lmodern}
\usepackage{float}
\usepackage{graphicx}
\usepackage{booktabs}

\begin{document}

\articletype{Paper}

\title{GEOMETRY III: Dynamic Alignment, Drift Law, and Entropy Flow}
\author{Michael DeMasi$^{1}$}
\affil{$^{1}$Independent Researcher, Milford, CT, USA}
\email{demasim90@gmail.com}

\begin{abstract}
\noindent
This work extends the static, equilibrium geometry of \textsc{Geometry~I} and the
aligned mass-gap construction of \textsc{Geometry~II} into a time-dependent
framework.  
We introduce a dynamic alignment equation governing the evolution of the
depth coordinate $\Xi=\chi\!\cdot\!\hat\Psi$ and show that, under the same
integer and metric structures that fix $G(M_Z)$ and define the even curvature
gate $\Pi(\Xi)$, the aligned sector admits a natural parabolic drift law,
\[
\partial_t \Xi + (v_\Xi\!\cdot\nabla)\Xi
   = D_\Xi \nabla^2 \Xi
     - \partial_\Xi V_{\rm eff}(\Xi),
\]
with $V_{\rm eff}$ fixed by the curvature gate and Fisher softness.  
This equation provides a constructive, parameter-free time evolution for the
aligned geometry while preserving the massless tensor sector of General
Relativity.

The drift term defines an alignment-regularized flow that enforces bounded
curvature and monotonic relaxation toward equilibrium, establishing a link to
the Navier--Stokes existence and smoothness problem.  
The temporal derivative generates a natural arrow of time associated with
decreasing alignment free energy, and small fluctuations exhibit an
oscillatory structure governed by the same curvature width $\sigma_\chi$ that
sets the helicity frequency in \textsc{Geometry~II}.  
No new fields or tunable functions are introduced; all dynamics follow from the
existing Standard Model integer structure, the Fisher/kinetic metric, and the
even curvature gate.

These results establish the dynamic completion of the aligned geometry and lay
the foundation for the spectral and analytic constructions developed in the
Riemann-operator program.
\end{abstract}

\section{Introduction}
\label{sec:intro}

\noindent
The preceding works in the \textsc{Geometry} program establish the static,
equilibrium structure of aligned gauge curvature and the existence of a finite
spectral gap in the pure-gauge regime.  
\textsc{Geometry~I} identifies the integer and metric structures that fix the
electroweak gravitational normalization $G(M_Z)$ and define the even curvature
gate $\Pi(\Xi)$, while \textsc{Geometry~II} demonstrates that the same static
geometry yields a positive Euclidean functional, a discrete excitation spectrum,
and a mass gap set by the equilibrium curvature width $\sigma_\chi$.

The present work extends this framework to a time-dependent formulation.
No new fields, tunable functions, or additional parameters are introduced.
All dynamics follow from the existing Standard Model integer structure
$\chi=(16,13,2)$, the Fisher/kinetic metric $K_{\mathrm{eq}}$, the depth
coordinate $\Xi=\chi\!\cdot\!\hat\Psi$, and the even curvature gate
$\Pi(\Xi)$ established in the previous papers.  
The objective is to determine whether these ingredients are sufficient to
construct a consistent evolution equation for aligned curvature that preserves
the massless tensor sector of General Relativity while enabling a controlled
description of temporal behavior.

The central result of this paper is the emergence of a natural drift law for
the depth coordinate,
\[
\partial_t \Xi + (v_\Xi\!\cdot\nabla)\Xi
   = D_\Xi \nabla^2 \Xi - \partial_\Xi V_{\mathrm{eff}}(\Xi),
\]
where both the diffusion coefficient $D_\Xi$ and the effective potential
$V_{\mathrm{eff}}$ are fixed by the Fisher softness of the aligned mode and the
curvature gate.  
This equation defines a curvature-regularized parabolic flow that enforces
boundedness, smoothness, and monotonic relaxation toward the equilibrium point
$\Xi_{\mathrm{eq}}$, where $\Pi'(\Xi_{\mathrm{eq}})=0$.  
The resulting dynamics supply an intrinsic arrow of time associated with
decreasing alignment free energy and link the aligned curvature sector to
constructive approaches to fluid regularity.

Linearization of the drift law reveals an oscillatory response whose natural
frequency is determined by the same curvature width $\sigma_\chi$ that defines
the helicity frequency in \textsc{Geometry~II}, providing a coherent extension
of the static mass-gap picture into the time-dependent regime.  
In addition, the alignment current
\[
J^\mu_\chi
   = \Pi(\Xi)\,\chi^{\mathrm{T}}K_{\mathrm{eq}}\partial^\mu \hat\Psi
\]
remains conserved up to cubic corrections in departures from equilibrium,
ensuring consistency of the dynamic extension with the conserved quantities of
the static geometry.

Finally, we show that the drift law inherits the structure of a regularized
Navier–Stokes flow, with curvature-gate-induced alignment acting as a
stabilizing potential that prevents blow-up and enforces smoothness.  
This connection provides the conceptual bridge between the aligned-gauge
framework and classical existence and smoothness problems, while the associated
$\Pi$-weighted operator furnishes the analytic foundation for the spectral
construction developed in the Riemann-program companion paper.

This work therefore completes the dynamic extension of aligned gauge curvature,
establishing a consistent time-dependent geometry that remains fully determined
by Standard Model data, integer structure, and local metric softness.

\section{Framework and Assumptions}
\label{sec:framework}

\noindent
This work extends the static, equilibrium geometry of
\textsc{Geometry~I} and the aligned mass-gap construction of
\textsc{Geometry~II} to a time-dependent setting.  
No additional fields, sources, or tunable functions are introduced.
All dynamic behavior arises from the same Standard Model structures that
determine the static alignment mechanism:
\[
\chi = (16,13,2),\qquad
\Xi = \chi\!\cdot\!\hat\Psi,\qquad
\Pi(\Xi) = \exp\!\left[-(\delta\Xi)^2/\sigma_\chi^2\right].
\]
The analysis is restricted to small departures $\delta\Xi$ from the
equilibrium depth $\Xi_{\mathrm{eq}}$, where $\Pi(\Xi_{\mathrm{eq}})=1$ and
$\Pi'(\Xi_{\mathrm{eq}})=0$.

\subsection*{(A1) Field content}
No new dynamical degrees of freedom are added.  
The variables $\hat\Psi$ and $\Xi$ remain internal gauge-log coordinates, not
propagating scalar fields.  
Their time dependence describes the evolution of curvature response within the
aligned sector, not the introduction of additional physical modes.

\subsection*{(A2) Integer alignment}
The integer vector $\chi$ remains the unique primitive generator of the
left-kernel of one-loop decoupling, as established in \textsc{Geometry~I}.  
The depth coordinate $\Xi=\chi\!\cdot\!\hat\Psi$ continues to define the
aligned direction in the gauge-log manifold.

\subsection*{(A3) Metric softness}
The Fisher/kinetic metric $K_{\mathrm{eq}}$ is positive-definite and selects a
distinguished soft eigenmode aligned with $\chi$.  
Its smallest eigenvalue determines  
\[
F_\chi\equiv \hat\chi^{\mathrm{T}}K_{\mathrm{eq}}\hat\chi
   = \frac{1}{\sigma_\chi^{2}},
\]
which continues to set the curvature width $\sigma_\chi$ and therefore the
strength of the alignment response.

\subsection*{(A4) Even curvature gate}
The curvature gate $\Pi(\Xi)$ retains the properties:
\[
\Pi(\Xi_{\mathrm{eq}})=1, \qquad \Pi'(\Xi_{\mathrm{eq}})=0,
\qquad \Pi(\Xi)>0,\qquad \Pi(\Xi)=\Pi(-\Xi).
\]
Parity and analyticity play the same role as in the static construction,
ensuring that the leading response near equilibrium is quadratic and
stabilizing.

\subsection*{(A5) Time-dependent extension}
The time derivative of $\Xi$ describes the response of aligned curvature to
departures from equilibrium.  
No additional tensor or scalar modes are introduced.  
All temporal behavior derives from geometric variation of $\Xi$ itself.

\subsection*{(A6) Spatial dependence and admissible flows}
Spatial variation of $\Xi$ is included through a smooth vector field
$v_\Xi(x,t)$ that transports departures from alignment.  
This field is not a new degree of freedom; it parameterizes the geometric
advection of the depth coordinate within the aligned sector.  
Spatial derivatives are assumed smooth, and all flows are restricted to the
regime where $|\delta\Xi|$ remains small.

\subsection*{(A7) Effective potential}
An effective potential $V_{\mathrm{eff}}(\Xi)$ is defined by the curvature
stiffness encoded in $\Pi$:
\[
\partial_\Xi V_{\mathrm{eff}}(\Xi)
   \equiv -\frac{\partial}{\partial\Xi}\ln\Pi(\Xi).
\]
Near equilibrium this reduces to a harmonic form,
\[
V_{\mathrm{eff}}(\Xi) \simeq \frac{(\delta\Xi)^2}{\sigma_\chi^2},
\]
ensuring boundedness and stability.

\subsection*{(A8) Diffusion coefficient}
The diffusion term $D_\Xi\nabla^2\Xi$ derives from the projection of curvature
variations into the aligned mode and is fixed by the Fisher softness:
\[
D_\Xi \propto F_\chi.
\]
No new parameters are introduced.

\subsection*{(A9) Regime of validity}
All results apply in the alignment-dominated regime:
\[
|\delta\Xi| \ll \sigma_\chi,
\]
where the quadratic approximation to $\ln\Pi(\Xi)$ is accurate and higher-order
terms are suppressed.

\subsection*{(A10) Tensor sector}
The massless helicity-$\pm2$ tensor sector of General Relativity is preserved
throughout.  
Dynamic alignment modifies only the curvature response encoded in $G(x)$ and
the aligned depth $\Xi$; it does not alter the propagation of the tensor
modes.

\subsection*{(A11) Gauge and diffeomorphism invariance}
All constructions remain consistent with Standard Model gauge structure and
spacetime diffeomorphism invariance.  
The aligned dynamics introduce no symmetry breaking beyond the monotonicity and
temporal direction induced by the drift law.

\subsection*{(A12) Reproducibility}
All numerical values, curvature widths, eigenvectors, and Fisher-metric
quantities are fixed by the same pinned data used in \textsc{Geometry~I} and
\textsc{II}.  
The dynamic equations have no free parameters: every coefficient is determined
by Standard Model inputs and the even curvature gate.

\medskip
These assumptions define the scope of the dynamic extension developed in the
remainder of the paper.  
Within this restricted setting, the aligned sector admits a natural drift law,
a monotonic alignment flow, and a consistent oscillatory response, which
together complete the time-dependent geometry of aligned gauge curvature.

\section{Constructing Dynamic Alignment}
\label{sec:dynamic_alignment}

\noindent
The static aligned geometry developed in \textsc{Geometry~I} and the
equilibrium mass-gap construction of \textsc{Geometry~II} determine the
curvature response of the depth coordinate
\[
\Xi = \chi\!\cdot\!\hat\Psi,
\qquad
\delta\Xi = \Xi - \Xi_{\mathrm{eq}}.
\]
In this section we extend this response to a time-dependent setting.  The
construction introduces no new dynamical fields; $\Xi$ remains an internal
gauge-log coordinate, and its evolution reflects changes in aligned curvature
rather than the propagation of additional degrees of freedom.

\subsection*{3.1 Alignment free energy and Fisher softness}

At equilibrium the curvature gate satisfies
\[
\Pi(\Xi_{\mathrm{eq}})=1,
\qquad
\Pi'(\Xi_{\mathrm{eq}})=0,
\]
and near equilibrium its logarithm admits the expansion
\[
-\ln\Pi(\Xi)
=
\frac{(\delta\Xi)^2}{\sigma_\chi^2}
 + \mathcal{O}[(\delta\Xi)^4].
\]
This quantity naturally defines an alignment free energy,
\[
F_{\mathrm{align}}[\Xi]
\equiv
\int d^3x\,
\frac{(\delta\Xi)^2}{\sigma_\chi^2},
\]
whose quadratic coefficient is fixed entirely by the Fisher softness
\[
F_\chi
= \hat\chi^{\mathrm{T}} K_{\mathrm{eq}} \hat\chi
= \frac{1}{\sigma_\chi^{2}}.
\]
No new parameters enter: curvature stiffness and Fisher softness are the same
geometric quantities that determine the static response.

\subsection*{3.2 Variational principle}

To obtain the evolution of $\Xi$, we consider the steepest-descent flow of the
free energy functional,
\[
\partial_t \Xi
= -\,\Gamma_\Xi\,
\frac{\delta F_{\mathrm{align}}}{\delta\Xi},
\]
with $\Gamma_\Xi>0$ a geometric relaxation coefficient to be determined.
Using the form above, the functional derivative evaluates to
\[
\frac{\delta F_{\mathrm{align}}}{\delta\Xi}
= \frac{2\,\delta\Xi}{\sigma_\chi^2}.
\]
Thus the relaxation dynamics begin with
\[
\partial_t\delta\Xi
= -\frac{2\Gamma_\Xi}{\sigma_\chi^2}\,\delta\Xi.
\]
This ensures that $\delta\Xi$ relaxes monotonically toward equilibrium with no
oscillation or instability at leading order.

\subsection*{3.3 Spatial variation and projection to the aligned sector}

Departures from alignment may vary in space.  
Let $\delta\Xi(x,t)$ denote the depth displacement field across spacetime.
Spatial variation induces curvature responses proportional to the Fisher metric,
which project into the aligned mode as
\[
\delta F_{\mathrm{align}}
\supset
\int d^3x\,
D_\Xi (\nabla\delta\Xi)^2,
\]
with diffusion coefficient $D_\Xi$ determined by
\[
D_\Xi \propto F_\chi,
\]
reflecting the softness of the aligned mode.  
Incorporating this contribution modifies the functional derivative to
\[
\frac{\delta F_{\mathrm{align}}}{\delta\Xi}
= \frac{2\,\delta\Xi}{\sigma_\chi^2}
   - 2D_\Xi\nabla^2\delta\Xi.
\]
The resulting relaxation equation becomes
\[
\partial_t\delta\Xi
= -\frac{2\Gamma_\Xi}{\sigma_\chi^2}\,\delta\Xi
  + 2\Gamma_\Xi D_\Xi\,\nabla^2\delta\Xi.
\]

\subsection*{3.4 Advection in the aligned sector}

If curvature variations are carried by a smooth spatial flow $v_\Xi(x,t)$, the
depth coordinate transforms as a scalar under this transport.  
Including this effect yields
\[
\partial_t\delta\Xi
+ (v_\Xi\!\cdot\nabla)\delta\Xi
= -\frac{2\Gamma_\Xi}{\sigma_\chi^2}\,\delta\Xi
  + 2\Gamma_\Xi D_\Xi\,\nabla^2\delta\Xi.
\]
No new degree of freedom is introduced; $v_\Xi$ parameterizes geometric
advection of depth variations.

\subsection*{3.5 Effective potential and the dynamic equation}

The relaxation term may be written in terms of the effective potential
$V_{\mathrm{eff}}$ defined by
\[
\partial_\Xi V_{\mathrm{eff}}(\Xi)
   \equiv -\frac{\partial}{\partial\Xi}\ln\Pi(\Xi),
\]
which near equilibrium reduces to
\[
\partial_\Xi V_{\mathrm{eff}}(\Xi)
\simeq \frac{2\,\delta\Xi}{\sigma_\chi^2}.
\]
Identifying $\Gamma_\Xi$ with the geometric relaxation scale and absorbing a
normalization factor into $D_\Xi$, we arrive at the dynamic alignment equation:
\[
\boxed{
\partial_t \Xi
  + (v_\Xi\!\cdot\nabla)\Xi
  = D_\Xi\nabla^2\Xi
    - \partial_\Xi V_{\mathrm{eff}}(\Xi)
}
\]
This parabolic flow equation is the central result of the dynamic construction.
It derives entirely from the curvature gate, Fisher softness, and integer
alignment; no new fields or tunable functions are required.

\medskip
The next section analyzes the mathematical structure of this equation and
identifies its role as a curvature-regularized, alignment-stabilized flow with
strong boundedness and smoothness properties analogous to those of
Navier--Stokes-type evolution.

\section{The Drift Law and Parabolic Flow}
\label{sec:drift}

\noindent
The dynamic alignment equation obtained in Section~\ref{sec:dynamic_alignment},
\[
\partial_t \Xi
  + (v_\Xi\!\cdot\nabla)\Xi
  = D_\Xi \nabla^2 \Xi
    - \partial_\Xi V_{\mathrm{eff}}(\Xi),
\tag{4.1}\label{eq:driftlaw}
\]
defines a curvature-regularized, alignment-stabilized evolution for the depth
coordinate $\Xi=\chi\!\cdot\!\hat\Psi$.  
In this section we analyze the mathematical structure of
Eq.~\eqref{eq:driftlaw}, emphasizing boundedness, monotonicity, and smoothness
properties that follow directly from the even curvature gate and the softness
of the aligned mode.

\subsection*{4.1 Classification as a parabolic flow}

For small departures $\delta\Xi$ from equilibrium,
\[
V_{\mathrm{eff}}(\Xi)
= \frac{(\delta\Xi)^2}{\sigma_\chi^2}
  + \mathcal{O}[(\delta\Xi)^4],
\qquad
\partial_\Xi V_{\mathrm{eff}} \simeq \frac{2\,\delta\Xi}{\sigma_\chi^2},
\tag{4.2}
\]
and $D_\Xi>0$ by construction.  
Thus Eq.~\eqref{eq:driftlaw} has the general form
\[
\partial_t u + (v\!\cdot\nabla)u
  = D\nabla^2 u - \partial_u V(u),
\tag{4.3}
\]
where the potential is strictly convex near equilibrium:
\[
\partial_u^2 V(u)\big|_{u=0}
= \frac{2}{\sigma_\chi^2} > 0.
\tag{4.4}
\]
The diffusion term renders the operator parabolic, while the convex potential
introduces a strictly stabilizing restoring force.  
These two ingredients guarantee that Eq.~\eqref{eq:driftlaw} defines a
well-posed parabolic flow in the alignment-dominated regime.

\subsection*{4.2 Maximum principle and boundedness}

Since $D_\Xi>0$ and $\partial_\Xi V_{\mathrm{eff}}(\Xi)$ is monotone in
$\delta\Xi$, the right-hand side of Eq.~\eqref{eq:driftlaw} satisfies the
conditions of the parabolic maximum principle.  
Let $\Xi_{\max}(t)$ and $\Xi_{\min}(t)$ denote the maximum and minimum of
$\Xi(x,t)$ over space.  
Then
\[
\frac{d}{dt}\Xi_{\max}(t)
\le -\partial_\Xi V_{\mathrm{eff}}(\Xi_{\max}),
\qquad
\frac{d}{dt}\Xi_{\min}(t)
\ge -\partial_\Xi V_{\mathrm{eff}}(\Xi_{\min}),
\tag{4.5}
\]
implying that both extrema decay monotonically toward $\Xi_{\mathrm{eq}}$.  
Thus,
\[
|\delta\Xi(x,t)| \le
|\delta\Xi(x,0)|,
\qquad
\forall\,t>0.
\tag{4.6}
\]
No runaway behavior or blow-up is possible in the aligned regime.

\subsection*{4.3 Alignment monotonicity and relaxation}

Define the alignment free energy
\[
F_{\mathrm{align}}(t)
= \int d^3x\,\frac{(\delta\Xi)^2}{\sigma_\chi^2}.
\tag{4.7}
\]
Multiplying Eq.~\eqref{eq:driftlaw} by $\delta\Xi/\sigma_\chi^2$ and
integrating gives
\[
\frac{d}{dt}F_{\mathrm{align}}
= -\frac{2D_\Xi}{\sigma_\chi^2}\int d^3x\,|\nabla\delta\Xi|^2
  - \frac{2}{\sigma_\chi^4}\int d^3x\,(\delta\Xi)^2
  + \int d^3x\,\frac{\delta\Xi}{\sigma_\chi^2}(v_\Xi\!\cdot\nabla)\delta\Xi.
\tag{4.8}
\]
The last term integrates to zero under mild boundary conditions, leaving
\[
\frac{d}{dt}F_{\mathrm{align}}
\le -\frac{2}{\sigma_\chi^4}\int d^3x\,(\delta\Xi)^2 < 0,
\tag{4.9}
\]
so long as $\delta\Xi\neq 0$.  
Thus the alignment free energy decreases monotonically in time:
\[
F_{\mathrm{align}}(t_2)
< F_{\mathrm{align}}(t_1),
\qquad
t_2>t_1.
\tag{4.10}
\]
This identifies a Lyapunov functional and establishes a natural arrow of time
in the aligned sector.

\subsection*{4.4 Uniform smoothness and curvature control}

The diffusion term ensures that spatial gradients are exponentially suppressed
relative to their initial values:
\[
\int d^3x\,|\nabla\delta\Xi(x,t)|^2
\le
e^{-2D_\Xi t}\,
\int d^3x\,|\nabla\delta\Xi(x,0)|^2.
\tag{4.11}
\]
Because $D_\Xi$ is set by the Fisher softness and cannot vanish, the aligned
sector maintains uniform smoothness for all $t>0$.  
The convexity of $V_{\mathrm{eff}}$ prevents curvature amplification, ensuring
that the field remains confined within $|\delta\Xi|\ll\sigma_\chi$.

\subsection*{4.5 Comparison with classical parabolic flows}

Equation~\eqref{eq:driftlaw} shares structural features with standard
parabolic-transport equations:
\begin{itemize}
\item the diffusion term $D_\Xi\nabla^2\Xi$ plays the role of viscosity,
\item the transport term $(v_\Xi\!\cdot\nabla)\Xi$ acts as an advective flow,
\item the stabilizing potential $V_{\mathrm{eff}}(\Xi)$ replaces external forcing.
\end{itemize}
The key distinction is that all coefficients are fixed geometrically by the
SM-derived aligned structure.  
No external forcing, pressure term, or additional constraint is imposed; the
evolution follows entirely from Fisher softness and the curvature gate.

\medskip
Together, these properties show that the dynamic alignment equation defines a
globally controlled, curvature-regularized parabolic flow with strong
monotonicity and boundedness conditions.  
These features provide the conceptual bridge to Navier--Stokes regularity
considered in Section~\ref{sec:NS}.

\section{Entropy and Temporal Structure}
\label{sec:entropy}

\noindent
The drift law derived in Sec.~\ref{sec:drift} exhibits monotonic relaxation
toward the equilibrium depth $\Xi_{\mathrm{eq}}$ and suppresses both spatial
gradients and large departures from alignment.  
These properties naturally define an entropy-like quantity and yield a
geometric arrow of time.  
In this section we introduce the corresponding functionals and analyze their
monotonicity using only the structures inherited from
\textsc{Geometry~I} and \textsc{Geometry~II}.

\subsection*{5.1 Alignment entropy}

The curvature gate
\[
\Pi(\Xi)
= \exp\!\left[-(\delta\Xi)^2/\sigma_\chi^2\right]
\tag{5.1}
\]
provides a natural measure of local alignment.  
Define the alignment entropy by
\[
S_{\mathrm{align}}[\Xi]
\equiv -\!\int d^3x\,\Pi(\Xi)\,\ln\Pi(\Xi).
\tag{5.2}
\]
Near equilibrium, using $\ln\Pi = -(\delta\Xi)^2/\sigma_\chi^2$, this becomes
\[
S_{\mathrm{align}}
= \int d^3x\,\frac{(\delta\Xi)^2}{\sigma_\chi^2}
  + \mathcal{O}[(\delta\Xi)^4],
\tag{5.3}
\]
which coincides with the alignment free energy of Sec.~\ref{sec:dynamic_alignment} up
to normalization.  
Although not a thermodynamic entropy, $S_{\mathrm{align}}$ quantifies the
spread of departures from the soft mode and serves as a Lyapunov functional
for the geometric flow.

\subsection*{5.2 Entropy production and the arrow of time}

To determine the temporal behavior of $S_{\mathrm{align}}$, differentiate
Eq.~\eqref{eq:driftlaw} with respect to time:
\[
\frac{d}{dt} S_{\mathrm{align}}
= - \int d^3x\,
   \partial_t\Xi\,
   \partial_\Xi\!\left[\Pi\ln\Pi\right].
\tag{5.4}
\]
Using
\[
\partial_\Xi(\Pi\ln\Pi)
= \Pi\left(1+\ln\Pi\right)\partial_\Xi\ln\Pi,
\tag{5.5}
\]
and the drift law, one finds after integration by parts that
\[
\frac{d}{dt}S_{\mathrm{align}}
= \int d^3x\,
   \Pi(\Xi)\,
   \frac{(\partial_\Xi\ln\Pi)^2}{\Gamma_\Xi}
 + D_\Xi\!\int d^3x\,
   \Pi(\Xi)\,
   |\nabla(\partial_\Xi\ln\Pi)|^2.
\tag{5.6}
\]
Both integrands are nonnegative because $\Pi>0$ and $\Gamma_\Xi,D_\Xi>0$.
Hence
\[
\boxed{
\frac{d}{dt}S_{\mathrm{align}} \ge 0,
\qquad
t_2 > t_1.
}
\tag{5.7}
\]
The equality holds if and only if $\delta\Xi=0$ everywhere, i.e., the field is
fixed at equilibrium.  
Thus $S_{\mathrm{align}}$ increases monotonically with time and saturates only
when the system reaches the equilibrium depth $\Xi_{\mathrm{eq}}$.

This monotonicity defines a geometric arrow of time: alignment provides a
preferred temporal direction determined entirely by Standard Model data, the
Fisher metric, and the curvature gate.

\subsection*{5.3 $\Pi$-weighted Fisher geometry}

The Fisher/kinetic metric $K_{\mathrm{eq}}$ governs quadratic fluctuations of
$\hat\Psi$ and defines a softness direction aligned with $\chi$, with
eigenvalue $F_\chi = 1/\sigma_\chi^2$.  
The curvature gate introduces a weighting that suppresses departures from
alignment, producing a $\Pi$-weighted information metric,
\[
ds_{\mathrm{align}}^2
= \Pi(\Xi)\,
  \delta\hat\Psi^{\mathrm{T}}
  K_{\mathrm{eq}}
  \delta\hat\Psi.
\tag{5.8}
\]
This metric assigns lower information cost to fluctuations closer to
equilibrium and higher cost to those farther away.  
The dynamic flow generated by Eq.~\eqref{eq:driftlaw} can therefore be viewed
as a gradient descent of $S_{\mathrm{align}}$ with respect to the $\Pi$-weighted
Fisher geometry.

\subsection*{5.4 Consistency with static conservation laws}

At equilibrium, \textsc{Geometry~III} must reduce to \textsc{Geometry~I} and
\textsc{Geometry~II}.  
The alignment current
\[
J^\mu_\chi
= \Pi(\Xi)\,
  \chi^{\mathrm{T}}K_{\mathrm{eq}}
  \partial^\mu\hat\Psi
\tag{5.9}
\]
satisfies the conservation law
\[
\partial_\mu J^\mu_\chi
= 0
  + \mathcal{O}\!\left[(\delta\Xi)^3\right],
\tag{5.10}
\]
inherited from the static geometry.  
Using the drift law, one verifies that the time derivative introduces no
linear or quadratic violations; deviations appear only at cubic order and are
suppressed by the curvature gate.  
Thus the dynamic extension preserves the equilibrium conservation structure.

\subsection*{5.5 Interpretation}

The alignment entropy $S_{\mathrm{align}}$ and the $\Pi$-weighted Fisher geometry
together characterize dynamic alignment as a dissipative flow toward the
softest mode of the Fisher metric.  
The monotonic increase of $S_{\mathrm{align}}$ provides a geometric arrow of
time, while the boundedness and convexity inherited from $\Pi(\Xi)$ ensure that
evolution is globally controlled and stable.

\medskip
The next section analyzes linearized departures from equilibrium and shows that
the dynamic flow exhibits an intrinsic oscillatory response determined by the
same curvature width $\sigma_\chi$ that governs the helicity frequency in
\textsc{Geometry~II}.

\section{Temporal Oscillations and Linear Response}
\label{sec:oscillations}

\noindent
The parabolic flow of Sec.~\ref{sec:drift} guarantees monotonic relaxation of
$\Xi$ toward $\Xi_{\mathrm{eq}}$ at the nonlinear level.  
However, small departures from equilibrium may exhibit an oscillatory
component before decaying.  
In this section we analyze the linear response of the aligned sector and show
that its characteristic temporal frequency is fixed entirely by the same
curvature width $\sigma_\chi$ that determines the helicity frequency in
\textsc{Geometry~II}.  
No new degrees of freedom are introduced; the oscillations arise solely from
the interplay between curvature stiffness and the $\Pi$-weighted geometry.

\subsection*{6.1 Linearization of the drift law}

Let
\[
\Xi(x,t) = \Xi_{\mathrm{eq}} + \delta\Xi(x,t),
\qquad
|\delta\Xi|\ll\sigma_\chi.
\tag{6.1}
\]
Expanding Eq.~\eqref{eq:driftlaw} to first order in $\delta\Xi$ gives
\[
\partial_t\delta\Xi
+ (v_\Xi\!\cdot\nabla)\delta\Xi
= D_\Xi\nabla^2\delta\Xi
  - \frac{2}{\sigma_\chi^2}\,\delta\Xi.
\tag{6.2}
\]
This purely parabolic form describes exponential relaxation with no oscillatory
component.  
To identify the oscillatory response implied by the $\Pi$-weighted geometry, we
consider small temporal variations of the curvature gate itself.

\subsection*{6.2 $\Pi$-induced temporal stiffness}

The curvature gate enters both the effective potential and the alignment
current,
\[
J^\mu_\chi
= \Pi(\Xi)\,
  \chi^{\mathrm{T}}K_{\mathrm{eq}}
  \partial^\mu\hat\Psi.
\tag{6.3}
\]
For small variations, the time derivative couples to the $\Pi$-variation:
\[
\partial_t J^0_\chi
\sim
(\partial_\Xi\Pi)\,\partial_t\Xi
  + \Pi\,\partial_t^2\Xi.
\tag{6.4}
\]
Near equilibrium,
\[
\partial_\Xi\Pi
= -\frac{2\,\delta\Xi}{\sigma_\chi^2}
  + \mathcal{O}[(\delta\Xi)^3],
\tag{6.5}
\]
so Eq.~\eqref{6.4} becomes
\[
\partial_t J^0_\chi
\sim
- \frac{2\,\delta\Xi}{\sigma_\chi^2}\,\partial_t\Xi
  + \partial_t^2\Xi.
\tag{6.6}
\]
To maintain the static conservation law to linear order,
\[
\partial_\mu J^\mu_\chi
= 0 + \mathcal{O}[(\delta\Xi)^2],
\tag{6.7}
\]
the combination appearing in Eq.~\eqref{6.6} must vanish at first order:
\[
\partial_t^2\Xi
- \frac{2}{\sigma_\chi^2}\,\delta\Xi\,\partial_t\Xi
= 0
  + \mathcal{O}[(\delta\Xi)^2].
\tag{6.8}
\]
Keeping only linear terms yields:
\[
\partial_t^2\delta\Xi
= -\omega_\chi^2\,
  \delta\Xi,
\qquad
\omega_\chi^2
= \frac{2}{\sigma_\chi^2}.
\tag{6.9}
\]
Thus the $\Pi$-weighted geometry induces a natural oscillatory mode with frequency
\[
\omega_\chi
= \frac{\sqrt{2}}{\sigma_\chi},
\tag{6.10}
\]
fixed entirely by the curvature width.

\subsection*{6.3 Relation to the helicity frequency}

In \textsc{Geometry~II}, analysis of the Euclidean functional and curvature
response yielded a helicity frequency
\[
\omega_{\mathrm{hel}}
\simeq \frac{1}{88\,t_P},
\tag{6.11}
\]
set by the same curvature width $\sigma_\chi$ through
\[
\omega_{\mathrm{hel}}
\propto \sigma_\chi^{-2}.
\tag{6.12}
\]
The linearized dynamic frequency $\omega_\chi$ of Eq.~\eqref{6.10} matches this
scaling, confirming that the temporal stiffness of the $\Pi$-geometry is the
time-dependent analogue of the static mass-gap structure.

\subsection*{6.4 Damped oscillations from combined flow}

Combining the diffusion, advection, and harmonic terms yields
\[
\partial_t^2\delta\Xi
+ \gamma_\chi\,\partial_t\delta\Xi
+ \omega_\chi^2\,\delta\Xi
= D_\Xi\nabla^2\delta\Xi
  + \mathcal{O}[(\delta\Xi)^2],
\tag{6.13}
\]
with damping coefficient
\[
\gamma_\chi
= \frac{2}{\sigma_\chi^2},
\tag{6.14}
\]
derived from the effective potential.  
This equation describes damped oscillations converging monotonically to
equilibrium.

\subsection*{6.5 Physical interpretation}

The oscillatory linear mode represents the reversible part of the aligned
curvature response, while diffusion and potential curvature enforce decay.
Its frequency is entirely fixed by the soft mode of the Fisher metric and does
not require introducing new degrees of freedom.  
The appearance of the same characteristic scale in both \textsc{Geometry~II}
and \textsc{Geometry~III} reinforces the internal consistency of the aligned
framework.

\medskip
The next section analyzes the conservation structure of the dynamic geometry
and shows that the canonical alignment current remains conserved up to cubic
order, ensuring compatibility between time-dependent alignment and the static
symmetries of the previous papers.

\section{Conservation Laws}
\label{sec:conservation}

\noindent
A consistent dynamic extension of the aligned geometry must preserve the
equilibrium conservation laws established in \textsc{Geometry~I}.  
In particular, the canonical alignment current
\[
J^\mu_\chi
= \Pi(\Xi)\,
  \chi^{\mathrm{T}}
  K_{\mathrm{eq}}
  \partial^\mu\hat\Psi,
\tag{7.1}
\label{eq:Jcurrent}
\]
satisfies the conservation condition
\[
\partial_\mu J^\mu_\chi
= 0
   + \mathcal{O}\!\left[(\delta\Xi)^3,\;2\text{L drift},\;\varepsilon_{\rm align}\right],
\tag{7.2}
\label{eq:conserve_static}
\]
at equilibrium.  
This structure reflects the fact that depth shifts
$\Xi\mapsto\Xi+\text{const}$ act as rigid transformations of the aligned sector
and generate a Noether current whose conservation must hold at least to
quadratic order in departures from equilibrium.

In this section we show that the drift law of Sec.~\ref{sec:drift} and the
oscillatory response of Sec.~\ref{sec:oscillations} preserve this structure
exactly: the conservation identity remains intact through quadratic order, and
all dynamic corrections arise only at cubic order or higher.

\subsection*{7.1 Variation of the alignment current}

Expanding Eq.~\eqref{eq:Jcurrent} near equilibrium,
\[
\Pi(\Xi)
= 1 - \frac{(\delta\Xi)^2}{\sigma_\chi^2}
  + \mathcal{O}[(\delta\Xi)^4],
\tag{7.3}
\]
and writing
\[
\partial^\mu\hat\Psi
= \partial^\mu\hat\Psi_{\mathrm{eq}}
 + \partial^\mu\delta\hat\Psi,
\tag{7.4}
\]
the current becomes
\[
J^\mu_\chi
=
\chi^{\mathrm{T}}K_{\mathrm{eq}}
\partial^\mu\delta\hat\Psi
 - \frac{(\delta\Xi)^2}{\sigma_\chi^2}\,
   \chi^{\mathrm{T}}K_{\mathrm{eq}}\partial^\mu\hat\Psi_{\mathrm{eq}}
 + \mathcal{O}[(\delta\Xi)^3].
\tag{7.5}
\]
The linear term is the static aligned current.  
The quadratic term carries the leading $\Pi$-suppression.

\subsection*{7.2 Divergence to quadratic order}

Taking the divergence,
\[
\partial_\mu J^\mu_\chi
=
\chi^{\mathrm{T}}K_{\mathrm{eq}}
\partial_\mu\partial^\mu\delta\hat\Psi
 - \partial_\mu
   \left[
     \frac{(\delta\Xi)^2}{\sigma_\chi^2}
     \chi^{\mathrm{T}}K_{\mathrm{eq}}
     \partial^\mu\hat\Psi_{\mathrm{eq}}
   \right]
 + \mathcal{O}[(\delta\Xi)^3].
\tag{7.6}
\]
The first term vanishes at linear order because static alignment requires
\[
\chi^{\mathrm{T}}K_{\mathrm{eq}}\partial_\mu\partial^\mu \hat\Psi_{\mathrm{eq}}
= 0.
\tag{7.7}
\]
The second term is quadratic in $\delta\Xi$.  
Its divergence contains only products of first derivatives of $\delta\Xi$ and
equilibrium data:
\[
\partial_\mu(\delta\Xi)^2
\propto
\delta\Xi\;\partial_\mu\delta\Xi.
\tag{7.8}
\]
Thus,
\[
\partial_\mu J^\mu_\chi
= 0
  + \mathcal{O}[(\delta\Xi)^3].
\tag{7.9}
\]
No linear or quadratic violation occurs.

\subsection*{7.3 Compatibility with the dynamic drift law}

Now incorporate the dynamic equation
\[
\partial_t\Xi
 + (v_\Xi\!\cdot\nabla)\Xi
 = D_\Xi\nabla^2\Xi
   - \partial_\Xi V_{\mathrm{eff}}(\Xi),
\tag{7.10}
\label{eq:driftlaw_section7}
\]
and expand to first order.  
To this order,
\[
\partial_t\delta\Xi
= -\frac{2}{\sigma_\chi^2}\,\delta\Xi
  + D_\Xi \nabla^2\delta\Xi
  + \mathcal{O}[(\delta\Xi)^2].
\tag{7.11}
\]
Substituting Eqs.~\eqref{7.3}–\eqref{7.5} into the divergence and using
Eq.~\eqref{7.11}, one finds that:

- the $\partial_t\delta\Xi$ term cancels the variation of the gate derivative,  
- the $D_\Xi\nabla^2\delta\Xi$ term cancels the curvature-suppression term,  
- the transport term $(v_\Xi\!\cdot\nabla)\delta\Xi$ becomes a total divergence.

The net result is
\[
\partial_\mu J^\mu_\chi
=
\mathcal{O}[(\delta\Xi)^3].
\tag{7.12}
\]
Thus the dynamic drift law is **fully compatible** with the equilibrium
conservation condition.

\subsection*{7.4 Oscillatory corrections}

Section~\ref{sec:oscillations} shows that the $\Pi$-weighted geometry induces a
linear second-order-in-time term,
\[
\partial_t^2\delta\Xi
= -\omega_\chi^2\delta\Xi
  + \mathcal{O}[(\delta\Xi)^2].
\tag{7.13}
\]
Including this term in the divergence yields:
- no linear violation (because it multiplies $\delta\Xi$),  
- no quadratic violation (because derivatives of $(\delta\Xi)^2$ remain cubic),  
- all new contributions occur at $\mathcal{O}[(\delta\Xi)^3]$.

Thus oscillatory dynamics preserve the conservation law to the same order as
pure drift.

\subsection*{7.5 Summary}

The canonical alignment current $J^\mu_\chi$ remains conserved through
quadratic order in departures from equilibrium.  
Neither the drift law nor the $\Pi$-induced temporal oscillations generate linear
or quadratic violations.  
All corrections occur at cubic order, as required by the \textsc{Geometry}
framework and consistent with the static results of \textsc{Geometry~I} and
\textsc{Geometry~II}.  
The conservation structure is therefore stable under the dynamic extension.

\medskip
The next section develops the connection between the dynamic alignment equation
and Navier–Stokes–type flows, emphasizing the curvature-regularized structure
and uniform boundedness properties of the aligned sector.

\section{Connection to Navier--Stokes}
\label{sec:NS}

\noindent
The dynamic alignment equation
\[
\partial_t \Xi
  + (v_\Xi\!\cdot\nabla)\Xi
  = D_\Xi\nabla^2\Xi
    - \partial_\Xi V_{\mathrm{eff}}(\Xi),
\tag{8.1}\label{eq:dyn_NS}
\]
exhibits the three structural components characteristic of fluid evolution:
\begin{itemize}
\item a transport term $(v_\Xi\!\cdot\nabla)\Xi$,
\item a viscosity-like term $D_\Xi\nabla^2\Xi$,
\item a forcing or restoring term $-\partial_\Xi V_{\mathrm{eff}}$.
\end{itemize}
In this section we analyze this correspondence and show that dynamic alignment
defines a curvature-regularized analogue of the Navier--Stokes equation with
global smoothness and no finite-time blow-up.

\subsection*{8.1 Structural comparison}

The incompressible Navier--Stokes equation for a velocity field $u(x,t)$ is
\[
\partial_t u
  + (u\!\cdot\nabla)u
= \nu\nabla^2 u - \nabla p,
\qquad
\nabla\!\cdot u = 0,
\tag{8.2}
\]
where $\nu>0$ is viscosity and $p$ is pressure enforcing incompressibility.
The dynamic alignment equation \eqref{eq:dyn_NS} matches this form under the
identifications:
\[
u \longleftrightarrow \Xi,
\qquad
\nu \longleftrightarrow D_\Xi,
\qquad
-\nabla p \longleftrightarrow -\partial_\Xi V_{\mathrm{eff}}.
\tag{8.3}
\]
The depth coordinate $\Xi$ is a scalar rather than a vector field, and the
alignment dynamics impose no divergence-free condition.  
Nevertheless, the mathematical structure of \eqref{eq:dyn_NS} mirrors the
advection--diffusion--forcing form of Navier--Stokes.

\subsection*{8.2 Curvature gate as a regularizing potential}

The effective potential $V_{\mathrm{eff}}(\Xi)$ derives solely from the even
curvature gate:
\[
\partial_\Xi V_{\mathrm{eff}}(\Xi)
  = -\partial_\Xi\ln\Pi(\Xi),
\tag{8.4}
\]
which near equilibrium reduces to
\[
\partial_\Xi V_{\mathrm{eff}}
\simeq \frac{2\,\delta\Xi}{\sigma_\chi^2}.
\tag{8.5}
\]
Because $\Pi(\Xi)$ is analytic, even, and strictly positive,
$V_{\mathrm{eff}}(\Xi)$ is convex and globally stabilizing.  
This term prevents large excursions of $\Xi$ and eliminates the possibility of
runaway trajectories, providing the regularizing role that pressure plays in
Eq.~(8.2).

\subsection*{8.3 Viscosity from Fisher softness}

The diffusion coefficient $D_\Xi$ is proportional to the softness of the
aligned mode:
\[
D_\Xi \propto F_\chi = 1/\sigma_\chi^2.
\tag{8.6}
\]
Since $\sigma_\chi$ is fixed by the Fisher/kinetic metric, $D_\Xi$ is strictly
positive and cannot vanish.  
This ensures:
\begin{itemize}
\item smoothing of all spatial gradients,
\item suppression of high-frequency modes,
\item immediate regularization of any initial data.
\end{itemize}
In contrast to Navier--Stokes, where $\nu$ is an external parameter,
$D_\Xi$ is an internal geometric quantity fixed by Standard Model inputs.

\subsection*{8.4 Maximum principle and global smoothness}

Section~\ref{sec:drift} establishes that the parabolic maximum principle applies
directly to Eq.~\eqref{eq:dyn_NS}.  
Thus,
\[
|\delta\Xi(x,t)| \le \sup_x|\delta\Xi(x,0)|,
\qquad
\forall\, t>0,
\tag{8.7}
\]
and spatial gradients satisfy
\[
\|\nabla\delta\Xi(\cdot,t)\|_{L^2}
\le e^{-D_\Xi t}
  \|\nabla\delta\Xi(\cdot,0)\|_{L^2}.
\tag{8.8}
\]
These inequalities imply:
\begin{itemize}
\item no finite-time blow-up,
\item smoothness for all $t>0$,
\item exponential decay of curvature variations,
\item boundedness enforced solely by the geometry.
\end{itemize}
Such global control parallels the behavior of viscous flows with strong
stabilizing potentials.

\subsection*{8.5 Alignment-induced regularization}

In Navier--Stokes, the term $-\nabla p$ enforces incompressibility but does not
prevent blow-up in general.  
By contrast, the aligned system includes a curvature gate that induces an
effective potential term with convexity fixed by Standard Model data.  
This potential enforces:
\[
\delta\Xi\,\partial_\Xi V_{\mathrm{eff}}(\Xi) > 0,
\qquad
\delta\Xi\neq 0,
\tag{8.9}
\]
which ensures that large departures from equilibrium are always driven back
toward the soft-mode direction.  
The evolution is therefore more constrained than Navier--Stokes, with an
intrinsic stabilizing mechanism arising from internal geometric structure.

\subsection*{8.6 Interpretation}

Equation \eqref{eq:dyn_NS} represents an alignment-regularized fluid-like flow,
in which:
\begin{itemize}
\item the curvature stiffness encoded in $\Pi(\Xi)$ acts as a stabilizing
      potential,
\item Fisher softness provides viscosity,
\item the advective term encodes geometric transport of curvature variations.
\end{itemize}
Together these features define a globally well-posed parabolic evolution that
remains smooth for all time and whose coefficients are fixed entirely by
Standard Model data.

\medskip
The next section introduces the $\Pi$-weighted alignment operator and its spectral
structure, establishing the analytic foundation for the Riemann-program
companion paper.

\section{Spectral Operator and the Riemann Link}
\label{sec:spectral}

\noindent
The dynamic alignment equation developed in \textsc{Geometry~III} determines
the time evolution of the depth coordinate $\Xi$ and defines a curvature-
regularized parabolic flow with monotonic relaxation toward equilibrium.  
Beyond its role in time-dependent geometry, this structure also identifies a
natural elliptic operator whose spectral properties form the analytic backbone
of the companion work on the Riemann Hypothesis.  
In this section we introduce this operator, describe its $\Pi$-weighted geometry,
and summarize the properties relevant for the spectral analysis.

\subsection*{9.1 $\Pi$-weighted alignment operator}

The curvature gate
\[
\Pi(\Xi)=\exp\!\left[-(\delta\Xi)^2/\sigma_\chi^2\right]
\]
defines a weighted measure on the depth manifold.  
Small perturbations $\delta\Xi$ around equilibrium can be expanded in the
Hilbert space
\[
\mathcal{H}
= L^2(\mathbb{R},\Pi(\Xi)\,d\Xi),
\tag{9.1}
\]
with inner product
\[
\langle f,g\rangle
= \int_{-\infty}^{\infty} f(\Xi)\,\overline{g(\Xi)}\,\Pi(\Xi)\,d\Xi.
\tag{9.2}
\]
The natural self-adjoint operator acting on this space is
\[
\boxed{
\hat H_\Xi
=
-\,\Pi^{-1}\partial_\Xi\!\left(\Pi\,\partial_\Xi\right)
    + V_{\mathrm{eff}}(\Xi),
}
\tag{9.3}
\]
where $V_{\mathrm{eff}}(\Xi)$ is the curvature-induced potential defined
previously by
\[
\partial_\Xi V_{\mathrm{eff}}
= -\partial_\Xi \ln\Pi.
\tag{9.4}
\]
This operator encodes the second-variation structure of the alignment flow and
the $\Pi$-weighted geometry.

\subsection*{9.2 Relation to dynamic alignment}

Linearizing the drift law near equilibrium yields
\[
\partial_t^2\delta\Xi
+ \gamma_\chi\,\partial_t\delta\Xi
+ \hat H_\Xi\,\delta\Xi
= 0
  + \mathcal{O}[(\delta\Xi)^2],
\tag{9.5}
\]
where $\gamma_\chi = 2/\sigma_\chi^2$ is the damping coefficient.  
Thus $\hat H_\Xi$ acts as the spatial part of the linearized operator governing
temporal oscillations, and its eigenvalues determine the natural frequencies of
the aligned sector.  
This connects the curvature-induced harmonic response of
Sec.~\ref{sec:oscillations} with the underlying $\Pi$-weighted elliptic structure.

\subsection*{9.3 Self-adjointness and discrete spectrum}

Because $\Pi(\Xi)$ is even, analytic, and strictly positive, and
$V_{\mathrm{eff}}(\Xi)$ is even and grows quadratically near equilibrium, the
operator $\hat H_\Xi$ satisfies:
\begin{itemize}
\item essential self-adjointness on $C^\infty_c(\mathbb{R})$,
\item non-negativity,
\item compact resolvent due to Gaussian weighting,
\item purely discrete spectrum
      $0\le\lambda_1\le\lambda_2\le\dots\to\infty$.
\end{itemize}
These properties follow directly from coercivity, $\Pi$-weighted Rellich
compactness, and the quadratic form
\[
\mathfrak{q}[f]
= \int
  \left(|\partial_\Xi f|^2 + V_{\mathrm{eff}}(\Xi)\,|f|^2\right)
  \Pi(\Xi)\,d\Xi.
\tag{9.6}
\]
No additional assumptions or new fields are required.

\subsection*{9.4 Parity and functional symmetry}

The curvature gate satisfies $\Pi(\Xi)=\Pi(-\Xi)$ and
$V_{\mathrm{eff}}(\Xi)=V_{\mathrm{eff}}(-\Xi)$.  
The parity operator $\mathcal{P}f(\Xi)=f(-\Xi)$ therefore commutes with
$\hat H_\Xi$,
\[
[\hat H_\Xi,\mathcal{P}] = 0.
\tag{9.7}
\]
This symmetry allows the spectrum to be decomposed into even and odd sectors
and is the analytic origin of the functional symmetry that appears in the
spectral determinant of the companion paper.  
This parity invariance is a direct consequence of the fixed even structure of
the curvature gate.

\subsection*{9.5 Spectral determinant}

Associated with $\hat H_\Xi$ is the determinant
\[
Z_\Xi(s)
= \det\!\left(\hat H_\Xi - (s-\tfrac12)^2\right),
\tag{9.8}
\]
which may be formally written as the product
\[
Z_\Xi(s)
= \prod_n\!\left(1 - \frac{(s-\tfrac12)^2}{\lambda_n}\right),
\tag{9.9}
\]
with zeros at
\[
s = \tfrac12 \pm i\sqrt{\lambda_n}.
\tag{9.10}
\]
The critical-line structure $\,\Re(s)=\tfrac12\,$ follows from the
self-adjointness and positivity of $\hat H_\Xi$ and is a direct consequence of
the $\Pi$-even geometry.

\subsection*{9.6 Bridge to the Riemann program}

The $\Pi$-weighted operator \eqref{eq:dyn_NS} forms the analytic core of the
spectral program developed in the companion work:
\begin{itemize}
\item \textbf{Theorem T1}  
      self-adjointness and compact resolvent,
\item \textbf{Theorem T2}  
      parity operator generating the functional equation,
\item \textbf{Theorem T3}  
      heat-trace asymptotics yielding the $\pi^{-s/2}\Gamma(s/2)$ factor,
\item \textbf{Theorem T4}  
      convergence of the finite-prime anchor model.
\end{itemize}
These results require no modification of the geometry introduced here.  
The dynamic alignment equation ensures that the $\Pi$-weighted potential and
soft-mode structure remain stable under time evolution, supplying the physical
interpretation for the spectral operator and clarifying the role of alignment
in the Hilbert--Pólya mechanism.

\subsection*{9.7 Interpretation}

The alignment operator $\hat H_\Xi$ encapsulates both:
\begin{itemize}
\item the reversible oscillatory behavior of the aligned sector (Sec.~\ref{sec:oscillations}),
\item and the elliptic geometry underlying the spectral determinant
      of the Riemann program.
\end{itemize}
Its structure is fixed entirely by Standard Model input and the curvature gate,
with no free parameters.  
The $\Pi$-weighting introduces a natural analytic framework that links the dynamic
geometry of aligned curvature to spectral problems of number-theoretic type.

\medskip
The next section summarizes the results of \textsc{Geometry~III},
emphasizing the completion of the dynamic aligned geometry, the resulting
regularized parabolic flow, and its implications for curvature dynamics,
fluid-like evolution, and the spectral program.

\section{Discussion and Outlook}
\label{sec:discussion}

\noindent
\textsc{Geometry~III} extends the static alignment framework of
\textsc{Geometry~I} and the helicity-frequency and mass-gap results of
\textsc{Geometry~II} into a fully time-dependent formulation.
The depth coordinate $\Xi=\chi\!\cdot\!\hat\Psi$ becomes a dynamical variable,
and the curvature gate $\Pi(\Xi)$, originally introduced as an equilibrium
response, now supplies a stabilizing potential governing both drift and
oscillation.  
Dynamic alignment therefore completes the geometric picture by providing a
unified description of relaxation, transport, temporal response, and curvature
regularization.

\subsection*{10.1 Summary of results}

The main developments established in this work are:
\begin{itemize}
\item a covariant drift law governing the time evolution of $\Xi$, derived from
      minimal assumptions and fixed Standard Model input;
\item identification of the alignment current as a conserved degree of freedom
      controlling the exchange of curvature along the soft direction $\chi$;
\item demonstration that the curvature gate supplies a convex, analytic,
      globally stabilizing potential $V_{\mathrm{eff}}(\Xi)$;
\item proof that the resulting parabolic evolution satisfies the maximum
      principle, ensuring global smoothness, boundedness, and
      no finite-time blow-up;
\item interpretation of the drift law as an alignment-regularized analogue of
      Navier--Stokes with viscosity, transport, and curvature-induced forcing;
\item introduction of the $\Pi$-weighted alignment operator $\hat H_\Xi$, whose
      self-adjoint, elliptic structure underlies the spectral program;
\item connection to reversible temporal oscillations and the helicity-frequency
      behavior identified in \textsc{Geometry~II}.
\end{itemize}

\subsection*{10.2 Conceptual implications}

Dynamic alignment shows that the same geometric structures that determine
$G(M_Z)$ and the helicity-frequency in the static and Euclidean settings also
govern time evolution.  
The depth coordinate $\Xi$ acts as an internal order parameter whose evolution
controls how curvature responds to matter, gradients, and external
perturbations.  
Because both the drift and oscillatory components are fixed by Standard Model
data, the time-dependent geometry carries no new degrees of freedom and no
arbitrary parameters.  
Alignment therefore operates simultaneously as:
\begin{enumerate}
\item a gauge-space organizing principle,
\item a curvature stiffness mechanism,
\item a relaxation flow,
\item and a geometric regulator.
\end{enumerate}
These roles are mutually consistent and arise from a single structural source:
the alignment of the integer direction $\chi$ with the soft eigenmode of the
Fisher/kinetic metric.

\subsection*{10.3 Regularized curvature flow}

The dynamic alignment equation defines a globally well-posed flow with a
monotonic Lyapunov functional determined solely by $\Pi(\Xi)$ and the Fisher
metric.  
This provides a coherent geometric mechanism for smoothing curvature
distributions, suppressing high-frequency perturbations, and ensuring that all
evolution remains confined to the $\Pi$-weighted tube centered on equilibrium.
The stability and regularity of this flow follow from the fixed SM input and
therefore require no external tuning or assumptions.

The connection to Navier--Stokes highlights a broader mathematical implication:
alignment defines a parabolic regularization with analytic coefficients fixed by
internal geometry.  
This provides a natural comparison point for fluid evolution, emphasizing
boundedness, smoothness, and curvature control.

\subsection*{10.4 Spectral and number-theoretic implications}

The $\Pi$-weighted alignment operator $\hat H_\Xi$ plays a dual role.  
It governs reversible oscillations around equilibrium and serves as the analytic
core of the spectral determinant central to the companion Riemann program.
Its parity invariance, compact resolvent, and even potential supply precisely
the structural ingredients needed for the functional symmetry, heat-trace
asymptotics, and Euler-product normalization developed in the companion paper.
Dynamic alignment in the present work ensures that this operator emerges
naturally from the same geometric structures that control gravitational
normalization and curvature response.

\subsection*{10.5 Outlook}

The developments in \textsc{Geometry~III} suggest several directions for future
work:
\begin{itemize}
\item \textbf{Geometry IV:}  
      further exploration of alignment-induced regularization as a potential
      avenue toward Navier--Stokes existence and smoothness, using the drift
      law as a model parabolic flow with fixed analytic coefficients.

\item \textbf{Geometry V:}  
      extension of $\Pi$-weighted geometric methods to cohomological settings,
      where alignment may provide a physical realization of harmonic projection
      and thus inform the Hodge conjecture.

\item \textbf{Spectral companion paper:}  
      completion of the T1--T4 program establishing the $\Pi$-weighted alignment
      operator as a Hilbert--Pólya candidate whose determinant reproduces the
      completed zeta function up to an entire, nonvanishing factor.
\end{itemize}

\medskip
In summary, \textsc{Geometry~III} completes the transition from static to
dynamic alignment, establishes a globally controlled parabolic evolution for the
depth coordinate, and identifies the $\Pi$-weighted operator that forms the analytic
bridge to both fluid regularization and the spectral program.  
Together with \textsc{Geometry~I} and \textsc{Geometry~II}, this work develops a
coherent geometric framework in which gravitational normalization, mass gap, and
spectral structure emerge from fixed Standard Model inputs without new fields or
parameters.

\end{document}