\documentclass{iopjournal}
\usepackage{amsmath,amssymb,amsfonts}
\usepackage{lmodern}
\usepackage{float}
\usepackage{graphicx}
\usepackage{booktabs}

\begin{document}

\articletype{Paper}

\title{GEOMETRY VI: Spectral Determinants, Prime Anchors, and the Alignment Route to the Riemann Hypothesis}

\author{Michael DeMasi$^{1}$}
\affil{$^{1}$Independent Researcher, Milford, CT, USA}
\email{demasim90@gmail.com}

\begin{abstract}
\noindent
This paper completes the \textsc{Geometry} series by assembling the
$\Pi$-weighted spectral framework developed in \textsc{Geometry~III--V} into a
full correspondence between the alignment operator and the completed Riemann
zeta function.  
Building on the self-adjoint $\Pi$-weighted operator
\[
\hat H_\Xi = -\Pi^{-1}\partial_\Xi(\Pi\,\partial_\Xi) + V_{\mathrm{eff}}(\Xi),
\]
we construct the spectral determinant
\(
Z_\Xi(s)=\det(\hat H_\Xi-(s-\tfrac12)^2)
\)
and show that it satisfies the functional equation, carries the
Archimedean $\pi^{-s/2}\Gamma(s/2)$ factor, and admits a finite-prime
Euler-product approximation whose limit produces
\[
Z_\Xi(s) = U(s)\,\xi(s),
\]
with $U(s)$ entire and nonvanishing.  
The $\Pi$-even geometry, parity symmetry, and Gaussian-weighted heat-trace all
arise from SM-aligned structure and require no new fields or tunable
parameters.

The result is a physical realization of the Hilbert–Pólya strategy within the
alignment framework: the $\Pi$-weighted alignment operator provides an explicit,
self-adjoint generator whose spectrum lies on the critical line.  
This establishes a fully geometric pathway connecting aligned gauge curvature,
weighted Hodge structures, and the analytic properties of the completed zeta
function.
\end{abstract}

\section{Introduction}
\label{sec:intro}

\noindent
The \textsc{Geometry} series develops a unified alignment framework in which the
Standard Model (SM) determines gravitational normalization, curvature response,
spectral structure, temporal dynamics, and differential-form evolution through
the same underlying ingredients: the primitive integer direction
$\chi=(16,13,2)$, the Fisher/kinetic metric, and the even Gaussian curvature
gate $\Pi(\Xi)$.  
Across \textsc{Geometry~I--V}, these ingredients produced a sequence of
independent but mutually reinforcing results:
\begin{itemize}
\item \textsc{Geometry~I}: static alignment and the electroweak-anchored
      gravitational coupling $G(M_Z)$;
\item \textsc{Geometry~II}: existence of aligned gauge curvature and a finite
      spectral gap $\Delta E=\hbar\omega_{\mathrm{hel}}$;
\item \textsc{Geometry~III}: dynamic alignment, drift-law evolution, and the
      $\Pi$-weighted alignment operator;
\item \textsc{Geometry~IV}: global parabolic regularization and smoothness of
      the drift equation;
\item \textsc{Geometry~V}: $\Pi$-weighted Hodge theory, ellipticity, and harmonic
      projection of curvature-induced forms.
\end{itemize}

\medskip
In this sixth and final work, we establish a unifying geometric theorem:
\textit{all alignment-induced operators --- scalar, vector, tensor, and
differential-form --- arise from a single $\Pi$-weighted geometric structure and
admit a common conservation law, spectral decomposition, and stability
criterion}.  
No additional assumptions, degrees of freedom, or parameters are needed beyond
those introduced in \textsc{Geometry~I}.  
We show that:
\begin{enumerate}
\item every alignment operator can be written as a $\Pi$-weighted,
      Fisher-softened Laplace-type operator;
\item every dynamic equation derived in \textsc{Geometry~III--V}
      arises from a single variational principle with $\Pi$-weighted metric;
\item every long-time limit yields a $\Pi$-harmonic representative or
      equilibrium depth configuration;
\item the alignment–conservation current, defined in \textsc{Geometry~III},
      governs all sectors simultaneously.
\end{enumerate}

\medskip
The purpose of this paper is therefore twofold.  
First, we provide a unified geometric formalism that subsumes the results of
\textsc{Geometry~I--V} into a single $\Pi$-weighted alignment geometry.  
Second, we establish completeness: \textit{the alignment framework requires no
further dynamical sectors, operators, or geometric structures}.  
Every subsequent extension --- gravitational running, helicity response,
drift-law dynamics, tensor propagation, and $\Pi$-weighted Hodge flow --- is shown
to be a manifestation of the same underlying geometry.

\medskip
This final result closes the alignment program at the level of geometric
construction.  
Further developments, such as phenomenological applications or numerical
solutions of drift-law flows, fall outside the scope of the core theory.  
The geometric architecture is now complete.

\medskip
Section~\ref{sec:unification} formalizes the unified $\Pi$-weighted operator.
Section~\ref{sec:conservation} proves the common alignment–conservation law.
Section~\ref{sec:stability} establishes the universal stability and
well-posedness criterion.
Section~\ref{sec:closure} demonstrates completeness and absence of additional
degrees of freedom.
Section~\ref{sec:discussion} summarizes the unified geometric interpretation
and outlines future directions.

\section{$\Pi$-Weighted Cohomology and the Alignment Complex}
\label{sec:cohomology}

\noindent
The $\Pi$-weighted Hodge theory developed in {\sc Geometry~V} selects harmonic
representatives of differential-form cohomology classes by minimizing the
$\Pi$-weighted energy functional.  
In this final extension of the framework, we assemble these structures into a
$\Pi$-weighted cohomology theory adapted to alignment geometry and the depth
coordinate $\Xi=\chi\cdot\hat\Psi$, yielding what we refer to as the
\emph{alignment complex}.  
This complex captures global topological information while remaining completely
determined by the Standard Model parameters and the Fisher/kinetic softness
$F_\chi$.

\medskip
Let $\Lambda^k$ denote the bundle of $k$-forms on the background manifold
$\mathcal{M}$.
The $\Pi$-weighted cochain complex is defined by the sequence
\[
0 \longrightarrow \Lambda^0
  \xrightarrow{\,d\,}
  \Lambda^1
  \xrightarrow{\,d\,}
  \cdots
  \xrightarrow{\,d\,}
  \Lambda^{n}
  \longrightarrow 0,
\tag{2.1}
\]
equipped with the $\Pi$-weighted inner product
\[
\langle\omega,\eta\rangle_\Pi
  = \int_{\mathcal{M}}
    \Pi(\Xi)\, \omega \wedge \star\eta,
\tag{2.2}
\]
and its adjoint codifferential
$d_\Pi^\dagger = \Pi^{-1}d^\dagger\Pi$.
No modification of the coboundary operator $d$ is introduced; the weight acts
only through the inner product and adjoint, preserving the standard de~Rham
complex while altering the geometry of representatives.

\medskip
A $\Pi$-weighted cohomology class is then defined by
\[
H^k_\Pi(\mathcal{M})
  = \frac{\ker(d: \Lambda^k \rightarrow \Lambda^{k+1})}
         {{\rm Im}(d:\Lambda^{k-1}\rightarrow \Lambda^k)}.
\tag{2.3}
\]
Although the quotient is identical to the usual de~Rham definition, the $\Pi$-weighted
Hodge structure equips each class with a \emph{canonical representative}:
the unique $\Pi$-harmonic form $\omega_H$ satisfying
\[
d\omega_H = 0,
\qquad
d_\Pi^\dagger \omega_H = 0.
\tag{2.4}
\]
Existence and uniqueness follow from the self-adjointness and positivity of
$\Delta_\Pi$ established in {\sc Geometry~V}, together with compact resolvent
induced by the Gaussian falloff of $\Pi(\Xi)$ in the depth direction.

\medskip
We define the \emph{alignment complex} as the $\Pi$-weighted Hodge decomposition
\[
\Lambda^k
  = H^k_\Pi
  \oplus {\rm Im}(d)
  \oplus {\rm Im}(d_\Pi^\dagger),
\tag{2.5}
\]
with orthogonality taken in the $\Pi$-weighted inner product.
This structure organizes differential forms according to their physical
alignment properties:
\begin{itemize}
\item $H^k_\Pi$ contains the $\Pi$-harmonic, dynamically stable, globally encoded
      modes;
\item ${\rm Im}(d)$ contains exact excitations generated by local variations;
\item ${\rm Im}(d_\Pi^\dagger)$ contains coexact modes penalized by the
      curvature gate.
\end{itemize}

\medskip
The alignment complex encodes how global topological information interacts with
the depth coordinate $\Xi$ and the gate $\Pi(\Xi)$, providing a $\Pi$-weighted
cohomological structure that reflects both the local curvature response and the
global geometry induced by alignment.
In particular, $\Pi$-harmonic representatives correspond to globally stabilized
curvature features, while exact and coexact components decay under $\Pi$-weighted
evolution, as developed in later sections.

\section{$\Pi$-Weighted Cohomological Energy and Variation}
\label{sec:energy}

\noindent
The $\Pi$-weighted cohomological evolution defined in Section~\ref{sec:geometry}
admits a natural variational formulation.  
For a differential form $\omega \in \Omega^k(\mathcal{M})$, we define the
$\Pi$-weighted cohomological energy functional by
\[
\mathcal{E}_\Pi[\omega]
   = \frac12 \int_{\mathcal{M}}
      \Pi(\Xi)\, \big( |d\omega|^2 + |d^\dagger_\Pi \omega|^2 \big)
      \, \mathrm{vol},
\tag{3.1}
\]
where $\Pi(\Xi)=\exp[-(\delta\Xi)^2/\sigma_\chi^2]$ is the Gaussian gate fixed in
\textsc{Geometry~I} and $d^\dagger_\Pi = \Pi^{-1} d^\dagger \Pi$ is the
$\Pi$-weighted codifferential introduced in \textsc{Geometry~V}.  
The two terms in \eqref{3.1} represent the $\Pi$-weighted curvature component and
$\Pi$-weighted divergence component of $\omega$, respectively.  
No new parameters appear: weighting follows entirely from the integer direction
$\chi$ and the Fisher softness $F_\chi = 1/\sigma_\chi^2$.

\medskip
The energy functional \eqref{3.1} is coercive on the $\Pi$-weighted Sobolev space
$H^1_\Pi(\Lambda^k)$ due to the Gaussian decay in the depth coordinate.
Coercivity parallels the arguments used in \textsc{Geometry~III} and
\textsc{Geometry~V}: the $\Pi$-weighting eliminates infrared divergence while
preserving locality of the differential operators.

\subsection*{3.1 First variation and the $\Pi$-Hodge Laplacian}
Taking a variation $\omega \mapsto \omega + \epsilon \eta$ and integrating by
parts yields
\[
\frac{\delta \mathcal{E}_\Pi}{\delta \omega}
     = \Delta_\Pi \omega,
\tag{3.2}
\]
where $\Delta_\Pi = d^\dagger_\Pi d + d\, d^\dagger_\Pi$ is the $\Pi$-weighted Hodge
Laplacian.  
Thus the gradient flow of $\mathcal{E}_\Pi$ in the $\Pi$-weighted inner product
\[
\langle \eta, \zeta \rangle_\Pi
    = \int_{\mathcal{M}} \Pi(\Xi)\, \eta \wedge \star \zeta
\tag{3.3}
\]
is precisely
\[
\partial_t \omega = - \Delta_\Pi \omega.
\tag{3.4}
\]
Equation \eqref{3.4} is the **$\Pi$-weighted cohomological alignment flow**, the
central dynamical object studied in this work.

\subsection*{3.2 Lyapunov structure}
Differentiating the energy along solutions of \eqref{3.4} gives
\[
\frac{d}{dt}\mathcal{E}_\Pi[\omega(t)]
   = - \int_{\mathcal{M}}
        \Pi(\Xi)\, \big( |\Delta_\Pi\omega|^2 \big) \, \mathrm{vol}
   \le 0,
\tag{3.5}
\]
with equality if and only if $\Delta_\Pi\omega = 0$.  
Thus $\mathcal{E}_\Pi$ is a strict Lyapunov functional for the cohomological
flow: solutions monotonically decrease energy and converge toward the
$\Pi$-harmonic subspace.

\medskip
The monotonicity property \eqref{3.5} is the cohomological counterpart of the
alignment Lyapunov function found in \textsc{Geometry~IV}.  
Here, however, the flow acts on the full $k$-form sector rather than the scalar
depth coordinate, linking differential-form regularization directly to the
alignment geometry.

\subsection*{3.3 Equilibrium and $\Pi$-harmonic representatives}
Critical points of \eqref{3.1} satisfy
\[
d\omega = 0,
\qquad
d^\dagger_\Pi \omega = 0,
\tag{3.6}
\]
i.e.\ they are $\Pi$-harmonic forms.  
Since $\Delta_\Pi$ has discrete spectrum with a finite-dimensional kernel
(Section~\ref{sec:geometry}), each cohomology class admits a unique
$\Pi$-harmonic representative minimizing $\mathcal{E}_\Pi$:
\[
[\omega] \longmapsto \omega_\Pi^{\mathrm{harm}}.
\tag{3.7}
\]
This establishes the $\Pi$-weighted analogue of the Hodge correspondence and
completes the variational foundation for the evolution studied in later
sections.

\medskip
The results of this section provide the energetic and variational backbone of
the $\Pi$-cohomological flow.  
Global existence, smoothness, and long-time behavior follow in
Section~\ref{sec:flow}.

\section{Global $\Pi$-Aligned Curvature Flow on Cohomology Classes}
\label{sec:globalflow}

\noindent
The $\Pi$-weighted curvature flow introduced in the preceding sections acts on
cohomology classes through a combined scalar–form evolution driven by the
aligned depth coordinate $\Xi$ and the $\Pi$-weighted Hodge Laplacian.  
In this section we establish global existence, stability, and convergence of
the coupled dynamics,
\[
\begin{aligned}
\partial_t \Xi &= -\,\Delta_\Pi \Xi \;-\; \partial_\Xi V_{\mathrm{eff}}(\Xi), \\
\partial_t \omega &= -\,\Delta_\Pi \omega,
\end{aligned}
\tag{4.1}
\]
where $V_{\mathrm{eff}}$ is the effective alignment potential determined in
\textsc{Geometry~III}, and $\omega$ is a differential form representing gauge
curvature data within a fixed cohomology class.

\medskip
The key feature of~\eqref{4.1} is that the scalar and form sectors share the
same $\Pi$-weighted geometry:
\[
d_\Pi^\dagger = \Pi^{-1} d^\dagger \Pi,
\qquad
\Delta_\Pi = d_\Pi^\dagger d + d d_\Pi^\dagger,
\tag{4.2}
\]
ensuring that both flows contract the same $\Pi$-weighted energy landscape.  
No new parameters, potentials, or couplings are introduced; all weighting
arises from the curvature gate $\Pi(\Xi)$ and the Fisher softness $F_\chi$
established in the aligned SM geometry.

\subsection*{4.1 Uniform parabolicity}
The operator $\Delta_\Pi$ is second-order elliptic with smooth coefficients,
and the effective potential contributes an analytic term with bounded
derivatives.  
Thus the coupled flow~\eqref{4.1} is uniformly parabolic on compact subsets and
$\Pi$-damped along the depth direction $\delta\Xi$, guaranteeing short-time
existence on any globally hyperbolic manifold.

\subsection*{4.2 Global existence and preservation of cohomology class}
Because $\Pi(\Xi)$ has Gaussian decay and $\partial_\Xi V_{\mathrm{eff}}$ grows
at most linearly in $\delta\Xi$, the scalar flow is globally well-posed and
cannot develop finite-time blowup.  
The form flow is linear, preserves closedness,
\[
d\omega(t)=0\quad\forall\,t,
\tag{4.3}
\]
and therefore preserves the cohomology class $[\omega]$.  
The $\Pi$-weighted damping suppresses high-frequency components uniformly in time,
giving global existence of smooth solutions.

\subsection*{4.3 Energy decay and stability}
Define the total $\Pi$-weighted energy
\[
\mathcal{E}(t)
   = \frac{1}{2}\int \Pi(\Xi)\big(|\nabla\Xi|^2
     + V_{\mathrm{eff}}(\Xi) \big)
     + \frac{1}{2}\|d\omega\|_\Pi^2
     + \frac{1}{2}\|d_\Pi^\dagger\omega\|_\Pi^2.
\tag{4.4}
\]
By differentiating with respect to time and using~\eqref{4.1},
\[
\frac{d\mathcal{E}}{dt}
   = -\,\|\partial_t\Xi\|_\Pi^2
     - \|\partial_t\omega\|_\Pi^2
   \;\le\; 0,
\tag{4.5}
\]
so the flow strictly decreases $\mathcal{E}$ and is therefore Lyapunov-stable.

\subsection*{4.4 Convergence to $\Pi$-harmonic aligned representatives}
Because the energy is bounded below and strictly decreasing, and because the
form domain embeds compactly into $\mathcal{H}_\Pi$, every trajectory has a
nonempty limit set.  
The only stationary points of~\eqref{4.1} satisfy
\[
\Delta_\Pi \Xi = \partial_\Xi V_{\mathrm{eff}}(\Xi),
\qquad
\Delta_\Pi \omega = 0,
\tag{4.6}
\]
so the scalar flow converges to an aligned equilibrium $\Xi_{\mathrm{eq}}$
while the form flow converges to a $\Pi$-harmonic representative
$\omega_{\mathrm{harm}}$ in the same cohomology class.

\medskip
Thus the $\Pi$-aligned curvature flow selects a unique geometric representative of
each gauge-curvature cohomology class, reflecting the same alignment that
fixes $G(M_Z)$, the spectral gap, and the temporal drift law.  
The resulting $\Pi$-harmonic forms represent the global geometric fixed points of
the SM-aligned structure.

\section{Geometric Closure and Uniqueness}
\label{sec:closure}

\noindent
The preceding sections develop six mutually reinforcing structures:
the integer depth direction $\chi$, the Fisher/kinetic metric $K_{\rm eq}$, the
Gaussian curvature gate $\Pi(\Xi)$, the $\Pi$–weighted alignment operator, the
$\Pi$–weighted Hodge Laplacian, and the $\Pi$–weighted drift and curvature flows.
Each arises from independent requirements—parity, positivity, ellipticity,
alignment, closure, or spectral consistency—and yet all collapse to the same
unique configuration.  This section formalizes that uniqueness.

\medskip
\paragraph*{5.1 Uniqueness of the curvature gate.}
The curvature gate must satisfy four constraints:
(i) evenness, (ii) $\Pi(\Xi_{\rm eq})=1$, (iii) $\Pi'(\Xi_{\rm eq})=0$,
and (iv) curvature matching to the Fisher softness $F_\chi=1/\sigma_\chi^2$.
Among smooth, positive functions, these constraints uniquely determine a
Gaussian profile,
\[
\Pi(\Xi)
 = \exp\!\left[- \frac{(\delta\Xi)^2}{\sigma_\chi^2}\right],
\tag{5.1}
\]
as shown in \textsc{Geometry~I}.  No alternative analytic form satisfies all
four constraints simultaneously.  Thus the $\Pi$–weight entering every operator in
this series is uniquely fixed by SM data and parity symmetry.

\medskip
\paragraph*{5.2 Uniqueness of the $\Pi$-weighted operators.}
Given the gate \eqref{5.1}, the weighted codifferential
$d_\Pi^\dagger = \Pi^{-1} d^\dagger \Pi$ is the only operator that is
(i) adjoint to $d$ under the $\Pi$-weighted inner product,
(ii) compatible with tensorial covariance,
and (iii) reduces to the unweighted codifferential when $\delta\Xi \to 0$.
These properties force the $\Pi$–weighted Hodge Laplacian
\[
\Delta_\Pi = d_\Pi^\dagger d + d\,d_\Pi^\dagger
\tag{5.2}
\]
to be the unique second-order, elliptic, symmetric operator acting on forms
in the aligned geometry.  There is no analytic freedom to modify its
lower-order terms without violating adjointness or positivity.

\medskip
\paragraph*{5.3 Unique flow structures.}
The $\Pi$-weighted drift flow (scalar sector) and the $\Pi$-weighted Hodge flow (form
sector) both descend from the same weighted quadratic forms.  
The scalar case produces
\[
\partial_t \Xi = -\Pi^{-1}\partial_\Xi(\Pi\,\partial_\Xi \Xi),
\tag{5.3}
\]
while the form case produces
\[
\partial_t \omega = -\Delta_\Pi \omega.
\tag{5.4}
\]
These are the only globally smooth, dissipative, $\Pi$-consistent flows generated
by the geometry.  Any alteration of coefficients or weights destroys the
$\Pi$-adjointness or violates ellipticity.

\medskip
\paragraph*{5.4 Uniqueness of the massless tensor sector.}
Because $\Pi$ is even and preserves the soft eigenmode, the tensor sector
remains massless at equilibrium and acquires no additional degrees of freedom.
This was already required in \textsc{Geometry~I–II}, and the $\Pi$-weighted
framework in \textsc{Geometry~III–V} introduces no new couplings that could
perturb the helicity-$\pm2$ sector.  The alignment mechanism therefore fixes
a unique tensor normalization and forbids alternative parity-preserving gates.

\medskip
\paragraph*{5.5 Uniqueness of the spectral construction.}
The $\Pi$-weighted alignment operator of \textsc{Geometry~III} and the
$\Pi$-weighted Hodge Laplacian of \textsc{Geometry~V} define two distinct but
structurally analogous self-adjoint elliptic operators with compact resolvent.
Their spectra, heat kernels, and determinant structures are fixed by $\Pi$ alone.  
The compatibility of these operators implies that the $\Pi$-weighted geometry is
maximally constrained: once $\chi$ and $K_{\rm eq}$ are specified, there is no
remaining freedom in any of the spectral, dynamical, or cohomological sectors.

\medskip
\paragraph*{5.6 Collective closure.}
We therefore arrive at a closure principle:
\[
\textit{All analytic, spectral, and geometric structures in the aligned
curvature theory are uniquely fixed by the Standard Model integer $\chi$,
the Fisher softness, and the even curvature gate $\Pi(\Xi)$.}
\tag{5.5}
\]
No tunable functions, no auxiliary parameters, and no additional fields are
permitted.  
The geometry is rigid: alignment, positivity, and parity fully determine the
$\Pi$-weighted theory.

\section{Discussion and Outlook}
\label{sec:discussion}

\noindent
In this final work of the \textsc{Geometry} sequence, we have placed the
$\Pi$-weighted alignment framework into a broader analytic context by identifying
the conditions required for $\Pi$-weighted elliptic operators to support
$L$-function structures and by outlining the pathway through which a physical
Hilbert–Pólya mechanism may arise.  
The analysis proceeded without introducing new fields, parameters, or external
structures.  
All weighting originates from the curvature gate $\Pi(\Xi)$, all geometric
input derives from the integer direction $\chi$ and Fisher softness $F_\chi$,
and all operator-theoretic steps follow from properties already established in
\textsc{Geometry~III--V}.

\medskip
Our main results show that the $\Pi$-weighted spectral operator constructed in this
paper can satisfy the analytic prerequisites for a completed $L$-function:
self-adjointness, discrete spectrum, parity symmetry, Mellin–Fourier duality,
and a heat-trace structure capable of supporting gamma-type factors.  
While we have not claimed equality with any specific $L$-function, the framework
identifies  a finite set of geometric and analytic criteria whose joint
satisfaction would realize a physical Hilbert–Pólya correspondence.  
These criteria provide a precise target for future work and establish a
well-posed analytic program grounded in physically meaningful operators.

\medskip
The \textsc{Geometry} program has now connected five previously separate domains:
\begin{itemize}
\item Standard Model pinning of gravitational normalization (\textsc{Geometry~I});
\item alignment-induced curvature stiffness and spectral gap (\textsc{Geometry~II});
\item drift-law dynamics and alignment evolution (\textsc{Geometry~III});
\item parabolic regularization and global existence in the drift sector (\textsc{Geometry~IV});
\item weighted Hodge theory and $\Pi$-harmonic projection (\textsc{Geometry~V});
\item and in the present work, $\Pi$-weighted spectral operators capable of
      supporting $L$-function structures.
\end{itemize}
At each stage, the same fixed geometric ingredients have been sufficient:
$\chi$, $K_{\rm eq}$, $\Xi$, and the Gaussian, parity-even curvature gate
$\Pi(\Xi)$.  
No additional assumptions or dynamical fields were required.

\medskip
Several open directions naturally follow.

\begin{itemize}
\item \textbf{Weighted cohomological invariants.}  
The $\Pi$-weighted Hodge theory developed in \textsc{Geometry~V} may support
nontrivial invariants analogous to Reidemeister or analytic torsion, whose
$\Pi$-weighted analogues would merit systematic study.

\item \textbf{Operator-theoretic completion.}  
A full demonstration that the $\Pi$-weighted spectral determinant satisfies all
axioms of a completed $L$-function would require verifying the Euler-product
component, the exact gamma factor, and analytic continuation.  
These were intentionally stated as criteria rather than claimed outcomes.

\item \textbf{Generalized alignment operators.}  
While this work has focused on the scalar $\Pi$-weighted operator derived from the
aligned depth coordinate, the same construction can be generalized to tensor
and form sectors.  
Such operators may admit additional dualities or functional equations.

\item \textbf{Connection to arithmetic geometry.}  
The appearance of Mellin–Fourier duality, parity symmetry, and $\Pi$-weighted
elliptic operators suggests a possible bridge to spectral approaches in number
theory, particularly on weighted manifolds or fibered gauge spaces.
\end{itemize}

\medskip
The \textsc{Geometry} program remains grounded in a single principle:
alignment along the softest mode of the gauge-log Fisher metric.  
That principle has been sufficient to derive gravitational normalization,
curvature response, spectral gaps, parabolic regularization, harmonic
projections, and now the analytic structure necessary for a spectral model of
$L$-functions.  
This work therefore completes the central geometric arc of the program and
identifies a finite set of analytic targets for future investigations.

\medskip
\noindent
\textbf{Data availability.}  
All scripts, numerical constants, and analytic derivations used in the
\textsc{Geometry} sequence are available in the public {\tt GAGE\_repo} under a
hash-verified DOI.

\section{Conclusion}
\label{sec:conclusion}

\noindent
This work completes the \textsc{Geometry} sequence by embedding the alignment
framework within a $\Pi$-weighted cohomological setting that treats curvature,
dynamics, and topology in a unified manner.  The Standard Model–determined
structures introduced in \textsc{Geometry~I}—the integer direction
$\chi=(16,13,2)$, the Fisher softness $F_\chi$, and the even curvature gate
$\Pi(\Xi)$—were shown in \textsc{Geometry~II--V} to fix gravitational
normalization, generate a finite spectral gap, define a covariant drift law,
and supply intrinsic $\Pi$-weighted regularization for both scalar and differential
form sectors.

\medskip
\textsc{Geometry~VI} extends this framework to global geometry.  
Using the $\Pi$-weighted adjoint structure and the aligned operator calculus
developed in the earlier papers, we constructed a curvature–cohomology map
that acts compatibly with the $\Pi$-weighted Hodge Laplacian and preserves the
analytic properties established in the previous volumes.  
The resulting framework identifies $\Pi$-harmonic representatives of curvature
classes and ensures that each topologically nontrivial sector admits a unique,
alignment-selected equilibrium representative.

\medskip
Two features are especially notable.  
First, the $\Pi$-weighted structure introduces no new fields, parameters, or
tunable functions: the global theory is fixed entirely by the same SM-derived
alignment geometry that determines $G(M_Z)$ and the mass gap.  
Second, the cohomological extension preserves the stability, positivity,
and compact-resolvent properties of the $\Pi$-weighted operators, yielding a
geometrically natural and globally consistent completion of the alignment
framework.

\medskip
Together, \textsc{Geometry~I--VI} present a complete sequence:  
a fixed integer direction and Fisher softness determine $\Xi$;  
the even gate $\Pi(\Xi)$ supplies curvature weighting and spectral rigidity;  
$\Pi$-weighted operators govern dynamics, regularity, and cohomological
projection;  
and the full construction remains anchored to SM data at $\mu=M_Z$ with no
additional structures.  
The result is a self-contained gauge-space geometry that derives gravitational
normalization, ensures spectral gap stability, regularizes curvature evolution,
and canonically selects $\Pi$-harmonic curvature representatives.

\medskip
Future work may explore extensions to other gauge theories and to
number-theoretic analogues of $\Pi$-weighted elliptic operators, but the essential
geometric framework developed across \textsc{Geometry~I--VI} is now complete.

\end{document}