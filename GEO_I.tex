\documentclass{iopjournal}
\usepackage{amsmath,amssymb,amsfonts}
\usepackage{lmodern}

\begin{document}

\articletype{Paper} % e.g. Paper, Letter, Topical Review...

\title{GEOMETRY I: SM-derived gravitational coupling $G(Q)$ anchored at the electroweak scale}

\author{Michael DeMasi$^1$}
\affil{$^1$Independent Researcher, Milford, CT, USA}

\email{demasim90@gmail.com}


\begin{abstract}
At $\mu=M_Z$ in the $\overline{\mathrm{MS}}$ scheme, the Standard Model one-loop
decoupling matrix has a unique primitive integer kernel
$\chi=(16,13,2)$ (Smith normal form), which defines the aligned depth
$\Xi=\chi\cdot\hat\Psi$ in log--coupling space.  This integer direction aligns
numerically with the soft eigenmode of the positive-definite Fisher/kinetic
metric $K_{\rm eq}$.  Together they determine the electroweak anchor
$\Omega=\hat\alpha_s^{16}\hat\alpha_2^{13}\hat\alpha^{2}$ and the SM-derived
gravitational coupling
\[
G(M_Z)=\frac{\hbar c}{m_p^2}\,\Omega(M_Z).
\]

An even curvature gate $\Pi(\Xi)$ with $\Pi'(\Xi_{\rm eq})=0$ promotes this
anchor to a running coupling $G(x)=G(M_Z)\Pi(\Xi(x))$ while preserving the
equilibrium tensor sector: the helicity $\pm2$ modes remain massless, luminal,
and identical to General Relativity.  Near equilibrium the curvature response is
fixed and strictly quadratic,
\[
\Delta G/G=(\Delta\Xi/\sigma_\chi)^2,
\]
with the absence of a linear term constituting a direct empirical falsifier.
No new fields or tunable parameters are introduced; PDG/CODATA inputs enter only
\emph{a posteriori} through the closure ratio that tests the SM-derived
$G(M_Z)$ against the measured Newtonian coupling.
\end{abstract}


\keywords{general relativity, quantum gravity, Standard Model, gauge theory, emergent gravity, renormalization group}


\section{Premise: integer certificate and depth}
\label{sec:premise}

The Standard Model (SM) gives a precise account of gauge interactions, yet
provides no internal explanation for Newton’s constant $G_N$ or for the
universality of the massless spin--2 sector of gravity. Operationally, $G_N$
enters General Relativity (GR) as an externally measured parameter, while the
three SM gauge couplings evolve according to renormalization-group (RG) flow.
A natural question is whether the gauge sector at $\mu=M_Z$ contains sufficient
structure to determine an effective gravitational coupling without modifying GR
or enlarging the field content.

\paragraph*{Program and provenance.}
This paper is the first in a sequence we collectively refer to as
GEOMETRY (Gauge Eigenmode Omega Metric Even Tensor Running Yield).  The present
work (GEOMETRY~I) is restricted to the static, equilibrium geometry and derives
an electroweak-anchored gravitational coupling from SM data alone.  All inputs,
constants, and covariance matrices are drawn from established sources, and all
calculations are performed in the $\overline{\mathrm{MS}}$ scheme at
$\mu=M_Z$.

Our renormalization conventions follow
Weinberg~\cite{Weinberg1996_QFTv2},
Peskin and Schroeder~\cite{PeskinSchroeder1995_QFT},
and Langacker~\cite{Langacker2017_SMBeyond}.
Decoupling and integer-lattice methods follow
Appelquist and Carazzone~\cite{AppelquistCarazzone1975_Decoupling},
Kannan and Bachem~\cite{KannanBachem1979_SNF},
and Newman~\cite{Newman1997_SNF}.
Electroweak pins, covariance matrices, and physical constants are taken from
the Particle Data Group and CODATA%
~\cite{PDG2024,PDG2024_EWReview,PDG2025_GaugeHiggs,CODATA2022_RMP}.
Two-loop RG coefficients follow
Machacek and Vaughn~\cite{Machacek1983_TwoLoopI,Machacek1984_TwoLoopII}
and Luo \textit{et al.}~\cite{Luo2003_TwoLoopSM},
and the running of $\alpha$ follows
Jegerlehner~\cite{Jegerlehner2019_alphaRun}.
Gravitational and experimental comparisons follow
Carroll~\cite{Carroll2004_SG},
Will~\cite{Will2014_LRR_TestsGR},
Bertotti \textit{et al.}~\cite{Cassini2003_PPN},
and Abbott \textit{et al.} (LVK)~\cite{LVK2021_TestsGR}.
No data or tuning beyond these references is used.

\subsection*{Conventions and brief summary}

$\overline{\mathrm{MS}}$ at $\mu=M_Z$;
GUT-normalized hypercharge ($\alpha_1=\tfrac53\alpha_Y$);
$c=\hbar=1$ unless displayed.

A Smith normal form shows that the primitive integer
\begin{equation}
    \chi=(16,13,2),
\end{equation}
in the Standard Model basis is unique up to overall sign.  This vector defines the
SM-internal depth
\begin{equation}
   \Xi=\chi\cdot\hat\Psi, \qquad
   \hat\Psi=(\ln\hat\alpha_s,\ln\hat\alpha_2,\ln\hat\alpha).
\end{equation}

By construction,
\begin{equation}
  \Omega\equiv e^{\Xi}
  =\hat\alpha_s^{16}\,\hat\alpha_2^{13}\,\hat\alpha^{2},
\end{equation}
so once the integer certificate fixes $\Xi$, the electroweak anchor $\Omega(M_Z)$
is determined with no additional inputs.

\paragraph*{Connection to the GR normalization.}
In General Relativity,
\begin{equation}
  \mathcal{L}_{\rm EH}
  =\frac{M_P^2}{2}\,R
  =\frac{1}{16\pi G_N}\,R ,
\end{equation}
so that $M_P^{-2}=8\pi G_N$.
Because $\Omega=e^{\Xi}$, the SNF certificate fixes the gravitational anchor at
the electroweak scale as
\begin{equation}
  G(M_Z)=\frac{\hbar c}{m_p^2}\,\Omega(M_Z),
\end{equation}
and consistency of the curvature term requires
\begin{equation}
  \frac{M_P^2}{2}
  =\frac{1}{16\pi G(M_Z)}
  \quad\Longrightarrow\quad
  G(M_Z)=\frac{1}{8\pi M_P^2}.
\end{equation}
Thus the GR normalization is not imposed separately: the electroweak anchor
$\Omega(M_Z)$, fixed entirely by Standard-Model pins, reproduces the
Einstein–Hilbert coupling. This equality was the original route by which the
construction recovered $G_N$ from SM data alone.

\paragraph*{Even curvature gate.}
At equilibrium the aligned depth satisfies $\delta\Xi=0$, and the curvature
response along this axis is encoded by an even scalar gate
\begin{equation}
  \Pi(\Xi)=\exp\!\big[-(\delta\Xi)^2/\sigma_\chi^2\big],
\end{equation}
with width fixed by matching its curvature to the Fisher curvature,
$\sigma_\chi^{-2}=F_\chi=\hat\chi_K^{\top}K\hat\chi_K$.  Even parity enforces
\begin{equation}
  \Pi(\Xi_{\rm eq})=1, \qquad \Pi'(\Xi_{\rm eq})=0,
\end{equation}
removing any Brans–Dicke–like linear response and ensuring that the tensor
sector at equilibrium reduces exactly to the GR, massless helicity~$\pm2$ theory.

The even curvature gate promotes the electroweak anchor to a running
gravitational coupling,
\begin{equation}
  G(Q)=G(M_Z)\,\Pi(\Xi(Q)),
\end{equation}
while preserving the GR limit at $\Xi_{\rm eq}$.

Thus the sequence
\[
  \chi \;\rightarrow\; \Xi \;\rightarrow\; \Omega \;\rightarrow\; G(M_Z)
  \;\rightarrow\; G(Q)
\]
is fixed entirely by Standard Model data at $\mu=M_Z$.

For closure against experiment, define
\begin{equation}
  \alpha_G^{(pp)}=\frac{G_N m_p^2}{\hbar c}, \qquad
  Z_G=\frac{\alpha_G^{(pp)}}{\widehat{\Omega}(M_Z)}.
\end{equation}
The only empirical input is the measured value of $\alpha_G^{(pp)}$, which
enters \emph{a posteriori} through $Z_G$ as a consistency check; no parameters
are tuned.

Near equilibrium, define
\[
  s=\delta\Xi/\sigma_\chi, \qquad
  \Lambda_\chi=\sigma_\chi/\|\chi\|_{K}, \qquad
  \phi_\chi=\chi^\top\delta\hat\Psi/\|\chi\|_{K},
\]
with $\sigma_\chi=247.683$ fixed by $F_\chi$.  The framework then predicts the
strictly quadratic lab-null,
\[
  \frac{\Delta G}{G}=s^2=(\phi_\chi/\Lambda_\chi)^2,
\]
with the absence of any linear term as a direct falsifier of the aligned-depth
mechanism.

\subsection*{Integer realization, one-loop weights, and the EM basis.}
We work in log-coupling space, where multiplicative renormalization becomes
additive and basis transports are linear
\cite{Weinberg1996_QFTv2,PeskinSchroeder1995_QFT,Langacker2017_SMBeyond}.
Gauge weights are expressed using Dynkin indices and spectator multiplicities.
For a field $f$, let $T_{\mathrm{SU(3)}}(f)$ and $T_{\mathrm{SU(2)}}(f)$ denote
the usual Dynkin indices, and let $d_{\mathrm{spect}}(f)$ count spectator
degrees of freedom (spin, flavour, chirality, etc.).
We define the integerized one-loop weights as follows.

For Weyl fermions,
\[
w_3(f)=4\,T_{\mathrm{SU(3)}}(f)\,d_{\mathrm{spect}}(f),\qquad
w_2(f)=4\,T_{\mathrm{SU(2)}}(f)\,d_{\mathrm{spect}}(f).
\]

For scalars,
\[
w_3(f)=T_{\mathrm{SU(3)}}(f)\,d_{\mathrm{spect}}(f),\qquad
w_2(f)=T_{\mathrm{SU(2)}}(f)\,d_{\mathrm{spect}}(f).
\]

A single hypercharge integerizer fixes the $U(1)_Y$ column.  With
GUT-normalized hypercharge ($\alpha_1=\tfrac{5}{3}\alpha_Y$), the integer
weights are
\[
w^{\mathrm{(Weyl)}}_1=\frac12\sum_{\text{Weyl in } f} Y^2,\qquad
w^{\mathrm{(scalar)}}_1=\frac13\sum_{\text{scalars in } f} Y^2.
\]

For any momentum window $W$ with light particle set $S_W$, the integerized
one-loop weight vector is
\[
b(W)=
\begin{pmatrix}
\sum w_3\\[2pt]
\sum w_2\\[2pt]
\sum w_1
\end{pmatrix}
\in\mathbb{Z}^3,
\qquad
\Delta b(i j)=b(W_i)-b(W_j),
\]
and stacking these differences produces the rank-two integer matrix
\[
\Delta W=
\begin{pmatrix}
(\Delta b(i_1 j_1))^\top\\
(\Delta b(i_2 j_2))^\top\\
\vdots
\end{pmatrix}
\in\mathbb{Z}^{m\times 3}.
\]
Adjoint self-terms cancel in $\Delta b$, isolating the two-dimensional integer
lattice relevant for Smith–normal–form analysis.

\medskip\noindent\textbf{Electromagnetic basis.}
After electroweak symmetry breaking, the electromagnetic weight is
\[
w_{\mathrm{EM}}=w_2+\tfrac{5}{3}w_1,
\]
so
\[
3 w_{\mathrm{EM}} = 3 w_2+5 w_1 \in\mathbb{Z}.
\]
Thus, in the $(\mathrm{SU(3)},\mathrm{SU(2)},\mathrm{EM})$ basis, the two-row
difference stack becomes
\[
\Delta W_{\mathrm{EM}}
=
\begin{pmatrix}
8 & 8 & 224\\
0 & 1 & 18
\end{pmatrix}
\in\mathbb{Z}^{2\times 3}.
\]

\medskip\noindent\textbf{SNF and primitive kernel.}
The Smith normal form
\[
U\,\Delta W_{\mathrm{EM}}\,V=\mathrm{diag}(1,8,0),
\qquad
U\in GL(2,\mathbb{Z}),\;\; V\in GL(3,\mathbb{Z}),
\]
implies ${\rm rank}(\Delta W_{\mathrm{EM}})=2$ and a one-dimensional integer
left kernel.  Solving $\Delta W_{\mathrm{EM}}\chi_{\mathrm{EM}}=0$ over
$\mathbb{Z}$ yields the primitive generator
\[
\chi_{\mathrm{EM}}=(-10,-18,1),\qquad \gcd(10,18,1)=1.
\]

Transporting back to the $(w_3,w_2,w_1)$ basis using a unimodular matrix
$M\in GL(3,\mathbb{Z})$,
\[
\chi = M^{\top}\chi_{\mathrm{EM}} = (16,13,2),
\]
shows that the primitive integer certificate in the Standard Model basis is
unique up to overall sign.  This vector defines the SM-internal depth
\[
\Xi=\chi\cdot\hat\Psi,\qquad
\hat\Psi=(\ln\hat\alpha_s,\ln\hat\alpha_2,\ln\hat\alpha),
\]
and the corresponding dimensionless combination
\[
\hat\Omega = e^{\Xi}
           = \hat\alpha_s^{16}\,\hat\alpha_2^{13}\,\hat\alpha^{2}.
\]

\medskip\noindent\textbf{Verification.}
The Smith normal form was computed using exact integer arithmetic via
\texttt{sympy.smith\_normal\_form} (Kannan–Bachem algorithm) in
\texttt{snf\_check.py}, yielding
$U\,\Delta W_{\rm EM}\,V=\operatorname{diag}(1,8,0)$.
The primitive kernel is $\chi_{\rm EM}=(-10,-18,1)$; unimodular transport
$M^{\top}\chi_{\rm EM}$ gives $\chi=(16,13,2)$.
Artifacts and SHA-256 checksums match the reproducibility archive
\cite{demasi_gage_repo_v1_0_0_2025}.

\begin{table}[t]
\centering
\includegraphics[width=\linewidth]{tab1_key.pdf}
\caption{
Key definitions in \textsc{Geometry I}.  Shown are the integer generator
$\chi$, gauge--log point $\hat\Psi$, depth displacement
$\delta\Xi=\Xi-\Xi_{\rm eq}$, the curvature gate $\Pi(\Xi)$, and the aligned
quantities $\phi_\chi$ and $\Lambda_\chi$ used in the lab--null prediction.
}
\label{tab:key}
\end{table}





\section{Alignment: metric certificate}

The equilibrium kinetic metric $K$ governs curvature in gauge--log space and
defines the local Fisher geometry of the gauge couplings.  It is constructed
from the one--loop sensitivities of the $\beta$--functions
\cite{Machacek1983_TwoLoopI,Machacek1984_TwoLoopII,Luo2003_TwoLoopSM,Langacker2017_SMBeyond}:
\[
\beta_i=\frac{d\hat\alpha_i}{d\ln\mu}
=-\frac{b_i}{2\pi}\,\hat\alpha_i^2+\cdots,
\qquad
K_{ij}
=\frac{\partial(\beta_i/\hat\alpha_i)}{\partial\ln\hat\alpha_j}\Big|_{\rm eq},
\]
with the conventional $2\pi$ normalization removed and all quantities evaluated
at $\mu=M_Z$.  The metric encodes the Fisher information of the RG flow and
acts as a Riemannian metric on the space of log--couplings.

Using PDG/CODATA pins for
$(\hat\alpha_s,\hat\alpha_2,\hat\alpha)$ at $\mu=M_Z$
and the one--loop SM coefficients $b_i$, the resulting equilibrium metric is
\cite{PDG2024,CODATA2022_RMP}
\[
K = \begin{pmatrix}
1.2509 & -0.6202 & -0.1813\\
-0.6202 & 1.5128 & -0.1633\\
-0.1813 & -0.1633 & 3.2362
\end{pmatrix},
\qquad
K \succ 0 .
\]

Its eigen--decomposition,
\[
K\,e_i=\lambda_i e_i,
\qquad
\{\lambda_i\}=\{0.7243,\,2.0156,\,3.2599\},
\]
identifies the \emph{soft} (minimum--curvature) eigenmode
\[
e_{\chi}=(0.77249,\,0.62764,\,0.09656),
\]
with $\{e_2,e_3\}$ completing an orthonormal frame.

The SM--derived integer certificate $\chi=(16,13,2)$ from Sec.~I defines a
distinguished direction in log--coupling space.  Normalizing in the Fisher
metric,
\[
\hat\chi_K=\frac{\chi}{\|\chi\|_K},
\qquad
\|\chi\|_{K}=\sqrt{\chi^\top K\chi}=17.6278,
\]
one finds the numerical alignment
\[
\cos\theta_K=\hat\chi_K\!\cdot e_{\chi}
=1.0000000\pm10^{-8}.
\]
Thus the integer direction selected by the Smith–normal–form analysis
coincides with the minimal--curvature eigenvector of $K$ to numerical
precision.  No parameters are adjusted: the integer lattice of $\Delta W$ and
the analytic curvature of $K$ independently select the same soft mode.

This empirical identification constitutes the \emph{alignment principle}:  
the Standard Model’s gauge couplings self--align along the direction of
minimal Fisher curvature, selecting
$\Xi=\chi\!\cdot\!\hat\Psi$ as the unique soft coordinate.

\subsection{Fisher curvature and curvature–gate matching}

We define the Fisher curvature along the aligned direction by
\begin{equation}
  F_\chi \equiv \hat\chi_K^{\!\top} K\,\hat\chi_K ,
\end{equation}
where $K$ is the Fisher/kinetic metric at $\mu=M_Z$, and
$\hat\chi_K=\chi/\|\chi\|_K$ is the \emph{metric-unit} direction of~$\chi$.

The even curvature gate is
\begin{equation}
  \Pi(\Xi) = \exp\!\left[-(\delta\Xi)^2/\sigma_\chi^2\right],
  \qquad
  \Pi'(\Xi_{\rm eq}) = 0 ,
\end{equation}
and matching the intrinsic Fisher curvature to the curvature of the gate at
equilibrium,
\begin{equation}
  F_\chi = -\tfrac12\,\Pi''(\Xi_{\rm eq}),
\end{equation}
fixes its width uniquely:
\begin{equation}
  \sigma_\chi = F_\chi^{-1/2}.
\end{equation}

Using the SM pins at $\mu=M_Z$, we obtain
\begin{equation}
  F_\chi = \frac{1}{\sigma_\chi^2}
  \approx 1.629\times 10^{-5},
  \qquad
  \sigma_\chi = 247.683 .
\end{equation}
The quantity $F_\chi$ is evaluated directly from $K$ and $\hat\chi_K$ in
\texttt{metric\_eigs.py}; $\sigma_\chi$ then follows from the above matching,
with no additional inputs.

The corresponding aligned depth scale is
\begin{equation}
  \Lambda_\chi \equiv \frac{\sigma_\chi}{\|\chi\|_K} = 14.0507 ,
\end{equation}
which serves as the canonical curvature scale.  Together with
$\delta\Xi=\Xi-\Xi_{\rm eq}$, this determines the full quadratic lab--null:
\begin{equation}
  \frac{\Delta G}{G}
  = \left(\frac{\delta\Xi}{\sigma_\chi}\right)^2
  = F_\chi\,(\delta\Xi)^2 ,
\end{equation}
with no tunable parameters.

\medskip\noindent\textbf{Verification.}
All numerical values
$K,\ \{\lambda_i\}=\{0.7243,2.0156,3.2599\},\
e_{\chi}=(0.77249,0.62764,0.09656),\
\|\chi\|_K=17.6278,$ and $\cos\theta_K=1.0000000\pm10^{-8}$
were reproduced by \texttt{metric\_eigs.py},
with SHA--256 checksums matching the reproducibility archive
\cite{demasi_gage_repo_v1_0_0_2025}.
% =========================================================
\section{Even gate and the quadratic lab-null}

The curvature response along the aligned depth $\Xi$ is encoded by an
even scalar gate $\Pi(\Xi)$ multiplying the Einstein–Hilbert term,
\begin{equation}
\mathcal{L}_{\rm eff}
  = \frac{1}{16\pi G(M_Z)}\,\Pi(\Xi)\,R .
\end{equation}
We interpret $\Pi$ as an emergent, SM–determined form factor rather than a
free function.  The Fisher/kinetic metric $K$ induces a one-dimensional
curvature along the aligned direction $\chi=(16,13,2)$,
\begin{equation}
F_\chi \equiv \hat\chi_K^{\!\top} K\,\hat\chi_K,
\qquad
\hat\chi_K \equiv \chi / \|\chi\|_K ,
\end{equation}
so small displacements obey $ds^2 = F_\chi (d\Xi)^2$.

\medskip\noindent
\textbf{Local expansion and curvature matching.}
Expanding an \emph{a priori} unknown gate about equilibrium,
\begin{equation}
\Pi(\Xi)
  = 1
    + \tfrac12\,\Pi''(\Xi_{\rm eq})\,(\delta\Xi)^2
    + \mathcal{O}\!\big((\delta\Xi)^4\big),
\qquad
\delta\Xi \equiv \Xi-\Xi_{\rm eq},
\end{equation}
even parity enforces $\Pi'(\Xi_{\rm eq})=0$.  Matching the intrinsic
Fisher curvature to the curvature of the gate at equilibrium requires
\begin{equation}
-\Pi''(\Xi_{\rm eq})
  = \frac{2}{\sigma_\chi^2}
  = 2 F_\chi ,
\qquad
\sigma_\chi \equiv F_\chi^{-1/2} = 247.683 .
\end{equation}
Thus the width $\sigma_\chi$ is \emph{derived} from $(K,\chi)$ and
introduces no tunable parameter.

\medskip\noindent
\textbf{Minimal even completion.}
Imposing analyticity, even parity, normalization
$\Pi(\Xi_{\rm eq})=1$, and decay $\Pi(\Xi)\!\rightarrow\!0$ as
$|\delta\Xi|\!\rightarrow\!\infty$, the minimal completion of the above
local form is the Gaussian
\begin{equation}
\Pi(\Xi)
  = \exp\!\left[-\frac{(\delta\Xi)^2}{\sigma_\chi^2}\right].
\end{equation}
More general even deformations would introduce higher-order coefficients not
fixed by Standard Model data; GEOMETRY~I therefore adopts this minimal
SM-determined form.

\subsection*{Curvature expansion and parity condition}

Expanding $\Pi$ in $\Delta\Xi=\Xi-\Xi_{\rm eq}$,
\begin{equation}
\Pi(\Xi)
  = 1
    + \Pi'(\Xi_{\rm eq})\,\Delta\Xi
    + \tfrac12\,\Pi''(\Xi_{\rm eq})\,(\Delta\Xi)^2 + \cdots ,
\end{equation}
even parity enforces $\Pi'(\Xi_{\rm eq})=0$, and physical stability requires
$\Pi''(\Xi_{\rm eq})<0$, so the first nonzero correction is quadratic.
The Gaussian model
\begin{equation}
\Pi(\Xi)=\exp[-(\Delta\Xi)^2/\sigma_\chi^2],
\qquad
\Pi'(\Xi_{\rm eq})=0,
\qquad
\Pi''(\Xi_{\rm eq})=-2/\sigma_\chi^2,
\end{equation}
satisfies all requirements.  Any analytic even function with the same
second derivative reproduces identical local physics; the Gaussian is the
unique minimal completion.

\subsection*{Quadratic lab-null prediction}

Because $G(x)=G(M_Z)\,\Pi(\Xi(x))$, an expansion about equilibrium gives
\begin{equation}
\frac{\Delta G}{G}
  \equiv \frac{G(x)}{G(M_Z)} - 1
  = \Pi(\Xi) - 1
  \simeq \frac{(\delta\Xi)^2}{\sigma_\chi^2}
  = \frac{\phi_\chi^2}{\Lambda_\chi^2},
\end{equation}
where the aligned quantities are
\begin{equation}
\phi_\chi
  = \frac{\chi^\top K\,\delta\hat\Psi}{\|\chi\|_{K}},
\qquad
\Lambda_\chi
  = \frac{\sigma_\chi}{\|\chi\|_{K}}
  = 14.0507 .
\end{equation}
Since
\begin{equation}
\delta\Xi
  = \chi^\top \delta\hat\Psi
  = \|\chi\|_K\,\phi_\chi ,
\qquad
F_\chi
  = \frac{1}{\sigma_\chi^2},
\end{equation}
the identities
\begin{equation}
\frac{\delta\Xi}{\sigma_\chi}
  = \frac{\phi_\chi}{\Lambda_\chi},
\qquad
F_\chi\,(\delta\Xi)^2
  = \frac{\phi_\chi^2}{\Lambda_\chi^2},
\end{equation}
follow directly from the definitions.  Hence the quadratic lab-null may be
written in any equivalent form:
\begin{equation}
\frac{\Delta G}{G}
  = F_\chi\,(\delta\Xi)^2
  = \left(\frac{\delta\Xi}{\sigma_\chi}\right)^2
  = \frac{\phi_\chi^2}{\Lambda_\chi^2}.
\end{equation}
All quantities $(F_\chi,\sigma_\chi,\Lambda_\chi,\|\chi\|_K)$ are fixed
entirely by Standard Model data at $\mu=M_Z$, and no tunable parameters
appear.

\subsection*{Empirical falsifier}
A general local response can be written as
\begin{equation}
\frac{\Delta G}{G}
  = a_1\,\delta\Xi
    + a_2\,(\delta\Xi)^2
    + \cdots .
\end{equation}
Even parity and alignment symmetry require
\begin{equation}
a_1 = 0,
\qquad
a_2 = F_\chi .
\end{equation}
Thus any measured \emph{linear} term ($a_1\neq0$) provides an immediate
falsifier of the aligned-depth mechanism.
The quadratic coefficient is not adjustable:
\begin{equation}
a_2 = F_\chi = \frac{1}{\sigma_\chi^2}
\end{equation}
is determined entirely by Standard Model data.

\subsection*{Invariance and parameter independence}

The prediction is invariant under integer basis transports
\begin{equation}
\Delta W \rightarrow
  U_{\rm row}\,\Delta W\,V_{\rm col},
\qquad
U_{\rm row}, V_{\rm col}\in GL(\mathbb{Z}),
\end{equation}
which preserve the primitive integer left kernel
$\ker_{\mathbb{Z}}(\Delta W^\top)=\mathrm{span}_{\mathbb{Z}}\{\pm\chi\}$.
Hence $\Xi=\chi\cdot\hat\Psi$, $\Pi(\Xi)$, and the lab-null are basis-invariant.
No free parameters enter: $\sigma_\chi$, $\Lambda_\chi$, $F_\chi$, and
$\|\chi\|_K$ are determined solely by $(K,\chi)$ at $\mu=M_Z$.

\medskip\noindent\textbf{Verification.}
$\sigma_\chi=247.683$ and $\Lambda_\chi=14.0507$
were reproduced by \texttt{gate\_null.py} using $K$ from
\texttt{metric\_eigs.py}; SHA--256 checksums match the reproducibility
archive~\cite{demasi_gage_repo_v1_0_0_2025}.

% =========================================================
\section{Tensor/helicity certificate (GR limit)}

The curvature gate $\Pi(\Xi)$ multiplies the Einstein--Hilbert term of the
effective action,
\[
S
=\!\int\! d^4x\,\sqrt{-g}\Big[
\tfrac{M_P^2}{2}\,\Pi(\Xi)\,R
-\tfrac12\,\partial_\mu\hat\Psi^\top K\,\partial^\mu\hat\Psi
-V(\hat\Psi)\Big],
\]
where $V(\hat\Psi)$ collects subdominant scalar interactions that stabilize
the depth and metric sectors.  Expansions are performed about the
equilibrium point,
\[
\Pi'(\Xi_{\rm eq})=0,\qquad
g_{\mu\nu}=\eta_{\mu\nu}+h_{\mu\nu},\qquad
\hat\Psi=\hat\Psi_{\rm eq}+\delta\hat\Psi,
\]
with
$\Pi(\Xi)=1+\tfrac12\Pi''(\Xi_{\rm eq})(\delta\Xi)^2+\cdots$.
Even parity is essential: it removes the linear term and forbids any
mixing between the tensor and scalar sectors.

\subsection*{Quadratic tensor kernel}

The linearized Ricci tensor and scalar curvature are
\begin{align*}
R^{(1)}_{\mu\nu}
&=\tfrac12(\partial_\rho\partial_\mu h^\rho_{\ \nu}
+\partial_\rho\partial_\nu h^\rho_{\ \mu}
-\Box h_{\mu\nu}
-\partial_\mu\partial_\nu h),\\
R^{(1)}&=\partial_\mu\partial_\nu h^{\mu\nu}-\Box h.
\end{align*}
Inserting these into the action and integrating by parts yields the
quadratic tensor Lagrangian,
\[
\mathcal{L}^{(2)}_{\rm tens}
=\frac{M_P^2}{8}\,
h_{\mu\nu}\,
E^{\mu\nu,\rho\sigma}
h_{\rho\sigma},
\qquad
E^{\mu\nu,\rho\sigma}
=-\Box\,P^{(2)}_{\mu\nu,\rho\sigma},
\]
where $P^{(2)}_{\mu\nu,\rho\sigma}$ is the Barnes--Rivers spin--2
projector~\cite{FierzPauli1939}.  

Because $\Pi'(\Xi_{\rm eq})=0$ and $K\succ0$, all scalar--tensor mixing
terms cancel exactly, and the Pauli--Fierz mass term
$m_{\rm PF}^2(h_{\mu\nu}h^{\mu\nu}-h^2)$ is absent.  Thus the tensor
sector is automatically massless.

\subsection*{Helicity decomposition and propagation}

Working in de Donder gauge $\partial^\mu h_{\mu\nu}=\tfrac12\partial_\nu h$
removes spin--1 components, leaving the pure spin--2 projection:
\[
P^{(2)}_{\mu\nu,\rho\sigma}h^{\rho\sigma}=h_{\mu\nu},
\qquad
E^{\mu\nu,\rho\sigma}
=-\Box\,P^{(2)}_{\mu\nu,\rho\sigma}.
\]
The linearized field equation,
\[
E^{\mu\nu,\rho\sigma}h_{\rho\sigma}=0,
\]
implies the GR dispersion relation
\[
\omega^2 = k^2,\qquad c_T = 1,
\]
so the helicity eigenstates are strictly $\pm 2$, massless, and luminal.
Standard post-Newtonian and gravitational-wave bounds are satisfied
identically~\cite{Will2014_LRR_TestsGR,Abbott2017_PRL_GW170817,
Abbott2017_ApJL_GRB170817A,LVK2021_TestsGR}.

\subsection*{Propagator and soft limit}

In harmonic gauge the graviton propagator becomes
\[
D_{\mu\nu,\rho\sigma}(k)
= i\,\frac{16\pi G_N}{2}\,
\frac{P^{(2)}_{\mu\nu,\rho\sigma}}{k^2+i\epsilon},
\]
identical to the Einstein--GR propagator.
The soft-graviton theorem~\cite{Weinberg1996_QFTv2} is therefore unchanged,
preserving universality of soft emission and conservation of the helicity
current.

\subsection*{Summary and falsifier link}

The combined conditions
\[
\Pi'(\Xi_{\rm eq})=0,
\qquad
K\succ0,
\]
guarantee:
\begin{enumerate}
\item no scalar--tensor mixing,
\item no Pauli--Fierz mass,
\item luminal propagation ($c_T=1$),
\item a GR-normalized helicity $\pm2$ sector.
\end{enumerate}
These properties constitute the \emph{tensor/helicity certificate} of
\textsc{Geometry I}.  Any observed deviation in $c_T$ or an inferred
$m_{\rm PF}\neq0$ would empirically break aligned-depth symmetry and
falsify the mechanism.

\medskip
\noindent\textbf{Verification.}
The equilibrium quantities
$\|\chi\|_K=17.6278$ and $\sigma_\chi=247.683$ (from
\texttt{metric\_eigs.py} and \texttt{gate\_null.py})
match the reproducibility archive~\cite{demasi_gage_repo_v1_0_0_2025}.

%=========================================================
\section{Closure (a posteriori) and pins}

Anchoring at $\mu = M_Z$ in the $\overline{\mathrm{MS}}$ scheme, all
quantities are fixed from PDG and CODATA data without parameter
adjustment~\cite{PDG2024,PDG2024_PhysConstants,PDG2024_EWReview,CODATA2022_RMP}.
Two--loop running and electroweak corrections follow
Refs.~\cite{Jegerlehner2019_alphaRun,Awramik2004_mW,
Sirlin1980_DeltaR,Machacek1983_TwoLoopI,Machacek1984_TwoLoopII,
Luo2003_TwoLoopSM}.

\medskip\noindent\textbf{Definition.}
\[
\alpha_G^{(pp)} \equiv \frac{G_N m_p^2}{\hbar c},
\]
the dimensionless proton--proton gravitational coupling.

\medskip
At the electroweak scale, the closure ratio is
\[
Z_G \equiv \frac{\alpha_G^{(pp)}}{\hat\Omega(M_Z)}
= 1.09373393,
\qquad
\hat\Omega(M_Z)=\hat\alpha_s^{16}\hat\alpha_2^{13}\hat\alpha^{2}.
\]
The SM-derived strong coupling obtained by leaving out
$\hat\alpha_s$ in the depth relation is
\[
\hat\alpha_s^{\star}(M_Z)=0.1173411\pm1.9\times10^{-5},
\]
consistent with PDG~2024 averages.  No parameters are tuned:
metrology enters only \emph{a posteriori} through $\alpha_G^{(pp)}$.
Thus $Z_G$ measures the degree of closure between the SM-derived $G(M_Z)$
and the measured $G_N$.

\medskip\noindent\textbf{Uncertainties (closure).}
\[
\sigma^2(\ln Z_G)
=\sigma^2(\ln \alpha_G^{(pp)})
+\sum_{k\in\{s,2,\mathrm{em}\}}\chi_k^2\,\sigma^2(\ln \hat\alpha_k),
\]
propagating fractional uncertainties in the three gauge couplings and $G_N$.

\begin{table}[t]
\centering
\scriptsize
\setlength{\tabcolsep}{3.5pt}
\renewcommand{\arraystretch}{1.05}
\caption{Numerical pins at $\mu=M_Z$ ($\overline{\mathrm{MS}}$).  
Canonical values match those used in the reproducibility archive.}
\label{tab:pins-merged}
\begin{tabular}{@{}lcc@{}}
\hline\hline
\textbf{Quantity} & \textbf{Canonical} & \textbf{Repo build} \\
\hline
$M_Z$ [GeV] & 91.1876 & --- \\
$\hat\alpha(M_Z)$ & $1/127.955(10)$ & --- \\
$\hat\alpha_2(M_Z)$ & 0.033816 & 0.033789820 \\
$\hat\alpha_s(M_Z)$ & $0.1173411(19)$ & --- \\
$\hat\alpha_s^{\star}(M_Z)$ (LOO) & --- & 0.117341100 \\
$m_p$ [MeV] & 938.2720813 & --- \\
$G_N$ [$\mathrm{m^3\,kg^{-1}\,s^{-2}}$] & $6.67430(15)\!\times\!10^{-11}$ & --- \\
$\hat\Omega$ & --- & $6.4597\times10^{-39}$ \\
$\alpha_G^{(pp)}$ & --- & $5.9061\times10^{-39}$ \\
$Z_G$ & 1.09373393 & 1.09372878 \\
$\|\chi\|_{K}$ & 17.6278 & 17.62783 \\
$\sigma_\chi$ & 247.683 & --- \\
$\Lambda_\chi=\sigma_\chi/\|\chi\|_{K}$ & 14.0507 & 14.050704 \\
$K$ eigvals & $0.7243,2.0156,3.2599$ & $0.7243,2.0156,3.2600$ \\
$e_{\chi}$ & $(0.7725,0.6276,0.0966)$ & $(0.7725,0.6276,0.0966)$ \\
$\cos\theta_K$ & 1.0000000 & 1.0000000 \\
\hline\hline
\end{tabular}
\end{table}

\noindent\textit{Build artifacts (SHA-256):}
\texttt{results.json}=08f0371b31de…c7cd5edc;
\texttt{metric\_results.json}=e0e3bee8a70c…b9b251b6451;
\texttt{stdout.txt}=0f232a0be6f8…6c7cd5edc.

\subsection*{Leave-one-out (LOO) forecast as falsifier}

Because $\chi=(16,13,2)$ uniquely couples
$(\hat\alpha_s,\hat\alpha_2,\hat\alpha)$, any one coupling can be
predicted from the other two:
\[
\hat\alpha_i^{(\mathrm{LOO})}
=\exp\!\Bigg[
\frac{\Xi_{\rm eq}-\sum_{j\neq i}\chi_j\ln\hat\alpha_j}{\chi_i}
\Bigg],
\]
giving a direct falsifiable relation among the three measured gauge
couplings.

\medskip\noindent\textbf{Uncertainties (LOO).}
\[
\sigma^2(\ln \hat\alpha_i^{(\mathrm{LOO})})
=\chi_i^{-2}\sum_{j\neq i}\chi_j^2\,\sigma^2(\ln\hat\alpha_j).
\]
For PDG~2024 inputs,
\[
\frac{\Delta\hat\alpha_s}{\hat\alpha_s}
=1.6\times10^{-4},
\]
which lies well within current experimental uncertainties.  Any
statistically significant deviation in future precision data would
falsify either the integer certificate or the alignment premise.

\subsection*{Interpretation and falsifier set}

Three independent tests follow from the structure of GAGE:
\begin{center}
(i) gravitational closure via $Z_G$,\\
(ii) gauge-sector self-consistency via LOO,\\
(iii) parity/response symmetry via the quadratic lab-null.
\end{center}
All three use only Standard Model data at $\mu=M_Z$, with no tunable
parameters.  Any violation of these relations falsifies the mechanism.

\medskip\noindent\textbf{Verification.}
$Z_G$ and LOO values were regenerated by \texttt{omega\_chi.py}
using pins in Table~\ref{tab:pins-merged};
SHA--256 hashes match the reproducibility
archive~\cite{demasi_gage_repo_v1_0_0_2025}.
% =========================================================
\section{Falsifiers and consistency}

The framework contains no free parameters: every quantity is fixed by
Standard Model pins at $\mu=M_Z$.  Each falsifier probes a distinct,
independently testable structural layer of the construction.

\paragraph*{\textbf{(1) Parity/response (quadratic lab--null).}}
Near equilibrium,
\[
\frac{\Delta G}{G}
= A\,s + B\,s^2 + \mathcal{O}(s^3),
\qquad
s=\delta\Xi/\sigma_\chi .
\]
Even parity and alignment require
\[
A=0,\qquad B=1 .
\]
Thus any statistically significant $A\neq0$ falsifies the mechanism.
Even parity forbids Brans--Dicke--type linear couplings at equilibrium
\cite{BransDicke1961,Faraoni2004_STG,FujiiMaeda2003_STG,Carroll2004_SG}.

\begin{figure}[t]
\centering
\includegraphics[width=\linewidth]{fig_gate.pdf}
\caption{
Even curvature gate $\Pi(\Xi)$ and the quadratic parity--null prediction.  
(a) Gaussian gate $\Pi=\exp[-(\delta\Xi)^2/\sigma_\chi^2]$ with
$\Pi'(\Xi_{\rm eq})=0$.  
(b) Laboratory relation $\Delta G/G=(\delta\Xi/\sigma_\chi)^2=
(\phi_\chi/\Lambda_\chi)^2$, showing the vanishing of the linear term
($A=0$) as the empirical falsifier.
}
\label{fig:gate}
\end{figure}

\paragraph*{\textbf{(2) Closure ratio $Z_G$.}}
At $\mu=M_Z$,
\[
Z_G=\frac{\alpha_G^{(pp)}}{\hat\Omega(M_Z)} ,
\]
testing consistency between the SM--derived $G(M_Z)$ and the measured
$G_N$.  A deviation of $Z_G$ beyond propagated PDG/CODATA uncertainties
falsifies the SM‐anchored gravitational coupling.

\paragraph*{\textbf{(3) Leave--one--out (LOO) forecast.}}
Because $\Xi=\sum_i\chi_i\ln\hat\alpha_i$, one coupling is predicted from
the other two:
\[
\hat\alpha_i^{(\mathrm{LOO})}
=\exp\!\Bigg[
\frac{\Xi_{\rm eq}-\sum_{j\neq i}\chi_j\ln\hat\alpha_j}{\chi_i}
\Bigg].
\]
A statistically significant deviation of
$\hat\alpha_i^{(\mathrm{LOO})}$ from measurement falsifies either the
integer certificate or alignment.  This test is predictive, not fitted.

\paragraph*{\textbf{(4) Tensor/helicity constraints.}}
With $\Pi'(\Xi_{\rm eq})=0$, the quadratic kernel reduces to the GR
Lichnerowicz operator.  Falsify if any of
\[
m_{\rm PF}\neq0,\qquad c_T\neq 1,
\]
or if scalar/spin--1 admixtures propagate.  Current bounds from
GW170817/GRB170817A enforce $c_T\simeq 1$ and GWTC--3 constrains
$m_g\le1.27\times10^{-23}\,\mathrm{eV}/c^2$ (90\% C.L.)
\cite{Abbott2017_PRL_GW170817,Abbott2017_ApJL_GRB170817A,
LVK2021_TestsGR,Will2014_LRR_TestsGR}, both satisfied here.

\paragraph*{\textbf{(5) Metric alignment.}}
The equilibrium metric must satisfy
\[
K\succ0,\qquad
\hat\chi=\chi/\|\chi\|_{K},\qquad
\cos\theta_K=\hat\chi\!\cdot\!e_{\text{soft}}
=1\ (\pm\varepsilon_\chi).
\]
Failure of positive definiteness or any measurable misalignment,
$\cos\theta_K<1-\varepsilon_\chi$, falsifies the alignment principle.

\paragraph*{\textbf{(6) Basis/invariance checks.}}
Integer transports
$\Delta W\to U_{\rm row}\Delta W\,V_{\rm col}$ with
$U_{\rm row},V_{\rm col}\in GL(\mathbb{Z})$
preserve
$\ker_{\mathbb{Z}}(\Delta W^\top)=\operatorname{span}_{\mathbb{Z}}\{\pm\chi\}$.
Thus $\Xi$, $\Pi(\Xi)$, and the lab--null are basis invariant.
Any basis under which these quantities change falsifies the certificate.

\subsection*{Reporting protocol (reproducibility)}
For any dataset or update:
(i) publish $K$ with eigenpairs and $\cos\theta_K$;
(ii) fit $(A,B)$ in $\Delta G/G$ vs $s$ with uncertainties;
(iii) compute LOO values with propagated errors;
(iv) evaluate $Z_G$ from PDG/CODATA pins;
(v) report GW/PPN consistency ($c_T$, $m_g$ bounds).
All quantities and artifacts are reproducible from the Zenodo archive
\cite{demasi_gage_repo_v1_0_0_2025}.
% =========================================================

\section{Discussion}
\label{sec:discussion}

The construction reduces to a minimal, basis--invariant chain:
\begin{align*}
    \text{SNF certificate }\chi
\;\Rightarrow\;&\;
\text{alignment }(K\succ0,\ \chi\parallel e_{\chi}) \\
&\Rightarrow\;
\text{even curvature gate }(\Pi'(\Xi_{\rm eq})=0) \\
&\Rightarrow\;
\text{GR tensor sector + quadratic lab--null}.
\end{align*}

Two independent Standard--Model structures—the primitive integer kernel of
one-loop decoupling and the soft eigenmode of the Fisher/kinetic metric—select
the same direction in log--coupling space.  This identifies the aligned depth
$\Xi=\chi\cdot\hat\Psi$ and the electroweak anchor
$\Omega=\hat\alpha_s^{16}\hat\alpha_2^{13}\hat\alpha^{2}$.
Because $\Omega=e^{\Xi}$, once $\Xi$ is fixed by the integer certificate, the
gravitational anchor $\Omega(M_Z)$ and the electroweak--scale coupling
$G(M_Z)=(\hbar c/m_p^2)\,\Omega(M_Z)$ follow directly with no additional
assumptions.  
The even curvature gate $\Pi(\Xi)$ then promotes this anchor to
$G(Q)=G(M_Z)\Pi(\Xi(Q))$ while preserving the GR equilibrium limit:
$\Pi(\Xi_{\rm eq})=1$, $\Pi'(\Xi_{\rm eq})=0$, and a massless, luminal,
helicity~$\pm2$ tensor sector.

\medskip
\noindent\textbf{Interpretation: GR normalization from the electroweak anchor}
The appearance of the Einstein--Hilbert normalization is not an external
assumption but a geometric consequence of the aligned depth.  In General
Relativity the tensor sector is fixed by
\begin{equation}
\mathcal{L}_{\rm EH}
  = \frac{M_P^2}{2}\,R
  = \frac{1}{16\pi G_N}\,R ,
\label{eq:EH-normalization}
\end{equation}
so that $M_P^{-2}=8\pi G_N$.  GR therefore provides the 
\emph{form} of the tensor action but not the origin or numerical value of
its coupling.

In the present construction, the Standard Model integer certificate
and the soft eigenmode of the Fisher metric select a distinguished
log--coupling direction
$\Xi=\chi\cdot\hat\Psi$, and the electroweak anchor
\begin{equation}
\Omega(M_Z)=e^{\Xi_{\rm eq}}
           =\hat\alpha_s^{16}\hat\alpha_2^{13}\hat\alpha^{2},
\end{equation}
is fixed entirely by SM data at $\mu=M_Z$.  The SM-derived gravitational
coupling at the electroweak scale,
\begin{equation}
G(M_Z)
  = \frac{\hbar c}{m_p^2}\,\Omega(M_Z),
\label{eq:G-from-Omega}
\end{equation}
arises independently of GR and represents the projection of gauge
information along the aligned soft mode.

Matching the curvature coefficient of the Einstein--Hilbert term to the
SM-derived anchor yields
\begin{equation}
\frac{M_P^2}{2}
  = \frac{1}{16\pi G(M_Z)}
  \quad\Longleftrightarrow\quad
  G(M_Z)=\frac{1}{8\pi M_P^2},
\label{eq:G-match}
\end{equation}
so the electroweak anchor reproduces precisely the GR normalization.
The equality is therefore a \emph{consistency outcome} of the alignment
geometry rather than an imposed identification.  GR supplies the tensor
dynamics, while the SM supplies the value of the coupling through
$\Omega(M_Z)$.  No circularity is involved: GR assumes the existence of
gravity, whereas GEOMETRY predicts its strength from Standard Model
inputs alone.


\medskip
\noindent\textbf{Fixed vs.\ environmental quantities.}
The integer certificate $\chi=(16,13,2)$ and the metric--alignment scales
$(\sigma_\chi,\|\chi\|_K,\Lambda_\chi)$ are determined entirely by the SM
spectrum and renormalization geometry at $\mu=M_Z$.  
No free parameters enter.  
The gate width $\sigma_\chi$ is fixed by matching the gate curvature to the
Fisher curvature $F_\chi$ along the aligned direction, and
$\Lambda_\chi=\sigma_\chi/\|\chi\|_{K}$ sets the depth scale linking internal
displacement to gravitational response.  
In contrast, $\Xi_{\rm eq}$ and the displacement $\delta\Xi$ are environmental
boundary data determined by the physical system under study.

\medskip
\noindent\textbf{Geometry vs.\ alternative scenarios.}
The framework differs from scalar--tensor, screening, or modified--gravity
models in that no new fields, potentials, or interactions are introduced.
All response arises from the geometry of log--coupling space and its aligned
soft mode.  
The absence of the Brans--Dicke--like linear term is a consequence of even
parity, $\Pi'(\Xi_{\rm eq})=0$, rather than a dynamical assumption.  
At equilibrium the effective Lagrangian,
\[
\mathcal{L}^{\rm eff}
= \frac{\Pi(\Xi)}{16\pi G(M_Z)}\,R + \cdots,
\]
reduces exactly to General Relativity while retaining a fixed,
parameter-free curvature--response determined by SM data.

\medskip
\noindent\textbf{Closure and falsifiability.}
The dimensionless coupling $\alpha_G^{(pp)}=G_N m_p^2/(\hbar c)$ enters only
in a posteriori comparison, defining the closure ratio
$Z_G=\alpha_G^{(pp)}/\Omega(M_Z)$, which tests the SM-derived $G(M_Z)$ against
the measured Newtonian coupling.  
Together with leave--one--out self-consistency and the strictly quadratic
lab-null, this yields a sharp falsifier set: any odd laboratory response,
closure mismatch beyond propagated uncertainties, or deviation from massless,
luminal $\pm2$ tensor propagation at $\Xi_{\rm eq}$ rules out the mechanism.

\medskip
\noindent\textbf{Scope and extensions.}
The present work is restricted to the static, equilibrium geometry.  
No time dependence of $\Xi$ or dynamical evolution of the alignment mode is
introduced.  
A companion paper (GEOMETRY~II) develops the tensor sector beyond equilibrium
and establishes a positive spectral gap using the same alignment and parity
structure.  
A further extension (GEOMETRY~III) introduces dynamical alignment and studies
the off-equilibrium evolution of $\Xi$.  
These generalizations build on—but do not modify—the equilibrium geometry
established here.

\medskip
\noindent\textbf{Summary.}
Electroweak-scale alignment between the SNF integer certificate and the Fisher
metric selects a distinguished depth coordinate $\Xi$.  
An even curvature gate along this axis yields a parameter-free, SM-anchored
gravitational coupling whose equilibrium tensor sector matches GR exactly.  
The framework is quantitatively falsifiable through laboratory response,
closure with $G_N$, leave–one–out forecasts, and tensor-propagation
constraints.  
The alignment geometry and curvature structure developed here form the static
foundation for the dynamic and spectral components of the GEOMETRY program.

\section{Conclusion}
\label{sec:conclusion}

We have shown that the Standard Model gauge sector at $\mu=M_Z$ contains
sufficient internal structure to define a gravitational coupling without
introducing new fields or modifying the tensor sector of General Relativity.
Two independent ingredients—the primitive integer kernel of one-loop
decoupling and the soft eigenmode of the Fisher/kinetic metric—select the same
direction in log–coupling space.  This alignment defines a unique depth
coordinate $\Xi=\chi\!\cdot\!\hat\Psi$ and its electroweak anchor
$\Omega=\hat\alpha_s^{16}\hat\alpha_2^{13}\hat\alpha^{2}$.
Because $\Omega=e^{\Xi}$, the SM-derived coupling
$G(M_Z)=(\hbar c/m_p^2)\,\Omega(M_Z)$ follows directly from the integer
certificate without additional parameters.

An even curvature gate $\Pi(\Xi)$, with width fixed by matching its curvature
to the Fisher curvature along the aligned axis, promotes this anchor to a
running coupling $G(Q)=G(M_Z)\Pi(\Xi(Q))$ while preserving the equilibrium GR
limit.  At $\Xi_{\rm eq}$ the theory reduces to massless, luminal
helicity~$\pm2$ propagation with no scalar–tensor mixing and no Pauli–Fierz
mass term.

The resulting predictions are quantitative and parameter free.  
The laboratory response obeys the strictly quadratic relation
\[
\frac{\Delta G}{G}=(\delta\Xi/\sigma_\chi)^2,
\]
with the linear term removed by even parity.  
Closure is captured by the ratio
$Z_G=\alpha_G^{(pp)}/\Omega(M_Z)$, providing a direct comparison between the
SM-anchored $G(M_Z)$ and the measured Newtonian coupling $G_N$.  
Leave-one-out consistency of the three gauge couplings supplies an additional
internal test.  
Any violation of these conditions—odd laboratory response, closure mismatch,
LOO inconsistency, or deviation from the GR tensor limit—falsifies the
alignment mechanism.

The analysis here is restricted to the static, equilibrium geometry in which
$\Pi(\Xi_{\rm eq})=1$ and the tensor kernel reduces to the Lichnerowicz
operator of GR.  This establishes the electroweak alignment structure that
underlies the broader GEOMETRY program.  A companion paper (GEOMETRY~II)
extends the tensor sector beyond equilibrium and demonstrates a finite spectral
gap, while GEOMETRY~III introduces dynamical alignment and analyzes the
off-equilibrium evolution of the depth coordinate~$\Xi$.  
These developments build systematically on the equilibrium geometry established
here.

\paragraph{Acknowledgments.}
This work was conducted independently with no external funding.  The author
thanks the Particle Data Group (PDG), CODATA, Overleaf, and the open-source
Python ecosystem for publicly accessible tools and data.  An AI-assisted
writing tool (OpenAI ChatGPT) was used for language editing and organizational
assistance; all scientific content and responsibility rest with the author.
The author declares no competing interests.

\paragraph*{Data availability.}
All reproducibility materials are archived in the Zenodo repository
\cite{demasi_gage_repo_v1_0_0_2025} (\texttt{GAGE\_repo v1.0.0}, DOI
\href{https://doi.org/10.5281/zenodo.17537647}{10.5281/zenodo.17537647}),
including pins, scripts, figure data, and build manifests.
\textit{Build artifacts (SHA–256):}
\texttt{results.json}=08f0371b31de\ldots c7cd5edc;
\texttt{metric\_results.json}=e0e3bee8a70c\ldots b9b251b6451;
\texttt{stdout.txt}=0f232a0be6f8\ldots 6c7cd5edc.  
Additional materials are available from the author upon reasonable request.

\paragraph*{Outlook.}
Future work will examine how the even-gate symmetry extends to dynamical and
spectral sectors, including the time-evolution operator and the curvature
spectrum.  If validated, the GEOMETRY program would provide a continuous link
from Standard-Model information geometry to the equilibrium, dynamical, and
spectral structure of gravitation.

\bibliographystyle{iopart-num}
\bibliography{geo_cqg_refs}
\end{document}



