\documentclass{iopjournal}
\usepackage{amsmath,amssymb,amsfonts}
\usepackage{lmodern}
\usepackage{float}
\usepackage{graphicx}

\begin{document}

\articletype{Paper}%

\title{GEOMETRY I: SM-derived gravitational coupling $G(M_Z)$ anchored at the electroweak scale}

\author{Michael DeMasi$^1$}
\affil{$^1$Independent Researcher, Milford, CT, USA}

\email{demasim90@gmail.com}

\begin{abstract}
Under the minimal internal constraints of the Standard Model (SM)---no new
fields, no tunable functions, and fixed renormalization at $\mu=M_Z$ in the
$\overline{\mathrm{MS}}$ scheme---we show that a gravitational normalization can
be derived from SM data alone, rather than introduced as an external input. This
construction does not introduce a propagating scalar or modify General
Relativity; $\Xi$ is an internal aligned coordinate and $G(M_Z)$ denotes an
electroweak-anchored normalization rather than a varying-$G$ framework.

At one loop, the SM decoupling matrix admits a unique primitive integer
left-kernel $\chi=(16,13,2)$ in Smith normal form (verified using a standard
integer-normal-form algorithm). This selects an aligned depth
$\Xi=\chi\!\cdot\!\hat\Psi$ in log-coupling space. Independently, the
positive-definite Fisher/kinetic metric identifies a soft eigenmode of maximal
responsiveness; the two structures align numerically ($\cos\theta\simeq1$),
fixing $\Xi$ as the unique admissible depth coordinate. Exponentiation then
yields a parameter-free electroweak anchor
$\Omega=e^{\Xi}=\hat\alpha_s^{16}\hat\alpha_2^{13}\hat\alpha^{2}$ and an
SM-derived gravitational normalization
\[
G(M_Z)=\frac{\hbar c}{m_p^{2}}\,\Omega(M_Z),
\]
with $m_p$ taken as the natural baryonic dimensional anchor.

An even, equilibrium-normalized curvature gate $\Pi(\Xi)$ satisfying
$\Pi'(\Xi_{\rm eq})=0$ and fixed curvature scale
$\sigma_\chi^{-2}=\hat\chi^{\!\top}K\hat\chi$ promotes $G(M_Z)$ to a
position-dependent coupling $G(x)=G(M_Z)\Pi(\Xi(x))$ while preserving the
massless, luminal helicity-$\pm 2$ tensor sector. Near equilibrium the curvature
response is strictly quadratic,
$\Delta G/G=(\delta\Xi/\sigma_\chi)^2$, providing a direct laboratory falsifier
with no linear term. Comparison with the measured Newtonian coupling enters only
\emph{a posteriori} through a closure ratio, not as an input or calibration.

The construction is therefore fixed entirely by SM data at $\mu=M_Z$, introduces
no tunable parameters, preserves the GR tensor limit, and yields a reproducible,
experimentally testable gravitational coupling anchored at the electroweak
scale. All figures, scripts, and numerical values are generated from public
inputs via an archived, hash-verified build workflow. All statements apply to
the equilibrium SM geometry; sourcing, stress--energy coupling, and dynamical
extensions are deferred to future work.
\end{abstract}

\keywords{general relativity, quantum gravity, Standard Model, gauge theory, emergent gravity, renormalization group}




\section{Introduction}
\label{sec:intro}

We restrict throughout to Standard Model (SM) data at $\mu = M_Z$ in the
$\overline{\mathrm{MS}}$ scheme, introduce no new fields, parameters, or
tunable functions, and retain the massless, luminal helicity-$\pm2$ tensor
sector of General Relativity (GR). The aim is to determine whether the SM
contains sufficient internal structure to define a gravitational
normalization without modifying GR or enlarging its field content. All
numerical values and figures presented below are generated directly from
public SM inputs using a reproducible, hash-verified build workflow archived
under a public DOI (see Data Availability).

The SM provides a quantitative description of the three gauge interactions
and their renormalization-group (RG) evolution, but it does not internally
specify Newton’s gravitational constant $G_N$. In GR, the Einstein--Hilbert
term
\begin{equation}
  \mathcal{L}_{\rm EH} = \frac{1}{16\pi G_N} R
\end{equation}
contains an empirically measured coupling: GR specifies how curvature
responds to stress--energy but does not determine the numerical normalization
of that response. By contrast, the three electroweak-scale SM couplings
$(\hat\alpha_s,\hat\alpha_2,\hat\alpha)$ are RG-predictive, experimentally
constrained, and scheme-consistent at $\mu = M_Z$. This motivates the central
question:
\begin{quote}
\emph{Does the SM gauge sector at $\mu = M_Z$ contain sufficient,
basis-invariant structure to fix a gravitational normalization without new
degrees of freedom or modification of GR?}
\end{quote}

Two rigid SM structures, normally analyzed separately, play the key role:
(i) the integer lattice arising from one-loop decoupling, and (ii) the
Fisher/kinetic metric on log--coupling space. Evaluated together at
$\mu = M_Z$, they identify a single aligned depth direction and thereby fix a
dimensionless electroweak anchor, allowing a gravitational normalization to
follow as a consequence rather than as an externally imposed parameter.

At one loop, the SM decoupling matrix is an exact integer matrix whose Smith
normal form (SNF) admits a unique primitive left-kernel generator (up to
overall sign), obtained using a standard integer-normal-form algorithm
implemented verbatim in the accompanying reproducibility scripts:
\begin{equation}
  \chi = (16,13,2).
  \label{eq:chi}
\end{equation}
This vector defines a depth coordinate
\begin{equation}
  \Xi = \chi \cdot \hat\Psi,
  \qquad
  \hat\Psi = (\ln\hat\alpha_s,\ln\hat\alpha_2,\ln\hat\alpha),
  \label{eq:xidepth}
\end{equation}
in log--coupling space. Independently, the positive-definite (equilibrium)
Fisher/kinetic metric $K$ (hereafter $K \equiv K_{\rm eq}$), constructed from
one-loop sensitivity data, possesses a soft eigenmode of maximal
responsiveness. Using the same SM input pins and renormalization scheme, we
find that the integer direction $\chi$ is numerically aligned with the
softest eigenvector of $K$, with $\cos\theta \simeq 1$. Thus the aligned depth
$\Xi$ is not a model assumption but the coordinate jointly selected by integer
rigidity and metric softness.

Exponentiating the aligned depth yields the electroweak anchor
\begin{equation}
  \Omega \equiv e^{\Xi}
  = e^{\chi\cdot\hat\Psi}
  = \hat\alpha_s^{16}\,\hat\alpha_2^{13}\,\hat\alpha^{2},
  \label{eq:omega}
\end{equation}
which defines an SM-derived electroweak-scale gravitational normalization
\begin{equation}
  G(M_Z) = \frac{\hbar c}{m_p^{2}}\,\Omega(M_Z),
  \label{eq:gmz}
\end{equation}
with no adjustable parameters. The proton mass $m_p$ is chosen as the
dimensional anchor because laboratory and astrophysical determinations of
$G_N$ predominantly probe baryonic (proton-dominated) matter; alternative
choices such as $m_e$ or $m_n$ simply rescale the same dimensionless anchor
$\Omega$.

The remainder of this work examines how this normalization is promoted to a
spacetime coupling consistent with GR, how parity constraints fix the
curvature response, and how the near-equilibrium prediction
\begin{equation}
  \frac{\Delta G}{G} = \left(\frac{\delta\Xi}{\sigma_\chi}\right)^{2}
\end{equation}
yields a direct empirical test.

\paragraph*{Summary of results and roadmap.}
Section~\ref{sec:lattice_depth} establishes the integer structure and the
unique primitive kernel $\chi$ of the one-loop decoupling matrix.
Section~\ref{sec:metric_softness} constructs the Fisher/kinetic metric and
verifies its numerical alignment with $\chi$, fixing the admissible depth
coordinate $\Xi$. Section~\ref{sec:curvature_gate} introduces the
parity-preserving curvature gate $\Pi(\Xi)$ and determines its fixed curvature
scale $\sigma_\chi$. Section~\ref{sec:SM_derived_G} derives the
electroweak-anchored normalization $G(M_Z)$ and promotes it to
$G(x)=G(M_Z)\Pi(\Xi(x))$ without modifying GR.
Section~\ref{sec:predictions} presents the quadratic lab-null, closure ratio,
and falsifiers. Section~\ref{sec:discussion} summarizes the implications and
outlines extensions to dynamical settings in future work.

\paragraph*{Program and provenance.}
This work is the first in a sequence collectively denoted
GEOMETRY (Gauge Exponential Omega Metric Even Tensor Running Yield).
GEOMETRY~I is restricted to the static, equilibrium geometry and derives an
electroweak-anchored gravitational coupling from SM data alone. All inputs,
constants, and covariance matrices are taken from established references, and
all calculations use the $\overline{\mathrm{MS}}$ scheme at $\mu = M_Z$.
Integer and metric verifications are reproduced automatically from the
archived build environment.

Renormalization conventions follow
Weinberg~\cite{Weinberg1996_QFTv2},
Peskin and Schroeder~\cite{PeskinSchroeder1995_QFT},
and Langacker~\cite{Langacker2017_SMBeyond}.
Decoupling and integer-lattice methods follow
Appelquist and Carazzone~\cite{AppelquistCarazzone1975_Decoupling},
Kannan and Bachem~\cite{KannanBachem1979_SNF},
and Newman~\cite{Newman1997_SNF}.
Electroweak pins, covariance matrices, and physical constants are taken from
the Particle Data Group and CODATA
\cite{PDG2024,PDG2024_EWReview,PDG2025_GaugeHiggs,CODATA2022_RMP}.
Two-loop RG coefficients follow
Machacek and Vaughn
\cite{Machacek1983_TwoLoopI,Machacek1984_TwoLoopII}
and Luo \textit{et al.}~\cite{Luo2003_TwoLoopSM},
and the running of $\hat\alpha$ follows Jegerlehner
\cite{Jegerlehner2019_alphaRun}.
Gravitational and observational constraints follow
Carroll~\cite{Carroll2004_SG},
Will~\cite{Will2014_LRR_TestsGR},
Bertotti \textit{et al.}~\cite{Cassini2003_PPN},
and Abbott \textit{et al.} (LVK)~\cite{LVK2021_TestsGR}.
No additional data, fitting, or tuning is employed.

\paragraph*{Interpretation of $G(M_Z)$.}
The normalization $G(M_Z)$ introduced in Eq.~\eqref{eq:gmz} is a
renormalization-anchored quantity, not a time- or space-varying gravitational
constant. No modification of General Relativity, its field equations, or the
equilibrium value of the Newtonian coupling is proposed. The construction
identifies a theoretically determined equilibrium normalization compatible
with GR, rather than a dynamical or varying-$G$ scenario.

\paragraph*{Interpretation of $\Xi$ and $\delta\Xi$.}
The aligned depth $\Xi = \chi\!\cdot\!\hat\Psi$ is an internal coordinate on
log--coupling space selected jointly by integer rigidity and metric softness.
It is not a propagating scalar field and carries no independent dynamics in
GEOMETRY~I. The displacement $\delta\Xi = \Xi - \Xi_{\rm eq}$ labels how local
curvature samples the aligned soft direction of the gauge sector; at static
equilibrium $\delta\Xi = 0$ everywhere. In this work $\Xi$ functions only as
an internal coordinate determining the curvature response through $\Pi(\Xi)$;
stress--energy sourcing and time dependence are deferred to dynamical
extensions in future work.

\begin{table}[H]
\centering
\includegraphics[width=\linewidth]{fig_ass.pdf}
\caption{
Assumptions and scope of \textsc{Geometry~I}.  Entries A1–A10 summarize
the framework, field content, equilibrium restriction, metric and integer
structures, gate shape, tensor sector, dimensional anchor, data inputs,
and perturbative stability assumptions used throughout.
}
\label{tab:assumptions}
\end{table}

\section{Integer lattice and the aligned depth coordinate}
\label{sec:lattice_depth}

We begin by identifying the structures that remain fixed once we restrict to
the Standard Model at $\mu = M_Z$ in the $\overline{\mathrm{MS}}$ scheme with
no new fields, tunable functions, or adjustable parameters. At this scale the
one-loop decoupling matrix has exactly integer entries determined solely by
representation content and spectator multiplicities. This endows
log--coupling space with a natural $\mathbb{Z}$-module structure and admits a
classification under $\mathrm{GL}(3,\mathbb{Z})$ via the Smith normal form
(SNF). Because the SNF is a unique canonical form over the integers, its
kernel is a fixed property of the SM representation lattice rather than a
model choice. Its computation uses only integer-preserving row/column
operations, reproduced verbatim in the accompanying reproducibility archive.
Since the SNF kernel is invariant under all unimodular basis changes, this
structure is basis-independent and scheme-consistent at one loop.

Applying SNF to the SM one-loop decoupling matrix yields a unique primitive
left-kernel generator (up to overall sign),
\begin{equation}
  \chi = (16,13,2),
  \label{eq:chi_def}
\end{equation}
which is the sole integer direction annihilating the decoupling matrix. No
additional integer kernel vectors exist, and unimodular transformations
(integer determinant $\pm 1$) cannot change the kernel rank or its primitive
representative. Thus $\chi$ is fixed by the SM’s integer structure and does
not depend on renormalization schemes, higher-loop corrections, numerical
fitting, or phenomenological input.

Let the renormalized gauge couplings at $\mu = M_Z$ be
$\hat\alpha_s$, $\hat\alpha_2$, and $\hat\alpha$, and define
log--coupling coordinates
\begin{equation}
  \hat\Psi = (\ln\hat\alpha_s,\,\ln\hat\alpha_2,\,\ln\hat\alpha),
  \label{eq:Psi_def}
\end{equation}
together with the associated depth coordinate
\begin{equation}
  \Xi = \chi \cdot \hat\Psi
  = 16\ln\hat\alpha_s + 13\ln\hat\alpha_2 + 2\ln\hat\alpha.
  \label{eq:Xi_def}
\end{equation}

\paragraph*{Interpretation of $\Xi$}
The scalar $\Xi$ defined in Eq.~\eqref{eq:Xi_def} is an internal coordinate on
log--coupling space and is not introduced as a propagating, canonical, or
dynamical scalar field. No new degrees of freedom are added, and no
scalar--tensor, dilaton, chameleon, or Brans--Dicke structure is implied.
Throughout GEOMETRY~I, $\Xi$ functions solely as an aligned internal
coordinate selected jointly by the integer lattice and the Fisher/kinetic
softness.

This coordinate is the sole nontrivial integer-invariant linear combination of
the log--couplings and therefore the unique depth coordinate compatible with
the SM integer lattice. Under the stated constraints, any function of the
three couplings that respects integer invariance must reduce to a function of
$\Xi$ alone; this is a direct consequence of $\mathrm{GL}(3,\mathbb{Z})$
rigidity and does not depend on fitting, phenomenology, or model-specific
choices.

Exponentiating transports $\Xi$ back into the coupling manifold and defines
the dimensionless electroweak anchor
\begin{equation}
  \Omega \equiv e^{\Xi}
  = e^{\chi\cdot\hat\Psi}
  = \prod_i e^{\chi_i \ln\hat\alpha_i}
  = \hat\alpha_s^{16}\,\hat\alpha_2^{13}\,\hat\alpha^{2}.
  \label{eq:Omega_def}
\end{equation}
Exponentiation is the natural map from log--coupling space to the
multiplicative coupling manifold, making $\Omega$ the uniquely associated
dimensionless quantity determined by the integer depth coordinate.

Up to this point, no geometric, dynamical, or gravitational assumptions have
been introduced. Equation~\eqref{eq:Omega_def} follows directly from the SM
representation content and the existence of a unique primitive integer kernel,
verified in a scheme-consistent, integer-preserving manner. The next section
shows that the same direction $\Xi$ is independently selected by the soft
eigenmode of the Fisher/kinetic metric, establishing that $\Xi$ is not only
algebraically admissible but also physically responsive and maximally
sensitive.

\section{Metric softness and alignment}
\label{sec:metric_softness}

The integer-aligned depth coordinate $\Xi$ identified in
Section~\ref{sec:lattice_depth} is fixed by the SM representation lattice and
is unique under $\mathrm{GL}(3,\mathbb{Z})$ invariance. We now show that the
same coordinate is independently selected by the geometric softness of the
gauge sector, quantified by a Fisher/kinetic metric constructed from the
one-loop sensitivity of the renormalization-group (RG) flow at $\mu = M_Z$ in
the $\overline{\mathrm{MS}}$ scheme, using the same inputs and electroweak
pins as in Section~\ref{sec:intro}. This metric is not an additional
structure; it is derived directly from the SM $\beta$ functions, so no new
dynamical fields or tunable functions enter this step.

Let $\beta_i(\hat\alpha_s,\hat\alpha_2,\hat\alpha)$ denote the RG flow of the
gauge couplings, and define log--coupling coordinates $\hat\Psi$ as in
Eq.~\eqref{eq:Psi_def}. Following the standard construction, the equilibrium
Fisher/kinetic metric on log--coupling space is defined by
\begin{equation}
  K_{ij}
  =
  \frac{\partial}{\partial \hat\Psi_j}
  \left(\frac{\beta_i}{\hat\alpha_i}\right)_{\!\rm eq},
  \label{eq:metric_def}
\end{equation}
with all quantities evaluated at $\mu = M_Z$. The symmetric matrix $K$ is
positive definite ($K \succ 0$), so it admits three orthonormal eigenvectors
with strictly positive eigenvalues; large eigenvalues correspond to stiff RG
response, and small eigenvalues correspond to soft RG response. In
particular, the smallest eigenvalue defines a distinguished soft direction in
log--coupling space fixed by SM dynamics alone.

Let $e_\chi$ denote the unit eigenvector associated with the smallest
eigenvalue of $K$, corresponding to the softest direction in log--coupling
space. By direct computation we find that the primitive integer kernel vector
$\chi$ from Eq.~\eqref{eq:chi_def} is aligned with $e_\chi$ to numerical
precision. Writing
\begin{equation}
  \hat\chi = \frac{\chi}{\|\chi\|}, \qquad
  \cos\theta \equiv \hat\chi \cdot e_\chi,
  \label{eq:alignment_def}
\end{equation}
one obtains
\begin{equation}
  \cos\theta = 1 - \varepsilon_\chi,
  \qquad
  \varepsilon_\chi \lesssim 10^{-8},
\end{equation}
in our numerical evaluation.\footnote{This computation uses the one-loop
$\beta$ functions, electroweak pins, and physical constants cited in
Section~\ref{sec:intro}, with all intermediate values and eigenvectors
generated reproducibly in the archived build workflow.} This agreement is not
imposed but arises from two independent structures: (i) the integer lattice
determined by SM field representations, and (ii) the Fisher/kinetic response
geometry of the RG flow. Their numerical coincidence identifies $\Xi$ as both
the unique integer-invariant depth coordinate and the physically responsive
depth direction. No other linear combination of the log--couplings satisfies
both criteria simultaneously.

To parameterize displacements along the soft direction, define the
Euclidean-normalized aligned vector $\hat\chi = \chi/\|\chi\|$ and write
\begin{equation}
  \delta\Xi
  =
  \hat\chi \cdot (\hat\Psi - \hat\Psi_{\rm eq}),
  \label{eq:deltaXi_def}
\end{equation}
where $\hat\Psi_{\rm eq}$ is evaluated at $\mu = M_Z$. The scalar $\delta\Xi$
therefore measures displacement strictly along the softest direction of the
Fisher/kinetic metric, while transverse components are suppressed by strictly
larger eigenvalues. Because $\Xi$ is simultaneously the unique
integer-invariant depth coordinate, any admissible curvature or response
function consistent with both integer invariance and metric softness must
depend only on $\delta\Xi$ under the stated SM-only assumptions. In
particular, no transverse combination can contribute without violating either
$\mathrm{GL}(3,\mathbb{Z})$ rigidity or the eigenvalue ordering of $K$.

This concurrence completes the structural irreversibility chain: no
alternative depth coordinate is compatible with both
$\mathrm{GL}(3,\mathbb{Z})$ invariance and Fisher/kinetic softness. The next
section introduces the curvature gate $\Pi(\Xi)$, which follows once parity
and equilibrium constraints are imposed.

\section{Curvature gate and parity constraint}
\label{sec:curvature_gate}

Since the aligned depth coordinate $\Xi$ is simultaneously the unique
$\mathrm{GL}(3,\mathbb{Z})$-invariant depth direction
(Section~\ref{sec:lattice_depth}) and the Fisher/kinetic soft eigenmode
(Section~\ref{sec:metric_softness}), any admissible curvature response
consistent with the stated SM-only internal constraints must depend only on the
scalar displacement $\delta\Xi$ of Eq.~\eqref{eq:deltaXi_def}. In this section
we show that the curvature response function $\Pi(\Xi)$ is fixed, up to
normalization and sign, by equilibrium, parity, and Fisher curvature, and that
these conditions select an even Gaussian with no tunable parameters under the
stated assumptions. Throughout, $\Pi$ is a function of the internal coordinate
$\Xi$ alone and does not represent a propagating scalar, auxiliary field, or
additional dynamical degree of freedom.

We consider a multiplicative scalar gate applied to the Einstein--Hilbert term:
\begin{equation}
  \mathcal{L}^{\rm eff}
  =
  \frac{1}{16\pi G(M_Z)}\,\Pi(\Xi)\,R,
  \label{eq:Leff_gate}
\end{equation}
where $G(M_Z)$ is defined in Eq.~\eqref{eq:gmz}, and no new fields, mass
terms, or independent kinetic terms are introduced. The gate must satisfy:
\begin{itemize}
  \item[(i)] \textbf{Equilibrium normalization:}
             $\Pi(\Xi_{\rm eq}) = 1$ so that GR is recovered at equilibrium.
  \item[(ii)] \textbf{Parity preservation:}
             $\Pi(\Xi_{\rm eq} + \delta\Xi) = \Pi(\Xi_{\rm eq} - \delta\Xi)$,
             forbidding odd powers of $\delta\Xi$ and ensuring a massless
             helicity-$\pm2$ tensor sector. This condition excludes any
             Brans--Dicke-type effective scalar that would induce a linear mode.
  \item[(iii)] \textbf{Curvature matching:}
             the second derivative $\Pi''(\Xi_{\rm eq})$ must reproduce the
             Fisher curvature along the aligned direction, ensuring that
             departures from equilibrium are penalized with the same softness
             scale as the RG geometry.
  \item[(iv)] \textbf{Analytic minimality:}
             no additional coefficients, tunable parameters, or non-analytic
             completions are permitted.
\end{itemize}

\subsection*{Lemma 1 (Parity restriction)}
Under (i)–(ii), the Taylor expansion of $\Pi$ about $\Xi_{\rm eq}$ is
\begin{equation}
  \Pi(\Xi)
  =
  1
  + \frac{1}{2}\,\Pi''(\Xi_{\rm eq})\,(\delta\Xi)^2
  + \mathcal{O}\!\big((\delta\Xi)^4\big),
  \label{eq:Pi_expansion}
\end{equation}
with all odd powers forbidden, so any departure from equilibrium begins at
quadratic order.

\subsection*{Lemma 2 (Fisher curvature matching)}
Along the aligned direction $\hat\chi$,
\begin{equation}
  \sigma_\chi^2 \equiv \frac{1}{\hat\chi^\top K\,\hat\chi},
  \label{eq:sigma_chi_def}
\end{equation}
and consistency with Fisher softness requires
\begin{equation}
  \Pi''(\Xi_{\rm eq})
  =
  -\frac{2}{\sigma_\chi^2}.
  \label{eq:Pi_curvature}
\end{equation}

\smallskip
\noindent\emph{Proof sketch.}
The Fisher/kinetic metric penalizes displacements along $\hat\chi$ in
proportion to $\hat\chi^\top K\,\hat\chi = \sigma_\chi^{-2}$. Matching this
penalty to the quadratic response term in Eq.~\eqref{eq:Pi_expansion} and
enforcing stability fixes both the curvature scale and sign, ensuring that
$\Pi$ decreases away from equilibrium and thus preserves the GR tensor limit.

\subsection*{Theorem (Uniqueness of the curvature gate)}
Under assumptions (i)–(iv), the unique analytic, parity-even,
curvature-matched response consistent with the stated SM-only constraints is
\begin{equation}
  \boxed{
  \Pi(\Xi)
  =
  \exp\!\left[-\,\frac{(\delta\Xi)^2}{\sigma_\chi^2}\right]
  }
  \label{eq:Pi_gaussian}
\end{equation}
with no tunable parameters and no dependence on additional fields or auxiliary
potentials.

\paragraph*{Remark on analytic completion}
Polynomial completions of Eq.~\eqref{eq:Pi_expansion} introduce undetermined
higher-order coefficients that are neither fixed by parity nor by Fisher
curvature. Exponentiation provides an analytic completion with a single
dimensionless scale $\sigma_\chi$ fixed by Eq.~\eqref{eq:sigma_chi_def},
consistent with equilibrium normalization, even parity, curvature matching,
and the absence of tunable parameters. Any alternative completion either
requires additional dimensionful coefficients or breaks analyticity, violating
assumption (iv).

\smallskip
\noindent\emph{Proof sketch.}
Equation~\eqref{eq:Pi_expansion} fixes the quadratic coefficient; parity
enforces evenness, equilibrium fixes normalization, and analytic
parameter-minimality completes the response via exponentiation of the quadratic
form. Alternative completions require additional coefficients or non-analytic
terms and therefore violate assumption (iv).

\subsection*{Running gravitational coupling}
Substituting Eq.~\eqref{eq:Pi_gaussian} into Eq.~\eqref{eq:Leff_gate} yields
\begin{equation}
  G(x)
  =
  G(M_Z)\,\Pi(\Xi(x))
  =
  \frac{\hbar c}{m_p^2}\,\Omega(M_Z)\exp\!\left[-\frac{(\delta\Xi(x))^2}{\sigma_\chi^2}\right],
  \label{eq:running_G}
\end{equation}
which preserves the equilibrium tensor sector and introduces no new fields or
adjustable parameters. This expression is an SM-derived equilibrium
normalization anchored at $\mu = M_Z$, not a varying-$G$ theory and not a
modification of GR. Near equilibrium, the strict quadratic prediction
\begin{equation}
  \frac{\Delta G}{G}
  =
  \left(\frac{\delta\Xi}{\sigma_\chi}\right)^{2}
  + \mathcal{O}\!\big((\delta\Xi)^4\big)
\end{equation}
follows immediately.

\section{Electroweak anchor and SM-derived gravitational coupling}
\label{sec:SM_derived_G}

With $\Xi$ uniquely fixed by the integer lattice and Fisher/kinetic softness,
and with $\Pi(\Xi)$ determined by equilibrium, parity, and curvature matching,
we now connect these internal structures to the gravitational normalization
multiplying the Einstein--Hilbert term. No new inputs, parameters, or external
assumptions are introduced in this section; all quantities are Standard Model
objects evaluated at $\mu = M_Z$ in the $\overline{\mathrm{MS}}$ scheme. The
construction defines a renormalization-anchored normalization, not a
time- or space-varying gravitational constant and not a modification of GR.

Exponentiating the aligned depth coordinate transports $\Xi$ back into
gauge--coupling space and defines the dimensionless electroweak anchor
\begin{equation}
  \Omega \equiv e^{\Xi}
  = \hat\alpha_s^{16}\,\hat\alpha_2^{13}\,\hat\alpha^{2},
  \label{eq:Omega_recall}
\end{equation}
where hatted quantities denote $\overline{\mathrm{MS}}$ couplings at
$\mu = M_Z$. Equation~\eqref{eq:Omega_recall} is not an ansatz: it is the
unique exponentiation of the integer-invariant depth scalar $\Xi$ and
introduces no free coefficients or additional scales. $\Omega$ is therefore a
pure, dimensionless SM construct fixed entirely by measured electroweak-scale
couplings.

Because $\Omega$ is dimensionless, the available conversion to a gravitational
normalization without introducing new parameters is provided by dimensional
analysis in Planck units. This yields
\begin{equation}
  G(M_Z)
  \,\equiv\,
  \frac{\hbar c}{m_p^2}\,\Omega(M_Z),
  \label{eq:G_MZ_def}
\end{equation}
which defines the gravitational coefficient appearing in
Eq.~\eqref{eq:Leff_gate}. No phenomenological $G_N$ value is inserted and no
tuning or matching step occurs here. Under the stated internal constraints,
Eq.~\eqref{eq:G_MZ_def} follows from dimensional consistency, integer
invariance, and the absence of additional scales. The choice of $m_p$ does not
introduce a free parameter: any Standard Model mass scale can serve as a
dimensional anchor, and all choices differ only by fixed, known SM mass ratios.

\medskip
\noindent\textbf{Theorem (Electroweak anchoring of gravitational normalization).}
\emph{Under SM-only constraints at $\mu = M_Z$, with no new fields, tunable
functions, or adjustable parameters, the effective gravitational normalization
is fixed by Eq.~\eqref{eq:G_MZ_def}. No alternative admissible, dimensionless,
parity-preserving, integer-aligned construction arises under the stated
assumptions; any modification requires introducing a new scale or violating
integer invariance.}

\medskip
\noindent\emph{Proof.}
Equation~\eqref{eq:Omega_recall} is the unique exponentiation of the primitive
integer-invariant scalar $\Xi$. The factor $(\hbar c/m_p^2)$ is the unique
dimensionally consistent conversion available without introducing new physical
scales. Any modification of exponents, prefactors, or functional form violates
either integer invariance, metric softness, dimensional consistency, or
analytic minimality; any additive constant introduces a new parameter.
\hfill$\square$

\bigskip
\paragraph*{Notation for expansion coefficients.}
For later empirical interpretation it is useful to write the fractional
gravitational response as a Taylor expansion around equilibrium,
\begin{equation}
  \frac{\Delta G}{G}
  =
  A\,\delta\Xi
  +
  B\,(\delta\Xi)^2
  +
  \mathcal{O}\!\big((\delta\Xi)^3\big),
  \label{eq:AB_def}
\end{equation}
where $A$ and $B$ are not tunable parameters but the fixed Taylor coefficients
implied by the curvature gate $\Pi(\Xi)$ under the assumptions of
Section~\ref{sec:curvature_gate}. Parity and equilibrium normalization require
\begin{equation}
  A = 0,
  \qquad
  B = \frac{1}{\sigma_\chi^2},
  \label{eq:AB_values}
\end{equation}
so the first nonzero deviation from equilibrium is strictly quadratic with a
unit-normalized curvature scale when expressed in the dimensionless coordinate
$s \equiv \delta\Xi/\sigma_\chi$. Therefore
\begin{equation}
  \frac{\Delta G}{G}
  =
  s^{2}
  +
  \mathcal{O}(s^{4}),
  \qquad
  s \equiv \frac{\delta\Xi}{\sigma_\chi}.
  \label{eq:lab_null_s}
\end{equation}

\bigskip
\noindent\textbf{Corollary (Running gravitational coupling).}
\emph{Using the curvature gate of Eq.~\eqref{eq:Pi_gaussian}, the
spacetime-dependent gravitational coupling is}
\begin{equation}
  G(x)
  =
  G(M_Z)\,\Pi(\Xi(x))
  =
  \frac{\hbar c}{m_p^2}\,
  \Omega(M_Z)\,
  \exp\!\left[-\frac{(\delta\Xi(x))^2}{\sigma_\chi^2}\right].
  \label{eq:G_running_final}
\end{equation}

At equilibrium, $\Pi(\Xi_{\rm eq}) = 1$ and $G(x)$ reduces to the constant
$G(M_Z)$ determined purely from SM couplings. Away from equilibrium the scalar
response is fixed and strictly quadratic:
\begin{equation}
  \frac{\Delta G}{G}
  =
  \left(\frac{\delta\Xi}{\sigma_\chi}\right)^{2}
  + \mathcal{O}\!\big((\delta\Xi)^4\big),
  \label{eq:quadratic_prediction}
\end{equation}
with no linear term, no tunable scale, and no phenomenological matching
coefficients. Equation~\eqref{eq:quadratic_prediction} is therefore an
immediate, laboratory-accessible falsifier rather than a fitting ansatz.

\paragraph*{GR compatibility at equilibrium.}
At $\Xi = \Xi_{\rm eq}$ one has $\Pi(\Xi_{\rm eq}) = 1$, so the
Einstein--Hilbert action, field equations, diffeomorphism symmetry, and the
massless luminal helicity-$\pm2$ tensor sector are identical to those of
General Relativity. No infrared mass term, kinetic modification, or
propagating scalar is introduced at equilibrium.

\section{Predictions, closure, and falsifiers}
\label{sec:predictions}

All empirical consequences follow directly from the fixed Standard Model
constraints established in
Sections~\ref{sec:lattice_depth}--\ref{sec:SM_derived_G}.
No phenomenological parameters, tunable coefficients, or adjustable functions
enter any expression. The empirical status of the construction is therefore
decided by direct tests of the statements below; violation of \emph{any} item
falsifies the framework. Nothing in this section introduces a varying-$G$
interpretation or a modification of GR: all predictions concern the internal,
dimensionless response encoded by $\delta\Xi$ and the fixed curvature gate.

\subsection*{Fixed predictions}

\begin{enumerate}
\item \textbf{Electroweak anchoring of gravitational normalization}
\begin{equation}
  G(M_Z)
  =
  \frac{\hbar c}{m_p^2}\,\Omega(M_Z),
  \qquad
  \Omega(M_Z)=\hat\alpha_s^{16}\hat\alpha_2^{13}\hat\alpha^{2}.
\end{equation}
No external $G_N$ value is inserted; $G(M_Z)$ is fixed entirely from SM inputs.
This normalization is an internal consequence of integer invariance and
dimensional consistency, not a fit to gravitational data.

\item \textbf{Even curvature gate and quadratic response}
\begin{equation}
  \Pi(\Xi)
  =
  \exp\!\left[-\,\frac{(\delta\Xi)^2}{\sigma_\chi^2}\right],
  \qquad
  \frac{\Delta G}{G}
  =
  \left(\frac{\delta\Xi}{\sigma_\chi}\right)^{2},
\end{equation}
so the first nonzero departure from equilibrium is strictly quadratic. No
linear term or cubic correction is admissible under parity, equilibrium
normalization, and analytic minimality.

\begin{figure}[t]
  \centering
  \includegraphics[width=\columnwidth]{fig_gate.pdf}
  \caption{Even curvature gate $\Pi(\Xi)$ and quadratic parity-null prediction
  $\Delta G/G = s^{2}$ on the normalized depth coordinate
  $s=\delta\Xi/\sigma_\chi=\phi_\chi/\Lambda_\chi$.
  Dashed and solid curves are grayscale-safe and remain distinguishable under
  monochrome print rendering.}
  \label{fig:gate}
\end{figure}

\item \textbf{Fixed curvature scale}
\begin{equation}
  \sigma_\chi^2 = (\hat\chi^\top K \hat\chi)^{-1},
  \qquad
  \sigma_\chi \simeq 247.683,
  \qquad
  \|\chi\|_K \simeq 17.6278,
  \qquad
  \Lambda_\chi \equiv \frac{\sigma_\chi}{\|\chi\|_K} \simeq 14.0507.
\end{equation}
The curvature scale is determined solely by the Fisher/kinetic metric along
$\hat\chi$; no adjustable scale enters $\Pi(\Xi)$, and no phenomenological
normalization is allowed.
\end{enumerate}

\subsection*{Numerical illustration (PDG/CODATA inputs)}

Using current PDG and CODATA pins at $\mu = M_Z$ in the
$\overline{\mathrm{MS}}$ scheme, the aligned Fisher curvature and gate width are
\begin{equation}
  F_\chi \equiv \frac{1}{\sigma_\chi^2} \simeq 1.629\times10^{-5},
  \qquad
  \sigma_\chi \simeq 247.683.
\end{equation}
The electroweak anchor and proton--proton gravitational coupling are
\begin{equation}
  \Omega(M_Z) \simeq 6.4597\times10^{-39},
  \qquad
  \alpha_G^{(\mathrm{pp})} \simeq 5.9061\times10^{-39},
\end{equation}
yielding a closure ratio
\begin{equation}
  Z_G \equiv \frac{\Omega(M_Z)}{\alpha^{(\mathrm{pp})}_G} \simeq 1.0937,
\end{equation}
which represents a percent-level deviation without any form of tuning and is
interpreted solely as an \emph{a posteriori} consistency check, arising from a
fixed, parameter-free construction rather than an input, matching criterion, or
fitted quantity. The closure ratio is not used to calibrate or adjust the
framework.

\medskip
\noindent\textbf{Leave-one-out (LOO) forecast.}
Holding $\hat\alpha_2$ and $\hat\alpha$ fixed, the implied strong coupling is
\begin{equation}
  \hat\alpha_s^{\star}(M_Z)
  =
  \left[
    \frac{\alpha_G^{(\mathrm{pp})}}{\hat\alpha_2^{13}\hat\alpha^{2}}
  \right]^{1/16}
  =
  0.1173411 \pm 1.86\times10^{-5},
\end{equation}
which lies in $\simeq 0.7\sigma$ agreement with the PDG world average. This is
a postdiction check of a fixed construction, not a fitted parameter, and it
introduces no freedom to alter either exponent or normalization.

\medskip
With current pinned inputs, both the $\sim9\%$ closure deviation and the
$\sim0.7\sigma$ leave-one-out result indicate percent-level empirical
pressure on the construction, with no tunable coefficients.

\subsection*{Empirical falsifiers}

\begin{enumerate}
\item \textbf{No linear term}
\begin{equation}
  \left.\frac{d}{d(\delta\Xi)}\frac{\Delta G}{G}\right|_{\delta\Xi=0}
  \neq 0
  \quad\Longrightarrow\quad
  \text{falsified}.
\end{equation}
Any detectable linear dependence violates parity, equilibrium normalization,
and the absence of propagating scalars.

\item \textbf{Quadratic coefficient fixed ($B=1$)}
\begin{equation}
  \frac{\Delta G}{G}
  =
  1\cdot\left(\frac{\delta\Xi}{\sigma_\chi}\right)^{2}
  + \mathcal{O}\!\big((\delta\Xi)^4\big),
\end{equation}
and any measurable deviation in the leading quadratic coefficient falsifies.
No alternative curvature scale or normalization can be introduced without
violating Section~\ref{sec:curvature_gate}.

\item \textbf{Closure ratio consistency}
\begin{equation}
  Z_G
  =
  \frac{\Omega(M_Z)}{\alpha_G^{(\mathrm{pp})}}
  \quad\text{must be statistically consistent (including uncertainties).}
\end{equation}
Persistent disagreement falsifies; this is a consistency test of a fixed
prediction, not a calibration step.

\item \textbf{Tensor-sector preservation (equilibrium)}
Helicity-$\pm2$ modes must remain massless and luminal at equilibrium. Any
effective mass or kinetic deformation at $\Xi_{\rm eq}$ falsifies the gate
construction.

\item \textbf{Alignment stability}
\begin{equation}
  \cos\theta \simeq 1,
  \qquad
  \text{(integer projector $\chi$ aligned with soft eigenmode $e_\chi$).}
\end{equation}
Sustained misalignment falsifies. This condition is fixed by the integer SNF
structure and the Fisher/kinetic metric; no compensatory freedom exists.
\end{enumerate}

\subsection*{Sufficiency}

The construction is falsified if \emph{any one} of items (1)–(5) above fails;
no auxiliary assumptions, retuning, replacement coefficients, or post hoc
adjustments are permitted. These falsifiers exhaust all degrees of freedom
available under the SM-only constraints.

\section{Discussion and consequences}
\label{sec:discussion}

Sections~\ref{sec:lattice_depth}--\ref{sec:predictions} show that, under
Standard Model-only constraints at $\mu = M_Z$ in the $\overline{\mathrm{MS}}$
scheme, a gravitational normalization can be fixed without introducing new
fields, tunable functions, or free parameters. The construction does \emph{not}
modify General Relativity (GR) or propose an alternative gravitational theory;
rather, it identifies an internally determined normalization for the
Einstein--Hilbert term arising from fixed gauge-sector structure. At
equilibrium, the tensor sector remains massless, luminal, and
parity-preserving, and $G(M_Z)$ plays the same functional role as the
Newtonian coupling $G_N$. No effective modification of the Einstein field
equations occurs at $\Xi = \Xi_{\rm eq}$.

This reframes the gauge--gravity interface: instead of treating $G_N$ as an
externally supplied empirical parameter, the Standard Model contains sufficient
fixed integer and geometric structure to produce an electroweak-scale anchor.
The mechanism rests on three independently determined ingredients:
(i)~the unique primitive integer kernel $\chi = (16,13,2)$ of the one-loop
decoupling lattice, (ii)~the soft eigenmode of the positive-definite
Fisher/kinetic metric $K$, and (iii)~the uniquely determined, parity-even,
curvature gate $\Pi(\Xi)$. None introduce model freedom; each follows from
existing Standard Model representation and one-loop RG sensitivity data, with
all quantities generated from public inputs using a reproducible, hash-verified
build workflow. Together they form a closed alignment chain:
integer rigidity $\rightarrow$ metric softness $\rightarrow$ even curvature
response.

To avoid misinterpretation, we emphasize that $\Xi$ is an internal, aligned
coordinate in log--coupling space and is \emph{not} introduced as a dynamical,
canonical, or propagating scalar field. No Brans--Dicke, scalar--tensor,
dilaton, chameleon, $f(R)$, or scalar-curvature kinetic structure is added.
Likewise, $\Pi(\Xi)$ is not a potential, not a Lagrangian degree of freedom,
and not a new matter or mediator field; it is a curvature-response gate that
arises solely from SM-internal integer and metric constraints. The construction
therefore remains strictly within the SM + GR field content at equilibrium.

Experimental meaning follows entirely from the equilibrium prediction
\begin{equation}
  \frac{\Delta G}{G} = \left(\frac{\delta\Xi}{\sigma_\chi}\right)^{2},
\end{equation}
which contains no linear term and a fixed unit quadratic coefficient. Any
measurable nonzero linear term, or any adjustable parameter introduced to
restore agreement, falsifies the construction. Comparison of $G(M_Z)$ with
$G_N$ enters only as an \emph{a posteriori} closure test rather than as an
input or calibration condition. Using current PDG/CODATA inputs, the closure
ratio
\begin{equation}
  Z_G \equiv \frac{\Omega(M_Z)}{\alpha_G^{(\mathrm{pp})}}
  \simeq 1.0937,
\end{equation}
corresponding to a $+9.37\%$ excess of the SM-derived anchor over the
proton--proton reference, or equivalently
\begin{equation}
  Z_G^{-1} \equiv \frac{\alpha_G^{(\mathrm{pp})}}{\Omega(M_Z)}
  \simeq 0.9143 \;(-8.57\%),
\end{equation}
shows percent-level consistency. The leave-one-out forecast
\begin{equation}
  \hat\alpha_s^\star(M_Z) = 0.1173411 \pm 1.86\times10^{-5}
\end{equation}
lies within $\sim 0.7\sigma$ of the PDG world average. These numerical
statements are postdictions of a fixed, parameter-free construction rather
than results of fitting or tuning.
Uncertainties arise solely from
experimental determinations of the electroweak couplings and from quoted
PDG/CODATA covariance structure; no theoretical or model-induced systematic
terms are introduced. The static construction is robust under higher-loop
corrections: the integer kernel and metric softness are properties of the
one-loop structure, but their combined alignment is numerically stable under
currently known variations of the input pins.

The present work is restricted to equilibrium or quasi-static configurations of
$\Xi(x)$ and does not attempt to specify a dynamical evolution law, identify
stress--energy sources for $\delta\Xi$, or characterize non-equilibrium
propagation or causal structure. These questions require extensions beyond the
static framework but leave its fixed internal ingredients unchanged. Natural
next steps include: (i)~a dynamical evolution equation for $\Xi$ away from
equilibrium, (ii)~identifying physical generators of $\delta\Xi$, and
(iii)~relating curvature response to stress--energy transport. These topics are
reserved for a separate follow-up manuscript provisionally titled
\emph{GEOMETRY~II: Dynamic Alignment and Stress--Energy Response}.
GEOMETRY~II will preserve all equilibrium pins and all integer/metric
structures established here.

Taken together, the results indicate that the Standard Model contains sufficient
internal algebraic and geometric structure to define a gravitational
normalization and parity-even curvature response without new degrees of
freedom at equilibrium. The framework is therefore best interpreted as a
Standard-Model-anchored mechanism consistent with GR, with empirical validation
dependent solely on the experimental tests stated in
Section~\ref{sec:predictions}. No auxiliary assumptions or tunable extensions
are available within the static sector.

\section{Conclusion}
\label{sec:conclusion}

This work identifies a Standard Model mechanism that fixes the gravitational
normalization at $\mu = M_Z$ using established gauge-sector structure, with no
new fields, tunable functions, or free parameters. A unique primitive integer
left-kernel of the one-loop decoupling matrix selects the depth direction
$\chi = (16,13,2)$ in log--coupling space, and the positive-definite
Fisher/kinetic metric independently selects the same soft eigenmode. Their
alignment defines the depth coordinate $\Xi = \chi \cdot \hat\Psi$ and the
dimensionless electroweak anchor
$\Omega = \hat\alpha_s^{16}\hat\alpha_2^{13}\hat\alpha^{2}$, yielding an
SM-derived gravitational normalization $G(M_Z)$. This normalization is therefore
a pure consequence of internal SM structure and dimensional consistency, not a
fitted parameter or a modification of GR.

An even, parity-preserving curvature gate $\Pi(\Xi)$ promotes this equilibrium
normalization to a spacetime-dependent coupling
\begin{equation}
  G(x) = G(M_Z)\,\Pi(\Xi(x)),
\end{equation}
while preserving the massless, luminal tensor sector of General Relativity.
Near equilibrium, the curvature response is fixed and strictly quadratic,
\begin{equation}
  \frac{\Delta G}{G}
  =
  \left(\frac{\delta\Xi}{\sigma_\chi}\right)^2,
\end{equation}
with a provably absent linear term. This absence provides a direct experimental
falsifier requiring no parameter adjustment, renormalization choice, or model
tuning. Consistency with the measured Newtonian coupling $G_N$ enters only as an
\emph{a posteriori} closure test, not as an input or calibration. With current
PDG/CODATA pins, the closure ratio $Z_G \simeq 1.0937$ and the leave-one-out
prediction $\hat\alpha_s^\star(M_Z) = 0.1173411 \pm 1.86\times10^{-5}$ show
percent-to-few-percent sensitivity without free parameters. All uncertainties
derive solely from experimental pins, not from theoretical degrees of freedom.

The present analysis applies to equilibrium or quasi-static configurations and
does not address non-equilibrium dynamics, sourcing of $\delta\Xi$, or
stress--energy evolution. These questions lie beyond the static framework but
can be pursued without altering the fixed equilibrium ingredients established
here. The integer structure, metric softness, and parity-even curvature gate are
rigid at equilibrium and provide the foundation on which any dynamical extension
must be built.

Taken together, the results indicate that the Standard Model contains sufficient
internal algebraic and geometric structure to define a gravitational
normalization compatible with GR, reframing the role of $G_N$ from a purely
external parameter to a quantity that can be tested against a theoretically
derived electroweak-scale value.

\paragraph*{Acknowledgments}
This work was conducted independently with no external funding. The author
thanks PDG, CODATA, Overleaf, and the open-source Python ecosystem for publicly
accessible tools and data. An AI-assisted writing tool (OpenAI ChatGPT) was used
for language optimization and workflow organization only; all scientific
content, calculations, and claims are the sole responsibility of the author.
The author declares no competing interests.

\paragraph*{Data availability}
All reproducibility materials are archived on Zenodo
\cite{demasi_gage_repo_v1_0_0_2025}
(\texttt{GAGE\_repo v1.0.0}, DOI:
\href{https://doi.org/10.5281/zenodo.17537647}{10.5281/zenodo.17537647}),
including pins, scripts, figure data, and build manifests.
\emph{Build artifacts (SHA--256):}
\texttt{results.json} = 08f0371b31de\ldots c7cd5edc;
\texttt{metric\_results.json} = e0e3bee8a70c\ldots b9b251b6451;
\texttt{stdout.txt} = 0f232a0be6f8\ldots 6c7cd5edc.
Additional materials are available from the author upon reasonable request.

\paragraph*{Outlook}
Future work will examine how the even-gate symmetry extends to dynamical and
spectral sectors, including the time-evolution operator, alignment-driven
transport, and curvature spectrum. If experimentally validated, the GEOMETRY
program may provide a continuous link from Standard-Model information geometry
to the equilibrium, dynamical, and spectral structure of gravitation.

\bibliographystyle{iopart-num}
\bibliography{geo_cqg_refs}
\end{document}








